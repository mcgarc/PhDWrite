%%%%%%%%%%%%%%%%%%%%%%%%%%%%%%%%%%%%%%%%%%%%
%                 Preamble                 %
%%%%%%%%%%%%%%%%%%%%%%%%%%%%%%%%%%%%%%%%%%%%


\documentclass[a4paper]{article}

\title{ESA research plan: Molecule chip}
\author{Cameron McGarry\\ 
Supervisors: Mike Tarbutt and Ben Sauer}

\pagenumbering{gobble}

\begin{document}

\maketitle

As described in the main ESA report, we have produced a design for a molecule
chip trap. This includes a main science chip, a multi-layer microfabricated chip
hosting trapping wires and microwave guides, and a holding-chip to be mounted on
a vacuum chamber flange. The next stage is to source and vacuum-test various
components that will be required in these designs. Once these are complete,
attention can be turned to manufacturing the science chip (to be undertaken at
the London Centre for Nanotechnology) and assembling the holding-chip and
flange inside the chip chamber.

Once assembled, the chip assembly can be tested under ultra-high vacuum. This
will not require the presence of any molecules, so can be conducted
independently of the molecule source chambers discussed in the report. We will
need to ensure that we are able to achieve the trapping currents predicted in
the main report, and that the microwave guides function as expected. It is
possible that at this stage we will need to alter the chip design, but the
modularity of the holding and science chips will allow us to swap out test
different designs with relative ease.

We will then go on to integrate the chip chamber with the existing CaF
experiment. Molecules will be loaded onto the chip, and initial tests will
involve fluorescence imaging of molecules held on the chip and then dropped.
We will be able to optimise the loading procedure and then continue to perform 
microwave spectroscopy using on-chip microwaves.

Further goals include loading the chip with a single molecule from optical
tweezers, building chips with multiple traps, and implementing a microwave
resonator on a superconducting chip.

\end{document}

