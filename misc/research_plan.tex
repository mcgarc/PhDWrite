%%%%%%%%%%%%%%%%%%%%%%%%%%%%%%%%%%%%%%%%%%%%
%                 Preamble                 %
%%%%%%%%%%%%%%%%%%%%%%%%%%%%%%%%%%%%%%%%%%%%


\documentclass[a4paper]{article}

\title{ESA research plan: Molecule chip}
\author{Cameron McGarry\\ 
Supervisors: Mike Tarbutt and Ben Sauer}

\pagenumbering{gobble}

\begin{document}

\maketitle

As described in the main ESA report, we have produced a design for a molecule
chip trap. This includes the main science chip, a multi-layer microfabricated
chip hosting trapping wires and microwave guides, and an accompanying
holding-chip, to be mounted on a vacuum chamber flange. The next stage of the
project is to source and vacuum-test the various components that will be
required in these designs. Once this is complete, attention can be turned to
manufacturing the science chip (to be undertaken at the London Centre for
Nanotechnology) and assembling the holding-chip and flange inside the chip
chamber.

Once constructed, the entire chip assembly can be tested under ultra-high
vacuum. This will not require the presence of any molecules, so can be conducted
independently of the molecule source chambers discussed in the report. We will
need to ensure that we are able to achieve the trapping currents predicted in
the main report, and that the microwave guides function as expected. It is
possible that at this stage alterations to the chip design will be required, but
the modularity of the holding and science chips will allow us to swap out test
different designs with relative ease.

We will then go on to integrate the chip chamber with the existing CaF
experiment. Molecules will be loaded onto the chip, held for a period of time
and then dropped. The falling cloud of molecules can be seen with fluorescence
imaging to determine its size and temperature. We will be able to optimise the
loading procedure and then continue to perform microwave spectroscopy using
on-chip microwaves, with the goal of observing the same coherence times as have
already been achieved in free space.

Further goals include loading the chip with a single molecule from optical
tweezers, building chips with multiple traps, and implementing a microwave
resonator on a superconducting chip.

\end{document}

