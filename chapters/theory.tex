\cm{Intro spiel}

\section{Molecule theory}

\cm{TODO}

\section{Chip trap theory}

\cm{TODO}

\section{Quantum optics}

It is instructive to begin with a discussion of the theory of quantum optics.
In this section we will focus on a general model, mostly considering a simple
two-level system interacting with light. Since this is a broad field, I will
focus the discussion only on the details that are directly used in this thesis.
Namely the calculation of the scattering rates for molecules in a light field
and the coupling of molecules to a light field in a resonator.

\cm{Do I also need to discuss the normal mw spectroscopy in free space?}

\subsection{Scattering rates}

\cm{To come later}

\subsection{Coupling with a resonator}

A key part of the motivation for a molecule chip trap is the idea that
integrated microwave guides can be used to couple photons to the rotational
states \cm{transitions?} of the molecules. Of particular interest is the idea
that a high-$Q$ resonator can be used to perform this coupling, leading to the
ability to perform sideband-cooling to the motional ground state, state readout
and coupling between individually-trapped molecules~\cite{Andre2006}. This is
similar to techniques used in atom chips~\cite{Treutlein2008} and for optical
resonators coupling to atomic energy levels~\cite{}.
%TODO can do numerous cites here

We will later see that for our purposes, the coupling of a single molecule and
the microwaves can be treated as the coupling of a two-level system to a quantum
mode of a cavity field. The canonical description of such a system is given by
the familiar Jaynes-Cummings Hamiltonian (JCH) in the rotating wave
approximation~\cite{gerry_knight_2004}
% TODO Note that s_z isn't defined here, will I define it above or have to add?
%
\begin{equation}
  H_\text{JC} = \hbar\omega_c a^\dagger a + \hbar \omega_0 s_z +
  \frac{\hbar\Omega}{2}(a^\dagger \sigma_- + a\sigma_+)
  \label{theory:eqn:JCH}
\end{equation}
%
where $a$ ($a^\dagger$) is the annihilation (creation) operator of the photons,
$\Omega$ is the Rabi frequency of the interaction, and $\sigma_\pm =
(\sigma_x \pm i\sigma_y)/2$ are the raising and lowering operators of spin. The
detuning of the cavity resonance from that of the spin is $\Delta = \omega_0 -
\omega_c$. The system is shown in \mysubfigref{theory:fig:JCHstates}{a}.

\begin{figure}
  %\includegraphics{}
  \cm{Part (a) is image of JCH system, (b) is state manifold without coupling
  or detuning, (c) is state manifold with coupling (d) adds detuning (similar
  to Bohi fig 1)}
  \caption{\cm{TODO}}
  \label{theory:fig:JCHstates}
\end{figure}

We denote the ground (exicted) state of the molecule as $\ket{g}$ ($\ket{e}$)
state. The light field state can be taken to be a Fock state ($\ket{n \in
\mathbb{Z}}$). Note that the final term in equation~\ref{theroy:eqn:JCH}) has
the effect of exciting the ground state while absorbing a photon
($\ket{g}\ket{n} \leftrightarrow \ket{e}\ket{n-1}$) or lowering the excited state
and releasing a photon ($\ket{e}\ket{n} \leftrightarrow \ket{g}\ket{n+1}$).

Following the procedure in \inlineref{gerry_knight_2004}, we can see that this
mixing of the states results in a shift of the energy levels to create the
dressed states
%
\begin{align}
  \ket{+, n} &= \cos\Phi_n \ket{g}\ket{n} + \sin\Phi_n \ket{e}\ket{n+1} \\
  \ket{-, n} &= -\sin\Phi_n \ket{g}\ket{n} + \cos\Phi_n \ket{e}\ket{n+1}
\end{align}
%
with
%
\begin{equation}
  \tan(2\Phi_n) = \frac{\Omega\sqrt{n+1}}{\Delta}
\end{equation}
%
and having shifted energies
%
\begin{equation}
  E_{\pm, n} = (n+1)\hbar\omega_c \pm \frac{\hbar}{2}\sqrt{\Omega^2(n+1) +
  \Delta^2}.
  \label{theory:eqn:JCHenergies}
\end{equation}
%
It is useful to consider the manifold of states as depicted in
\myfigref{theory:fig:JCHstates}.  Note that in the limit of no coupling
($\Omega = 0$) and no detuning ($\Delta = 0$) the energies are that of the bare
states, and $\ket{g}\ket{n+1}$ is degenerate with $\ket{e}\ket{n}$, as in part
(b) of the subfigure. Introducing coupling ($\Omega \neq 0$) lifts this
degeneracy, as in part (c). When the detuning is non-zero ($\Delta \neq 0$)
there is additional offset due to the second term in
\myeqref{theory:eqn:JCHenergies}, see part (d) of the figure.

\cm{Now justify need to be in strong-coupling r\'egime.}
