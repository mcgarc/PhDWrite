\ph{A cloud of cold molecules will be produced by the methods either discussed
above, or in another chapter somewhere... Referencing the relevant CCM papers.}

\ph{Ongoing experiments in capturing \CaF{} with an optical tweezer} do not
take place in the same chamber where the MOT is produced. Instead, a cloud of
ultracold molecules are produced as above, and are then transported to an
auxilliary chamber by means of a magnetic transport trap (MTT) similar to the
one described in \inlineref{Lewandowski2003}. This is formed of a pair of
anti-Helmholtz coils mounted on a transport stage external to the chamber. By
ramping down the current in the internal trapping coils and ramping up the
current in the MTT, molecules can be transfered, into the MTT. The stage can
then be used to transport the molecules into the tweezer chamber. We have
extended this setup to include an additional downstream chamber for experiments
with the chip trap, as shown in \myfigref{experiment:fig:MTTsetup}.
%
\cm{Will likely need more discussion of transport in the thesis. For one thing
it seems like it does cause a bit of heating for tweezers at the moment.}

As well as transporting \CaF{}, we are also able to transport \esRb{} (either
mixed with \CaF{} or alone). We can also use the infrastructure of the other
experiments to support the chip, \cm{especially the dipole trap in the
tweezer chamber as discussed somewhere else}. 

\begin{figure}[ht]
  \begin{overpic}[width=0.25\textwidth]{figs/vacuum_setup.pdf}
    \put(1,-5){Chip chamber}
    \put(34,-5){Tweezer chamber}
    \put(73,-5){MOT chamber}
  \end{overpic}
  \includegraphics[width=0.25\textwidth]{./figs/existing_chambers.png}
  \caption{
    \cm{Need to do a proper CAD for my chamber or similar cartoon to replace
    these.}
    The MOT and tweezer chambers for existing experiments. The magnetic
    transport trap (MTT) is used made up of a pair of transport coils arranged
    in anti-Helmholtz configuration. These generate a quadrupole field, which
    will move with the coils on a translation stage, allowing the transfer of
    molecules trapped in the MOT chamber into the tweezer chamber. The chip
    chamber will be positioned further downstream (left) of the tweezer chamber.
    This CAD was undertaken by Kyle Jarvis.
  }
  \label{experiment:fig:MTTsetup}
\end{figure}

Once inside the chip chamber, the molecules can be loaded from the MTT onto the
chip by the process described in chapter~\ref{design}.
