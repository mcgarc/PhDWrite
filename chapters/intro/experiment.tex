\ph{A cloud of cold molecules will be produced by the source described
above...}

The cloud will then be transferred into a magnetic transport trap (MTT) formed of
a pair of anti-Helmholtz coils external to the chambers. The dipole trap will be
switched off and the current in the coils ramped up to form a quadrupole trap.
The coils are mounted on a transport stage, such that they and hence the centre
of the quadrupole, can be moved across to neighbouring chambers. The trapped
cloud of \CaF{} and \esRb{} will be transported in the magnetic trap.
The MTT has been designed for loading into a neighbouring tweezer experiment,
however this will be extended to allow loading onto the chip. The MOT, MTT and
tweezer experiment are shown in \myfigref{experiment:fig:MTTsetup}. 

\begin{figure}[ht]
  \includegraphics[width=0.8\textwidth]{./figs/existing_chambers.png}
  \caption{
    \cm{Need to do a proper CAD for my chamber.}
    The MOT and tweezer chambers for existing experiments. The magnetic
    transport trap (MTT) is used made up of a pair of transport coils arranged
    in anti-Helmholtz configuration. These generate a quadrupole field, which
    will move with the coils on a translation stage, allowing the transfer of
    molecules trapped in the MOT chamber into the tweezer chamber. The chip
    chamber will be positioned further downstream (left) of the tweezer chamber.
    This CAD was undertaken by Kyle Jarvis.
  }
  \label{experiment:fig:MTTsetup}
\end{figure}

The chip flange will be mounted in a third chamber downstream from the MOT and
tweezer chambers. The flange is to be aligned such that the cloud can pass above
the surface with good clearance, entering on the side of the holding chip opposite
the flange. A macroscopic trap, aligned with the centre of the MTT, can then be
generated by the onboard U-wire and surrounding bias coils. This intermediary
trapping stage will be used to align the cloud with those on the science chip.

The cloud can be brought closer to the surface of the chip by a series of
current ramps in the bias coils and chip wires. By adiabatically changing the
bias fields and currents it is possible to transform the trapping potential
without ever distorting it to the point that it is no longer a trap.

A ramp is performed by linearly changing the trapping currents (those in the
Z-wires and bias coils). The time over which a ramp should occur is expected to
be on the order of tens of microseconds, so that the molecules experience an
adiabatic change of the potential.~\cite{Boehi2009} The proposed series of
trapping stages for loading into the dimple is shown in
\mytabref{experiment:table:loading}, along with required trapping currents and
bias fields.

\thesis{Temperature/ compression? Proportion of molecules we expect to be able
to keep? Also just realy need to review this. Often don't need an x bias and
$B_0$ is basically always zero. Also qudrupole trap depth.}

