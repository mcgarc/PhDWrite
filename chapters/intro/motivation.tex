Cold atoms and molecules systems are promising for the implementation of new
quantum technologies. They have been used for devices in fields of
measurement~\cite{PhysRevLett.120.103201}, simulation~\cite{Gross995} and testing of
fundamental physics~\cite{DeMille990}. However, one of the main disadvantages
when compared to, for example, superconductor~\cite{Wallraff2004} or pure
optical~\cite{Browne2017} systems is the lack of scalability and
miniaturization.~\cite{nielsenandchuang}

In 1995, Weinstein and Libbrecht proposed an experiment for
studying cold atoms in environments with extremely high magnetic
gradients.~\cite{PhysRevA.52.4004} These gradients can be obtained by using
microfabricated wires in combination with a biasing field to form a trap above
a surface. Not only does this lead to an interesting environment for studying
new physics, but it introduces miniaturization into the cold-atom
toolkit.~\cite{2011Ac} There is also potential for integration of atomic or
molecular systems on a chip with other chip-based quantum
systems.~\cite{2011Ac, Kubo2011}

The first atomic chip trap was demonstrated four years later by Reichel et
al.~\cite{Reichel1999}. Their chip trap was shown to be a robust environment for
atomic experiments. The field of atom chips has since expanded~\cite{2011Ac},
with use as an architecture for creation of Bose-Einstein condensates being of
particular interest to many physicists.~\cite{Ott2001}

Although laser cooling of atoms has been commonplace since the late
20\textsuperscript{th} century, cooling of molecules has taken longer to develop
due to their more complex internal energy structure. However, this same rich
energy structure and other features of the molecules could be exploited to
produce new interesting technologies.~\cite{Tarbutt2018} We are now able to
produce \CaF{} molecules cooled below the Doppler limit~\cite{Truppe2017},
coherently control the states of such molecules~\cite{Williams2018}
and load molecules into a tweezer array.~\cite{Anderegg2019} These developments
open the door to the realization of a molecule chip.

A proposal by Andr\'e et al. in 2006~\cite{Andre2006} suggested integrating an
electrostatic trap with a microwave resonator on a chip. A single trapped
molecule would experience a strong coupling between its rotational states and
photons in the resonator, which could be used for long-range coupling of
molecular qubits, coherent control and measurement. Coupling to other quantum
architectures via flying photon qubits may also be
possible.~\cite{PhysRevLett.92.063601}

% I would like to find a citation for this, but Mike has previously said (lit
% review email) that it isn't necessary.
Microwave transitions in atoms could have been suggested, however these are
magnetic dipole transitions between hyperfine states. The coupling from an
electric dipole transition between rotational states of a molecule is orders of
magnitude stronger. 

In the following subsection I will further discuss the field of laser cooling of
atoms and molecules. The rest of the report will focus on chip trapping, the
design of our molecule chip for \CaF{} and the outlook for the remainder of the
project.
