Cold atomic and molecular systems are promising architectures for the
implementation of new quantum technologies. They have been used for devices in
fields of measurement~\cite{PhysRevLett.120.103201}, simulation~\cite{Gross995}
and testing of fundamental physics~\cite{DeMille990}.  In particular,
experiments using microscopic traps built on chips have proven to be a robust
way of studying high-density clouds of atoms~\cite{Reichel1999, Ott2001} which
can be shown to interact with the surface~\cite{} or couple to electromagnetic
fields also carried on the chip surface.~\cite{Treutlein2008, Hinds??}

A proposal by Andr\'e et al. in 2006~\cite{Andre2006} suggested integrating an
electrostatic trap with a microwave resonator on a chip. The trap could be used
to confine a single molecule in the strong field region of the mircorwace
field, allowing strong coupling between its rotational states and photons in
the resonator. This could be exploited for long-range coupling of molecular
qubits, coherent control and measurement. Coupling to other quantum
architectures via flying photon qubits may also be
possible.~\cite{PhysRevLett.92.063601}

Recent developments in the field of ultracold molecules suggest that calcium
monofluoride would be an interesting case study for a molecular chip trap,
analogous to the atomic ones which have already been developed. \CaF{} has been
directly laser-cooled to \SI{50}{\micro\kelvin} ~\cite{}, is capable of being
trapped magnetically, with cohrence times between rotational states
being as long as
\SI{0.61}{\milli\second}~\cite{Blackmore_2018}. 

In the centre for cold matter, we have been working on developing a chip trap
for \CaF{}, using magnetic traps instead of the electrostatic ones as in the
Andr\'e proposal. In an earlier report, I described a prototype chip design,
including posible methods for loading from our source of ultracold molecules,
and how microwave guides could be integrated with the trapping wires.

This document reports on the progress of this project since submission of the
that report.  In chapter~\ref{design} I will report on the outcome of
simulations that have helped to further refine the design for the chip trap, in
chapter~\ref{fab} I will describe the fabrication methods that have been used
to build a prototype chip. Finally in chapter~\ref{outlook} I will discuss the
outlook for the remainder of the project and beyond.

\subsection*{Magnetic chip traps}

For the sake of completeness, the remainder of this chapter provides a brief
overview of the operating principles of a magnetic chip trap. This restates the
more complete discussion that was presented in the earlier report. More details
can be found in, for example, \inlineref{2011Ac}.

The basic principle of a magnetic wire trap is to superimpose the magnetic
field of a wire with a homogenous bias field, creating a minimum. This is
illustrated in \myfigref{chiptraps:fig:reicheltrap}, where a two dimensional
trap is formed from a straight wire carrying current $I$, at a height $h$ from
the trap. 
The bias field's direction must oppose the
direction of the wire field. We can immediately observe from basic
electromagnetism that the bias field relates to the current and the height
as~\cite{Reichel1999}
%
\begin{equation}
  \tilde{B} = \frac{\mu_0 I}{2\pi h}.
  \label{into:eq:trapbias}
\end{equation}

As with other magnetic traps, atoms (or molecules) can be optically pumped into
weak-field seeking states such that they favourably occupy the region surround
the trap minimum.~\cite{Metcalf1999} The depth of the trap is determined by the
strength of the bias field and the magnetic moment of the particle ($\mu$),
%
\begin{equation}
  U_\text{depth} = \mu \tilde{B}. 
  \label{intro:eq:trapdepth}
\end{equation}

\begin{figure}
  \centering
  \includegraphics[width=0.75\textwidth]{./figs/chiptraps/reicheltrap.pdf}
  \caption{The combination of the magnetic field due to a straight wire and a
  constant bias are shown to produce a zero, forming a two-dimensional trap for
  weak field seekers. This is the basic principle underlying all magnetic wire
  traps. Reproduced from Reichel et al.~\cite{Reichel1999}
  }
  \label{intro:fig:reicheltrap}
\end{figure}

There are numerous ways to introduce confinement in the axial direction. We
will be primarily concerned with the Z-trap and the U-trap. In these variations,
a bent wire is used instead of a straight one, as is shown in
\myfigref{intro:fig:reichel_UZ}. The magnetic field then provides some confinement in the
axial direction, although the depth is not as strong as it is in the other
directions. Note that the U trap forms a quadrupole trap (where the minimum
field is zero) and the Z trap forms a Ioffe-Pritchard trap (where the field
minimum is non-zero).~\cite{} We refer to the length of the centre section as
the axial-length.

\begin{figure}
  \centering
  \includegraphics[width=0.75\textwidth]{./figs/chiptraps/reichel_UZ.png}
  \caption{
  The trapping field of a U and a Z trap are shown in the left and right
  columns respectively. Bending a straight wire as in row (a) introduces weak
  confinement along the axis of the trap.  The $x$ and $z$ components of the
  field along the trap axis are shown in rows (b) and (c) respectively. This
  figure is reproduced from \inlineref{2011Ac}. 
  }
  \label{intro:fig:reichel_UZ}
\end{figure}

As discussed above, atom chips have been used in numerous experiments, for
example as clocks~\cite{RAMIREZMARTINEZ2011247} or for robust production of Bose-Einstein
condensates~\cite{Ott2001}. Atom chips have also been used to bring magnetically
trapped atoms into with microwave fields in on-chip wave
guides~\cite{Treutlein2008}, in close analogy with what we aim to accomplish
for molecules. The recent improvements in cold molecular science discussed
above mean that realisation of a molecule chip is now within the realms of
possibility.
