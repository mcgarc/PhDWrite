\cm{Use intro to couch thesis in terms of what others have already done}

Opening spiel
\cite{Andre2006}

\section{Ultracold atoms}

\cm{
(also bring in idea of UHV here somewhere)
- cf ions \cite{RevModPhys.62.531} (this is a nice review of chips \cite{Romaszko2020}, keep emphasis on neutral atoms however
}

Laser cooling of atoms is a well-explored field, with the 1997 Nobel Prize in
Physics having been awarded to Cohen-Tannoudji, Chu and Phillips for
`development of methods to cool and trap atoms with laser
light'~\cite{RevModPhys.70.721}. The subject is reviewed in various texts, with
the canonical reference being Metcalf and van der Straten~\cite{Metcalf1999}.
Here I will give a brief qualitative overview of the topic and its many
applications in modern physics, which include quantum: information,
communication and metrology. Further details on these cooling mechanisms are
given in section~\ref{theory:cooltrap}.

In laser slowing an atom's velocity is reduced in one direction by the absorption and
subsequent re-emission of a photon from a laser beam, that is the scattering of
the photon by the atom. The emission of the
photon has no preferred direction, and so over a number of these scattering events there
is a net reduction of the atom's velocity in the direction of the laser
beam~\cite{PhysRevLett.40.1639}. This principle of slowing with radiation
pressure was first observed by Wineland et al.~\cite{PhysRevLett.40.1639} in
ions, but one of the key goals of laser slowing was to sufficiently reduce the
velocity of a beam of neutral atoms (produced in, for example, an oven) to
sufficiently low velocities for trapping. To achieve this it is essential to
account for the change in Doppler shift that occurs as the atom's velocity is
reduced. Phillips and Metcalf~\cite{PhysRevLett.48.596} were able to achieve this by
the Zeeman slowing technique, where a changing magnetic field across the
beamline is used to ensure the relevant transition is always resonant with the
light. Later the same group would achieve the same effect by chirping the
frequency of the light~\cite{Prodan1984}, so that the light is sufficiently
red-shifted to be resonant with the atoms as they slow down.

Magnetic trapping of the atoms soon followed, with the slow atomic beam
directed into a quadrupole field, generated by current-carrying coils, the
atoms could be pumped into a weak-field seeking state, and trapped in the field
minimum~\cite{PhysRevLett.54.2596}. The next key development was the
implementation of the magneto-optical trap (MOT) where atoms were confined by a
quadrupole magnetic field and red-detuned laser light incident from every
direction. This scheme is designed so that atoms that are away from the trap
centre experience a Zeeman shift that brings them into resonance with the
light. The polarisations are then chosen so that the atom will preferentially
scatter photons from the beam that will result in a restoring force towards the
trap centre. The benefit of this over the magnetic trap being that as well as
a position-dependent restoring force, there is a velocity-dependent damping to
reduce the atoms' temperature. This was first implemented by by Raab et
al.~\cite{PhysRevLett.59.2631} in \Na{}, although subsequently many different
species have been laser-cooled and confined in a MOT \cm{CITE?}.

Due to the stochastic nature of Doppler-cooling, there is a fundamental limit
on the lowest temperature that can be achieved using just the Doppler slowing
mechanism. This can be reached by turning the magnetic field of the MOT off,
leaving just the red-detuned laser beams to form a molasses. Early
implementations of this scheme found that in fact anomalously low temperatures
were reached in a molasses~\cite{} % https://journals.aps.org/prl/abstract/10.1103/PhysRevLett.61.169
This was eventually explained by Dalibard et al.~\cite{Dalibard:89} as being
due to sub-Doppler cooling mechanisms, which rely on polarisation gradients
set up in the light field. This reduced the atom temperature to the so-called
recoil limit, with this being related to the energy imparted by a single photon
scatter.

It is possible to cool beneath even the recoil limit by evaporative cooling,
where the hottest atoms are ejected from the trap. The remaining atoms
thermalise at a lower temperature than the original cloud. This technique was
employed by Anderson et al.~\cite{Anderson198} to cool \esRb{} to sufficiently
low temperature and density to form a Bose-Einstein condensate (BEC). Another
now-popular method of producing high-density atomic clouds is the optical
dipole trap~\cite{Chu1986}, where atoms are confined by intense,
far-off-resonant light which induces an electric dipole in the atom. The
resulting energy shift is proportional to $-E^2$, where $E$ is the strength of
the electric component of the light field. The atoms are therefore confined in
the region where the light's intensity is strongest, that is around the beam
waist. For a sufficiently tight trap, an optical tweezer is formed, where a
single atom can be trapped and easily manipulated by controlling the
light~\cite{Schlosser2001}.  A series of tweezer traps can be used to form a
lattice of traps~\cite{Schlosser2001}, with individually-trapped atoms able to
interact with each other by \cm{some force, CITE}. A lattice can also be formed
in a standing wave optical trap (SWOT), where light reflected from a surface
creates a series of local field maxima for trapping of atoms~\cite{Wu2017}.

With the abilitity to reliably create low-temperature atomic clouds confined in
a trap; and, in the case of tweezers, individually trapped atoms, comes the
ability to create quantum devices for simulation, communication and metrology,
with promises of future applications in quantum computing. A typical example is
the use of \cm{strontium} atoms held in a lattice for precise measurement of the
second~\cite{PhysRevX.8.021036}, but there are many other examples of atoms being used for
precisely sensing, for example, acceleration~\cite{} including gravity~\cite{}.
% TODO cite something and Birmingham?
Cold atoms can also be employed to exploring fundamental physics, such as
\cm{search for dark matter?}. Future experiments plan to use cold atoms to
measureme of gravitational waves in a novel form of \cm{grav wave telescope
  that fills a hole not covered by other plans, and so could actually be useful
  if it worked...}
\cm{coherent control e.g. simulation in lattice \cite{Schäfer2020}}

% miniaturisation e.f. grating MOT \cite{Nshii2013}
These experiments, and those like them have been made easier by technological
developments such as atom dispensers~\cite{}, which remove the need to cool
a hot beam of atoms, and vapour cells~\cite{}, which can entirely remove the
need for a complicated vacuum system. This has lead the way to miniaturisation
and scalability of cold atom experiments, which is an active field of research.
The complexity of these experiments can be further reduced by using, for
example mirror~\cite{} or grating~\cite{} MOTs, where some of the MOT light
beams are produced by reflection or diffraction from a surface. These are
strongly linked to the main focus of this thesis: the chip trap.

\section{Chip traps}

%\cite{2011Ac}

The chip trap was proposed as a mechanism for studying atoms in very
high-gradient magnetic traps~\cite{PhysRevA.52.4004}, but it has evloved into a
useful tool for miniaturised and robust cold atom experiments, which can be
integrated with, for example, microwave components, to form hybrid quantum
systems. Further details will be presented in section~\ref{theory:chips}, but
here I will give a brief overview of the operating principles and history of
atomic chip traps.

We have already discussed the idea of trapping atoms in a quadrupole magnetic
field, however in the above-described experiments the fields were generated by
macroscopic current-carrying coils. It is also possible to generate magnetic
traps from the field of a wire combined with a homogenous bias field. In a chip
trap experiment, the wires are positioned on the surface of a substrate, and
can be made very small (on the scale of a few microns) by the use of common
microfabrication techniques. Various microscopic atom traps were implemented,
for example in Ref.s~\cite{PhysRevLett.80.1634, PhysRevLett.81.5310,
Drindic1998} \cm{Introduce MPI (HAnsch/ Zimmerman) and Harvard (Drindic)} ,
followed by magnetic guiding using a two-dimensional chip trap in
\inlineref{PhysRevLett.82.2014} \cm{Innsbruk}. % The first chip traps operating
in three dimensions were reported by \cm{MPI group} al.~\cite{Reichel1999}
\cm{MPI}, who also implmenented a mirror-MOT for loading their traps. This was
shortly followed by devlopments in contolling and guiding atoms trapped above a
chip's surface~\cite{Folman2000} and the \cm{Folman: INsbruck}
%
atom chips trapping and guiding \cite{Folman2000}

Atom chips were found to be a robust tool for cold atom experiments, including
the production of BECs, with groups from \cm{Hansel group} and \cm{Ott group}
both reporting \cm{Rb? BECs} \cm{at the same conference} in 2001~\cite{Hansel2001,
dtt2001}. In 2004 it was demonstrated by the MPI group that coherence times 
of around \SI{1}{\second} could be achieved for \esRB{} atoms held in a chip
trap within \SI{100}{\micro\meter} of the surface. In this experiment microwave
fields were used to drive hyperfine tranisitons in the molecules, with the
microwave radiation delivered from an external source. This was soon extended to
experiments where the microwaves were delivered by on-chip microwave
components~\cite{Treutlein2008, Boehi2009}, with a view to develop a microwave
trap~\cite{} and perform quantum gates entirely on-chip. It was also shown that
the coherence was sufficient to implement an atomic clock on a
chip~\cite{Knappe2004}.
%
\cm{ramsey infterferometery?}
%
\cm{and later freq. standard \cite{RAMIREZMARTINEZ2011247}}

Michelson interfferometer \cite{Wang2005}
Long coherence times (58s) \cite{Deutsch2010}
superconducting (sc) atom chip \cite{Nirrengarten2006}
sc chip for BEC
(Roux, C., Emmert, A., Lupascu, A., Nir- rengarten, T., Nogues, G., Brune, M.,
Raimond, J.-M., and Haroche, S. (2008) Eur. Phys. Lett. 81, 56004.)
Meissner effect \cite{Cano2008}
Couple BEC to sc loop (Singh, M. (2009) Opt. Express 17, 2600)
...more superconducting \cite{Bernon2013}

INteraction with surface (Casmir-polder)
\cite{PhysRevA.72.033610}
\cite{PhysRevLett.98.063201}


Lower dimensional gasses
\cite{PhysRevLett.116.030402}
\cite{Hofferberth2007}
\cite{Yuen2015}

Squeezing BEC
\cite{PhysRevLett.105.080403}

Squeezing in cavity, Monica S-S (chip for preparation - useful example of experiment
easier) \cite{PhysRevLett.104.073604}

Similar vein...  Atom chip on ISS \cite{Frye2021}

\section{Ultracold molecules}


Cold molecules also well established, but younger
- \emph{Direct} (review: \cite{Tarbutt2018} first cooling with SrF \cite{Shuman2009}, followed by
slowing with E field \cite{Bethlem1999} (mention this because relevant later)
radiation pressure paper \cite{PhysRevLett.108.103002}) vs. indirect cooling
(producing RbCs Durham\cite{PhysRevA.89.033604}, I think this is the original indirect \cite{Moses2017} (can mention briefly if at all)  
- Direct cooling into storage rings (Thesis \cite{Crompvoets2005} review of mol beams \cite{vandeMeerakker2012})
- Longitudinal laser slowing of supersonic CaF \cite{PhysRevA.89.053416}
- SrF MOT \cite{Barry2014} and improved in ~\cite{PhysRevLett.116.063004}
- CaF buffer gas source (details, maybe not for here): \cite{Truppe2018} % Cite
info on TandF says 2018 but think it's 2017? Original bufgas source
\cite{Barry2011} or not?? I think this one is first? \cite{Maxwell2005}
- CaF slowing Truppe: \cite{Truppe2017a} Doyle: \cite{0953-4075-49-17-174001}
- CaF MOT Doyle: r.f. MOT \cite{PhysRevLett.119.103201}
- Subdoppler cooling \cite{Truppe2017}
- Deep cooling of CaF \cite{PhysRevLett.123.033202}
- CaF dipole traps (Doyle group)
- CaF trap and coherent control ~\cite{WilliamsMagnetic2018}
- CaF (and RbCs) coherent control ~\cite{Blackmore_2018}, long lived states
(Caldwell) ~\cite{PhysRevLett.124.063001}
- Non-destructive imaging of cloud \cite{PhysRevLett.121.083201}
- Tweezer array of CaF Doyle \cite{Anderegg2019}, undergoing coherent control
\cite{PhysRevLett.127.123202} (2021)
- Ongoing cooling investigation - transverse (similar to what has been done for
YbF \cite{Alauze2021}) and ZS proposal \cite{Fitch2016}, experiment \cite{PhysRevLett.127.263002}

Some problems:
- Harder to cool due to complex energy structure
- Typically lower density clouds than atoms (makes things more challenging and
e.g. evaporative cooling impossible)

Newer things, polyatomic MOTs (Doyle: \cite{Vilas2021}), which have some uses maybe? ~\cite{DoylePolyatomic2022}
collisions first: \cite{son2019collisional}, us \cite{Jurgilas2021, JurgilasPRL_2021}, next step sympathetic cooling

Useful features of molecules
- computing \cite{PhysRevLett.88.067901}
- Permanent electric dipole moment, long range interaction -> simulation
- Various novel ideas:
   qudits \cite{Sawant_2020}
   topological qubits: theory (Zoller) \cite{Micheli2006};  experiment ~\cite{Gross995}
- Interesting to study for Q. chemistry (see collisions above)
- spectrosocpy: eEDM
(latest ACME \cite{Andreev2018}, review \cite{ACMEreview})
- Long lived states (above) + strong mw coupling in rot states (segway to chip)


\section{Molecule chip trap}

Justify

back to idea above of long states and strong coupling to mws, promising
architecture \cite{Andre2006}

Meijer group
- warm molecules on chips, eg declerate \cite{Meek2008} and \cite{Meek2009}
(c.f. afore-mentioned \cite{Bethlem1999})
- imaging \cite{Marx2013}
- measuring and manipulating \cite{PhysRevA.92.063408} -- seems to be their
last one
(Note this is all still at very low density and at temperatures of tens of mK)
(I think the measurements are also destructive, whereas the Andre paper
promises readout too...)
% Remember I need to really sell that these are ultracold molecules on chip


ready to implement this in CaF with our experiment (phase space densities,
moreo promised with ODT)

\section{The structure of this thesis}
% TODO combine with above section?

This thesis will present the design and construction of a \CaF{} chip trap
experiment. In chapter~\ref{theory} we will introduce key background theory: 
laser-cooling of simple systems, the operation of chip traps, and the physics
of diatomic molecules. Next chapter~\ref{overview} gives an overview of the
existing \CaF{} experiment, as well as some particulars of the laser-cooling
methods used specifically for \CaF{}. Here we will also outline the new chip
experiment, and how it will integrate with the existing apparatus. We have
demonstrated the loading procedure for the trap with simulations, and these are
presented in chapter~\ref{sim}.

We microfabricated chip traps, as will be discussed in chapter~\ref{fab}, and
then loaded these into a vacuum chamber for initial testing. This included
testing that the current capcity of the trapping wires was as expected, that
ultra-high vacuum could be reached, and that it will be possible to image
molecules without too much background scatter created by the chip. These tests,
and a scheme to reduce the background scatter are discussed in
chapter~\ref{experiment}. Chapter~\ref{mws} describes how a microwave guide and
resonator can be implemented as part of the experiment, as well as the
behavoiur of a single molecule coupled to a microwave cavity. We will extend
this idea for ensembles of molecules coupled to a cavity in
chapter~\ref{squeeze}, where I propose a scheme to use a quantum non-demolition
measurement to create a spin-squeezed state in such an ensemble. Finally,
chapter~\ref{outlook} describes the future prospects for this project.
