% Could have mentioned ion chips. Reviews:
%\cite{RevModPhys.62.531}
%\cite{Romaszko2020} (mostly chips?)

One of the reasurring things about new experiments with ultracold molecules is
that a lot of what researchers may wish to accomplish has an
already-accomplished analogue in the well-established field of cold atoms. On
the whole, researchers of molecules have been able to foolow this existing
roadmap, from laser slowing, to magneto-optical trapping, to optical dipole
trapping; all of which will be discussed in the next section. A next obvious goal may
be evaporative cooling and the creation of a Bose-Einstein condensate, but
there is another convenient and useful tool of cold atom experiments whose
molecular parallel has yet to be realised.

This tool is the atom chip which, we will see below, has proven to be of great
use as a robust, compact and portable device for cold atom experiments. A
molecular counterpart could prove to be equally invaluable, and offer
improvements on existing atom chips, by allowing strong coupling to microwave
fields on the chip, including the realisation of a cavity quantum
electrodynamics system. Such a device was originally proposed by Andre et
al.~\cite{Andre2006}, as a `coherent all-electrical interface between polar
molecules and mesoscopic superconducting resonators.'

This chapter provides an outline as to why a molecule chip trap for ultracold
molecules is a desirable goal, starting from a review of cold atoms, to atom
chips and then cold molecules. I will then summarise the motivation for a
molecule chip and outline the structure of the thesis.

\section{Ultracold atoms}

Laser cooling of atoms is a developed field, with the 1997 Nobel Prize in
Physics having been awarded to Cohen-Tannoudji, Chu and Phillips for
`development of methods to cool and trap atoms with laser
light'~\cite{RevModPhys.70.721}. The subject is reviewed in various texts, with
the canonical reference being Metcalf and van der Straten~\cite{Metcalf1999}.
Here I will give a brief qualitative overview of the topic and its many
applications in modern physics, which include quantum: information,
communication and metrology. Further details on these cooling mechanisms are
given in section~\ref{theory:cooltrap}.

In a typical laser slowing experiment, the forward velocity of a beam of atoms
is reduced by resonant, counter-propagating laser light. An atom can scatter
a photon from the laser, upon re-emission the photon has no preferred
direction of travel. Over a large number of photon scatters this results in a
net reduction in the forward velocity of the atoms~\cite{PhysRevLett.40.1639}.
This principle of slowing with radiation pressure was first observed by
Wineland et al.~\cite{PhysRevLett.40.1639} in \Mg{} ions, but one of the key
goals of laser slowing was to sufficiently reduce the velocity of a beam of
neutral atoms (produced in, for example, an oven) to sufficiently low
velocities for trapping. 

To achieve this it is essential to account for the change in the resonant
frequency of the atoms due to the Doppler shift that occurs as the atom's
velocity is reduced. Phillips and Metcalf~\cite{PhysRevLett.48.596} were able
to achieve this for \Na{} atoms by the Zeeman slowing technique, where a
changing magnetic field across the beamline is used to ensure the relevant
transition is always resonant with the light. Later the same group would
achieve the same effect by chirping the frequency of the
light~\cite{Prodan1984}, so that the light is sufficiently red-shifted to be
resonant with the atoms as they slow down.  Such experiments must take place
under ultra high vacuum to prevent loss by collisions with a background gas.

Magnetic trapping of the atoms soon followed, with the slow atomic beam
directed into a quadrupole field generated by current-carrying coils. The atoms
could be pumped into a weak-field seeking state, and trapped in the field
minimum~\cite{PhysRevLett.54.2596}. The next key development was the
implementation of the magneto-optical trap (MOT) where atoms were confined by a
quadrupole magnetic field and red-detuned laser light incident from every
direction. This scheme is designed so that atoms that are away from the trap
centre experience a Zeeman shift that brings them into resonance with the
light. The polarisations are then chosen so that the atom will preferentially
scatter photons from the beam that will result in a restoring force towards the
trap centre. The benefit of this over the magnetic trap being that as well as a
position-dependent restoring force, there is a velocity-dependent damping to
reduce the atoms' temperature. This was first implemented by by Raab et
al.~\cite{PhysRevLett.59.2631}, also in \Na{}, although subsequently many
different species have been laser-cooled and confined in a MOT.

Due to the stochastic nature of Doppler-cooling, there is a fundamental limit
on the lowest temperature that can be achieved using just the Doppler slowing
mechanism. This can be reached by turning the magnetic field of the MOT off,
leaving just the red-detuned laser beams to form an optical molasses. Early
implementations of this scheme found that in fact, anomalously low temperatures
were reached in a molasses~\cite{Lett1988}. This was eventually explained by
Dalibard et al.~\cite{Dalibard:89} as being due to sub-Doppler cooling
mechanisms, which rely on polarisation gradients set up in the light field.
This reduced the atom temperature to the so-called recoil limit, which is
related to the energy imparted by a single photon scatter.

It is possible to cool beneath even the recoil limit by evaporative cooling,
where the hottest atoms are ejected from the trap. The remaining atoms
thermalise at a lower temperature than the original cloud. This technique was
employed by Anderson et al.~\cite{Anderson198} to cool \esRb{} to sufficiently
low temperature and density to form a Bose-Einstein condensate (BEC). Another
now-popular method of producing high-density atomic clouds is the optical
dipole trap~\cite{Chu1986}, where atoms are confined by intense,
far-off-resonant light. Atoms in such a trap are confined by the a.c.\ Stark
shift, which creates an attractive potential towards the region where the
light's intensity is greatest.

For a sufficiently tight optical dipole trap, an optical tweezer is formed,
where a single atom can be trapped and easily manipulated by controlling the
light~\cite{Schlosser2001}. A series of tweezer traps can be used to form a
lattice of traps~\cite{Schlosser2001}, with individually-trapped atoms able to
interact with each other by dipole-dipole interactions. A lattice can also be
formed in a standing wave optical trap (SWOT), where light reflected from a
surface creates a series of local field maxima for trapping of
atoms~\cite{Wu2017}.

With the ability to reliably create low-temperature atomic clouds (or even a
single atom) confined in traps and lattices, comes the ability to create
quantum devices for simulation, communication and metrology, with promises of
future applications in quantum computing. Coherent control of atoms in optical
lattices for simulation has been demonstrated~\cite{Schafer2020}, and optical
lattice clocks have become the cutting edged in
timekeeping~\cite{PhysRevX.8.021036}. There are numerous examples of atoms
being used for precise sensing, for example, acceleration~\cite{Chen2019}
including gravity~\cite{Stray2022}. Cold atoms can also be employed to
explore fundamental physics, such as the search for dark
matter~\cite{Wcislo2018}. Future experiments also plan to utilise cold atoms
for new gravitational wave detectors operating in frequency bands that are
inaccessible to existing detectors~\cite{Badurina_2020}.

These experiments, and those like them, have been made easier by technological
developments such as atom dispensers, which remove the need to cool
a hot beam of atoms, and vapour cells, which can entirely remove the
need for a complicated vacuum system. This has lead the way to miniaturisation
and scalability of cold atom experiments, which is an active field of research.
The complexity of these experiments can be further reduced by using, for
example pyramid~\cite{Lee:96}, mirror~\cite{Reichel1999, 4797887} or
grating~\cite{Nshii2013} MOTs, where some of the MOT light beams are produced
by reflection or diffraction from a surface. These are strongly linked to the
main focus of this thesis: the chip trap.

\section{Chip traps}

The chip trap was proposed as a mechanism for studying atoms in very
high-gradient magnetic traps~\cite{PhysRevA.52.4004}, but it has evolved into a
useful tool for miniaturised and robust cold atom experiments. These can be
integrated with, for example, microwave components, to form hybrid quantum
systems. Further details will be presented in section~\ref{theory:chips}, but
here I will give a brief overview of the operating principles and history of
atomic chip traps.

We have already discussed the idea of trapping atoms in a quadrupole magnetic
field, however in the above-described experiments the fields were generated by
macroscopic current-carrying coils. It is also possible to generate magnetic
traps from the field of a wire combined with a homogeneous bias field. In a
chip trap experiment, the wires are positioned on the surface of a substrate,
and can be made very small (on the scale of a few microns) by the use of common
microfabrication techniques. Various microscopic atom traps were pioneered by
by H\"ansch and Zimmermann at the Max Planck Institute of Quantum Optics (MPQ)
and T\"ubingen~\cite{PhysRevLett.80.1634, PhysRevLett.81.5310}, as well as in
Harvard~\cite{Drindic1998}. This was followed by work in Innsbruck, where \Li{}
atoms were guided using a two dimensional trap formed with a free-standing
wire~\cite{PhysRevLett.82.2014}. 

The first true chip traps, operating in three dimensions were reported by the
MPQ group~\cite{Reichel1999}, who also implemented a mirror-MOT for loading
\esRb{} into their traps. This was shortly followed by developments in
controlling and guiding \Li{} atoms trapped above a chip's
surface~\cite{Folman2000} from Innsbruck.
%
Atom chips were soon found to be a robust tool for cold atom experiments,
including the production of BECs, with groups from the Ludwig Maximilian
University of Munich and T\"ubingen both reporting \esRb{} BECs at the same
conference in 2001~\cite{Hansel2001, dtt2001}. Since then atom chips have been
used to investigate lower dimensional gasses, as reported in
\inlinerefs{PhysRevLett.116.030402, Hofferberth2007, Yuen2015}, and have found
uses in various other experiments as a convenient method of preparing cold
atomic clouds~\cite{PhysRevLett.104.073604}.
%
This includes experiments creating non-classical spin-squeezed states (see
chapter~\ref{squeeze}), implementing magnetic lattices~\cite{Gerritsma2007} and
performing interferometry~\cite{Wang2005}.

2004 saw the first implementation of an on-chip Michelson interferometer, based
on the creation, splitting and guiding of BECs all on a chip. In the same year,
prospect of integrated atom chip devices was fully realised at the National
Institute of Standards and Technology, where a compact and portable atomic chip
clock was developed~\cite{Knappe2004} based on the hyperfine transitions in
\Cs{}. This has since become a commercial product with uses across a variety of
fields, such as military and space science~\cite{RAMIREZMARTINEZ2011247}
including an atom chip experiment conducted on board the International Space
Station~\cite{Frye2021}.

Also in 2004, it was demonstrated by the MPQ group that coherence times of
around \SI{1}{\second} could be achieved for \esRb{} atoms held in a chip trap
within \SI{100}{\micro\meter} of the surface~\cite{Treutlein2004}. In this
experiment microwave fields were used to drive hyperfine tranisitons in the
molecules, with the microwave radiation delivered from an external source. This
was soon extended to experiments where the microwaves were delivered by on-chip
microwave components built on a two-layer device~\cite{Treutlein2008,
Boehi2009}, with a view to develop a microwave trap and perform quantum
gates entirely on-chip.

The work of the Munich group built towards coherent control of an atom coupled
to on-chip microwaves, however a more ambitious goal is the coupling of an atom
to a microwave cavity, which can lead to powerful control over the spin system
by leveraging the effects of cavity quantum electrodynamics. One requirement of
this is so-called strong coupling between the microwave photons and the atomic
transition, which is challenging due to the comparatively small transition
matrix element of the hyperfine transitions in the atoms, and the need for a
high quailty microwave cavity. The latter of these, as we will discuss in
chapter~\ref{mws}, requires the use of superconducting microwave components.

A superconducting atom chip is an attractive idea, since as well as the
potential for low-loss microwave components, sueprconducting wires can produce
deeper magnetic traps. The first such device was produced in first realised by
Haroche's group in Paris in 2006~\cite{Nirrengarten2006}, with a BEC on a
superconducting chip following in 2008~\cite{Roux2008}. Similar superconducting
wire traps were used to investigate the Meissner
effect~\cite{PhysRevLett.101.183006}, and subsequent development of these
superconducting atom chips lead to full quantum control of an atom cloud held
in place above a superconducting microwave resonator~\cite{Bernon2013}. Here
the resonator was designed to be off the atomic resonance, so that the internal
state of the trapped atoms were unaffected by its presence. Of additional note
in this paper is the loading scheme used for the chip -- an optical tweezer is
used to bring the atoms from a room-temperature environment to within
\SI{400}{\micro\meter} of the chip surface, before being transferred to
surface-based microtraps.

A later experiment~\cite{Hatterman2017} would see the coupling of atoms to the
microwave resonator, realising a hybrid atom-microwave system, and the
investigation of microwave cavity quantum electrodynamics. Rabi oscillations
between hyperfine transitions could be driven, but strong coupling was not, and
has not been achieved. It has been possible to couple atoms strongly to, for
example monolithic microresonators~\cite{Aoki2006}, but only in freefall, not whilst
trapped. Other efforts to reach this strong coupling regime come from Hogan's
group, who use Rydberg atoms, utilising the increased electric dipole moment of
the transition (typically a factor of 1000 larger than for ground-state atoms)
to achieve stronger coupling~\cite{PhysRevLett.124.193604}. Again, such
experiments are performed with atoms that are in flight, and in this case
travelling at supersonic speeds.

An improvement to such experiments would be to confine the atom close to the
microwave field, near the anti-node of the resonator to maximise the coupling
strength. However, another potential method is to use a molecule with high
electric dipole moment instead of a Rydberg atom. Recent advances in the
laser-cooling and trapping of diatomic molecules make these an exciting
prospect for implementing a microwave cavity QED device, as was originally
proposed by Andre et al.~\cite{Andre2006}. In the next section we will briefly
review this field, before further outlining the prospects for a molecule chip
trap.

\section{Ultracold molecules}

There are a number of potential benefits that can come from cooling molecules
rather than atoms. As already mentioned, molecules can have a permanent
electric dipole moment in their ground state, allowing for strong dipole-dipole
coupling in a lattice, or to external fields.
%
Molecules also allows the investigation of fundamental physics, such as the
measurement of the electron's electric dipole moment, measurement of
which has helped to eliminate particle physics theories that go beyond the
standard model~\cite{ACMEreview}.

The energy structure of molecules is far richer than that of atoms; a diatomic
molecule will have both vibrational and rotational energy levels, which can be
leveraged for novel imaging schemes, strong coupling to microwave transitions,
amongst other applications. However these introduce additional complications
that make experiments with ultracold molecules more challenging than their
atomic equivalents.

There are two main schools of ultracold molecule experiments, in the first of
these gases of cold atoms are used to synthesise cold molecules, usually by
association by Feshbach resonance~\cite{Moses2017,PhysRevA.89.033604}. However
in this thesis we will focus on the direct cooling of hot diatomic molecules.
The first experiment to successfully trap molecules was reported by Weinstein
et al.~\cite{Weinstein1998}, who created calcium monohydride (\CaH{}) molecules
by laser ablation, and then to below the depth of a magnetic trap by buffer-gas
cooling. In this technique, an inert buffer gas is used as an intermediary to
thermalise the molecules with a cold copper cell typically held at a few
kelvin.
%
An alternative technique to slow a beam of molecules was developed in Gerard
Meijer's group, who leveraged the permanent electric dipole moment of molecules
used Stark deceleration to first slow a beam of carbon
monoxide~\cite{Bethlem1999}, and later to slow then electrostatically trap
ammonia in both a quadrupole trap~\cite{Bethlem2000} and a storage
ring~\cite{Crompvoets2001, Crompvoets2005}. However these methods produce
molecules that are orders of magnitude warmer than the atoms produced by laser
slowing.

Buffer gas cooling became a staple of cold molecule experiments, being
routinely used in the production of slow, high-flux molecular
beams~\cite{Maxwell2005, Patterson2007, Barry2011}.  Optical cycling in a
diatomic molecule with the potential of laser slowing was first seen using
strontium monofluoride (\SrF{}) by DeMille's group~\cite{Shuman2009}. To
account for the more complex energy structure of the molecule, additional
lasers were used to repump the molecule from states that were dark to the
cooling transition.  Despite this difficulty, the DeMille group had soon enough
applied the technique to achieve laser-slowing~\cite{PhysRevLett.108.103002}
and eventually an \SrF{} MOT~\cite{Barry2014, PhysRevLett.116.063004}. The
initial report boasted temperatures less than \SI{1}{\milli\kelvin} and there
was scope to reduce this further.

Alongside the development of laser-cooling \SrF{}, there have been developments
in the laser-cooling of various other molecules, notably ytterbium monofluoride
(\YbF{}) and calcium monofluoride (\CaF{}), the latter being the main molecule
of focus in this thesis. YbF has been used in various experiments to measure
the electric dipole moment of the electron, and it is hoped that laser-cooling
could be a method for further reducing the associated uncertainty~\cite{Fitch2020}. Meanwhile, \CaF{} shows promise as a molecule of
itnerest for investigating quantum information and the fundamentals of quantum
chemisty. Laser-slowing of \CaF{} was first observed in the Centre for
Cold Matter (CCM) at Imperial College London~\cite{PhysRevA.89.053416}, where a
supersonic beam was slowed by \SI{20}{\meter\per\second} using a chirped beam of
light. The \CaF{} experiment was further developed with the implementation of a
buffer gas source for \CaF{}~\cite{Truppe2018}, with further developments in
laser slowing coming from both CCM~\cite{Truppe2017a} and the Doyle group in
Harvard~\cite{0953-4075-49-17-174001}.

A \CaF{} MOT was first reported by the Harvard
group~\cite{PhysRevLett.119.103201}, followed shortly by
CCM~\cite{Williams2017}. Both experiments taking slightly different approaches
to avoid the pumping of molecules into dark states, using r.f.\ switching and
dual-frequency techniques respectively. The \CaF{} MOT has opened the door to
experiments with large clouds of \CaF{} molecules, with today's MOTs capable of
loading $\sim10^5$ molecules~\cite{PhysRevLett.119.103201}. The temperatures of
these clouds can be further reduced to below the Doppler
limit~\cite{Truppe2017, PhysRevLett.123.033202} and can then be further trapped
and controlled by some of the same techniques already discussed above for
atoms. This includes confinement in magnetic~\cite{WilliamsMagnetic2018} and in
a lattice~\cite{Anderegg2019a}. Futher cooling schemes such as transverse
cooling (similar to that seen for \YbF in \inlineref{Alauze2021}) and
Zeeman-Sisyphys cooling have been demonstrated\cite{Fitch2016,
PhysRevLett.127.263002}.  Combining these various schemes with ongoing research
into collisions with atoms~\cite{PhysRevLett.126.153401} and sympathetic
cooling could see even lower \CaF{} temperatures achieved in the near future.

\CaF{} has shown to have promise in quantum information and simulation, with
coherent control in a magnetic trap having been
demonstrated~\cite{WilliamsMagnetic2018, Blackmore_2018} with coherence times
of over \SI{1}{\second} plausibly
achievable~\cite{PhysRevLett.124.063001}.Coherent control of the molecule has
also been demonstrated in a dipole trap~\cite{PhysRevLett.127.123202}, showing
that there is significant potential for the implementation of quantum
simulations and computing. Further novel schemes for molecule-based quantum
information have been proposed, such as those described for topological quantum
computing in \inlineref{Micheli2006} or for the implementation of qudits in
\inlineref{Sawant_2020}.

Another topic of interest is collisions between cold molecules and cold atoms.
This has recently been observed in sodium lithium molecules by Son et
al.~\cite{son2019collisional}, and in \CaF{} in CCM~\cite{Jurgilas2021,
JurgilasPRL_2021}. The thermalisation of molecules with atoms could lead to
further temperature reduction by sympathetic cooling, which may be enhanced in
a dipole trap. It also paves the way for investigations into quantum chemistry
at a very fundamental level. Recent research into the production of polyatomic
MOTs by the Harvard group~\cite{DoylePolyatomic2022} will no-doubt also play an
important role in such investigations.

To summarise, research into ultracold molecules is rapidly developing, with
methods for producing cold, dense clouds of diatomic molecules already
available. The rich energy structure, and permanent electric dipole moment of
molecules makes them a promising candidate for use in quantum information,
communications and sensing experiments including in chip-based devices.

\section{Molecule chip trap}

We have seen in the above discussion that ultracold molecules can be a useful
tool in quantum information and simulations. Further, recent developments in
cooling techniques mean that we are now able to produce dense clouds of
molecules such as \CaF{}, which exhibit long coherence times in magnetic traps.
Various techniques that have been applied to cold atoms have been equally
successfully applied to cold molecules, but a conspicuous absence in the
preceding discussion is a chip trap for ultracold molecules.

Such a device is of significant interest, since the various capabilities of
atom chips: a robust and stable architecture for quantum experiments could
prove extremely useful when applied to cold molecules. Diatomic molecules, such
as \CaF{} have now been laser-cooled and magnetically confined. In modern
systems dense clouds can be created, which are suitable for loading into
magnetic chip traps (see chapter~\ref{sim}). Further, polar molecules can be
contained in electrostatic traps, which could be leveraged to construct traps
similar to existing ion chip traps~\cite{Andre2006, Romaszko2020}. Such devices
are inherently scalable, with the ability to constrain multiple molecules at
various sites across the chip surface. They can then be transported by changing
the electric potentials, which simply amounts to changing the voltages on
trapping electrodes.

Another benefit of a molecule chip system is that the rotational transitions in
a diatomic molecule can couple very strongly to microwave fields, much more so
than the hyperfine transitions in atoms. Andr\'e et al.~\cite{Andre2006}
propose that building a molecule chip with integrated microwave guides, similar
to equivalent atom chips reported in \inlineref{Nirrengarten2006} could provide
access to powerful tools for quantum information and communication. Amongst
these are the ability to carry out a non-demolition measurement of the molecule
state. This, along with the ability to drive microwave transitions using the
guide, allows the device to be totally self contained -- once molecules
are loaded all interactions can be performed electrically. The long-lived
rotational states of \CaF{} make this molecule a promising candidate such a
device.

Along with readout, strong coupling to a resonator could allow for the
implementation of a novel sideband cooling scheme to reduce the molecule's
motional energy to the ground state of the trapping potential. It would also be
possible to couple neighbouring trapping sites using the resonator as a bus.
Similarly it may be possible to extend the interaction to longer ranges, since
photons can be used as flying qubits to couple quantum mechanical systems over
long distances. The schemes mentioned in this section will be explored further
in chapter~\ref{mws}.

There has been some work investigating molecules trapped close to chips, mainly by
the Meijer group, who have designed and implemented microfabricated Stark
decelerators~\cite{Meek2008}, and even trapped molecules on a
chip~\cite{Meek2009}. However, similar to their other work, these molecules are
much hotter than can be achieved with laser-cooling.

This thesis describes the work that has been undertaken in CCM to build a
microfabricated chip trap for \CaF{} molecules at ultracold temperatures. I
will outline a design inspired by existing atom chips, and the proposals in
\inlineref{Andre2006}. This design will provide a stepping stone towards more
advanced devices, which will be able to implement the various schemes described
above. It will integrate with the existing experiment, and take advantage of
the properties of \CaF{}, including the long coherence times of rotational
transitions that can be achieved in magnetic traps.

\section{Structure of the thesis}

In chapter~\ref{theory} we will introduce key background theory: 
laser-cooling of simple systems, the operation of chip traps, and the physics
of diatomic molecules. Next chapter~\ref{overview} gives an overview of the
existing \CaF{} experiment, as well as some particulars of the laser-cooling
methods used specifically for \CaF{}. Here we will also outline the new chip
experiment, and how it will integrate with the existing apparatus. We have
demonstrated the loading procedure for the trap with simulations, and these are
presented in chapter~\ref{sim}.

We microfabricated chip traps, as will be discussed in chapter~\ref{fab}, and
then loaded these into a vacuum chamber for initial testing. This included
testing that the current capcity of the trapping wires was as expected, that
ultra-high vacuum could be reached, and that it will be possible to image
molecules without too much background scatter created by the chip. These tests,
and a scheme to reduce the background scatter are discussed in
chapter~\ref{experiment}. Chapter~\ref{mws} describes how a microwave guide and
resonator can be implemented as part of the experiment, as well as the
behavoiur of a single molecule coupled to a microwave cavity. We will extend
this idea for ensembles of molecules coupled to a cavity in
chapter~\ref{squeeze}, where I propose a scheme to use a quantum non-demolition
measurement to create a spin-squeezed state in such an ensemble. Finally,
chapter~\ref{outlook} describes the future prospects for this project.
