\cm{Use intro to couch thesis in terms of what others have already done}

Opening spiel
\cite{Andre2006}

\section{Ultracold atoms}

\cm{
(also bring in idea of UHV here somewhere)
- cf ions \cite{RevModPhys.62.531} (this is a nice review of chips \cite{Romaszko2020}, keep emphasis on neutral atoms however
}

Laser cooling of atoms is a well-explored field, with the 1997 Nobel Prize in
Physics having been awarded to Cohen-Tannoudji, Chu and Phillips for
`development of methods to cool and trap atoms with laser
light'~\cite{RevModPhys.70.721}. The subject is reviewed in various texts, with
the canonical reference being Metcalf and van der Straten~\cite{Metcalf1999}.
Here I will give a brief qualitative overview of the topic and its many
applications in modern physics, which include quantum: information,
communication and metrology. Further details on these cooling mechanisms are
given in section~\ref{theory:cooltrap}.

In laser slowing an atom's velocity is reduced in one direction by the absorption and
subsequent re-emission of a photon from a laser beam, that is the scattering of
the photon by the atom. The emission of the
photon has no preferred direction, and so over a number of these scattering events there
is a net reduction of the atom's velocity in the direction of the laser
beam~\cite{PhysRevLett.40.1639}. This principle of slowing with radiation
pressure was first observed by Wineland et al.~\cite{PhysRevLett.40.1639} in
ions, but one of the key goals of laser slowing was to sufficiently reduce the
velocity of a beam of neutral atoms (produced in, for example, an oven) to
sufficiently low velocities for trapping. To achieve this it is essential to
account for the change in Doppler shift that occurs as the atom's velocity is
reduced. Phillips and Metcalf~\cite{PhysRevLett.48.596} were able to achieve this by
the Zeeman slowing technique, where a changing magnetic field across the
beamline is used to ensure the relevant transition is always resonant with the
light. Later the same group would achieve the same effect by chirping the
frequency of the light~\cite{Prodan1984}, so that the light is sufficiently
red-shifted to be resonant with the atoms as they slow down.

Magnetic trapping of the atoms soon followed, with the slow atomic beam
directed into a quadrupole field, generated by current-carrying coils, the
atoms could be pumped into a weak-field seeking state, and trapped in the field
minimum~\cite{PhysRevLett.54.2596}. The next key development was the
implementation of the magneto-optical trap (MOT) where atoms were confined by a
quadrupole magnetic field and red-detuned laser light incident from every
direction. This scheme is designed so that atoms that are away from the trap
centre experience a Zeeman shift that brings them into resonance with the
light. The polarisations are then chosen so that the atom will preferentially
scatter photons from the beam that will result in a restoring force towards the
trap centre. The benefit of this over the magnetic trap being that as well as
a position-dependent restoring force, there is a velocity-dependent damping to
reduce the atoms' temperature. This was first implemented by by Raab et
al.~\cite{PhysRevLett.59.2631} in \Na{}, although subsequently many different
species have been laser-cooled and confined in a MOT \cm{CITE?}.

Due to the stochastic nature of Doppler-cooling, there is a fundamental limit
on the lowest temperature that can be achieved using just the Doppler slowing
mechanism. This can be reached by turning the magnetic field of the MOT off,
leaving just the red-detuned laser beams to form a molasses. Early
implementations of this scheme found that in fact anomalously low temperatures
were reached in a molasses~\cite{} % https://journals.aps.org/prl/abstract/10.1103/PhysRevLett.61.169
This was eventually explained by Dalibard et al.~\cite{Dalibard:89} as being
due to sub-Doppler cooling mechanisms, which rely on polarisation gradients
set up in the light field. This reduced the atom temperature to the so-called
recoil limit, with this being related to the energy imparted by a single photon
scatter.

It is possible to cool beneath even the recoil limit by evaporative cooling,
where the hottest atoms are ejected from the trap. The remaining atoms
thermalise at a lower temperature than the original cloud. This technique was
employed by Anderson et al.~\cite{Anderson198} to cool \esRb{} to sufficiently
low temperature and density to form a Bose-Einstein condensate (BEC). Another
now-popular method of producing high-density atomic clouds is the optical
dipole trap~\cite{Chu1986}, where atoms are confined by intense,
far-off-resonant light which induces an electric dipole in the atom. The
resulting energy shift is proportional to $-E^2$, where $E$ is the strength of
the electric component of the light field. The atoms are therefore confined in
the region where the light's intensity is strongest, that is around the beam
waist. For a sufficiently tight trap, an optical tweezer is formed, where a
single atom can be trapped and easily manipulated by controlling the
light~\cite{Schlosser2001}.  A series of tweezer traps can be used to form a
lattice of traps~\cite{Schlosser2001}, with individually-trapped atoms able to
interact with each other by \cm{some force, CITE}. A lattice can also be formed
in a standing wave optical trap (SWOT), where light reflected from a surface
creates a series of local field maxima for trapping of atoms~\cite{Wu2017}.

With the abilitity to reliably create low-temperature atomic clouds confined in
a trap; and, in the case of tweezers, individually trapped atoms, comes the
ability to create quantum devices for simulation, communication and metrology,
with promises of future applications in quantum computing. A typical example is
the use of \cm{strontium} atoms held in a lattice for precise measurement of the
second~\cite{PhysRevX.8.021036}, but there are many other examples of atoms being used for
precisely sensing, for example, acceleration~\cite{} including gravity~\cite{}.
% TODO cite something and Birmingham?
Cold atoms can also be employed to exploring fundamental physics, such as
\cm{search for dark matter?}. Future experiments plan to use cold atoms to
measureme of gravitational waves in a novel form of \cm{grav wave telescope
  that fills a hole not covered by other plans, and so could actually be useful
  if it worked...}
\cm{coherent control e.g. simulation in lattice \cite{Schäfer2020}}

% miniaturisation e.f. grating MOT \cite{Nshii2013}
These experiments, and those like them have been made easier by technological
developments such as atom dispensers~\cite{}, which remove the need to cool
a hot beam of atoms, and vapour cells~\cite{}, which can entirely remove the
need for a complicated vacuum system. This has lead the way to miniaturisation
and scalability of cold atom experiments, which is an active field of research.
The complexity of these experiments can be further reduced by using, for
example mirror~\cite{} or grating~\cite{} MOTs, where some of the MOT light
beams are produced by reflection or diffraction from a surface. These are
strongly linked to the main focus of this thesis: the chip trap.

\section{Chip traps}

Big review: \cite{2011Ac}
% atom chips (mention ions and ion chips)

Basic operating principles
16 CalTech chip proposals original idea for atoms in high field gradients \cite{PhysRevA.52.4004}
- electrostatic trapping also possible, as in Andre paper, and e.g.  \cite{Bethlem2000}
- microfabrication of the wires allows for highly local traps
- can trap within microns of a surface
17, 18 first demos of atom chips in 3d \cite{PhysRevLett.80.1634} \cite{PhysRevLett.81.5310}
% TODO Maybe some unicode in 17?
this \cite{Drindic1998} implmenets CalTech design% TODO Fix unicode in this one
U and Z traps \cite{Reichel1999} (including novel mirror MOT)
on chip guiding \cite{PhysRevLett.82.2014}
atom chips trapping and guiding \cite{Folman2000}
joint first BECs \cite{Hansel2001} \cite{Ott2001}
Fermi gas in$^{40}K$ \cite{Aubin2006}
Treutlein coherent manipulation on cip \cite{Treutlein2004}
later developments (I think? should look at this Truetlein2004 paper...)
  > multi-level (Truetlein thesis \cite{Treutlein2008} but paper maybe better... \cite{Boehi2009})
?? ramsey infterferometery
clocks \cite{Knappe2004}
...and later freq. standard \cite{RAMIREZMARTINEZ2011247}
Michelson interfferometer \cite{Wang2005}
Long coherence times (58s) \cite{Deutsch2010}
superconducting (sc) atom chip \cite{Nirrengarten2006}
sc chip for BEC
(Roux, C., Emmert, A., Lupascu, A., Nir- rengarten, T., Nogues, G., Brune, M.,
Raimond, J.-M., and Haroche, S. (2008) Eur. Phys. Lett. 81, 56004.)
Meissner effect \cite{Cano2008}
Couple BEC to sc loop (Singh, M. (2009) Opt. Express 17, 2600)
...more superconducting \cite{Bernon2013}

INteraction with surface (Casmir-polder)
\cite{PhysRevA.72.033610}
\cite{PhysRevLett.98.063201}


Lower dimensional gasses
\cite{PhysRevLett.116.030402}
\cite{Hofferberth2007}
\cite{Yuen2015}

Squeezing BEC
\cite{PhysRevLett.105.080403}

Squeezing in cavity, Monica S-S (chip for preparation - useful example of experiment
easier) \cite{PhysRevLett.104.073604}

Similar vein...  Atom chip on ISS \cite{Frye2021}

\section{Ultracold molecules}

% TODO Ensure I have discussed Rydberg atoms and coupling to hyperfine levels
There are a number of potential benefits that can come from cooling molecules
rather than atoms. For example with molecules it is possible to have a
permanent electric dipole moment, allowing for the various benefits discussed
for Rydberg atom experiments but in the ground state of the molecule.
%
\cm{Explore fundamental physics, eEDM, etc.}
%
Molecules also have a far richer energy structure than atoms; a diatomic
molecule will have both vibrational and rotational energy levels, which can be
leveraged for novel imaging schemes, strong coupling to microwave transitions,
amongst other applications. However these introduce additional complications
that make experiments with ultracold molecules more challenging than their
atomic equivalents.

There are two main schools of ultracold molecule experiments, in the first of
these gases of cold atoms are used to synthesise cold molecules, ususally by
association by Feschbach resonance~\cite{Moses2017,PhysRevA.89.033604}. However
in this thesis we will focus on the direct cooling of hot diatomic molecules.
The first experiment to successfully trap molecules was reported by Weinstein
et al.~\cite{Weinstein1998}, who created calcium monohydride (\CaH{}) molecules
by laser ablation, and then to below the depth of a magnetic trap by buffer-gas
cooling. In this technique, an inert buffer gas is used as an intermediary to
thermalise the molecuels with a cold copper cell typically held at a few
kelvin.
%
% TODO Can talk about Stark deceleration of rydberg atoms above at some point
An alternative technique to slow a beam of molecules was developed in Gerard
Meijer's group, who leveraged the permanent electric dipole moment of molecules
used Stark deceleration to first slow a beam of carbon
monoxide~\cite{Bethlem1999}, and later to slow then electrostatically trap
ammonia in both a quadrupole trap~\cite{Bethlem2000} and a storage
ring~\cite{Crompvoets2001, Crompvoets2005}.  However \cm{both of these are
still quite hot}.

Buffer gas cooling became a staple of cold molecule experiments, being
routinely used in the production of slow, high-flux molecular
beams~\cite{Maxwell2005, Barry2011}.
% CITE https://aip.scitation.org/doi/abs/10.1063/1.2717178
Optical cycling in a diatomic molecule with the potential of laser slowing was
first seen using strontium monofluoride (\SrF{}) by DeMille's
group~\cite{Shuman2009}. To account for the more complex energy structure of
the molecule, additional lasers were used to repump the molecule from states
that were dark to the cooling transition.
% Can I rephrase this to highlight additional experimental complication?
Soon enough the DeMille group had applied the technique to achieve
laser-slowing~\cite{PhysRevLett.108.103002} and eventually an \SrF{}
MOT~\cite{Barry2014, PhysRevLett.116.063004}. \cm{These were colder with
potential for further reducing temp with other schemes...}

Alongside the development of laser-cooling \SrF{}, there have been developments
in the laser-cooling of various other molecules, notably ytterbium monofluoride
(\YbF{}) and calcium monofluoride (\CaF{}), the latter being the main molecule
of focus in this thesis. YbF has been used in various experiments to measure
the electric dipole moment of the electron, with important implications for
high-energy physics~\cite{}. Meanwhile, \CaF{} shows promise as a molecule of
itnerest for investigating quantum information and the fundamentals of quantum
chemisty~\cite{}. Laser-slowing of \CaF{} was first observed in the Centre for
Cold Matter (CCM) at Imperial College London~\cite{PhysRevA.89.053416}, where a
supersonic beam was slowed by \SI{20}{\meter\per\second} using a chirped beam of
light. The \CaF{} experiment was further developed with the implementation of a
buffer gas source for \CaF{}~\cite{Truppe2018}, with further developments in
laser slowing coming from both CCM~\cite{Truppe2017a} and the Doyle group in
Harvard~\cite{0953-4075-49-17-174001}.
% TODO Check Truppe2017a is right cite for here and that the entry is correct

A \CaF{} MOT was first reported by the Harvard
group~\cite{PhysRevLett.119.103201}, followed shortly by CCM~\cite{}. Both
experiments taking slightly different approaches to avoid the pumping of
molecules into dark states, using r.f.\ switching and dual-frequency techniques
respectively. The \CaF{} MOT has opened the door to experiments with large
clouds of \CaF{} molecules, with today's MOTs capable of loading \cm{number}
molecules. The temperatures of these clouds can be further reduced to below the
Doppler limit~\cite{Truppe2017, PhysRevLett.123.033202} and can then be further
trapped and controlled by some of the same techniques already discussed above
for atoms. This includes confinement in magnetic~\cite{WilliamsMagnetic2018}
and optical dipole traps~\cite{}, in a
lattice~\cite{} and recently with a single \CaF{} molecule in optical
tweezers\cm{?? \cite{}}. Futher cooling schemes such as transverse
cooling (similar to that seen for \YbF in \inlineref{Alauze2021}) and
Zeeman-Sisyphys cooling have been demonstrated\cite{Fitch2016,
PhysRevLett.127.263002}. Combining these various schemes with ongoing research
into collisions with atoms~\cite{PhysRevLett.126.153401} and sympathetic
cooling could see even lower \CaF{} temperatures achieved in the near future.
% TODO Check tweezer cite and what it says before they go into the above
%Tweezer array of CaF Doyle \cite{Anderegg2019}

\CaF{} has shown to have promise in quantum information and simulation, with
coherent control in a magnetic trap having been
demonstrated~\cite{WilliamsMagnetic2018, Blackmore_2018} with long coherence
times~\cite{PhysRevLett.124.063001} \cm{how long?} Coherent control of the
molecule is also possible in a tweezer trap~\cite{PhysRevLett.127.123202},
% TODO Check this PRL cite is legit for what I am saying (coherent control in
% tweeze)
showing that there is significant potential for the implementation of
simulations such as those described in \cm{Zoller paper? and other?}

\cm{Another para on why this is actually useful, maybe say a bit more at the
start of this section too...}

- Non-destructive imaging of cloud \cite{PhysRevLett.121.083201}

Newer things, polyatomic MOTs (Doyle: \cite{Vilas2021}), which have some uses maybe? ~\cite{DoylePolyatomic2022}
collisions first: \cite{son2019collisional}, us \cite{Jurgilas2021, JurgilasPRL_2021}, next step sympathetic cooling

Useful features of molecules
- computing \cite{PhysRevLett.88.067901}
- Permanent electric dipole moment, long range interaction -> simulation
- Various novel ideas:
   qudits \cite{Sawant_2020}
   topological qubits: theory (Zoller) \cite{Micheli2006};  experiment ~\cite{Gross995}
- Interesting to study for Q. chemistry (see collisions above)
- spectrosocpy: eEDM
(latest ACME \cite{Andreev2018}, review \cite{ACMEreview})
- Long lived states (above) + strong mw coupling in rot states (segway to chip)


\section{Molecule chip trap}

% TODO Check that I do actually say this...
We have seen in the above discussion that ultracold molecules can be a useful
tool in quantum information and simulations. Further, recent developments in
cooling techniques mean that we are now able to produce dense clouds of
molecules such as \CaF{}, which exhibit long coherence times in magnetic traps.
Various techniques that have been applied to cold atoms have been equally
succesfully applied to cold molecules, but a conspicuous absence in the
preceeding discussion is a chip trap for ultracold molecules.

% TODO Need to build up a bit better that it has been hard to do this so far
% because of lack of molecuel psd
Such a device is of significant interest, since the various capabilities of
atom chips: a robust and stable \cm{architecture} for performing cold molecule
experiments \cm{would be super useful}. Further, the rotational transitions in
a diatomic molecule couple very strongly to microwave fields, much more so than
the hyperfine transitions in atoms. It has been proposed that building a
molecule chip with integrated microwave guides, similar to equivalent atom
chips reported in \cm{inlinerefs treutelin (Paper not thesis) and superconducting chip}
could be used for non-demolition state readout, \cm{somethign else} and
sideband cooling into the ground state of the trap~\cite{Andre2006}.

There has been some existing work investigating molecules trapped close to
chips, mainly by the Meijer group, who have designed and implemented
microfabricated Stark decelerators~\cite{}, and even trapped molecules on a
chip~\cite{}. However, similar to their other work, these molecules are much
hotter than can be achieved with laser-cooling. In this thesis, I will discuss
the work that has been undertaken towards loading ultracold \CaF{} molecules
into a microfabricated magnetic chip trap, with potential for the future
integration of microwave guides, and the implementation of an integrated
molecule chip trap as was proposed in \inlineref{Andre2006}.



Justify

back to idea above of long states and strong coupling to mws, promising
architecture \cite{Andre2006}

Meijer group
- warm molecules on chips, eg declerate \cite{Meek2008} and \cite{Meek2009}
(c.f. afore-mentioned \cite{Bethlem1999})
- imaging \cite{Marx2013}
- measuring and manipulating \cite{PhysRevA.92.063408} -- seems to be their
last one
(Note this is all still at very low density and at temperatures of tens of mK)
(I think the measurements are also destructive, whereas the Andre paper
promises readout too...)
% Remember I need to really sell that these are ultracold molecules on chip


ready to implement this in CaF with our experiment (phase space densities,
moreo promised with ODT)

\section{The structure of this thesis}
% TODO combine with above section?

In chapter~\ref{theory} we will introduce key background theory: 
laser-cooling of simple systems, the operation of chip traps, and the physics
of diatomic molecules. Next chapter~\ref{overview} gives an overview of the
existing \CaF{} experiment, as well as some particulars of the laser-cooling
methods used specifically for \CaF{}. Here we will also outline the new chip
experiment, and how it will integrate with the existing apparatus. We have
demonstrated the loading procedure for the trap with simulations, and these are
presented in chapter~\ref{sim}.

We microfabricated chip traps, as will be discussed in chapter~\ref{fab}, and
then loaded these into a vacuum chamber for initial testing. This included
testing that the current capcity of the trapping wires was as expected, that
ultra-high vacuum could be reached, and that it will be possible to image
molecules without too much background scatter created by the chip. These tests,
and a scheme to reduce the background scatter are discussed in
chapter~\ref{experiment}. Chapter~\ref{mws} describes how a microwave guide and
resonator can be implemented as part of the experiment, as well as the
behavoiur of a single molecule coupled to a microwave cavity. We will extend
this idea for ensembles of molecules coupled to a cavity in
chapter~\ref{squeeze}, where I propose a scheme to use a quantum non-demolition
measurement to create a spin-squeezed state in such an ensemble. Finally,
chapter~\ref{outlook} describes the future prospects for this project.
