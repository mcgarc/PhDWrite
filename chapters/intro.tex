\cm{Use intro to couch thesis in terms of what others have already done}

Opening spiel
\cite{Andre2006}

\section{Ultracold atoms}

% Keep this somewhat short, it doesn't need to be long but hit all the main
% points here.

Cold atoms and laser cooling -- well established technologies
Nobel speech: \cite{RevModPhys.70.721}
Can also cite Metcalf \cite{Metcalf1999}
(also bring in idea of UHV here somewhere)
- cf ions \cite{RevModPhys.62.531} (this is a nice review of chips \cite{Romaszko2020}, keep emphasis on neutral atoms however
- laser slowing
  > radiation pressure \cite{PhysRevLett.40.1639}
  > zeeman slowing \cite{PhysRevLett.48.596}
  > chirped light \cite{Prodan1984}
- magnetic trapping \cite{PhysRevLett.54.2596}
- MOT \cite{PhysRevLett.59.2631}
- subdoppler \cite{Dalibard:89}
- evaporative cooling to BEC \cite{Anderson198}
- dipole trap \cite{Chu1986}
- tweezer \cite{Schlosser2001}
- lattice \cite{PhysRevLett.81.3108}
- SWOT \cite{Wu2017}
- coherent control e.g. simulation in lattice \cite{Schäfer2020}
- measurement, e.g. (clocks 2018 lattice clock \cite{PhysRevX.8.021036})
- miniaturisation e.f. grating MOT \cite{Nshii2013}
- atom chips (mention ions and ion chips)

\section{Chip traps}

Basic operating principles
- original idea for atoms in high field gradients ~\cite{PhysRevA.52.4004}
- magnetic trap from wires and bias fields
- first trap \cite{Reichel1999} (inc novel mirror-MOT) precise control of
atoms in U-trap and some evaporation
- nice trapping and guiding \cite{Folman2000} more trapping and guiding
- microfabrication of the wires allows for highly local traps
- can trap within microns of a surface
- electrostatic trapping also possible, as in Andre paper, and e.g.
\cite{Bethlem2000}

Uses
- robust, lots of BECs (first (check?) \cite{Haensel2001} but close is \cite{Ott2001})
- compact, integrated device for e.g. frequency standard \cite{RAMIREZMARTINEZ2011247}
- investigate physics with quantum object close to macroscopic
- integrate with mw devices on the chip
    > set up that we can do this already for beams, but better to trap, e.g.
      Rydberg atoms, Hogan -- there are better cites including in jabref
      https://journals.aps.org/prl/pdf/10.1103/PhysRevLett.125.073201
    > multi-level (Truetlein thesis \cite{Treutlein2008} but paper maybe
      better... \cite{Boehi2009})
    > superconducting \cite{Bernon2013}
- ideas of scalability?

Big review: \cite{2011Ac}

\section{Ultracold molecules}


Cold molecules also well established, but younger
- \emph{Direct} (review: \cite{Tarbutt2018} first cooling with SrF \cite{Shuman2009}, followed by
radiation pressure paper \cite{PhysRevLett.108.103002}) vs. indirect cooling
(producing RbCs Durham, I think this is the original indirect (can mention briefly if at all) \cite{PhysRevA.89.033604}
https://www.nature.com/articles/nphys3985)
- Direct cooling into storage rings (Thesis \cite{Crompvoets2005} review of mol beams \cite{vandeMeerakker2012})
- Longitudinal laser slowing of supersonic CaF \cite{PhysRevA.89.053416}
- SrF MOT \cite{Barry2014} and improved in ~\cite{PhysRevLett.116.063004}
- CaF buffer gas source (details, maybe not for here): \cite{Truppe2018} % Cite
info on TandF says 2018 but think it's 2017? Original bufgas source
\cite{Barry2011} or not?? I think this one is first? \cite{Maxwell2005}
- CaF slowing Truppe: \cite{Truppe2017a} Doyle: \cite{0953-4075-49-17-174001}
- CaF MOT Doyle: r.f. MOT \cite{PhysRevLett.119.103201}
- Subdoppler cooling \cite{Truppe2017}
- Deep cooling of CaF \cite{PhysRevLett.123.033202}
- CaF dipole traps (Doyle group)
- CaF trap and coherent control ~\cite{WilliamsMagnetic2018}
- CaF (and RbCs) coherent control ~\cite{Blackmore_2018}, long lived states
(Caldwell) ~\cite{PhysRevLett.124.063001}
- Non-destructive imaging of cloud \cite{PhysRevLett.121.083201}
- Tweezer array of CaF Doyle \cite{Anderegg2019}, undergoing coherent control
\cite{PhysRevLett.127.123202} (2021)
- Ongoing cooling investigation - transverse (similar to what has been done for
YbF \cite{Alauze2021}) and ZS proposal \cite{Fitch2016}, experiment \cite{PhysRevLett.127.263002}

Some problems:
- Harder to cool due to complex energy structure
- Typically lower density clouds than atoms (makes things more challenging and
e.g. evaporative cooling impossible)

Newer things, polyatomic MOTs (Doyle: https://arxiv.org/abs/2112.08349),
which have some uses maybe? https://projects.iq.harvard.edu/files/jdoyle/files/jps_cp_spin2021.pdf
collisions first: \cite{son2019collisional}, us \cite{Jurgilas2021, JurgilasPRL_2021}, next step sympathetic cooling

Useful features of molecules
- Permanent electric dipole moment, long range interaction -> simulation
- Various novel ideas:
   qudits \cite{Sawant_2020}
   topological qubits: theory (Zoller) \cite{Micheli2006};  experiment ~\cite{Gross995}
- Interesting to study for Q. chemistry (see collisions above)
- spectrosocpy: eEDM
(latest ACME \cite{Andreev2018},
review https://arxiv.org/abs/2203.08103, 
- Long lived states (above) + strong mw coupling in rot states (segway to chip)


\section{Molecule chip trap}
% TODO Combine with structure section to be one 'this thesis'/ 'CaF chip'
% section?

- back to idea above of long states and strong coupling to mws, promising
architecture \cite{Andre2006}
- warm molecules on chips, eg declerate \cite{Meek2008} and \cite{Meek2009}
% Remember I need to really sell that these are ultracold molecules on chip
- ready to implement this in CaF with our experiment (phase space densities,
moreo promised with ODT)

\section{The structure of this thesis}

This thesis will present the design and construction of a \CaF{} chip trap
experiment. In chapter~\ref{theory} we will introduce key background theory: 
laser-cooling of simple systems, the operation of chip traps, and the physics
of diatomic molecules. Next chapter~\ref{overview} gives an overview of the
existing \CaF{} experiment, as well as some particulars of the laser-cooling
methods used specifically for \CaF{}. Here we will also outline the new chip
experiment, and how it will integrate with the existing apparatus. We have
demonstrated the loading procedure for the trap with simulations, and these are
presented in chapter~\ref{sim}.

We microfabricated chip traps, as will be discussed in chapter~\ref{fab}, and
then loaded these into a vacuum chamber for initial testing. This included
testing that the current capcity of the trapping wires was as expected, that
ultra-high vacuum could be reached, and that it will be possible to image
molecules without too much background scatter created by the chip. These tests,
and a scheme to reduce the background scatter are discussed in
chapter~\ref{experiment}. Chapter~\ref{mws} describes how a microwave guide and
resonator can be implemented as part of the experiment, as well as the
behavoiur of a single molecule coupled to a microwave cavity. We will extend
this idea for ensembles of molecules coupled to a cavity in
chapter~\ref{squeeze}, where I propose a scheme to use a quantum non-demolition
measurement to create a spin-squeezed state in such an ensemble. Finally,
chapter~\ref{outlook} describes the future prospects for this project.
