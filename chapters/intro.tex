Opening spiel

\section{Ultracold molecules}

Cold atoms and laser cooling -- well established technologies
- evaporative cooling to BEC
- coherent control
- dipole trap, tweezer, single atom, lattice
- magnetic trapping
- atom chips
(also bring in idea of UHV here somewhere)

Cold molecules also well established, but younger
- SrF MOT
- CaF MOT
- CaF dipole traps (Doyle group)
- CaF coherent control, long lived states

Some problems:
- Harder to cool due to complex energy structure
- Typically lower density clouds than atoms (makes things more challenging and
e.g. evaporative cooling impossible)

Newer things, polyatomic MOTs, collisions, next step sympathetic cooling

Useful features of molecules
- Permanent electric dipole moment, long range interaction -> simulation
- Interesting to study for Q. chemistry (see collisions above)
- spectrosocpy: eEDM
- Long lived states (above) + strong mw coupling in rot states (segway to chip)

\section{Chip traps}

Basic operating principles
- magnetic trap from wires and bias fields
- microfabrication of the wires allows for highly local traps
- can trap within microns of a surface

Uses
- investigate physics with quantum object close to macroscopic
- integrate with mw devices on the chip
- ideas of scalability?

Atom chips

Molecule chip trap
- back to idea above of long states and strong coupling to mws, promising
architecture \cite{Andre2006}
- ready to implement this in CaF with our experiment (phase space densities,
moreo promised with ODT)

\section{The structure of this thesis}

This thesis will present the design and construction of a \CaF{} chip trap
experiment. In chapter~\ref{theory} we will introduce key background theory: 
laser-cooling of simple systems, the operation of chip traps, and the physics
of diatomic molecules. Next chapter~\ref{overview} gives an overview of the
existing \CaF{} experiment, as well as some particulars of the laser-cooling
methods used specifically for \CaF{}. Here we will also outline the new chip
experiment, and how it will integrate with the existing apparatus. We have
demonstrated the loading procedure for the trap with simulations, and these are
presented in chapter~\ref{sim}.

We microfabricated chip traps, as will be discussed in chapter~\ref{fab}, and
then loaded these into a vacuum chamber for initial testing. This included
testing that the current capcity of the trapping wires was as expected, that
ultra-high vacuum could be reached, and that it will be possible to image
molecules without too much background scatter created by the chip. These tests,
and a scheme to reduce the background scatter are discussed in
chapter~\ref{experiment}. Chapter~\ref{mws} describes how a microwave guide and
resonator can be implemented as part of the experiment, as well as the
behavoiur of a single molecule coupled to a microwave cavity. We will extend
this idea for ensembles of molecules coupled to a cavity in
chapter~\ref{squeeze}, where I propose a scheme to use a quantum non-demolition
measurement to create a spin-squeezed state in such an ensemble. Finally,
chapter~\ref{outlook} describes the future prospects for this project.
