Above we have presented our design for the chip trap and our initial
fabrication of prototypes. The next stage will be to complete fabrication of a
chip with trapping wires, which can be used for initial testing under vacuum
conditions. The entire chip chamber can be assembled in isolation from the rest
of the molecule experiment. We will then be able to test the currents that can
be acieved on the chip, and design the control system for the electronics.

In parallel we can begin to work on the rest of the fabrication process: first
spin coating ployimide onto a chip protoytpe with good planarisation, and then 
fabricating coplanar waveguides on this upper layer.

We will then move on to testing the chip with \Rb{} atoms. These will behave
much like the \CaF{} molecules, but with a much higher phase space density.
This will allow us to implement and understand the loading technique before
moving on to trapping \CaF{}. Depending on how succesful these remaining
fabrication steps are, this could be performed with the complete chip or with a
version that does not feature microwave guides. In the former case, microwave
guides would be added at a later stage, after trapping of a cloud of molecules 

Once molecules have been trapped on the chip, we will attempt to control them
using the field from the CPW, using the same \SI{41}{\giga\hertz} microwave
transition that has already been measured in the macroscopic trap.

If this stage is reached, then there will be an enormous number of ways to
extend the technology. An experiment that could be conceived without requiring
much more equipment would be using a mixture of \esRb{} and \CaF{} in the chip
trap to perform sympathetic evaporative cooling. This could help offset
adiabatic heating that will occur during chip trap compression. Another option
is to design a chip with multiple traps, and see if they can be loaded and then
coupled reliably.

There is also scope for more theoretical work. One example is the investigation
of a chip that is integrated with an optical tweezer. This would provide
stronger confinement than the magnetic traps, whilst still allowing us to
strongly couple the molecule to microwave guides.

Looking even further forward, it may be possible to add cooling apparatus to
the experiment so that the chip can be fabricated using superconducting
materials. Such a chip would be capable of higher trapping currents and hosting
CPWs of high enough quality factor to implement a resonator. True cavity-QED
experiments could then be performed, including investigation of hybrid quantum
systems.
