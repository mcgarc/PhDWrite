In this thesis I have outlined the work that has been undertaken to design and
build a microfabricated chip trap for ultracold molecules. This device
implements a series of magnetic traps for loading from the existing \CaF{}
experiment and bringing the molecules close to the chip surface by a series of
current ramps. Molecules can be dropped from the chip for imaging with the RROC
scheme described in chapter~\ref{experiment}. The chip has been designed so 
that a second generation device can include an additional layer for microwave
components which could be used to directly drive rotational transitions in the
molecules. Such experiments will be useful in exploring the dynamics of cold
molecules in the proximity of macroscopic bodies and in the microwave near
field.

Future experiments could seek to implement a superconducting microwave cavity
and realise the procedures described in chapters~\ref{mws} and~\ref{squeeze}
for performing state readout, sideband cooling and preparation of non-classical
states. There are various technical obstacles that must be overcome to achieve
this, chiefly the requirement of high cavity microwave cavities, which can only
feasibly be implemented by the use of superconductors.  This would require the
installation of additional cooling for the superconductors, which would not be
possible without significant changes to the existing apparatus. This
remains a feasible long term goal for a third generation device, and a
single level device with superconducting trapping wires could simultaneously
simplify and enhance the experiment.

Along with the integration of superconducting microwave components, it may be
useful to develop other tools for cold molecules to enable such an experiment.
One example is the use of an optical dipole trap before the trapping stage to
increase phase-space density before loading.  Alternatively a mirror or grating
MOT integrated with a chip, similar to those that were described in
section~\ref{intro:atoms} could be a convenient and effective way for forming a
\CaF{} MOT for loading into the chip trap. By loading such a MOT directly from
a source this could avoid the losses that occur during molecular transport.
Similarly, it would be useful to develop a method for cooling the molecules on
or near the chip, which is prohibited by the limited optical access in the
experiment described here.

Another potential consideration is the implementation of transport or loading
via an optical dipole trap, which could achieve efficient transport, or
directly load a single molecule into an on-chip trap. Another idea is to use
the dipole trap to contain the molecules in close proximity to a substrate with
only microwave components. This could be a useful scheme for achieving strong
coupling between rotational transitions and microwave photons, but would
sacrifice some of the simplicity of an entirely integrated electronic device.

The molecule chip is a promising device for future experiments, with potential
applications in quantum information and communications, as well as promising a
robust platform for future experiments with ultracold molecules. This thesis
has presented a prototype device with trapping capabilities that are ready to
be demonstrated. It is the first step towards future generations of the
experiment, bringing us closer to a fully-integrated electrical experiment for
trapping and control of ultracold molecules.
