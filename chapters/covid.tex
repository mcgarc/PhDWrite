In this thesis I will describe the creation of a microfabricated chip trap for
laser-cooled \CaF{} molecules. The original intention was to design and build
such a device, and incorporate it into the existing \CaF{} experiment that we
operatre in the Centre for Cold Matter at Imperial College.

In March 2020 I was focusing on the fabrication of the
chip trap (described in chapter~\ref{fab}), which took place in the cleanroom
facilities at the London Centre for Nanotechnology. These were closed to
external users from 19\textsuperscript{th} March -- 1\textsuperscript{st}
October, completely halting development of this key aspect of the project. When
access resumed it was limited due to the requirement to social distance. This
meant that developing the fabrication procedure for the chip was severely
delayed.

The chip experiment has been designed to integrate with the existing \CaF
experiment (described in chapter~\ref{overview}) however the development of
this machine was delayed closure of the College laboratories starting on
18\textsuperscript{th} March. Since the existing experiments was not
sufficiently developed, it was impossible to add the chip chamber to the
apparatus. 

The combined result of these two effects was that the molecule chip experiment
was not fully implemented, and we were limited to the preliminary experiments
which are reported in chapter~\ref{experiment}. The fabrication was completed,
but only to the point of implementing a single-layer trapping chip, without
integrated microwave guides. During the lab closures, the simulations in
chapter~\ref{sim} were conducted, and the chip design was improved, but this
effort was not sufficient to fully mitigate the considerable delays.
