One of the key motivations for developing a molecule chip is the ability to
integrate microwave guides for strong coupling between microwave photons and
the rotational transitions in molecules. In this chapter I will present how
such a scheme can be realised.  I will begin with a summary of the coplanar
waveguide (CPW), which can be used to guide microwaves across a dielectric
surface. I will then move on to discuss the logistics of integrating a CPW into
the experiment, and especially on the delivery of microwaves to the chip. I
will then present two uses of single \CaF{} molecules coupled to microwave
guides: sideband cooling and state readout.

\section{The coplanar waveguide}
\label{mws:CPW}

The CPW was originally proposed by Cheng P.~Wen as a means
of guiding microwaves across the surface of a dielectric
substrate~\cite{1127105}. It consists of a central conductor with a ground
plane on either side, as pictured in~\mysubfigref{mws:fig:CPW}{a}. CPWs have become
prolific, since they allow the creation of robust microwave devices, and offer
some benefits over other waveguide architectures, such as the stripline
waveguide, because they can provide circularly polarised fields, and also
provide easy access to the ground plane for shunt connections.

The CPW's geometry is defined by the centre conductor width ($S$) and the
channel width ($W$). The height of the conductor is $t$ and the dielectric
height is $h$.  The geometry of the CPW determines the region the microwave
field occupies, as illustrated in \mysubfigref{mws:fig:CPW}{b}.  The CPW
can therefore be designed to maximise overlap between the microwave field and a
cloud of molecules trapped nearby.  As a rough approximation the molecules
should be trapped at a distance $h\sim S$ above the centre of the CPW to
achieve a good overlap~\cite{Boehi2009}. The electric field surrounding the CPW is shown
in \mysubfigref{mws:fig:CPW}{b}. The equations defining this field can be found
analytically~\cite{Simons2004}, or the field can be determined by a
finite-element simulation, as is done here.

\begin{figure}[ht]
  \centering
  \begin{subfigure}[b]{0.45\textwidth}
  \centering
    \begin{overpic}[abs, height=0.7\textwidth]{figs/mws/CPW_lines.pdf}
      \put(25, 130){(a)}
    \end{overpic}
  \end{subfigure}
  \hspace{0.7cm}
  \begin{subfigure}[b]{0.45\textwidth}
  \centering
    \begin{overpic}[abs,
      height=0.7\textwidth]{figs/mws/2022-02-06_CPWresonators_arrows.pdf}
      \put(-20, 130){(b)}
    \end{overpic}
  \end{subfigure}
	\caption[Waveguide geometry and field]{
    Subfigure (a) shows a perspective view of a segment of CPW on dielectric substrate, with
pertinent geometry and dielectric constants labeled. The field lines are
sketched, and shown in detail in the cut-through in (b). Here the field lines
are calculated analytically for arbitrary parameters.}
	\label{mws:fig:CPW}
\end{figure}

\subsection{CPW properties}

We will now consider the electric properties of a CPW. These will largely
depend on the planar geometry of the waveguide, which we express in terms
of the ratio~\cite{1127105, Simons2004}
%
\begin{equation}
  k_0 = \frac{S}{S+2W}.
  \label{eqn:k0def}
\end{equation}
%
For convenience, we also define the value
%
\begin{equation} 
  k'_0 = \sqrt{1-{k_0}^2}.
\end{equation}
%
For a waveguide made of a material with conductivity $\sigma$
and permeability $\mu$ and propagating field with frequency $\omega$
there is a skin depth~\cite{Simons2004}
%
\begin{equation}
  \delta = \sqrt{\frac{2}{\omega\mu\sigma}}
\end{equation}
%
and skin effect surface resistance~\cite{Simons2004}
%
\begin{equation}
  R_s = \frac{1}{\sigma\delta}.
\end{equation}

Since the field of the CPW will pass through both the dielectric and the
surrounding air, the CPW's capacitance is the sum of the capacitance of each of
these two parts
%
\begin{equation}
  C_\text{CPW} = C_\text{dielectric} + C_\text{air}.
\end{equation}
%
The method of finding these individual contributions is somewhat involved, so
we simply state the results here in terms of the elliptic integral of the
first kind, $K(k)$~\cite{Simons2004}. The capacitance due to the dielectric is
%
%
\begin{equation}
  C_\text{dielectric} = 2\epsilon_0(\epsilon_\text{r}-1)\frac{K(k_0)}{K(k'_0)}
\end{equation}
%
and the capacitance of the air region is
%
\begin{equation}
  C_\text{air} = 4\epsilon_0 \frac{K(k_0)}{K(k'_0)}.
\end{equation}

We can use this to find the effective permittivity~\cite{Simons2004}
%
\begin{align}
  \epsilon_\text{eff} &= \frac{C_\text{dielectric}}{C_\text{air}} \\
    &= \frac{1+ \epsilon_\text{r}}{2}
\end{align}
%
where we have assumed that we are in the limit of a thick dielectric layer ($h
\gg W$) so that most of the field density below the surface is inside the
dielectric.
%
The phase velocity  is~\cite{Simons2004}
%
\begin{align}
  v_\text{ph} &= \frac{c}{\sqrt{\epsilon_\text{eff}}} \\
    &= \frac{c}{\sqrt{(1 + \epsilon_\text{r})/2}}.
\end{align}

Now using the approximation~\cite{Collin2007}
$\sqrt{\mu_0/\epsilon_0}\approx120\pi\si{\theohm}$ we have the line
impedance~\cite{Simons2004}
\begin{align}
  Z_0 &= \frac{1}{C_\text{air} v_\text{ph}} \\
    &= \frac{30 \pi}{\sqrt{(\epsilon_\text{r}+1)/2}} \frac{K(k_0)}{K(k'_0)}
    \si{\theohm}.
    \label{mws:eqn:Z0}
\end{align}
Note that the impedance of the waveguide has dependence only on the geometry in
the form of the ratio $k_0$, and the relative permittivity of the
substrate.~\cite{Simons2004} This means that for any substrate we choose, the
value of $k_0$ can be chosen to fix the impedance at the standard $Z_0 =
\SI{50}{\ohm}$. This is useful for building macroscopic input ports to the CPW,
and then tapering for a highly localised field near the molecule trap.
%
Therefore as long as $k_0$ is held constant the CPW can be tapered to change the
size of the centre conductor, and hence control the region the field occupies.

A CPW has a cut-off frequency for the lowest transverse electric mode, given
by~\cite{Simons2004}
%
\begin{equation}
  f_\text{cutoff} = \frac{c}{4h\sqrt{\epsilon_r - 1}}.
\end{equation}
%
The height $h$ in this context will be the thickness of the dielectric layer
between the wires and the chips which, as was discussed in
section~\ref{fab:planned},
will be at minimum a few microns.  Even in the extreme case of
$h\sim\SI{10}{\micro\meter}$, this cutoff will always be several orders of
magnitude above our operating frequency.  The cutoff frequency can therefore be
ignored for the purposes of our discussion.

\subsection{CPW attenuation}

It is important to understand the level of attenuation through a CPW. Not only
is it important that we can transmit enough signal to the molecules to drive 
rotational transitions, but the attenuation in a resonator will be the dominant
factor in determining its quality factor, and hence whether we are in the
strong coupling regime. Begin by considering the electric field inside a CPW.
Say that the field at a position $z$ along the axis of the guide is $E(z)$.
This amplitude attenuates according to
%
\begin{equation}
  E(z) = E(0)e^{-\alpha z}
\end{equation}
%
where we call $\alpha$ the attenuation constant.

There are two contributing terms to the attenuation constant
%
\begin{equation}
  \alpha = \alpha_d + \alpha_c,
\end{equation}
%
which are the contributions from the dielectric and the conductor respectively. We
neglect small contributions from radiative losses~\cite{Frankel1991}. Bending
losses can also be ignored as long as the radius of the bends are much larger
than the wavelength of the propagating wave. We must be mindful of this during
design, since the typical wavelengths for rotational transitions are on the
order of centimeters.  We also do not consider the insertion loss of the
resonator, since this does not affect the quality
factor~\cite{doi:10.1063/1.3010859}.

\subsubsection{Dielectric losses}

Dielectric losses in the thick dielectric limit are described
by ~\cite{Collin2007}
\begin{equation}
  \alpha_d =
  \frac{\omega_0}{4c}\frac{\epsilon_r}{\sqrt{\epsilon_\mathrm{eff}}}
  \tan \delta_e
\end{equation}
%
where $\tan\delta_e$ is the dielectric loss tangent. Common dielectrics for
microwave guides include aluminium nitride (\AlN{}) and high-resistivity
silicon.  For our purposes we wish to consider waveguides situated on
dielectric layers that can be easily deposited above the trapping wires.
Following the work in \inlineref{Treutlein2008}, we primarily consider polyimide, which
can be depositied by spin-coating. Other options are available, such as
polyethylene naphthalate (PEN) ~\cite{WEI20169937}.
% TODO Is it possible to grow some other dielectric?

Typical values of dielectric constants and $\alpha_d$ are shown in
\mytableref{mws:table:diprops}. Note that these parameters depend on the
frequency of the microwave field, and to some extent the temperature of the
dielectric, and so are presented to illustrate the amount of loss that can
typically be expected for our experiment at room temperature and frequencies in
the \SI{10}{\giga\hertz} regime.

% For Edwards cite, seee Appendix B saved in Physics Resources/CPW
\begin{table}[tb!]
  \caption[Various dielectric constants]{Dielectric constants and loss for various substrates in the
  \SI{10}{\giga\hertz} regime at room temperature}
\centering
\begin{tabular}{l c c c c }
\hline\hline
  Material & $\epsilon_r$ & $\tan\delta_e$ & $\alpha_d$ (\si{\per\meter}) & Ref. \\ [ 0.5ex]
\hline
  Polyimide & 3.4 & \SI{1.8E-2}{} & 0.45 & \cite{DuPontKapton} \\
  PEN & 2.56 & 0.003 & 0.63 & \cite{WEI20169937} \\
  Silicon (high resistivity)& $\sim{12}$ & \SI{2E-4}{} & 0.02 & \cite{Simons2004, 1717770, doi:10.1063/1.4929503} \\
  Sapphire & $\sim10$ & $2\times10^{-5}$ & $10^{-3}$ & \cite{edwards2016foundations}\\
  Aluminium nitride & 8.9 & \SI{5E-4}{} & 0.22 & \cite{edwards2016foundations} \\
  Arlon AD1000 & 10.2 & 0.0023 & 1.0 & \cite{arlon}\\
\hline
\end{tabular}
\label{mws:table:diprops}
\end{table}

\subsubsection{Conductor losses}

Conductor losses arise due to dissipation in the centre conductor and ground
plane of the CPW~\cite{Simons2004}.  The conductor attenuation constant is
%
\begin{equation}
  \alpha_c = \frac{R_c +R_g}{2Z_0}.
\end{equation}
%
where $R_c$ and $R_g$ are the series resistances per unit length of the centre
conductor and the ground plane respectively.
%
For a waveguide with height $t$, these are given by
%
\begin{equation}
  R_c = \frac{R_s}{4 S(1-k_0^2)K^2(k_0)}\left[ \pi + \log\left(\frac{4\pi
  S}{t}\right) - k_0\log\left(\frac{1+k_0}{1-k_0}\right) \right],
\end{equation}
%
and
%
\begin{equation}
  R_g = \frac{k_0 R_s}{4S(1-k_0^2)K^2(k_0)}\left[\pi +
  \log\left(\frac{4\pi(S+2W)}{t}\right) -
  \frac{1}{k_0}\log\left(\frac{1+k_0}{1-k_0}\right)\right].
\end{equation}
%
The electrical properties of silver and gold are given in
\mytableref{mws:table:metalprops}, and the computed conductor attenuations are
shown in \myfigref{mws:fig:conductorloss}. The latter of these shows the
conductor losses for varying conductor thickness (a) and centre conductor width
(b). Each of these are shown for gold on sapphire, silver on sapphire and gold
on polyimide CPWs. In (a) we take $S=\SI{10}{\micro\meter}$ and in (b) we take
$t=\SI{5}{\micro\meter}$. In both the free parameters are chosen to match an
impedance of $Z_0=\SI{50}{\ohm}$.

% https://www.tibtech.com/conductivite.php?lang=en_US
% 
\begin{table}[tb!]
  \caption[Various electrical constants]{Electrical constants for various conductors in the
  \SI{10}{\giga\hertz} regime at room temperature}
\centering
\begin{tabular}{l c c c}
\hline\hline
Material & $\sigma$ ($\times10^6\si{\siemens\per\meter}$) & $\mu_r$ & Ref. \\ [ 0.5ex]
\hline
  Gold & 44.2 & 1.0 & \cite{edwards2016foundations}\\
  Silver & 62.1 & 1.0 & \cite{edwards2016foundations}\\
\hline
\end{tabular}
\label{mws:table:metalprops}
\end{table}

% Data generated by nbs/2022-02-06_CPWresonators.nb
\begin{figure}[h]
  \centering
  \begin{subfigure}[b]{0.48\textwidth}
  \begin{tikzpicture}
    \begin{axis}[
        xmode=log,
        enlargelimits=true,
        xlabel=$t$ (\si{\nano\meter}),
        ylabel=$\alpha_c$ (\si{\per\meter}),
        width=\textwidth,
        height = 0.7\textwidth,
        x tick label style={/pgf/number format/.cd, set thousands separator={}}
    ]
      \addplot [thick, black] table {figs/mws/conductorlossvaryt.dat};
      \addplot [style=thick, color=blue] table [y index=2] {figs/mws/conductorlossvaryt.dat};
      \addplot [style=thick, color=pink] table [y index=3] {figs/mws/conductorlossvaryt.dat};
      \node [] at (4.5cm, 2.6cm) {(a)};
    \end{axis}
  \end{tikzpicture}
  \end{subfigure}
  \begin{subfigure}[b]{0.48\textwidth}
  \begin{tikzpicture}
    \begin{axis}[
        enlargelimits=true,
        xlabel=$S$ (\si{\micro\meter}),
        %ylabel=$\alpha_c$ (\si{\per\meter}),
        width=\textwidth,
        height = 0.7\textwidth,
        x tick label style={/pgf/number format/.cd, set thousands separator={}}
    ]
      \addplot [thick, black] table {figs/mws/conductorlossvarys.dat};
      \addplot [style=thick, color=blue] table [y index=2] {figs/mws/conductorlossvarys.dat};
      \addplot [style=thick, color=pink] table [y index=3] {figs/mws/conductorlossvarys.dat};
      \node [] at (4.5cm, 2.6cm) {(b)};
    \end{axis}
  \end{tikzpicture}
  \end{subfigure}
    \caption[Conductor losses for a CPW]{
      Conductor losses are plotted for (a) varying the conductor height of the
      CPW and (b) varying the centre conductor width. Losses are shown for gold
      on sapphire (black), silver on sapphire (blue) and gold on polyimide
      (red). In (a) a typical conductor width of $S=\SI{10}{\micro\meter}$ is
      assumed, and in (b) the thickness of the conductors is taken to be
      $t=\SI{5}{\micro\meter}$. In both cases the impedance is set to $Z_0 =
      \SI{50}{\ohm}$ by choosing $k_0$, according to \myeqref{mws:eqn:Z0}.
    }
  \label{mws:fig:conductorloss}
\end{figure}

As can be seen by comparing the values in \mytableref{mws:table:diprops} and
\myfigref{mws:fig:conductorloss}, the conductor losses clearly dominate over
the dielectric losses.
%
With loss of $\alpha \approx \alpha_c$, and waveguides of lengths on
the scale of a few centimeter, we can expect a total loss of around
$\SI{400}{\per\metre} \times \SI{0.01}{\metre} = \SI{35}{\decibel}$. This
should be sufficiently small that microwaves of the required power to drive
rotational transitions will be able to reach molecules
trapped on the chip.~\cite{Treutlein2008}
%
However, such high conductor losses will prevent the implmentation of a
high-$Q$ microwave resonator on the chip, as we will discuss in the next
section.

\subsection{CPW resonators}
\label{mws:resonators}

A microwave resonator can be formed from a section of CPW that is capacitively
coupled to another driving segment~\cite{Day2003}. The resonant frequency is
determined by the resonator's length, $L$, and the waveguide's phase
velocity~\cite{Simons2004}
%
\begin{equation}
  \omega_0 = \frac{\pi v_\text{ph}}{L} = \frac{\pi
  c}{\sqrt{\epsilon_\text{eff}} L}
\end{equation}
%
Microwaves can be coupled to the resonator by positioning the resonator inline
with a microwave guide, or in parallel as shown in
\myfigref{mws:fig:resonators}. The strength of the coupling can be controlled by the shape of the
interface between the resonator and the waveguide~\cite{doi:10.1063/1.3010859}.

\begin{figure}
  \centering
  \includegraphics[width=.6\textwidth]{figs/mws/resonators.pdf}
  \caption[Capacative coupling schemes for resonator]{
    Different schemes for capacative coupling into a microwave
    resonator. Gold represents conductor and black a dielectric substrate. In
    (a) the resonator runs parallel to a waveguide that feeds in microwaves.
    The input port is denoted by the arrow. In (b) the resonator is in series
    with the input port, and is also coupled to an output port.
  }
  \label{mws:fig:resonators}.
\end{figure}

The quality factors for CPW resonators can be expressed in terms of an
attenuation constant
%
\begin{equation}
  Q = \frac{\omega_0}{2c\alpha}.
  \label{mws:eqn:Qalpha}
\end{equation}
%
From \myfigref{mws:fig:conductorloss} and \mytableref{mws:table:diprops} we
find that using conventional materials, the attenuation will be dominated
by conductor losses, and $Q\sim10$. In section~\ref{mws:coupling} that we
require $Q > 1.7 \times 10^5$.

% TODO Could improve the prose here
Conductor losses can be eliminated by use of a superconducting CPW. Such
devices are a developed technology, which have been used for coupling to
solid-state qubits~\cite{Wallraff2004}. Operating at the low temperatures
required for superconductors allows the achievement of very high quality
factors of order $10^6$~\cite{doi:10.1063/1.3552890, Day2003}. For such devices
the quality factor is dominated by the dielectric losses, and as such a
low-loss dielectric is usually chosen for the substrate (sapphire being a
common choice). An alternative method to produce a multi-layer resonator could
potentially be etching the polyimide in the slot gaps, as in
\inlineref{920142}. Otherwise, a single-layer superconducting design such as
that shown in \inlineref{Hattermann2017} could be adopted. Regardless, the
multilayer design remains a useful tool for microwave-molecule interactions,
and investigating lower quality resonators coupled to molecules. 

% I have this point from Goppl that at low drive powers Q is somehow reduced. I
% don't understand this really, and don't think it's that relevant because we
% are so far off single-photon regime here anyway. But didn't want to entirely
% forget, so now this comment exists.

\section{How to couple a molecule to a coplanar waveguide}
\label{mws:coupling}

Having established the principles of the CPW, and that such a device can
operate on a polyimide substrate, we can now consider the logistics of
implementing such a device with our design. In this section we will consider
the details of the how the coupling of molecules to a CPW can work in practice,
including the coupling Hamiltonian, tuning the molecular rotational frequency
using the Stark shift,  the delivery of microwaves to the chip, and
schemes for sideband cooling and state readout with the microwaves.

\subsection{Cavity quantum electrodynamics}
\label{mws:CQED}

For our purposes, the coupling of a single molecule and the microwave field can
be treated as the coupling of a two-level system to a quantum mode of a cavity
field. The canonical description of such a system is given by the familiar
Jaynes-Cummings Hamiltonian (JCH) in the rotating wave
approximation~\cite{gerry_knight_2004}, which we will now briefly review. The
relevant Hamiltonian is
%
\begin{equation}
  H_\text{JC} = \hbar\omega_c a^\dagger a + \frac{\hbar \omega_0}{2} \sigma_z +
  \frac{\hbar\Omega}{2}(a^\dagger \sigma_- + a\sigma_+)
  \label{theory:eqn:JCH}
\end{equation}
%
where $a$ ($a^\dagger$) is the annihilation (creation) operator of the photons,
$\Omega$ is the Rabi frequency of the interaction, $\sigma_i$ with $i\in{x, y,
z}$ are the Pauli matrices, and $\sigma_\pm = (\sigma_x \pm i\sigma_y)/2$ are
the raising and lowering operators of the molecule state. The detuning of the
cavity resonance from that of the spin is $\Delta = \omega_0 - \omega_c$. The
system is shown in \mysubfigref{theory:fig:JCHstates}{a}.

\begin{figure}
  \includegraphics[width=\textwidth]{figs/mws/JCH.pdf}
  \caption[Two-lvel cavity QED system]{
    A two-level cavity QED system. Subfigure (a) shows a schematic of
    the system, where microwaves interact with a rotational transition in
    \CaF{}. This implements the JCH, whose energy levels are shown both on (b)
    and off (c) resonance. Energy spacings are highlighted in purple. This
    figure is inspired by one in \inlineref{PhysRevA.69.062320}.
  }
  \label{theory:fig:JCHstates}
\end{figure}

We denote the ground (exicted) state of the molecule as $\ket{g}$ ($\ket{e}$).
The light field state can be taken to be a Fock state ($\ket{n}$ with $n \in
\mathbb{Z}$). Note that the final term in equation~\ref{theory:eqn:JCH}) has
the effect of exciting the ground state while absorbing a photon
($\ket{g}\ket{n} \leftrightarrow \ket{e}\ket{n-1}$) or lowering the excited state
and releasing a photon ($\ket{e}\ket{n} \leftrightarrow \ket{g}\ket{n+1}$).

Following the procedure in \inlineref{gerry_knight_2004}, we can see that this
mixing of the states results in a shift of the energy levels to create the
dressed states
%
\begin{align}
  \ket{+, n} &= \cos\Phi_n \ket{g}\ket{n} + \sin\Phi_n \ket{e}\ket{n+1} \\
  \ket{-, n} &= -\sin\Phi_n \ket{g}\ket{n} + \cos\Phi_n \ket{e}\ket{n+1}
\end{align}
%
with
%
\begin{equation}
  \tan(2\Phi_n) = \frac{\Omega\sqrt{n+1}}{\Delta}
\end{equation}
%
and having shifted energies
%
\begin{equation}
  E_{\pm, n} = (n+1)\hbar\omega_c \pm \frac{\hbar}{2}\sqrt{\Omega^2(n+1) +
  \Delta^2}.
  \label{theory:eqn:JCHenergies}
\end{equation}
%
It is useful to consider the manifold of states as depicted in
\myfigref{theory:fig:JCHstates}.  
Note that in the limit of no coupling
($\Omega = 0$) and no detuning ($\Delta = 0$) the energies are that of the bare
states, and $\ket{g}\ket{n+1}$ is degenerate with $\ket{e}\ket{n}$, as in part
(b) of the subfigure. Introducing coupling ($\Omega \neq 0$) lifts this
degeneracy, as in part (c). When the detuning is non-zero ($\Delta \neq 0$)
there is additional offset due to the second term in
\myeqref{theory:eqn:JCHenergies}, see part (d) of the figure.

The strong coupling r\'egime is reached when the coupling $g=2\Omega$ is
greater than the rate of decay from the cavity $\kappa = \omega_0 / Q$, where
$Q$ is called the quality factor of the cavity. The coupling parameter is
related to the transition dipole moment $d$ and the amplitude of the electric
field $E_0$ by
%
\begin{equation}
  \hbar g = \frac{d E_0}{2}.
\end{equation}
%
For the resonator, the amplitude of the electric field can be expressed in
terms of the cavity parameters by considering the electric field density
%
\begin{equation}
  \frac{1}{2} \epsilon_0 E_0 = \frac{\hbar \omega_0}{V}
\end{equation}
%
where $V$ is the volume of the mode in the cavity. The idea is to confine the
molecules in a trap that is on the $w=\SI{10}{\micro\meter}$ scale (see
section~\ref{overview:design}) and the resonator will necessarily have a length
on the scale of $\lambda_0 = 2\pi c / \omega_0$, therefore $V\approx
w^2\lambda_0$. Hence we have that
%
\begin{equation}
  g = \sqrt{\frac{2\pi c d^2}{\hbar \epsilon_0 w^2 \lambda_0^2}}.
\end{equation}

% TODO maybe move this whole sentence above
For the rotational \CaF{} transitions that we introduced above, in
section~\ref{theory:molecules} $d = \mu/\sqrt{3}$ with $\mu =
\SI{31}{\debye}$.
%
The coupling strength is therefore expected to be
%
% See nbs/2022-02-08_coupling.nb
\begin{equation}
  \frac{g}{2\pi} = \SI{20}{\kilo\hertz}
\end{equation}
%
and for strong coupling a cavity quality of
%
\begin{equation}
  Q = \frac{\omega_0}{g} > 1.7 \times 10^5
\end{equation}
%
is required.

\subsection{Stark shift}

Since the resonant frequency of the resonator is fixed by its length, it will
be useful to be able to tune the frequency of the molecule transition, 
brining the molecules into or out of resonance with the microwaves. We will
see that this is key for state readout in section~\ref{mws:readout}. We propose
that the detuning can be controlled by the Stark shift due to a d.c.~field.  By
positioning voltage-biased pads near to the molecule trap, the d.c.~field can
be used to shift the transition frequency.

When the shift is sufficiently small compared to the unperturbed energy, the
Stark shift can be found by second-order perturbation theory.
%
We start with the rotational eigenstates $\ket{N,m_N}$ whose energies are
$E_{rot}(N)=B N(N+1)$, and add the perturbation $H'=\mu_e*E \cos(theta)$, where
$\mu_e$ is the dipole moment in the frame of the molecule, $E$ is the electric
field applied along $z$, and $\theta$ is the angle between the dipole moment
and the $z$-axis. Recognizing that $\cos(\theta)$ is the reduced spherical
harmonic $C^{(1)}_0$, and applying second-order perturbation theory, we find
that the energy shift of $\ket{N,m_N}$ is
%
\begin{equation}
  \Delta_{N, m_N} = \sum_{N', m_N'} \frac{(\mu_e E)^2}{E_\text{rot}(N) -
  E_\text{rot}(N')} |{\bra{N', m_N'}}C_0^{(1)} \ket{N, m_N}|^2.
\label{mws:eqn:Stark}
\end{equation}
%
The matrix element here can be found by application of the Wigner-Eckart
theorem~\cite{edmonds1996}.
%
\begin{multline} {\bra{N', m_N'}}C_0^{(1)} \ket{N, m_N} =
  \\ (-1)^{m_N'}\sqrt{(2N+1)(2N'+1)}
  % 3 j symbols
  \begin{pmatrix} N' & 1 & N \\ -m_N' & 0 & m_N \end{pmatrix} \begin{pmatrix}
N' & 1 & N \\ 0 & 0 & 0 \end{pmatrix}
\end{multline}
%
where the last two quantities represent the usual 3j-symbols.


The summation in \myeqref{mws:eqn:Stark} can be simplified when we note that the 3j-symbols
are only non-zero when $m_N' = m_N$ and $N=N'$. It
is simple to compute the resulting energy shift for arbitrary states, and the
results are shown for the $N=0, 1, 2$ states of \CaF{} in \myfigref{mws:fig:stark}. We
note that it is possible to induce shifts
of several gigahertz with a modest electric field strength. We anticipate that
only shifts of $<\SI{10}{\mega\hertz}$ will be required in an
experiment, and for such small perturbations, any additional effects of mixing
hyperfine states, or effects due to the magnetic field can be ignored.

% Data generated by nbs/2022-02-16_stark.nb
\begin{figure}[h]
  \centering
  \begin{tikzpicture}
    \begin{axis}[
        enlargelimits=true,
        xlabel=Field strength (\si{\kilo\volt\per\centi\meter}),
        ylabel=State energy (\si{\giga\hertz}),
        width=0.7\textwidth,
        height = 0.4\textwidth,
        legend pos=outer north east
        %x tick label style={/pgf/number format/.cd, set thousands separator={}}
    ]
      \addplot [thick, black] table {figs/mws/starkReal.dat};
      \addlegendentry{$\ket{0,0}$}
      \addplot [style=thick, color=blue] table [y index=2] {figs/mws/starkReal.dat};
      \addlegendentry{$\ket{1,0}$}
      \addplot [style=thick, color=pink] table [y index=3] {figs/mws/starkReal.dat};
      \addlegendentry{$\ket{1,\pm1}$}
      \addplot [style=thick, color=purple] table [y index=4] {figs/mws/starkReal.dat};
      \addlegendentry{$\ket{2,0}$}
      \addplot [style=thick, color=gold] table [y index=5] {figs/mws/starkReal.dat};
      \addlegendentry{$\ket{2,\pm1}$}
      \addplot [style=thick, color=grey] table [y index=6] {figs/mws/starkReal.dat};
      \addlegendentry{$\ket{2,\pm2}$}
    \end{axis}
  \end{tikzpicture}
  \caption[The Stark effect in \CaF{}]{
    The Stark effect for low-lying rotational states of the \CaF{} molecule $\ket{N,
    m_N}$ found by second order perturbation theory.}
  \label{mws:fig:stark}
\end{figure}

\subsection{Integrating microwave components}
\label{mws:components}

% COMPONENTS
% Vacuum-side barrel connector
% https://www.mouser.co.uk/ProductDetail/Radiall/R127703001?qs=exnfk7tZe0CIIzofDocGig%3D%3D
% Vacuum-side launcher
% https://www.pasternack.com/2.92mm-female-end-launch-pcb-connector-pe45507-p.aspx
% Feedthroughs
% Kyle wrote in one note it's MDC SMA45-GS-WELD 
% I can't quite find this, think the new version is this one
% https://www.mdcprecision.com/9251004-sma-coaxial-feedthrough-1-pin-grounded-shield-double-ended-weldable

The chip flange (shown in \myfigref{overview:fig:chipchamber}) has been
designed to incorporate MDC Precision's SMA45-GS-WELD \SI{2.92}{\milli\meter}
high-frequency SMA feedthroughs. These welded connectors can be used to bring
microwaves inside the chip chamber, where they can be launched onto a CPW on
the subchip. Microwaves will be transferred through from the feedthrough to the
subchip using a SMA male-male barrel adapter from Pasternak, and a female-clamp
PCB launcher from Pasternack. The launching assembly can be seen in
\mysubfigref{mws:fig:implement}{a}. These components have been tested for UHV
compatibility, as discussed in section~\ref{exper:vacuum}.  The flange
assembly has been designed to allow microwaves to exit the chamber, where they
can be terminated or measured.

Once launched onto the subchip, the microwaves will be delivered by CPW to the
subchip. 
%
In future iterations of the experiment the aluminium-core PCB can be replaced
with, for example, an Arlon AD1000 board, which will be a suitable carrier for
both trapping currents and microwaves~\cite{Morgan2020}. 
%
The subchip CPW will be connected to the chip CPW by wire-bonds.  At
the point where the wire-bonds connect, the CPW can be chosen to be 
large enough so that it is easy to make good connection with the wire bonds. The CPW
can then tapered down to reduce the field size while maintaining the same $k_0$
and hence the same impedance. This CPW can be capacitively coupled to a
microwave resonator, but for a simple example we show a multi-layer design for
a single CPW in \myfigref{mws:fig:implement}.

\begin{figure}[htb]
    \centering
    \begin{tabular}[t]{cc}
\begin{subfigure}{0.4\textwidth}
    \centering
    \smallskip
    \begin{overpic}[abs,
      width=\textwidth]{figs/mws/FlangeAssemblyLabels.pdf}
      \put(10, 172){(a)}
    \end{overpic}
\end{subfigure}
    &
        \begin{tabular}{c}% if you add [t], than sub images are pushed down
        \smallskip
            \begin{subfigure}[t]{0.4\textwidth}
                \centering
                \begin{overpic}[abs, width=0.8\linewidth]{figs/mws/present_CPW2_IS.pdf}
                  \put(5, 123.8){(b)}
                \end{overpic}
            \end{subfigure}\\
            \begin{subfigure}[t]{0.4\textwidth}
                \centering
                \begin{overpic}[abs, width=0.8\textwidth]{figs/mws/present_CPW2_inset.pdf}
                  \put(5, 123.8){(c)}
                \end{overpic}
            \end{subfigure}
        \end{tabular}\\
    \end{tabular}
  \caption[Microwave components of the flange assembly]{
    The microwave components of the flange assembly (flange, feedthroughs,
    barrel connectors, launchers, PCB and chip) are shown in (a). In (b) we
    have the chip design from \myfigref{overview:fig:chipchamber} (not to
    scale) with the
    additional microwave guide layer overlayed (red). Conversely to the lower
    layer, the microwave layer shows areas where the dielectric is exposed, so
    regions that aren't pink are covered with conductor. Example voltage pads
    for Stark-shifting the molecules are also shown. Subfigure (c) shows a
    detailed view of the trapping region.
  }
  \label{mws:fig:implement}
\end{figure}


To form an even smaller trap than discussed above, the CPW centre conductor can
be biased, to create a dimple trap along with the underlying axis of the
Z-wire, as was done in \inlineref{Treutlein2008}. For a resonator it would also
be possible to introduce a bias field, but this would have to be done on-chip
with a band stop filter~\cite{doi:10.1063/1.4808364}. This could also be used
to implement the electrostatic traps described in \inlineref{Andre2006}.

\section{Sideband cooling}

This subsection summarises the sideband cooling scheme presented in
\inlineref{Andre2006} and presents cooling rates for our trapping scheme with
\CaF{}.  Throughout the section we assume that we operate with a single
molecule coupled to a high-quality cavity ($Q\sim10^6$, $\kappa \approx
2\pi\times \SI{20}{\kilo\hertz}$) in the strong coupling regime ($g\sim10^5$).
We take the molecule to act as a two-level system in the stretched state, which
we introduced in section~\ref{overview:existing}.  For our example we choose
$\ket{g}=\ket{0}_\text{str}$ and $\ket{e}=\ket{1}_\text{str}$.

In the macroscopic magnetic trap, the
$\ket{e}\rightarrow\ket{g}$ decay is very slow ($\Gamma \sim
10^{-5}\si{\hertz}$), but when a molecule is in close proximity to a microwave
resonator at the transition frequency, the decay rate is strongly enhanced
with rate
%
\begin{equation}
  \Gamma_c = \frac{\kappa}{2}.
\end{equation}
%
This enhanced decay rate opens the door to sideband cooling into the motional
ground state of the trap.

We write the state of the molecule as $\ket{i, m}$ where $i\in\{e,g\}$ is the
state of the molecule, and $m$ is the quantum number of the motion
in the trap. For any ground state $\ket{g,m}$ there is a carrier frequency
transition to $\ket{e, m}$ at the resonant frequency $\omega_0$, and sidebands
to the $\ket{e, m+m'}$ states at frequency $\omega_0 + m'\omega_t$, where
$\omega_t$ is the trap frequency and $m'\in \mathbb{Z}$. There are similar
carrier and sideband transitions for $\ket{e,m} \rightarrow \ket{g, m+m'}$.

We imagine that an external driving field is applied on the
$\ket{g, m} \rightarrow \ket{e, m-1}$ sideband as pictured in
\myfigref{mws:fig:sideband}. Combined with the enhanced decay, this allows
sideband cooling into low motional ground states of the trap. The molecule will
be transferred by the driving field into a lower motional state, and will
rapidly decay into the ground state by releasing a photon into the resonator.
Each time this happens the molecule's energy is reduced by $\hbar\omega_t$.
This results in a cooling rate
%
\begin{equation}
  R_\text{sbc} = \frac{\hbar\Gamma_c\omega_t}{k_B}.
\end{equation}


\begin{figure}[ht]
  \centering
  \includegraphics[width=0.6\textwidth]{figs/mws/sidebandcool.pdf}
  \caption[Sideband cooling scheme]{
    The sideband cooling scheme is shown, with an external driving field
    applied to the $\ket{g, m}\rightarrow\ket{e, m-1}$ transition. Decay on
    $\ket{e,m}\rightarrow\ket{g,m}$ is strongly enhanced by the resonator,
    allowing sideband cooling to low motional states.
  }
  \label{mws:fig:sideband}
\end{figure}


\begin{figure}[htb]
  \centering
  \begin{subfigure}[b]{0.44\textwidth}
  \begin{tikzpicture}
    \begin{axis}[
        enlargelimits=true,
        xlabel=Trapping current (\si{\ampere}),
        ylabel=$R_\text{sbc}$ (\si{\milli\kelvin\per\second}),
        width=\textwidth,
        height = 0.7\textwidth,
        x tick label style={/pgf/number format/.cd, set thousands separator={}}
    ]
      \addplot [thick, black] table {figs/mws/coolrate.dat};
    \end{axis}
    \node [] at (0.6, 2.7) {(a)};
  \end{tikzpicture}
  \end{subfigure}
  \hspace{1cm}
  \begin{subfigure}[b]{0.44\textwidth}
  \begin{tikzpicture}
    \begin{axis}[
        ymode=log,
        enlargelimits=true,
        xlabel=Trapping current (\si{\ampere}),
        ylabel=$T_\text{min}$ (\si{\nano\kelvin}),
        width=\textwidth,
        height = 0.7\textwidth,
        x tick label style={/pgf/number format/.cd, set thousands separator={}}
    ]
      \addplot [thick, black] table {figs/mws/mintemps.dat};
      \addplot [style=thick, color=blue] table [y index=2] {figs/mws/mintemps.dat};
      \addplot [style=thick, color=pink] table [y index=3] {figs/mws/mintemps.dat};
      %\addlegendentry{\SI{4}{\kelvin}}
    \end{axis}
    \node [] at (0.6, 2.7) {(b)};
  \end{tikzpicture}
  \end{subfigure}
  \caption[Calculated sideband cooling rates]{
    Subfigure (a) shows the cooling rate for varying trapping current for a
    $Q=10^6$ resonator and dimple trap (see section~\ref{theory:simple}).
    Subfigure (b) shows the minimum temperature that can be achieved as a
    function of trapping current for different resonator temperatures:
    \SI{4}{\kelvin} (black), \SI{400}{\milli\kelvin} (blue) and
    \SI{40}{\milli\kelvin} (red).
    }
    \label{mws:fig:sbtemps}
\end{figure}

% TODO Mike thinks the freq I state here is very high, and I think I don't
% ever really give any typical frequencies in theory, so I should check this
% point carefully
As per the discussion in section~\ref{theory:chips}, we expect typical trap
frequencies $\omega_t\sim 2\pi \times 10^5 \si{\hertz}$ and therefore the
cooling rate will be of order \SI{10}{\milli\kelvin\per\second}. This is shown
for a molecule in a dimple trap  at various trapping currents in
\mysubfigref{mws:fig:sbtemps}{a}.

The lowest motional state that can be reached is determined by the background
photon number in the resonator. A typical high-$Q$ cavity will operate at a few
tens of millikelvin~\cite{doi:10.1063/1.3010859} but these temperatures require
a dilution fridge, and high-temperature superconductors can operate at typical
cryocooler temperatures of $T_\text{c}=\SI{4}{\kelvin}$. This corresponds
to a background photon number of
%
\begin{equation}
  \overline{m} = \frac{k_B T_c}{\hbar \omega_0}.
\end{equation}
%
It has been commonly shown that other sources of background photons can be
eliminated~\cite{Wallraff2004}, in particular it can be useful to filter
high-frequency photons that can enter from the input
ports~\cite{doi:10.1063/1.3638063}. 

We expect the minimum temperature achievable by sideband cooling to be reached
when the molecule motion and photon energy are in thermal equilibrium, so that
%
\begin{align}
  T_\text{min} &= \frac{\hbar \omega_t \overline{m}}{k_B}
               &= \frac{\omega_t}{\omega_0}T_c
\end{align}
%
which is shown as a function of the trapping current for various resonator
temperatures in \mysubfigref{mws:fig:sbtemps}{b}.

\section{Entangling light and photon states}

Reference~\cite{Andre2006} also proposes entangling the molecule state with the
quantum state of photons in the resonator. This is inspired by the methodology
of \inlineref{PhysRevA.69.062320} where the state of a Cooper-pair box is
entangled with a resonator state. This entanglement can be used to perform
state readout, state preparation and coupling to neighbouring qubits. In this
section we will mainly discuss state readout, since this framework will be
useful in chapter~\ref{squeeze}.

We again assume that the molecule is trapped in the $N=0$ or $N=1$ stretched
states. In this case the resonator is driven by a microwave field, so that the
molecule-resonator coupling is described by the Jaynes-Cummings
Hamiltonian that was discussed in section~\ref{mws:CQED}.
%
Entangling of the light and molecule state can be achieved in the dispersive
regime, where the transition frequency is tuned so that $|\Delta|\gg g$.
Reference~\cite{PhysRevA.69.062320} tells us that we can gain insight into the
dispersive behaviour by applying the unitary transformation
%
\begin{equation}
  U = \exp \left[\frac{g}{\Delta}(a\sigma_+ - a^\dagger\sigma_-)\right].
  \label{mws:eqn:Utransform}
\end{equation}
%
With expansion up to second order in $g/\Delta$, we obtain
%
\begin{equation} H= UH_\text{JC}U^\dagger \approx \hbar \omega_c
  a^\dagger a + \frac{\hbar}{2}\left(\omega_c +
  \frac{g^2}{\Delta}\right)s_z + \frac{\hbar
  g^2}{\Delta}s_z a^\dagger a.
  \label{mws:eqn:UHU}
\end{equation}
%
% Could have s_i in theory, just pass this to mike and see what he says...
Here we have introduced the spin operator $s_i = \sigma_i/2$ for convenience.

The three terms of $H$ describe the oscillation of light in the cavity, the
energy of the spin and the interaction of the photons with the spin. Note that
the interaction of the photons and the spins induces the usual AC Stark shift
proportional to $(n+\frac{1}{2})$. The last term can be used to make a quantum
non-demolition (QND) measurement, since it will enable exchange of information
between the z-component of the spin with the photons.

We take the state of the photons in the resonator to be in a canonical coherent
state~\cite{Gazeau2009}
%
\begin{equation}
  \ket{\alpha} = e^{-\frac{|\alpha|^2}{2}}\sum_{n=0}^\infty \frac{\alpha^n}{\sqrt{n!}} \ket{n}
  \label{mws:eqn:coherent}
\end{equation}
%
with $\alpha\in\mathbb{C}$, and $\ket{n}$ representing the $n^\text{th}$
Fock state of the light~\cite{agarwal2012}. For such a state $|\alpha|^2$ is
the average photon number, with $a^\dagger a$ being the number operator such
that $\bra{\alpha}a^\dagger a \ket{\alpha} = |\alpha|^2$.

The interaction of the light with the molecule is described by the last term in
$H$, so for an interaction over time $T$, we have
%
\begin{equation}
  \ket{\Psi(T)} = \exp\left(-iH_\text{int}T/\hbar\right)\ket{\Psi(0)}
\end{equation}
%
where
%
\begin{equation}
  H_\text{int} = \hbar \frac{g^2}{\Delta} s_z a^\dagger a
\end{equation}
%
and $\ket{\Psi(0)} = \ket{\psi}\ket{\alpha}$ is the state of the system at the
time of measurement. This can be expanded by inserting the definition of the
coherent and molecule states,
%
\begin{equation}
  \ket{\Psi(T)} = e^{-\frac{|\alpha|^2}{2}}\sum_{n=0}^\infty
   \frac{\alpha^n}{\sqrt{n!}} e^{-i\nu T s_z a^\dagger a} \ket{n} (\cos\theta
   \ket{g} + e^{i\phi}\sin\theta\ket{e})
   \label{mws:eqn:evolve1}
\end{equation}
%
where $\nu = g^2/\Delta$.
Now the number operator in the exponent acts on $\ket{n}$, and the spin
operator acts on $\ket{N}$ for
%
\begin{equation}
   \ket{\Psi(T)} = e^{-\frac{|\alpha|^2}{2}}\sum_{n=0}^\infty
   \frac{\alpha^n}{\sqrt{n!}} (e^{i\nu T n \hbar/2}\cos\theta\ket{n}\ket{g} +
   e^{-i\nu T n \hbar/2}e^{i\phi}\sin\theta\ket{n}\ket{e}).
\end{equation}
%
Collecting the coefficients of the molecule states yields
%
\begin{equation}
  \ket{\Psi(T)} = \cos\theta\left(e^{-\frac{|\alpha|^2}{2}}\sum_{n=0}^\infty
   \frac{(\alpha e^{i\nu T \hbar/2})^n}{\sqrt{n!}}\ket{n}\right)\ket{g} +  
    e^{i\phi}\sin\theta\left(e^{-\frac{|\alpha|^2}{2}}\sum_{n=0}^\infty
   \frac{(\alpha e^{-i\nu T \hbar/2})^n}{\sqrt{n!}}\ket{n}\right)\ket{e}.
\end{equation}
%
Finally, note that each of the photon states (in parentheses) defines a
coherent state, so the resulting state after interaction is
%
\begin{equation}
  \ket{\Psi(T)} = \cos\theta\ket{\alpha_+}\ket{g} +
  e^{i\phi}\sin\theta\ket{\alpha_-}\ket{e}
  \label{mws:eqn:entangled}
\end{equation}
%
where $\alpha_\pm = \alpha \exp(\pm i \nu T \hbar/2)$.

\subsection{State readout}
\label{mws:readout}

The state of the molecule is now entangled with the state of the light in such
a way that measuring the phase of the light will perform a readout of the
molecule state. In this subsection I will present the homodyne
measurement~\cite{agarwal2012} technique in the context of performing a
measurement of our molecule state.

The homodyne measurement is illustrated in \myfigref{mws:fig:homodyne}. The
light to be measured, here labelled $\ket{\Psi_a}$, is incident on one port (a)
of a beam splitter. On the other port (b) we have a strong local oscillator in
a coherent state $\ket{\beta}$, with large amplitude, in this case meaning that
$|\beta| \gg |\alpha|$. We set the relative phases of $\ket{\alpha}$ and
$\ket{\beta}$ so that $\arg(\alpha)=0$ and $\arg(\beta)=-\varphi$.

\begin{figure}
  \centering
  \includegraphics[width=0.4\textwidth]{figs/mws/homodyne.pdf}
  \caption[Schematic of a homodyne measurement]{
    Schematic of a homodyne measurement. The state to be measured
    $\ket{\Psi_a}$ is incident on port (a) of the beamsplitter, and the local
    oscillator $\ket{\beta}$ is incident on port (b). The signal from
    photodiodes at the output ports is summed to produce a homodyne
    measurement.
  }
  \label{mws:fig:homodyne}
\end{figure}

The annihilation operators associated with the input ports are related to those
of the output ports (c and d) by the usual relation for a balanced beam
splitter~\cite{agarwal2012}
%
\begin{equation}
  \label{squeeze:eqn:bsmat}
  \begin{pmatrix} c \\ d \end{pmatrix} = \frac{1}{\sqrt{2}}\begin{pmatrix}
    1 & i \\ i & 1 
  \end{pmatrix}  \begin{pmatrix} a \\ b \end{pmatrix}.
\end{equation}

The difference in the expected photon numbers arriving at each
detector is therefore
%
\begin{align}
  \langle c^\dagger c - d^\dagger d\rangle &= i\langle a^\dagger b-
  ab^\dagger \rangle \\
  &= i \bra{\Psi_a}\bra{\beta}(a^\dagger b-
  ab^\dagger)\ket{\beta}\ket{\Psi_a} \\
  & = i|\beta| \bra{\Psi_a}(a^\dagger e^{i\varphi} - a
  e^{-i\varphi})\ket{\Psi_a}.
\end{align}
%
Note that in the last equality we have used $b\ket{\beta}=\beta\ket{beta}$, and
also the equivalent conjugate expression $\bra{\beta}b^\dagger =
\beta^*\ket{\beta}$ to remove the $b$ and $b^\dagger$ operators and replace
them with their eigenvalues.
%
We now introduce the canonical quadratures of the light field, corresponding to
its real and imaginary parts. They are defined by~\cite{gerry_knight_2004}
%
\begin{align}
  X = \frac{a + a^\dagger}{2} && Y = \frac{a - a^\dagger}{2i}.
\end{align}
%
The expected photon difference is now
\begin{equation}
  \langle c^\dagger c - d^\dagger d\rangle =
  2|\beta|\bra{\Psi_a}(Y\cos\varphi - X\sin\varphi)\ket{\Psi_a} \\
  \label{squeeze:eqn:homoquads}
\end{equation}

% For transmission coefficient, can see Mauro's paper (Q. Tomography), eqn.
% 2.37 - 2.42 but I am ignoring it here (just reduces $\alpha$)
Hence measuring the intensity of each output of the beamsplitter can give us
information on the phase of the light. We choose $\varphi = 0$ so that the
measurement is of the $Y$ quadrature, i.e.\ we measure the imaginary part of
$\ket{\Psi_a}$
%
\begin{equation}
  \langle c^\dagger c - d^\dagger d\rangle =  2
  |\alpha||\beta|\left\langle\sin(\nu T s_z)\right\rangle.
\end{equation}
%
For a short pulse of light, the interrogation time will be the lifetime of the
photons in the cavity, $T = \kappa^{-1}$. In this regime we can expand sine
to first order, so that
%
\begin{equation}
  \langle c^\dagger c - d^\dagger d\rangle = 2|\alpha||\beta|
  \frac{g^2}{\Delta\kappa}\langle s_z\rangle.
  \label{mws:eqn:homomeas}
\end{equation}
%
In other words, the expectation value of the photon measurement is linked
directly to that of the molecule's state. A measurement of $\langle c^\dagger c
- d^\dagger d\rangle$ will also measure $s_z$. This allows for readout of the
spin state via the microwave ports.

For this readout procedure, we have made the assumptions that we are operating
in the dispersive regime ($|\Delta| \gg g$) but in the final step
(\myeqref{mws:eqn:homomeas}) the expansion of sine require a stricter
condition, that 
%
\begin{equation}
  \frac{g^2}{\Delta\kappa} \ll 1.
\end{equation}
%
The detuning required for our $g$ and $Q$ is therefore
%
% From calculation in 2022-02-08_coupling
\begin{equation}
  \frac{\Delta}{2 \pi} \gg \SI{5}{\mega\hertz}
\end{equation}
%
which is well within the bounds achievable with the Stark shift.

\subsection{State preparation}

The engtangling of the molecule and light state will produce a powerful system
with applications beyond simple state readout. First, the system can be used to
produce Schr\"odinger cat states in the light field~\cite{Andre2006}. This can be achieved
by pulsing the microwaves to produce the entangled state in
\myeqref{mws:eqn:entangled}. Expressing the molecule state in the $x$ basis
$\ket{\pm} = (\ket{g} \pm \ket{e})/2$, this becomes
%
\begin{equation}
  \ket{\Psi(T)} = \ket{\Alpha_+(\theta, \phi) }\ket{+} + \ket{\Alpha_-(\theta,
  \phi)}\ket{-}
\end{equation}
%
with
%
\begin{equation}
  \ket{\Alpha_\pm(\theta, \phi)} = \cos\theta\ket{\alpha_+} \pm
  e^{i\phi}\sin\theta\ket{\alpha_-}.
\end{equation}
%
Now independent measurement of the molecule state in the $x$ basis will yield
$\ket{\pm}$, heralding the creation of the $\ket{\Alpha_\pm(\theta, \phi)}$
Schr\"odinger cat state in the light field. This is an attractive method for
producing such states in the microwave regime, in analogy to what has
previously been achieved using visible light~\cite{Hacker2019}.

\subsection{Coupling between molecules}

As reported in \inlineref{Andre2006}, the entanglement can also be used to
couple between molecules trapped near the same resonator. We write the state of
the two molecules as $\ket{\Psi}_i$, $i\in\{1,2\}$, with the detunings
individually controlled labelled $\Delta_i$. Molecule $1$ can be addressed
independently of $2$ by setting $\Delta_2 \gg \Delta_1$ and visa versa. 

When $\Delta_1 = \Delta_2$ the interaction Hamiltonian between the two
molecules can be found by adiabatic elimination of the photon state to
be~\cite{Andre2006}
%
\begin{equation}
  H_\text{int} = \hbar \frac{g^2}{\Delta} (\sigma_+^1\sigma_-^2 +
  \sigma_-^1\sigma_+^2)
\end{equation}
%
where $\sigma^i_k$ is the $\sigma_k$ operator acting on the
$i^\text{th}$ molecule. It can be shown that such an interaction can be used to
implement high-fidelity two-qubit operation on a chip~\cite{Andre2006,
PhysRevA.69.062320}.
