% TODO Better intro spiel
\cm{
In chapter (probably the introduction) I introduced the concept of microwave
guides integrated into chip traps. Previous experiments have used on-chip
microave guides to couple to \cm{(I think this is right)} the hyperfine structure
of atoms. However, the coupling to heteronuclear dipolar molecules is much
stronger \cm{Why?}. In the strong coupling regime it may be possible to \cm{(do
cavity QED, sideband cooling, etc...)}
%
Need to briefly say that we will use the stretched states, but the detail on
this comes later.
}

\section{The coplanar waveguide}

\cm{The following is mostly from my ESA so will need editing! Important
takeaways are the field strcutre, where the molecules need to sit in this
field, and what we need to do to get a cavity. What can we do if we don't have
a cavity?}

The coplanar waveguide (CPW) was originally proposed by Cheng P.~Wen as a
means of guiding microwaves across the surface of a dielectric
substrate~\cite{1127105}. It consists of a central conductor with a ground
plane on either side, as pictured in~\myfigref{mws:fig:CPW}. CPWs have become
prolific, since they allow the creation of robust microwave devices, and offer
some benefits over other waveguide \cm{architectures} such as the stipline
waveguide because they can provide circularly polarised fields, and also
provide easy access to the ground plane for shunt connections.

The CPW's geometry is defined by the centre conductor width ($S$) and the
channel width ($W$). The height of the conductor is $t$ and the dielectric
height is $h$. 
The geometry of the CPW determines the region the
microwave field occupies, as illustrated in \myfigref{experiment:fig:CPWfield}.
The CPW can therefore be designed to maximise overlap between the microwave
field and a cloud of molecules trapped nearby.  As a rough approximation the
molecules should be trapped at a distance $h\sim S$ above the centre of the CPW
to achieve a good overlap.~\cite{Boehi2009} The field surrounding the CPW is
shwon in \mysubfigref{mws:fig:CPW}{b}, the equations defining this field can be
found analytically~\cite{Simons2004}, or the field can be determined by a
finite-element simulation.

% TODO 
\begin{figure}[ht]
  \centering
  \begin{subfigure}[b]{0.45\textwidth}
    \begin{overpic}[abs, width=\textwidth]{figs/mws/CPW_dims.pdf}
      \put(20, 180){(a)}
      \put(100, 25){$S$}
      \put(126, 12){$W$}
      \put(15, 43){$h$}
      \put(193, 58){$t$}
      \put(40, 43){$\epsilon_r$}
    \end{overpic}
  \end{subfigure}
  \begin{subfigure}[b]{0.45\textwidth}
    \begin{overpic}[abs, width=\textwidth]{figs/mws/CPWMode_legend.png}
      \put(20, 180){\textcolor{white}{(b)}}
      \put(165, 180){\textcolor{white}{High}}
      \put(165, 110){\textcolor{white}{Low}}
		  \put(20, 20){\textcolor{white}{\SI{20}{\micro\meter}}}
    \end{overpic}
  \end{subfigure}
	\caption{Subfigure (a) shows a perspective view of a segment of CPW on dielectric substrate, with
pertinent geometery and dielectric constants labeled. The conductor is in gold
  and the dielectric is left transparent. A COMSOL finite-element simulation of the
resulting field density near such a CPW is shown in (b), with
$S=\SI{20}{\micro\meter}$, $W=\SI{10}{\micro\meter}$, $\epsilon_r=10$,
$h=\SI{10}{\micro\meter}$. Conductor regions are highlighted in gold and the
end of the bottom of the substrate is marked with the black line. The field is
presented in arbitrary units to illustrate where strong coupling can be
achieved. Note that in a resonator the field will depend on the losition in the
longitudinal direction.}
	\label{mws:fig:CPW}
\end{figure}

\subsection{CPW properties}

We will now consider the electric properties of a CPW. These will largely
depend on the the planar geometry of the waveguide, which we express in terms
of the ratio~\cite{1127105, Simons2004}
%
\begin{equation}
  k_0 = \frac{S}{S+2W} = \sqrt{1-{k'_0}^2}
  \label{eqn:k0def}
\end{equation}
%
For a waveguide made of a material with conductivity $\sigma$
and permeability $\mu$ and propagating field with frequency $\omega$
there is a skin depth~\cite{Simons2004}
%
\begin{equation}
  \delta = \sqrt{\frac{2}{\omega\mu\sigma}}
\end{equation}
%
and skin effect surface resistance~\cite{Simons2004}
%
\begin{equation}
  R_s = \frac{1}{\sigma\delta}.
\end{equation}

Since the field of the CPW will pass through both the dielectric and the
surrouding air, the CPW's capacitance is the sum of the capacitance of each of
these two parts
%
\begin{equation}
  C_\text{CPW} = C_\text{dielectic} + C_\text{air}.
\end{equation}
%
The method of finding these individual contributions is somewhat involved, so
we simply state the results here in terms of the elliptic integral of the
first kind $K(k)$~\cite{Simons2004}. The capacitance due to the dielectric is
%
%
\begin{equation}
  C_\text{dielectric} = 2\epsilon_0(\epsilon_\text{r1}-1)\frac{K(k_0)}{K(k'_0)}
\end{equation}
%
and the capacitance of the air region is
%
\begin{equation}
  C_\text{air} = 4\epsilon_0 \frac{K(k_0)}{K(k'_0)}.
\end{equation}

We can use this to find the effective permittivity~\cite{Simons2004}
%
\begin{align}
  \epsilon_\text{eff} &= \frac{C_\text{dielectric}}{C_\text{air}} \\
    &= \frac{1+ \epsilon_\text{r1}}{2} \\
\end{align}
%
where we have assumed that we are in the limit of a thick dielectric layer ($h
\gg W$ so that most of the field density is inside the dielectric.
%
The phase velocity (using $c$ as the speed of light) is~\cite{Simons2004}
%
\begin{align}
  v_\text{ph} &= \frac{c}{\sqrt{\epsilon_\text{eff}}} \\
    &= \frac{c}{\sqrt{(1 + \epsilon_\text{r1})/2}}.
\end{align}

Now using the approximation~\cite{Collin2007}
$\sqrt{\mu_0/\epsilon_0}\approx120\pi\text{Ohms}$ we have the line
impedance~\cite{Simons2004}
\begin{align}
  Z_0 &= \frac{1}{C_\text{air} v_\text{ph}} \\
    &= \frac{30 \pi}{\sqrt{(\epsilon_\text{r1}+1)/2}} \frac{K(k_0)}{K(k'_0)}
    \text{Ohms}
    \label{mws:eqn:Z0}
\end{align}
Note that the impedance of the waveguide has dependence only on the geometry in
the form of the ratio $k_0$, and the relative permittivity of the
substrate.~\cite{Simons2004} This means that for any substrate we choose the
value of $k_0$ can be chosen to fix the impedance at the standard $Z_0 =
\SI{50}{\ohm}$.

Therefore as long as $k_0$ is held constant the CPW can be tapered to change the
size of the centre conductor, and hence control the region the field occupies.

\subsection{CPW attenuation}

The electric field inside a CPW evolves according to
%
\begin{equation}
  \widetilde{E}(z) = \widetilde{E}(0)e^{-\gamma z}.
  \label{mws:eqn:Eloss}
\end{equation}
%
where $\gamma = \alpha + i\beta$, $\beta = 2\pi / \lambda$ is the wave
number and $\alpha$ is the attenuation constant.~\cite{Simons2004}
Taking the absolute value, the amplitude falls off as
%
\begin{equation}
  E(z) = E(0)e^{-\alpha z}.
\end{equation}

There are two contributing terms to the attenuation constant
%
\begin{equation}
  \alpha = \alpha_d + \alpha_c,
\end{equation}
%
there are the contributions from dielectric and conductor respectively.
Negligible contributions from radiative losses are
neglected~\cite{Frankel1991}.  Bending losses can be neglected as long as
the radius of the bends are much larger than the wavelength of the propagating
wave. We must be mindful of this during design, since the typical wavelengths
for rotational transitions are on the order of centimeters.

\cm{Fix, there is a bit about this in Goppl}
We also do not consider the insertion loss of the
resonator, since this does not affect the quality
factor~\cite{doi:10.1063/1.3010859}.

\subsubsection{Dielectric losses}

Dielectric losses in the thick dielectric limit are described
by ~\cite{Collin2007}
\begin{equation}
  \alpha_d =
  \frac{\omega_0}{4c}\frac{\epsilon_\mathrm{r1}}{\sqrt{\epsilon_\mathrm{eff}}}
  \tan \delta_e
\end{equation}
%
where $\tan\delta_e$ is the dielectric loss tangent. Common dielectrics for
microwave guides include aluminium nitride (\AlN{}) and high-resitivity
silicon.  For our purposes we wish to consider waveguides situated on
dielectrics layers that can be easily deposited above the trapping wires.
Following the work in \inlineref{Treutlein2008}, we primarily consider polyimide, which
can be despoitied by spoin-coating. Other options are available, such as
polyethylene naphthalate (PEN) ~\cite{WEI20169937}.

Typical values of dielectric constants and $\alpha_d$ are shown in
\mytableref{mws:table:diprops}. Note that these parameters have some dependence
on the frequency and temperature of the dielectric, and so are presented to
illustrate the amount of loss that can typically be expected for our experiment
at room temperature and frequencies in the \SI{10}{\giga\hertz} r\'egime.

\begin{table}[ht]
  \caption{Dielectric constants and loss for various substrates in the
  \SI{10}{\giga\hertz} r\'egime at room temperature}
\centering
\begin{tabular}{l c c c c }
\hline\hline
  Material & $\epsilon_r$ & $\tan\delta_e$ & $\alpha_d$ (\si{\per\meter})& Ref. \\ [ 0.5ex]
\hline
  Polyimide & 3.4 & \SI{1.8E-2}{} & 0.45 & \cite{DuPontKapton} \\
  PEN & 2.56 & 0.003 & 0.63 & \cite{WEI20169937} \\
  Silicon (high resistivity)& $\sim{12}$ & \SI{2E-4}{} & 0.02 & \cite{Simons2004, 1717770, doi:10.1063/1.4929503} \\
  Sapphire & $\sim10$ & $2\times10^{-5}$ & $10^{-3}$ & \cite{mw101}\\ % TODO Better cite?
  Aluminium nitride & 8.9 & \SI{5E-4}{} & 0.22 & \cite{mw101} \\ % TODO Better cite?
\hline
\end{tabular}
\label{mws:table:diprops}
\end{table}

\subsubsection{Conductor losses}

Conductor losses arise due to dissipation in the centre conductor and ground
plane of the CPW~\cite{Simons2004}.
The conductor attenuation constant is
\begin{equation}
  \alpha_c = \frac{R_c +R_g}{2Z_0}.
\end{equation}
Where $R_c$ and $R_g$ are the series resistances of the centre conductor and
the ground plance respectively.
For a waveguide with height 
$t$, theses are given by
\begin{equation}
  R_c = \frac{R_s}{4 S(1-k_0^2)K^2(k_0)}\left[ \pi + \log\left(\frac{4\pi
  S}{t}\right) - k_0\log\left(\frac{1+k_0}{1-k_0}\right) \right],
\end{equation}
and
\begin{equation}
  R_g = \frac{k_0 R_s}{4S(1-k_0^2)K^2(k_0)}\left[\pi +
  \log\left(\frac{4\pi(S+2W)}{t}\right) -
  \frac{1}{k_0}\log\left(\frac{1+k_0}{1-k_0}\right)\right].
\end{equation}
%
The electrical properties of silver and gold are given in
\mytableref{mws:table:metalprops}, and the computed conductor attenuations are
shown in \myfigref{mws:fig:conductorloss}.

\begin{table}[ht]
  \caption{Electrical constants and loss for various conductors in the
  \SI{10}{\giga\hertz} r\'egime at room temperature}
\centering
\begin{tabular}{l c c c}
\hline\hline
  Material & $\sigma$ (\si{\siemens\per\meter}) & $\mu_r$ & Ref. \\ [ 0.5ex]
\hline
  Gold & & & \cite{} \\
  Silver & & & \cite{} \\
\hline
\end{tabular}
\label{mws:table:metalprops}
\end{table}

% Data generated by nbs/2022-02-06_CPWresonators.nb
\begin{figure}[h]
  \centering
  \begin{subfigure}[b]{0.48\textwidth}
  \begin{tikzpicture}
    \begin{axis}[
        xmode=log,
        enlargelimits=true,
        xlabel=$t$ (\si{\nano\meter}),
        ylabel=$\alpha_c$ (\si{\neper\per\meter}),
        width=\textwidth,
        height = 0.7\textwidth,
        x tick label style={/pgf/number format/.cd, set thousands separator={}}
    ]
      \addplot [thick, black] table {figs/mws/conductorlossvaryt.dat};
      \addplot [style=thick, color=blue] table [y index=2] {figs/mws/conductorlossvaryt.dat};
      \addplot [style=thick, color=pink] table [y index=3] {figs/mws/conductorlossvaryt.dat};
      \node [] at (5cm, 3.1cm) {(a)};
    \end{axis}
  \end{tikzpicture}
  \end{subfigure}
  \begin{subfigure}[b]{0.48\textwidth}
  \begin{tikzpicture}
    \begin{axis}[
        enlargelimits=true,
        xlabel=$S$ (\si{\micro\meter}),
        %ylabel=$\alpha_c$ (\si{\neper\per\meter}),
        width=\textwidth,
        height = 0.7\textwidth,
        x tick label style={/pgf/number format/.cd, set thousands separator={}}
    ]
      \addplot [thick, black] table {figs/mws/conductorlossvarys.dat};
      \addplot [style=thick, color=blue] table [y index=2] {figs/mws/conductorlossvarys.dat};
      \addplot [style=thick, color=pink] table [y index=3] {figs/mws/conductorlossvarys.dat};
      \node [] at (5cm, 3.1cm) {(b)};
    \end{axis}
  \end{tikzpicture}
  \end{subfigure}
    \caption{
      Conductor losses are plotted for (a) varying the conductor height of the
      CPW and (b) varying the centre conductor width. Losses are shown for gold
      on sapphire (black), silver on sapphire (blue) and gold on polyiide
      (red). In (a) a typical conductor width of $S=\SI{10}{\micro\meter}$ is
      assumed, and in (b) the thickness of the conductors is taken to be
      $t=\SI{5}{\micro\meter}$. In both cases the impedance is set to $Z_0 =
      \SI{50}{\ohm}$ by choice $k_0$ by \myeqref{mws:eqn:Z0}.
    }
  \label{mws:fig:conductorloss}
\end{figure}

The conductor losses clearly dominate over the dielectric losses. For this
reason it is common to use superconductors for CPWs~\cite{}, since for a
superconductor $\sigma \to \infty$ and hence $R_s$ and $\alpha_c \to 0$.
This will not prevent the implementation of a microwave guide on a multi-layer
chip trap, but it will impact our ability to build incorporate a microwave
resonator as we will now discuss.

\cm{Clear up this para, but use as basis for explaining that a waveguide can be
built}
That said, it should be noted that the high loss should not prevent us from
using the CPW to directly drive microwave transitions in trapped molecules, as
we will still be able to pass signal through the waveguide without the need to
construct a resonator. With loss of $\alpha \approx \alpha_c$, and waveguides of
lengths on the scale of a few centimeter, we can expect a total loss of around
$\SI{400}{\neper\per\metre} \times \SI{0.01}{\metre} = \SI{35}{\decibel}$. This
should be sufficiently small that microwaves will be able to reach molecules
trapped on the chip.~\cite{Treutlein2008}
One of the later goals of the project will be to characterise microwave losses
on the chip and compare these with those expected from theory.

\subsection{CPW resonators}
\label{mws:resonators}

We discussed in chapter~\ref{theory} that strong-coupling between a microwave
resonator and a trapped molecule has the potential to be a powerful tool for
\cm{new physics}. These proposals will be explored further in the next section, but
first we present how a microwave resonator can be implemented with a CPW.

A microwave resonator can be formed from a section of CPW that is capacitively
coupled to another driving segment~\cite{Day2003}. The resonant frequency is
determined by the resonator's length, $L$, and the waveguide's phase
veloctiy~\cite{Simons2004}
%
\begin{equation}
  \omega_0 = \frac{\pi v_\text{ph}}{L} = \frac{\pi
  c}{\sqrt{\epsilon_\text{eff}} L}
\end{equation}
%
Microwaves can be coupled to the resonator by positioning the resonator inline
with a microwave guide, or in parallel as shown in \cm{some figure that I need
to make}. The strength of the coupling can be controlled by the shape of the
interface between the resonator and the waveguide~\cite{doi:10.1063/1.3010859}

Recall from the discussion in chapter~\ref{theory} that for strong coupling we
require $Q > 1.7 \times 10^5$. The quality factors for CPW resonators can be
expressed in terms of an attenuation constant
%
\begin{equation}
  Q = \frac{\omega_0}{2c\alpha}.
  \label{mws:eqn:Qalpha}
\end{equation}
%
From \myfigref{mws:fig:conductorloss} and \mytableref{mws:table:diprops} it is
clear that without the use of superconductors, the attenuation will be
dominated by conductor losses, and $Q\sim10$. This is clearly not sufficient
for our purposes.

Conductor losses can be eliminated by use of a superconducting CPW. CPW
resonators are a developed technology, commonly used for coupling to
solid-state qubits~\cite{Wallraff2004}. Operating at the low temperatures
required for superconductors allows the achievement of very high quality
factors of order $10^6$~\cite{doi:10.1063/1.3552890, Day2003}. For such devices
a low-loss dielectric is usually chosen for the substrate (sapphire being a
common choice) however a multi-layer resonator could potentially be achieved by
etching the polyimide in the slot gaps, as in \inlineref{920142}. Otherwise, a
single-layer superconducting design such as that shown in
\inlineref{Hattermann2017} could be adopted. Regardless, the multilayer design
remains a useful tool for prototyping a \cm{strong-coupling experiment.}
\cm{Rephrase this to stress that it might still work with resonator.}

% I have this point from Goppl that at low drive powers Q is somehow reduced. I
% don't understand this really, and don't think it's that relevant because we
% are so far off single-photon regime here anyway. But didn't want to entirely
% forget, so now this comment exists.

\section{Implementation of CPW-molecule coupling}
\label{mws:integrating}

Having established the CPW, and that such a device can operate on a polyimide
substrate, we can now consider the logistics of implementing such a device
with our design. In this section we will consider the \cm{states that we will
use} and \cm{how the transition frequency can be tuned to (or away from) the
resonator frequency}. We have already outlined how to fabricate a multi-layer device
in chater~\ref{fab}, but the problem of delivering microwaves to the CPW
will also be discussed in this section.

\cm{I like this para, but I think it doesn't really go here... Definitely need
to say something about resonator vs. waveguide}
%
Molecules trapped on a chip near a CPW (not a resonator) could be used for
microwave spectroscopy experiments, similar to the free space experiments
described in section~\ref{theory:molecules} and
\inlineref{WilliamsMagnetic2018}. Coupling molecules to a resonator however
promises a variety of new tools for quantum control of the molecules. Although
there are various obstacles to overcome in building such a device, it is
instructive to consider its operation. \cm{Better phrasing}

\subsection{Rotational transition}

\cm{
Specify the states of the molecule that we will use
%
Specify coupling parameters, etc.
}

\subsection{Stark shift}

Since the resonant frequency of the resonator is fixed by its length, it will
be useful to be able to tune the frequency of the molecule transition, either
to bring the molecules \cm{into} or out of resonance with the microwaves. We
propose that this can be done by using the Stark shift due to a d.c.~field.
By positioning voltage-biased pads near to the molecule trap, the d.c.~field
can be used to shift the transition by \cm{how much?}.

\cm{Stark shift theory here}

\subsection{Integrating microwave components}

% TODO Fill in blanks, note the chapter ref. is likely to change in the near
% future (might discuss subship in fab)
% TODO Add arlon thing to dielectrics table above
The chip flange (shown in \myfigref{overview:figs:chipexperiment}) has been
designed to incorporate high-frequency coaxial feedthroughs. \cm{Company name/
spec. here.} These can be used to bring microwaves inside the chip chamber,
where they can be launched onto a CPW on the subchip. As discussed in
chapter~\ref{} the subchip in its current form is \cm{an aluminium-core PCB},
however in future iterations of the experiment this can be replaced with, for
example, an \cm{Arlon something somthing} board, which will be a suitable
carrier for both trapping currents and microwaves. This is inspired by the
atom-chip design presented in \cm{Hogan paper}. A schematic of the proposed
microwave delivery system is shown in \mysubfigref{mws:fig:implement}{a}.
\cm{Need to mention that I vacuum-tested these components (point to experiment
chapter?)}

\begin{figure}[ht]
  \centering
  \begin{subfigure}[b]{0.3\textwidth}
    \begin{overpic}[abs,
      width=\textwidth]{figs/mws/FlangeAssemblyFinalMWPresent.png}
      \put(10, 132){(a)}
    \end{overpic}
  \end{subfigure}
  \begin{subfigure}[b]{0.3\textwidth}
    \begin{overpic}[abs, width=\textwidth]{figs/mws/present_CPW2_IS.pdf}
      \put(10, 132){(b)}
    \end{overpic}
  \end{subfigure}
  \begin{subfigure}[b]{0.3\textwidth}
    \begin{overpic}[abs, width=\textwidth]{figs/mws/present_CPW2_inset.pdf}
      \put(10, 132){(c)}
    \end{overpic}
  \end{subfigure}
  \caption{
    The microwave components of the flange assembly (flange, feedthroughs,
    barallel connectors, launchers, PCB and chip) are shown in (a). In (b) we
    have the chip design from \myfigref{overview:figs:chipexperiment} (not to
    scale) with the
    additional microwave guide layer overlayed (red). Conversely to the lower
    layer, the microwave layer shows areas where the dielectric is exposed, so
    regions that aren't pink are covered with conductor. Example voltage pads
    for Stark-shifting the molecules are also shown. Subfigure (c) shows a
    detailed view of the trapping region.
  }
  \label{mws:fig:implement}
\end{figure}

The subchip CPW will be connected to the \cm{science chip} CPW by wire-bonds.
These can be chosen to be arbitrarily large s that it is easy to make good
connection with the wire bonds, and then tapered down to reduce the field size
while maintaining the same $k_0$ and hence the same impedance. This CPW can be
capacitively coupled to a microwave resonator, but for a simple example we show
a multi-layer design for a single CPW in \mysubfigref{mws:fig:implement}{b}.

To form an even smaller trap than discussed above, the CPW centre conductor can
be biased, to create a dimple trap along with the underlying axis of the
Z-wire, as was done in \inlineref{Treutlein2008}. For a resonator it would also
be possible to introduce a bias field, but this would have to be done on-chip
with a band stop filter~\cite{doi:10.1063/1.4808364}. This could also be used
to implement the electrostatic traps described in \inlineref{Andre2006}.




\section{Sideband cooling}


This subsection summarises the sideband cooling scheme presented in
\inlineref{Andre2006} and presents cooling rates for our trapping scheme with
\CaF{}.  Throughout section we assume that we operate with a single molecule
coupled to a high-quality cavity ($Q\sim10^6$, $\kappa \sim 10^4$) in the
strong coupling r\'egime ($g\sim10^5>Q$)

In the macroscopic magnetic trap, the
$\ket{e}\rightarrow\ket{g}$ decay is very slow ($\Gamma \sim
10^{-5}\si{hertz}$), but when a molecule is in close proximity to a microwave
resonator at the transition frequency, the decay rate is strongly enhanced. 
with rate
%
\begin{equation}
  \Gamma_c = \frac{\kappa}{2}.
\end{equation}

This enhanced decay rate opens the door to sideband cooling into the motional
ground state of the trap.

% TODO, this really isn't explained very well. Need to go into a fair bit more
% detail
We write the state of the molecule as $\ket{N, m}$ where $N\in\{e,g\}$ is the
rotational state of the molecule, and $m$ is the quantum number of the motion
in the trap. We imagine that an external driving field is applied on the
$\ket{g, m} \rightarrow \ket{e, m-1}$ transition as pictured in
\myfigref{mws:fig:sideband}. Combined with the enhanced decay, this allows
sideband cooling into low motional ground states of the trap.

\begin{figure}[htb]
  \centering
  \begin{subfigure}[c]{0.4\textwidth}
  %\begin{overpic}[abs, width=\textwidth]{figs/mws/sidebandcool.pdf}
  %    %\put(10, 105){(a)}
  %  \end{overpic}
      \begin{tikzpicture}
        \node[anchor=south west,inner sep=0] (image) {\includegraphics[width=\textwidth]{figs/mws/sidebandcool.pdf} };
        %\node [] at (0.1, 4.3) {(a)};
        %\node [] at (9.5, 4.3) {(b)};
      \end{tikzpicture}
  \end{subfigure}
  \hspace{1cm}
  \begin{subfigure}[c]{0.48\textwidth}
  \begin{tikzpicture}
    \begin{axis}[
        ymode=log,
        enlargelimits=true,
        xlabel=Trapping current (\si{\ampere}),
        ylabel=$T_\text{min}$ (\si{\micro\kelvin}),
        width=\textwidth,
        height = 0.7\textwidth,
        x tick label style={/pgf/number format/.cd, set thousands separator={}}
    ]
      \addplot [thick, black] table {figs/mws/mintemps.dat};
      \addplot [style=thick, color=blue] table [y index=2] {figs/mws/mintemps.dat};
      \addplot [style=thick, color=pink] table [y index=3] {figs/mws/mintemps.dat};
      %\addlegendentry{\SI{4}{\kelvin}}
    \end{axis}
    %\node [] at (0.1, 0.9) {(b)};
  \end{tikzpicture}
  \end{subfigure}
  \caption{The sideband cooling scheme is shown in (a) with an external driving field applied to the
    $\ket{g, m}\rightarrow\ket{e, m-1}$ transition. Decay on
    $\ket{e,m}\rightarrow\ket{g,m}$ is strongly enhanced by the resonator,
    allowing sideband cooling to low motional states. Subfigure (b) shows the
    minimum temperature that can be achieved as a function of trapping current
    for different resonator temperatures: \SI{4}{\kelvin} (black),
    \SI{400}{\milli\kelvin} (blue) and \SI{40}{\milli\kelvin} (red).
    }
  \label{mws:fig:sideband}
\end{figure}
%
% TODO Check trap frequency is what we expect, see sideband cooling notebook,
% think it should be more like a MHz
The cooling rate is given by the enhanced scattering rate times the energy
reduced per scatter. The latter of these being given by $\hbar\omega_t$, with
$\omega_t$ being the trap frequency. As per discussion in
section~\ref{theory:chips}, we expect typical trap frequencies $\omega_t\sim
2\pi \times 10^4 \si{\hertz}$ and therefore the cooling rate will be
\SI{10}{\milli\kelvin\per\second}.

\cm{Change where this little discussion about temperature goes?}
\cm{EVEN BETTER: change to do a graph of resonator temp vs minimum temperature
that can be achieved at various trap frequencies.}
The lowest motional state that can be reached is determined by the background
photon number in the resonator. A typical high-$Q$ cavity will operate at a few
tens of millikelvin~\cite{doi:10.1063/1.3010859} but these temperatures require
a dilution fridge, and high-temperature superconductors can operate at typical
cryocooler temperatures of $T_\text{res}=\SI{4}{\kelvin}$. This corresponds
to a background photon number of
%
\begin{equation}
  \overline{m} = \frac{k_B T_\text{res}}{\hbar \omega_0}
\end{equation}
%
\cm{and can eliminate other sources of noise, cite Wallraff strong coupling
paper}
We expect the minimum temperature achievable by sideband cooling to be reached
when the molecule motion and photon energy are in thermal equilibirum, so that
%
\begin{equation}
  T_\text{min} = \frac{\hbar \omega_t \overline{m}}{k_B}
\end{equation}
%
which is shown as a function of the trapping current for various resonator
temperatures in \mysubfigref{mws:fig:sideband}. \cm{Discuss lowering current
to decrease temp and that this is actually really cold in tight confinement,
not worth comparing to free space.}

\section{Entangling light and photon states}

Reference~\cite{Andre2008} also proposes entangling the molecule state with the
quantum state of photons in the resonator. This is inspired by the methodology
of \inlineref{PhysRevA.69.062320} where the state of a Cooper-pair box is
entangled with a resonator state. This entanglement can be used to perform
state readout or coupling to neighbouring qubits, as will be discussed further
below.

% TODO Bring in any more necessary cites (from squeeze ch)

\cm{Entangling} can be achieved in the dispersive r\'egime, where the transition
frequency is tuned so that $|\Delta|\gg g$. 
Reference~\cite{PhysRevA.69.062320} tells us that we can gain insight into the
dispersive behaviour by applying the unitary transformation
%
\begin{equation} U = \exp \left[\frac{g}{\Delta}(a\sigma_+ -
a^\dagger\sigma_-)\right] \end{equation}
%
to the Jaynes-Cummings Hamiltonian described in section~\ref{theory:QO}. With
expansion up to second order in $g/\Delta$, we obtain
%
\begin{equation} H= UH_\text{JC}U^\dagger \approx \hbar \omega_c
  a^\dagger a + \hbar\left(\omega_c +
  \frac{g^2}{\Delta}\right)s_z + \frac{\hbar
  g^2}{\Delta}\sigma_z a^\dagger a.
\end{equation}
%
% TODO Probably better to start with s_i in theory
Here we have introduced the spin operator $s_i = \sigma_i/2$.

The three terms of $H$ describe the oscillation of light in the cavity, the energy
of the spin and the interaction of the photons with the spin. Note that the
interaction of the photons and the spins induces the usual AC start shift
proportional to $(n+\frac{1}{2})$. The last term allows the pefromance of 
QND measurement, since it will enable exchange of information
between the z-component of the spin with the photons.

We take the state of the photons in the resonator to be in a canonical coherent
state~\cite{Gazeau2009}
%
\begin{equation}
  \ket{\alpha} = e^{-\frac{|\alpha|^2}{2}}\sum_{n=0}^\infty \frac{\alpha^n}{\sqrt{n!}} \ket{n}
\end{equation}
%
% TODO Wording here, again representing? Was it before? Is it clear?
% Should I change n to m (including above) for consistency?
with $\alpha\in\mathbb{C}$, and $\ket{n}$ again representing the $n^\text{th}$
Fock state of the light~\cite{agarwal2012}. The interaction of the light with
the molecule is desribed by the last term in $H$, so for an interaction over
time $T$, we have
%
\begin{equation}
  \ket{\Psi(T)} = \exp\left(-iH_\text{int}T/\hbar\right)\ket{\Psi(0)}
\end{equation}
%
where
%
\begin{equation}
  H_\text{int} = \hbar \frac{g^2}{\Delta} s_z a^\dagger a
\end{equation}
%
and $\ket{\Psi(0) = \ket{\psi}\ket{\alpha}}$ is the state of the system at the
time of measurement. This can be expanded by inserting the definition of the
coherent and molecule states,
%
\begin{equation}
  \ket{\Psi(T)} = e^{-\frac{|\alpha|^2}{2}}\sum_{n=0}^\infty
   \frac{\alpha^n}{\sqrt{n!}} e^{-i\nu T s_z a^\dagger a} \ket{n} (\cos\theta
   \ket{g} + e^{i\phi}\sin\theta\ket{e})
\end{equation}
%
where $\nu = g^2/\Delta$.
% TODO Did I introduce number operator already?
Now the number operator in the exponent acts on $\ket{n}$, and the spin
operator acts on $\ket{N}$ for
%
\begin{equation}
   \ket{\Psi(T)} = e^{-\frac{|\alpha|^2}{2}}\sum_{n=0}^\infty
   \frac{\alpha^n}{\sqrt{n!}} (e^{i\nu T n \hbar/2}\cos\theta\ket{n}\ket{g} +
   e^{-i\nu T n \hbar/2}e^{i\phi}\sin\theta\ket{n}\ket{e}).
\end{equation}
%
Collecting the coefficients of the molecule states yields
%
\begin{equation}
  \ket{\Psi(T)} = \cos\theta\left(e^{-\frac{|\alpha|^2}{2}}\sum_{n=0}^\infty
   \frac{(\alpha e^{i\nu T \hbar/2})^n}{\sqrt{n!}}\ket{n}\right)\ket{g} +  
    e^{i\phi}\sin\theta\left(e^{-\frac{|\alpha|^2}{2}}\sum_{n=0}^\infty
   \frac{(\alpha e^{-i\nu T \hbar/2})^n}{\sqrt{n!}}\ket{n}\right)\ket{e}.
\end{equation}
%
Finally, note that each of the photon states (in parentheses) defines a
coherent state, so the resulting state after interaction is
%
\begin{equation}
  \ket{\Psi(T)} = \cos\theta\ket{\alpha_+}\ket{g} +
  e^{i\phi}\sin\theta\ket{\alpha_-}\ket{e}
\end{equation}
%
where $\alpha_\pm = \alpha \exp(\pm i \nu T \hbar/2)$.

\subsection{State readout}

The state of the molecule is now entangled with the state of the light in such
a way that measuring the phase of the light will perform a readout of the
molecule state. In this subsection I will present the homodyne
measurement~\cite{agarwal2012} technique in the context of performing a
measurement of our molecule state.

The homodyne \cm{experiment} is illustrated in \myfigref{mws:fig:homodyne}. The
light to be measured, here labelled $\ket{\Psi_a}$, is incident on one port (a)
of a beam splitter. On the other port (b) we have a strong local oscillator in
a coherent state $\ket{\beta}$, with large amplitude, in
this case meaning that $|\beta| \gg |\alpha|$. We set the relative phases of
$\ket{\alpha}$ and $\ket{\beta}$ so that $\arg{\alpha}=0$ and
$\arg{\beta}=-\varphi$.

\begin{figure}
  \centering
  \includegraphics[width=0.3\textwidth]{figs/mws/homodyne.pdf}
  \caption{Schematic of a homodyne measurement. The state to be measured
    $\ket{\Psi_a}$ is incident on port (a) of the beamsplitter, and the local
    oscillator $\ket{\beta}$ is incident on port (b). The signal from
    photodiodes at the output ports is summed to produce a homodyne
    measurement.
  }
  \label{mws:fig:homodyne}
\end{figure}

The annihilation operators associated with the input ports are related to those
of the output ports (c and d) by the usual relation for a balanced beam
splitter~\cite{agarwal2012}
%
\begin{equation}
  \label{squeeze:eqn:bsmat}
  \begin{pmatrix} c \\ d \end{pmatrix} = \frac{1}{\sqrt{2}}\begin{pmatrix}
    1 & i \\ i & 1 
  \end{pmatrix}  \begin{pmatrix} a \\ b \end{pmatrix}.
\end{equation}
%
The difference in the expected photon numbers arriving at each
detector is
%
\begin{align}
  \langle c^\dagger c - d^\dagger d\rangle &= i\langle a^\dagger b- ab^\dagger
  \rangle \\
  &= i \kappa \bra{\Psi_a}\bra{\beta}(a^\dagger b-
  ab^\dagger)\ket{\beta}\ket{\Psi_a} \\
  & = i|\beta|\kappa \bra{\Psi_a}(a^\dagger e^{i\varphi} - a
  e^{-i\varphi})\ket{\Psi_a} \\
\end{align}
%
where we have used $b\ket{\beta} = \beta\ket{\beta}$ and the factor of $\kappa$
arises from the fact that we are looking at the output of the cavity light,
which takes the state from $\ket{\gamma}_\text{in}$ to $\ket{\gamma}_\text{out}
= \sqrt{\kappa}\ket{\gamma}_\text{in}$~\cite{Vanner16182}.
% TODO I need to find better cite, could also cite Blais and that might be
% enough?

We now intoduce the canonical quadratures of the light field, corresponding to
its real and imaginary parts. They are defined by~\cite{gerry_knight_2004}
%
\begin{align}
  X = \frac{a + a^\dagger}{2} && Y = \frac{a - a^\dagger}{2i}.
\end{align}
%
The expected photon difference is now
\begin{align}
  \langle c^\dagger c - d^\dagger d\rangle &=
  2|\beta|\kappa\bra{\Psi_a}(Y\cos\varphi - X\sin\varphi)\ket{\Psi_a} \\
  &= 2|\beta|\kappa\bra{\Psi_a}(X\sin(\varphi+\frac{\pi}{2}) +
  Y\cos(\varphi+\frac{\pi}{2}))\ket{\Psi_a}.
  \label{squeeze:eqn:homoquads}
\end{align}

% TODO Need to account for transmission coefficient but probably later, see
% Mauro's paper (Q. Tomography), eqn. 2.37 - 2.42
Hence measuring the intensity of each output of the beamsplitter can give us
information on the phase of the light. We choose $\varphi = 0$ so that the
measurement is of the $Y$ quadrature, i.e.\ we measure the imaginary part of
$\ket{\Psi_a}$
%
\begin{equation}
  \langle c^\dagger c - d^\dagger d\rangle =  2\kappa
  |\alpha||\beta|\left\langle\sin(\nu T s_z)\right\rangle.
\end{equation}
%
For a short pulse of light, the interrogation time will be the lifetime of the
photons in the cavity, $T = \kappa^{-1}$. In this r\'egime we can expand sine
to first order, so that
%
\begin{equation}
  \langle c^\dagger c - d^\dagger d\rangle = 2|\alpha||\beta|
  \frac{g^2}{\Delta}\langle s_z\rangle.
  \label{eqn:homomeas}
\end{equation}
%
% TODO This isn't really a spin state, need to make this more clear above
In other words, the expectation value of the photon measurement is linked
directly to that of the molecule's state. A measurement of $\langle c^\dagger c
- d^\dagger d\rangle$ will also measure $s_z$. This allows for readout of the
spin state via the microwave ports.

\cm{TODO A bit on implementation with specifics in \CaF{}}

\subsection{State preparation}

\cm{What is this?}

\cm{Maybe something about preparing light states too?}

\subsection{Coupling between molecules}

\cm{How will this work?}


