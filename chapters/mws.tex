% TODO Better intro spiel
\cm{
In chapter (probably the introduction) I introduced the concept of microwave
guides integrated into chip traps. Previous experiments have used on-chip
microave guides to couple to \cm{(I think this is right)} the hyperfine structure
of atoms. However, the coupling to heteronuclear dipolar molecules is much
stronger \cm{Why?}. In the strong coupling regime it may be possible to \cm{(do
cavity QED, sideband cooling, etc...)}
}

\section{The coplanar waveguide}

\cm{The following is mostly from my ESA so will need editing! Important
takeaways are the field strcutre, where the molecules need to sit in this
field, and what we need to do to get a cavity. What can we do if we don't have
a cavity?}

The coplanar waveguide (CPW) was originally proposed by Cheng P.~Wen as a
means of guiding microwaves across the surface of a dielectric
substrate~\cite{1127105}. It consists of a central conductor with a ground
plane on either side, as pictured in~\myfigref{mws:fig:CPW}. CPWs have become
prolific, since they allow the creation of robust microwave devices, and offer
some benefits over other waveguide \cm{architectures} such as the stipline
waveguide because they can provide circularly polarised fields, and also
provide easy access to the ground plane for shunt connections.

The CPW's geometry is defined by the centre conductor width ($S$) and the
channel width ($W$). The geometry of the CPW determines the region the
microwave field occupies, as illustrated in \myfigref{experiment:fig:CPWfield}.
The CPW can therefore be designed to maximise overlap between the microwave
field and a cloud of molecules trapped nearby.  As a rough approximation the
molecules should be trapped at a distance $h\sim S$ above the centre of the CPW
to achieve a good overlap.~\cite{Boehi2009} The field surrounding the CPW is
shwon in \mysubfigref{mws:fig:CPW}{b}, the equations defining this field can be
found analytically~\cite{Simons2004}, or the field can be determined by a
finite-element simulation.

% TODO Need to make it clear that we are in a limiting geometry with large
% dielectirc compared to conductor height

\begin{figure}
  %\includegraphics{}
  \cm{Need a drawing of CPW and a drawing of field lines. In former ensure we
  define the permittivities. Like simons 2.2(a) and 2.25}
  \caption{}
  \label{mws:fig:CPW}
\end{figure}

\subsection{CPW properties}

We will now consider the electric properties of a CPW. These will largely
depend on the the planar geometry of the waveguide, which we express in terms
of the ratio~\cite{1127105, Simons2004}
%
\begin{equation}
  k_0 = \frac{S}{S+2W} = \sqrt{1-{k'_0}^2}
  \label{eqn:k0def}
\end{equation}
%
where the second equality defines $k'_0$.
%
We assume that the waveguide is made of a material with conductivity $\sigma$
and permeability $\mu$. For an angular frequency $\omega$ passing through the
CPW there is a skin depth~\cite{Simons2004}
%
\begin{equation}
  \delta = \sqrt{\frac{2}{\omega\mu\sigma}}
\end{equation}
%
and skin effect surface resistance~\cite{Simons2004}
%
\begin{equation}
  R_s = \frac{1}{\sigma\delta}.
\end{equation}

Since the field of the CPW will pass through both the dielectric and the
surrouding air, the CPW's capacitance is the sum of the capacitance of each of
these two parts
%
\begin{equation}
  C_\text{CPW} = C_\text{dielectic} + C_\text{air}.
\end{equation}
%
The method of finding these individual contributions is somewhat involved, so
we simply state the results here in terms of the elliptic integral of the
first kind $K(k)$~\cite{Simons2004}. The capacitance due to the dielectric id 
%
%
\begin{equation}
  C_\text{dielectric} = 2\epsilon_0(\epsilon_\text{r1}-1)\frac{K(k_0)}{K(k'_0)}
\end{equation}
%
and the capacitance of the air region is
%
\begin{equation}
  C_\text{air} = 4\epsilon_0 \frac{K(k_0)}{K(k'_0)}.
\end{equation}

We can use this to find the effective permittivity~\cite{Simons2004}
%
\begin{align}
  \epsilon_\text{eff} &= \frac{C_\text{dielectric}}{C_\text{air}} \\
    &= \frac{1+ \epsilon_\text{r1}}{2} \\
\end{align}
%
and the phase velocity (using $c$ as the speed of light)~\cite{Simons2004}
%
\begin{align}
  v_\text{ph} &= \frac{c}{\sqrt{\epsilon_\text{eff}}} \\
    &= \frac{c}{\sqrt{(1 + \epsilon_\text{r1})/2}}.
\end{align}

Now using the approximation~\cite{Collin2007}
$\sqrt{\mu_0/\epsilon_0}\approx120\pi\text{Ohms}$ we have the line
impedance~\cite{Simons2004}
\begin{align}
  Z_0 &= \frac{1}{C_\text{air} v_\text{ph}} \\
    &= \frac{30 \pi}{\sqrt{(\epsilon_\text{r1}+1)/2}} \frac{K(k_0)}{K(k'_0)}
    \text{Ohms}
\end{align}
Note that the impedance of the waveguide has dependence only on the geometry in
the form of the ratio $k_0$, and the relative permittivity of the
substrate.~\cite{Simons2004} This means that for any substrate we choose the
value of $k_0$ can be chosen to fix the impedance at the standard $Z_0 =
\SI{50}{\ohm}$.

Therefore as long as $k_0$ is held constant the CPW can be tapered to change the
size of the centre conductor, and hence control the region the field occupies.

\subsection{CPW Resonators}
\label{mws:resonators}

A microwave resonator can be formed from a section of CPW that is capacitively
coupled to another driving segment~\cite{Day2003}. The resonant frequency is
determined by the resonator's length, $L$, and the waveguide's phase
veloctiy~\cite{}
%
\begin{equation}
  \omega_0 = \frac{\pi v_\text{ph}}{L} = \frac{\pi
  c}{\sqrt{\epsilon_\text{eff}} L}
\end{equation}
%
Microwaves can be coupled to the resonator by positioning the resonator inline
with a microwave guide, or in parallel as shown in \cm{some figure that I need
to make}. The strength of the coupling can be controlled by the shape of the
interface between the resonator and the waveguide~\cite{doi:10.1063/1.3010859}
\cm{and another cite for when they are in parallel?}.

\cm{Alex: Limit? I would have thought only need Q up to Q of a rotational
transition, what is that?}

To achieve strong coupling we require \cm{something about quality factor I
probably said somewhere else that I should reference}. The quality factor of a
CPW resonator is given in terms of an attenuation constant
$\alpha$~\cite{Simons2004}
% Simons pg. 409-410
%
\begin{equation}
  Q = \frac{\omega_0}{2c\alpha}.
  \label{mws:eqn:Qalpha}
\end{equation}
%
This attenuation constant is defined as the real part of the propagation
constant $\gamma = \alpha + i\beta$, where $\beta = 2\pi / \lambda$ is the wave
number.~\cite{Simons2004} Propagation through a waveguide induces evolution
described by
%
\begin{equation}
  \widetilde{E}(z) = \widetilde{E}(0)e^{-\gamma z}.
  \label{mws:eqn:Eloss}
\end{equation}
%
Taking the absolute value, the amplitude falls off as
%
\begin{equation}
  E(z) = E(0)e^{-\alpha z}.
\end{equation}
%
Finding \cm{high enough Q resonator, c.f. Alex's comment below} is essential
for the \cm{strong-coupled proposals} that will be discussed in \cm{the next
section}. Hence we will discuss the sources of attenutation that contribute to
$\alpha$, demonstrating how to implement a resonator of sufficient $Q$ to reach
the strong coupling r\'egime.
\cm{Should strong coupling be hyphenated?}

There are two contributing terms to the attenuation constant
%
\begin{equation}
  \alpha = \alpha_d + \alpha_c,
\end{equation}
%
there are the contributions from dielectric and conductor respectively.
Negligible contributions from radiative losses are
neglected~\cite{Frankel1991}. We also do not consider the insertion loss of the
resonator, since this does not affect the quality
factor~\cite{doi:10.1063/1.3010859}.

\cm{Mike: This section is very useful. I would find even more useful if you had
included graphs to show how the various loss modes depend on the key parameters
(frequency, conductor width/thickness etc).}

\subsubsection{Dielectric losses}

Dielectric losses \cm{in our limit (this is really important!)} are described
by ~\cite{Collin2007}
\begin{equation}
  \alpha_d =
  \frac{\omega_0}{4c}\frac{\epsilon_\mathrm{r1}}{\sqrt{\epsilon_\mathrm{eff}}}
  \tan \delta_e
\end{equation}
%
where $\tan\delta_e$ is the dielectric loss tangent. Common dielectrics for
microwave guides include aluminium nitride (\AlN{}) and high-resitivity silicon
(\hiresSi{}).  For our purposes we wish to consider waveguides situated on
dielectrics layers that can be easily deposited above the trapping wires.
Following the work in \inlineref{Treutlein2008}, we consider polyimide, which
can be despoitied by spoin-coating. Other options are available, such as
polyethylene naphthalate (PEN) which has been used in \inlineref{} to create
CPW resonators on a flexible substrate.

Typical values of dielectric constants and $\alpha_d$ are shown in
\mytableref{mws:table:diprops}. Note that these parameters have some dependence
on the frequency and temperature of the dielectric, and so are presented to
illustrate the amount of loss that can typically be expected from each one.
Reducing the temperature may lower dielectric loss and losses in experiment may
differ from those that are predicted (as they were for example in \inlineref{}
\cm{PEN paper}).

\begin{table}[ht]
  \caption{Dielectric constants and loss for various substrates in the
  \SI{10}{\giga\hertz} r\'egime at room temperature}
\centering
\begin{tabular}{l c c c c }
\hline\hline
  Material & $\epsilon_r$ & $\tan\delta_e$ & $\alpha_d$ & Ref. \\ [ 0.5ex]
\hline
  Polyimide & 3.4 & \SI{1.8E-2}{} & 0.45 & \cite{} \\
  PEN & 2.56 & 0.003 & 0.63 & \cite{} \\
  Silicon & 11.65 & \SI{5E-5}{} & 0.02 & \cite{} \\
  \hiresSi{} & ?? \\
  Saphire & ?? \\
  Aluminium nitride & 8.9 & \SI{5E-4}{} & 0.22 & \cite{} \\
\hline
\end{tabular}
\label{mws:table:diprops} % is used to refer this table in the text
\end{table}

The data in the table indicates that dielectric losses are much more
significant when using polyimide or PEN -- the types of dielectrics that would
be required for use with a multi-layer chip. Such losses could potenitally be
mitigated by operating at cryogenic temperatures, since this can reduce the
dielectric loss tangent of some materials~\cite{}. However the high-$Q$
resonators implemented in for example ~\cite{}
% https://aip.scitation.org/doi/10.1063/1.3693409
require the lower dielectric losses of other dielectrics. We will continue this
discussion after we have considered the
contribution of conductor losses.

\subsubsection{Conductor losses}

Conductor losses arise due to dissipation in the centre conductor and ground
plane of the CPW~\cite{Simons2004}.
The conductor attenuation constant is
\begin{equation}
  \alpha_c = \frac{R_c +R_g}{2Z_0}.
\end{equation}
Where $R_c$ and $R_g$ are the series resistances of the centre conductor and
the ground plance respectively.
For a waveguide with height 
$t$, theses are given by
\begin{equation}
  R_c = \frac{R_s}{4 S(1-k_0^2)K^2(k_0)}\left[ \pi + \log\left(\frac{4\pi
  S}{t}\right) - k_0\log\left(\frac{1+k_0}{1-k_0}\right) \right],
\end{equation}
and
\begin{equation}
  R_g = \frac{k_0 R_s}{4S(1-k_0^2)K^2(k_0)}\left[\pi +
  \log\left(\frac{4\pi(S+2W)}{t}\right) -
  \frac{1}{k_0}\log\left(\frac{1+k_0}{1-k_0}\right)\right].
\end{equation}

% TODO Replace with figure showing values of alpha for various different
% situations, varying the conductor thickness, or maybe S? Bottom line is that
% we need a superconducting resonator
%
%
%Consider an example case of gold conductor on HiRes Si substrate. Gold has
%resistivity $\rho_\mathrm{Au} = \SI{2.4E-8}{\ohm\metre}$ at room temperature,
%and as above, we will have $\epsilon_\mathrm{r1} \approx 10$. This requires $k_0
%\approx 1/3$ to achieve impedance matching at $Z_0 = \SI{50}{\ohm}$. The only
%free parameters are the conductor thickness $t$, and the width of the centre
%conductor $S$, both of which must be maximised to reduce loss. Typical values
%for the smallest CPWs will be $S\sim\SI{1}{\micro\metre}$,
%%
%\cm{Mike: surely 10um is the smallest we are fabricating? Cameron: maybe need
%different values for different x-sections (or widths with set height?) see above
%comment from Mike about more plots}
%%
%with a height of approximately \SI{6}{\micro\metre}
%(discussed further below). Typical conductor losses are therefore expected to be
%of order
%\begin{equation}
%  \alpha_c \sim \SI{400}{\neper\per\metre}
%\end{equation}
%or as a quality factor
%\begin{equation}
%  Q_c \sim 1.
%\end{equation}

Conductor losses for gold CPWs are large, but it is possible to use
superconductors to produce resonators with $Q$ on the order of
$1000$~\cite{Booth1999, Wallraff2004}. \cm{Much higher than this is possible!}
However as discussed above, cooling of the chip is not currently possible due
to experimental restrictions, and hence we are not able to make use of
superconductors to reach these high quality factors.

\subsubsection{Summary}

It is clear from the above discussion that a superconducting CPW is required to
implement a resonator. In this case we will be limited by the dielectric losses
described above. However, the dielectric properties of the substrates must be
determined experimentally in this low temperature r\'egime. Further, it is
known that the interface between the conductor and dielectric can cause an
anomolous decrease in quality factor, which must also be found experimentally.

For this reason, integrating a resonator into the multi-level design is not
feasible. Potential solutions will be discussed further in
section~\ref{mws:integrating}. With this in mind, the dielectric and conductor
losses do not prevent the implementation of a microwave guide for interacting
with molecules trapped on a chip. Such a device can operate at room temperature
with for example a polyimide dielectric and gold conductor. This would be
similar to the device described in \cm{can I not reference Treutlein thesis
here?}

\cm{Alex: also need to mention bending losses}

\cm{Raise point from Gopl about at low drive powers Q factor is reduced, so not
suitable for single photon work? c.f. ref 28 therein}

\cm{Alex: should re-calculate for the actual loss over the length of small
waveguide we are using and include that. Also use realistic $S$.}

\section{Integrating microwave structures on a chip}
\label{mws:integrating}

That said, it should be noted that the high loss should not prevent us from
using the CPW to directly drive microwave transitions in trapped molecules, as
we will still be able to pass signal through the waveguide without the need to
construct a resonator. With loss of $\alpha \approx \alpha_c$, and waveguides of
lengths on the scale of a few centimeter, we can expect a total loss of around
$\SI{400}{\neper\per\metre} \times \SI{0.01}{\metre} = \SI{35}{\decibel}$. This
should be sufficiently small that microwaves will be able to reach molecules
trapped on the chip.~\cite{Treutlein2008}
One of the later goals of the project will be to characterise microwave losses
on the chip and compare these with those expected from theory.
%
\cm{I might be able to expand on this further down the line, but it seems
unnecessary to include now. Maybe need a fig. to explain too?
%
\emph{
A consequence of being unable to implement a microwave resonator is that it
becomes challenging to implement the electrostatic trap proposed by Andr\'e et
al.~\cite{Andre2006} as this relied on the ability to introduce a bias voltage
on the centre conductor of the resonator in order to form the trap. It is
possible to bias the centre conductor (as detailed in
reference~\cite{doi:10.1063/1.3573824}) but it is not possible to do the same
for a waveguide.
}}


% TODO
\cm{This is from the ESA, so need to fix up}

\cm{Mike:This bit doesn't make sense to me. The logic here seems to imply that
the  electrostatic trap relies on a resonator but the magnetic trap doesn't.
Which isn't true!  We chose the magnetic trap because we think it will give us
longer coherence times. \\
%
Cameron: I think my reasoning here glossed over the details from the COMSOL
models Kyle made for brining the bias onto the chip. I thought our conclusion
was that the electrostatic trap required a resonator to bring the bias in, but
maybe I am wrong? Mike has probably forgotten this, and a new reader wouldn't
know it to start with, so I need to figure out how best to explain myself. Maybe
forget the bias stuff and just argue that magentically insensetive transitions
lead to long coherence times, which is what we want.
}

As the microwave attenuation prevents implementation of a resonator (and hence
implementation of an electrostatic trap integrated with the waveguide) another
solution for trapping and control of the microwaves must be found. Since
magnetic trapping of atoms above a chip has been extensively researched (see
section~\ref{chiptraps}), we have chosen this mechanism for trapping of molecules
above the chip.

Various possible geometries for combining magnetic trapping wires and a CPW have
been extensively explored. It transpires that it is extremely challenging to
construct a geometry with both of these on the same layer that achieves a
satisfactory overlap between the CPW field and the trapping region. To get
around this, a chip with two layers has been designed: the lower layer consists
of an array of DC trapping wires, and an upper layer hosts the CPW. The two are
to be separated by an insulating layer of resin.

Multi-layer chips have been constructed before, notably those by
Treutlein~\cite{Treutlein2008} and B\"ohi~\cite{rohtua}, whose work has heavily
inspired our design. Treutlein and B\"ohi used photolithography followed by
electroplating to create the lower layer (see section \ref{chiptraps:wiretraps})
then formed the insulating layer by spin-coating polyimide resin in several
stages. This allows the creation of a smooth layer, effectively covering bumps
that can be formed from the underlying wires which avoids significant
discontinuities in the waveguide.

Polyimide is used as the resin as it is highly resistant to cleaning techniques
commonly used in photolithography, including piranha clean.  It is also
effective as a microwave substrate~\cite{Simons2004} \cm{Mike seems concerned
about this being high}
($\epsilon_\mathrm{r1} = 3.3$, $\tan\delta_e = 0.016$) with conductor losses
still dominating those caused by the dielectric.

\cm{Alex: were these the limits or the max they used?}
Using a multi-layer chip introduces thermal constraints on the currents that can
flow through the trapping wires. Current densities achieved by Treutlein and
B\"ohi \cite{} were \SI{2.8E10}{\ampere\per\metre\squared} in the upper layer and
\SI{5.5E10}{\ampere\per\metre\squared} in the lower layer.

\section{Coupling with a single molecule on a chip}

\cm{
  \begin{itemize}
    \item Build on maths in theory chapter
    \item Specify the states of the molecule that we will use
    \item Specify coupling parameters, etc.
    \item Discuss coupling to waveguide and to a resonator
  \end{itemize}
}

\section{Experiments with a single molecule coupled to a mw resonator}

\subsection{Sideband cooling}

\subsection{State readout}

\subsection{State preparation}

\cm{What is this?}

\subsection{Coupling between molecules}

\cm{How will this work?}


