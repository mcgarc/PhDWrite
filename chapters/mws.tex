% TODO Better intro spiel
\cm{
In chapter (probably the introduction) I introduced the concept of microwave
guides integrated into chip traps. Previous experiments have used on-chip
microave guides to couple to \cm{(I think this is right)} the hyperfine structure
of atoms. However, the coupling to heteronuclear dipolar molecules is much
stronger \cm{Why?}. In the strong coupling regime it may be possible to \cm{(do
cavity QED, sideband cooling, etc...)}
}

\section{The coplanar waveguide}

\cm{The following is mostly from my ESA so will need editing! Important
takeaways are the field strcutre, where the molecules need to sit in this
field, and what we need to do to get a cavity. What can we do if we don't have
a cavity?}

The coplanar waveguide (CPW) was originally proposed by Cheng P.~Wen as a
means of guiding microwaves across the surface of a dielectric
substrate~\cite{1127105}. It consists of a central conductor with a ground
plane on either side, as pictured in~\myfigref{mws:fig:CPW}. CPWs have become
prolific, since they allow the creation of robust microwave devices, and offer
some benefits over other waveguide \cm{architectures} such as the stipline
waveguide because they can provide circularly polarised fields, and also
provide easy access to the ground plane for shunt connections.

The CPW's geometry is defined by the centre conductor width ($S$) and the
channel width ($W$). The geometry of the CPW determines the region the
microwave field occupies, as illustrated in \myfigref{experiment:fig:CPWfield}.
The CPW can therefore be designed to maximise overlap between the microwave
field and a cloud of molecules trapped nearby.  As a rough approximation the
molecules should be trapped at a distance $h\sim S$ above the centre of the CPW
to achieve a good overlap.~\cite{Boehi2009} The field surrounding the CPW is
shwon in \mysubfigref{mws:fig:CPW}{b}, the equations defining this field can be
found analytically~\cite{Simons2004}, or the field can be determined by a
finite-element simulation.

% TODO Need to make it clear that we are in a limiting geometry with large
% dielectirc compared to conductor height

\begin{figure}
  %\includegraphics{}
  \cm{Need a drawing of CPW and a drawing of field lines. In former ensure we
  define the permittivities}
  \caption{}
  \label{mws:fig:CPW}
\end{figure}

\subsection{CPW properties}

We will now consider the electric properties of a CPW. These will largely
depend on the the planar geometry of the waveguide, which we express in terms
of the ratio~\cite{1127105, Simons2004}
%
\begin{equation}
  k_0 = \frac{S}{S+2W} = \sqrt{1-{k'_0}^2}
  \label{eqn:k0def}
\end{equation}
%
where the second equality defines $k'_0$.

Since the field of the CPW will pass through both the dielectric and the
surrouding air, the CPW's capacitance is the sum of the capacitance of each of
these two parts
%
\begin{equation}
  C_\text{CPW} = C_\text{dielectic} + C_\text{air}.
\end{equation}
%
The method of finding these individual contributions is somewhat involved, so
we simply state the results here in terms of the elliptic integral of the
first kind $K(k)$~\cite{Simons2004}. The capacitance due to the dielectric id 
%
%
\begin{equation}
  C_\text{dielectric} = 2\epsilon_0(\epsilon_\text{r1}-1)\frac{K(k_0)}{K(k'_0)}
\end{equation}
%
and the capacitance of the air region is
%
\begin{equation}
  C_\text{air} = 4\epsilon_0 \frac{K(k_0)}{K(k'_0)}.
\end{equation}

We can use this to find the effective permittivity~\cite{Simons2004}
%
\begin{align}
  \epsilon_\text{eff} &= \frac{C_\text{dielectric}}{C_\text{air}} \\
    &= \frac{1+ \epsilon_\text{r1}}{2} \\
\end{align}
%
and the phase velocity (using $c$ as the speed of light)~\cite{Simons2004}
%
\begin{align}
  v_\text{ph} &= \frac{c}{\sqrt{\epsilon_\text{eff}}} \\
    &= \frac{c}{\sqrt{(1 + \epsilon_\text{r1})/2}}.
\end{align}

Now using the approximation~\cite{Collin2007}
$\sqrt{\mu_0/\epsilon_0}\approx120\pi\text{Ohms}$ we have the line
impedance~\cite{Simons2004}
\begin{align}
  Z_0 &= \frac{1}{C_\text{air} v_\text{ph}} \\
    &= \frac{30 \pi}{\sqrt{(\epsilon_\text{r1}+1)/2}} \frac{K(k_0)}{K(k'_0)}
    \text{Ohms}
\end{align}
Note that the impedance of the waveguide has dependence only on the geometry in
the form of the ratio $k_0$, and the relative permittivity of the
substrate.~\cite{Simons2004} This means that for any substrate we choose the
value of $k_0$ can be chosen to fix the impedance at the standard $Z_0 =
\SI{50}{\ohm}$.

Therefore as long as $k_0$ is held constant the CPW can be tapered to change the
size of the centre conductor, and hence control the region the field occupies.

\subsection{CPW Resonators}
\label{mws:resonators}

Here I will present the implementation and theory of CPW microwave resonators,
with a view to determine whether implementation of the Andr\'e design is
feasible.

A microwave resonator can be formed from a section of CPW that is capacitively
coupled to another driving segment.~\cite{Day2003} The properties of such a
resonator are determined by the geometry, including its length and the nature
of the capacitor structures, which can be made up of gaps, or overlapping
fingers. An example of such a resonator is shown in
\myfigref{experiment:fig:resonator}.~\cite{doi:10.1063/1.3010859, Pain1999}

\begin{figure}
  \includegraphics[width=0.8\textwidth]{./figs/gopl_cpw.png}
  \caption{
    \cm{Aim to use one of my own CPWs here. Might be hard as I probably
    won't have all the different coupling methods, so maybe alter this figure,
    if that is allowed??}
    \cm{Tight bend radisu here? I thought $\lambda/3$ was the limit but this
    might be smaller?}
    A CPW resonator is shown (D) along with various capacitor structures: 8
    finger capacitor (E), two finger capacitor (H) and gap capacitors (I and K).
    The capacitors isolate a segment of the central conductor to form the
    resonator. Note that the curving of the CPW allows longer lengths to fit
    into a shorter region without adversely affecting
    transmission.~\cite{Simons2004} A taper can be seen at each end of the
    resonator, with $k_0$ and hence $Z_0$ conserved.
    \cm{Metion bend loss, to be expanded in body text.}
    This figure is reproduced from reference~\cite{doi:10.1063/1.3010859}.
  }
  \label{experiment:fig:resonator}
\end{figure}

A CPW resonator of length $L$ has fundamental angular frequency
\begin{equation}
  \omega_0 = \frac{\pi v_\mathrm{ph}}{L} = \frac{\pi
  c}{\sqrt{\epsilon_\text{eff}} L}
\end{equation}

To determine the feasibility of implementing such a resonator, the quality
factor should be considered. By definition the quality factor of a perfect
damped resonator is proportional to the ratio of energy stored and energy lost
per cycle, that is~\cite{Pain1999}
\begin{equation}
  Q = 2\pi\frac{U_\mathrm{max}}{U_\mathrm{lost}}.
  \label{experiment:mw:eqn:Qdef}
\end{equation}
It can readily be shown that the quality factor can be expressed in terms of the
attenuation constant $\alpha$ (defined below) and the resonant frequency
$\omega_0$, so that~\cite{Simons2004}
% Simons pg. 409-410
%
\begin{equation}
  Q = \frac{\omega_0}{2c\alpha}.
  \label{experiment:mw:eqn:Qalpha}
\end{equation}
%
\cm{Alex: Limit? I would have thought only need Q up to Q of a rotational
transition, what is that?}
It is desirable to maximise $Q$ (minimise damping). It has been shown that CPWs
with $Q$ on the order of $1000$ can be fabricated.~\cite{doi:10.1063/1.3010859,
Hattermann2017} Below we will discuss whether this can be achieved in our case.

\subsubsection{Microwave loss modes}

\cm{Mike: This section is very useful. I would find even more useful if you had
included graphs to show how the various loss modes depend on the key parameters
(frequency, conductor width/thickness etc).}

The attenuation constant is defined as the real part of the propagation
constant $\gamma = \alpha + i\beta$, where $\beta = 2\pi / \lambda$ is the wave
number.~\cite{Simons2004} Propagation through a waveguide induces evolution
described by
%
\begin{equation}
  \widetilde{E}(z) = \widetilde{E}(0)e^{-\gamma z}.
  \label{experiment:mw:eqn:Eloss}
\end{equation}
%
Taking the absolute value, the amplitude falls off as
%
\begin{equation}
  E(z) = E(0)e^{-\alpha z}
\end{equation}
%
\cm{Alex asked if $\alpha_r$ includes input/output coupling. I think this is the
same as insertion loss and I have a target to talk more about that above, but
maybe reinforce here?}
where the attenuation is given by the sum of attenuation from three loss modes:
dielectric loss ($\alpha_d$), conductor loss ($\alpha_c$) and radiation loss
($\alpha_r$). The propagation constant can be written as
%
\begin{equation}
  \alpha = \alpha_d + \alpha_c + \alpha_r.
\end{equation}

It is instructive to consider the quality factor of a resonator for one loss
mechanism, ignoring the effect of the others. These are given
as~\cite{Simons2004}
\begin{equation}
  Q_i = \frac{\omega_0}{2c\alpha_i}
\end{equation}
with the total quality factor being
\begin{equation}
  Q = \left(\sum Q_i^{-1} \right)^{-1}.
\end{equation}

\cm{Talk about "insertion loss" which is in Simon's 12. (pg. 410??)}
Note that we will not consider the dependence of the quality factor the
capacitive coupling into the resonator (insertion loss).~\cite{Simons2004,
doi:10.1063/1.3010859} Broadly speaking, a higher coupling capacitance
corresponds to a wider resonance peak, however in our case this effect is
dominated by other forms of loss.

\cm{Alex: also need to mention bending losses}

\subsubsection*{Dielectric losses}

Dielectric losses vary linearly with $\tan \delta_e$, the dielectric loss
tangent of the substrate~\cite{Collin2007}
\begin{equation}
  \alpha_d =
  \frac{\omega_0}{4c}\frac{\epsilon_\mathrm{r1}}{\sqrt{\epsilon_\mathrm{eff}}}
  \tan \delta_e.
\end{equation}
A typical microwave substrate is chosen to minimise these dielectric losses. As
such we would expect $\tan\delta_e\leq10^{-3}$ and
$\epsilon_\mathrm{r1} \sim 10$ so the limit on the dielectric loss component is
\begin{equation}
  \alpha_d \leq \SI{0.9}{\neper\per\meter}
\end{equation}
for the \SI{41}{\giga\hertz} transition in \CaF{}. As a Q-factor, that is
\begin{equation}
  Q_d \geq 460.
\end{equation}

Higher $Q$ is achievable depending on the substrate that is chosen. For example,
aluminium nitride~\cite{mw101}  ($\epsilon_\mathrm{eff}=5$, $\tan\delta_e =
5\times10^{-4}$) or high-resistivity silicon (HiRes Si)~\cite{1717770}
($\epsilon_\mathrm{eff}=5$, $\tan\delta_e =2\times10^{-4}$) are both good
choices in the high-frequency regime.  Dielectric quality is usually higher
below room temperature, but experimental limitations prevent us from cooling the
chip at this stage.
% https://www.microwaves101.com/encyclopedias/aluminum-nitride
% https://www.microwaves101.com/encyclopedias/hard-substrate-materials

\subsubsection*{Conductor losses}

Conductor losses arise due to dissipation in the centre conductor and ground
plane of the CPW.~\cite{Simons2004} We define the height of these structures to
be $t$, so that the series resistance of the centre conductor is
\begin{equation}
  R_c = \frac{R_s}{4 S(1-k_0^2)K^2(k_0)}\left[ \pi + \log\left(\frac{4\pi
  S}{t}\right) - k_0\log\left(\frac{1+k_0}{1-k_0}\right) \right],
\end{equation}
and the corresponding series resistance of the ground plane
\begin{equation}
  R_s = \frac{k_0 R_s}{4S(1-k_0^2)K^2(k_0)}\left[\pi +
  \log\left(\frac{4\pi(S+2W)}{t}\right) -
  \frac{1}{k_0}\log\left(\frac{1+k_0}{1-k_0}\right)\right].
\end{equation}
The conductor attenuation constant is
\begin{equation}
  \alpha_c = \frac{R_c +R_g}{2Z_0}.
\end{equation}

Consider an example case of gold conductor on HiRes Si substrate. Gold has
resistivity $\rho_\mathrm{Au} = \SI{2.4E-8}{\ohm\metre}$ at room temperature,
and as above, we will have $\epsilon_\mathrm{r1} \approx 10$. This requires $k_0
\approx 1/3$ to achieve impedance matching at $Z_0 = \SI{50}{\ohm}$. The only
free parameters are the conductor thickness $t$, and the width of the centre
conductor $S$, both of which must be maximised to reduce loss. Typical values
for the smallest CPWs will be $S\sim\SI{1}{\micro\metre}$,
%
\cm{Mike: surely 10um is the smallest we are fabricating? Cameron: maybe need
different values for different x-sections (or widths with set height?) see above
comment from Mike about more plots}
%
with a height of approximately \SI{6}{\micro\metre}
(discussed further below). Typical conductor losses are therefore expected to be
of order
\begin{equation}
  \alpha_c \sim \SI{400}{\neper\per\metre}
\end{equation}
or as a quality factor
\begin{equation}
  Q_c \sim 1.
\end{equation}

Conductor losses for gold CPWs are large, but it is possible to use
superconductors to produce resonators with $Q$ on the order of
$1000$.~\cite{Booth1999, Wallraff2004} However as discussed above, cooling of
the chip is not currently possible due to experimental restrictions, and hence
we are not able to make use of superconductors to reach these high quality
factors.

\subsubsection*{Radiative losses}

The radiation losses of a CPW have been calculated and measured by Frankel et
al.~\cite{Frankel1991}, who show that
\begin{equation}
  \alpha_r = \frac{\pi^2 \omega_0^3}{2^7}\left(\frac{\left(1 -
  \frac{\epsilon_\mathrm{eff}}{\epsilon_\mathrm{r1}}\right)^2}{\sqrt{\frac{\epsilon_\mathrm{eff}}{\epsilon_\mathrm{r1}}}}\right)
  \frac{(S+2W)^2\epsilon_\mathrm{r1}^{3/2}}{c^3 K(k'_0)K(k_0)}.
\end{equation}.
Inserting typical values, we retrieve \cm{Mike: too vague, need frequency for
example as it goes like $f^3$...}
\begin{equation}
  \alpha_r \sim \SI{0.4}{\neper\per\metre}
\end{equation}
corresponding to a minimum quality factor
\begin{equation}
  Q_r \geq 1000
\end{equation}
which agrees with the measurements made by Frankel et al. which were for CPWs of
similar geometry and frequency domain to our intended usage.

\subsubsection*{Comparison of loss mechanisms}

It is clear that without use of superconductors, the conductor loss will far
dominate the other sources, with an overall quality factor $Q\sim1$. Typical
resonators will operate with much higher $Q$~\cite{Hattermann2017}, so it will
not be possible to implement a resonator.

\cm{Alex: should re-calculate for the actual loss over the length of small
waveguide we are using and include that. Also use realistic $S$.}
That said, it should be noted that the high loss should not prevent us from
using the CPW to directly drive microwave transitions in trapped molecules, as
we will still be able to pass signal through the waveguide without the need to
construct a resonator. With loss of $\alpha \approx \alpha_c$, and waveguides of
lengths on the scale of a few centimeter, we can expect a total loss of around
$\SI{400}{\neper\per\metre} \times \SI{0.01}{\metre} = \SI{35}{\decibel}$. This
should be sufficiently small that microwaves will be able to reach molecules
trapped on the chip.~\cite{Treutlein2008}
One of the later goals of the project will be to characterise microwave losses
on the chip and compare these with those expected from theory.
%
\cm{I might be able to expand on this further down the line, but it seems
unnecessary to include now. Maybe need a fig. to explain too?
%
\emph{
A consequence of being unable to implement a microwave resonator is that it
becomes challenging to implement the electrostatic trap proposed by Andr\'e et
al.~\cite{Andre2006} as this relied on the ability to introduce a bias voltage
on the centre conductor of the resonator in order to form the trap. It is
possible to bias the centre conductor (as detailed in
reference~\cite{doi:10.1063/1.3573824}) but it is not possible to do the same
for a waveguide.
}}

\section{Coupling with a single molecule on a chip}

\cm{
  \begin{itemize}
    \item Build on maths in theory chapter
    \item Specify the states of the molecule that we will use
    \item Specify coupling parameters, etc.
    \item Discuss coupling to waveguide and to a resonator
  \end{itemize}
}

\section{Experiments with a single molecule coupled to a mw resonator}

\subsection{Sideband cooling}

\subsection{State readout}

\subsection{State preparation}

\cm{What is this?}

\subsection{Coupling between molecules}

\cm{How will this work?}


\section{Integrating microwave structures on a chip}
\label{mws:integrating}

% TODO
\cm{This is from the ESA, so need to fix up}

\cm{Mike:This bit doesn't make sense to me. The logic here seems to imply that
the  electrostatic trap relies on a resonator but the magnetic trap doesn't.
Which isn't true!  We chose the magnetic trap because we think it will give us
longer coherence times. \\
%
Cameron: I think my reasoning here glossed over the details from the COMSOL
models Kyle made for brining the bias onto the chip. I thought our conclusion
was that the electrostatic trap required a resonator to bring the bias in, but
maybe I am wrong? Mike has probably forgotten this, and a new reader wouldn't
know it to start with, so I need to figure out how best to explain myself. Maybe
forget the bias stuff and just argue that magentically insensetive transitions
lead to long coherence times, which is what we want.
}

As the microwave attenuation prevents implementation of a resonator (and hence
implementation of an electrostatic trap integrated with the waveguide) another
solution for trapping and control of the microwaves must be found. Since
magnetic trapping of atoms above a chip has been extensively researched (see
section~\ref{chiptraps}), we have chosen this mechanism for trapping of molecules
above the chip.

Various possible geometries for combining magnetic trapping wires and a CPW have
been extensively explored. It transpires that it is extremely challenging to
construct a geometry with both of these on the same layer that achieves a
satisfactory overlap between the CPW field and the trapping region. To get
around this, a chip with two layers has been designed: the lower layer consists
of an array of DC trapping wires, and an upper layer hosts the CPW. The two are
to be separated by an insulating layer of resin.

Multi-layer chips have been constructed before, notably those by
Treutlein~\cite{Treutlein2008} and B\"ohi~\cite{rohtua}, whose work has heavily
inspired our design. Treutlein and B\"ohi used photolithography followed by
electroplating to create the lower layer (see section \ref{chiptraps:wiretraps})
then formed the insulating layer by spin-coating polyimide resin in several
stages. This allows the creation of a smooth layer, effectively covering bumps
that can be formed from the underlying wires which avoids significant
discontinuities in the waveguide.

Polyimide is used as the resin as it is highly resistant to cleaning techniques
commonly used in photolithography, including piranha clean.  It is also
effective as a microwave substrate~\cite{Simons2004} \cm{Mike seems concerned
about this being high}
($\epsilon_\mathrm{r1} = 3.3$, $\tan\delta_e = 0.016$) with conductor losses
still dominating those caused by the dielectric.

\cm{Alex: were these the limits or the max they used?}
Using a multi-layer chip introduces thermal constraints on the currents that can
flow through the trapping wires. Current densities achieved by Treutlein and
B\"ohi \cite{} were \SI{2.8E10}{\ampere\per\metre\squared} in the upper layer and
\SI{5.5E10}{\ampere\per\metre\squared} in the lower layer.
