The chip apparatus was assembled for preliminary tests, but isolated from the 
main \CaF{} experiment. This modest setup was useful for a variety of
measurements, which will be discussed below, including vacuum compatibility,
current testing under vacuum, and looking at the amount of background scatter
from the apparatus during imaging.

\section{Experiment assembly}




\section{Vacuum testing}


\section{Current testing}


\section{Scatter testing}


\section{Background-free imaging}

\cm{Normally we just image with 606nm light and do light-induced fluorescence
imaging, but this creates a load of scatter which reduces SNR.}

One possible method of reducing background whilst imaging molecules is Raman
Resonance Optical Cycling (RROC), an background-free imaging scheme recently
proposed and demonstrated using \SrF{} in \inlineref{Shaw2021}. In this scheme,
off-diagonal ($v\neq v'$) vibrational transitions are driven to excite the
molecule, which will decay primarily on the diagonal ($v=v'$) transitions. The
off-diagonal transitions are separated from the diagonal ones by
$>\SI{10}{\tera\hertz}$, and so it is possible to use a bandpass filter to
exclude the imaging light from any measurement. For an ideal bandpass filter
this would remove any background from imaging light scattered by the apparatus.

Reference~\cite{Shaw2021} reports a suppersion of scattered light by a factor
of $\sim10^6$, and an average emission of 20 photons per molecule. In this
experiment, a beam of molecules was imaged, and this emission is limited by the
interaction time during travel. For imaging a stationary cloud of molecules we
would expect instead to be limited by the reduced scattering rate in the
off-diagaonal transitions. In this section I will \cm{either just describe the
technique, or do this and show some results (fat chance...)}

\subsection{Implementation in \CaF{}}

% TODO Show name and model of bandpass filter

This scheme for \CaF{} is shown in \myfigref{exper:fig:bgfreelevels}. The
\pewpew{}{01} and \pewpew{}{10} light is used to drive the transitions at
\SI{585}{\nano\meter} and \SI{628}{\nano\meter} respectively (see
\mytableref{overview:table:lasers}). Fluorescenc from the decay on the
\SI{606}{\nano\meter} transitions $v'=0\rightarrow v=0$ and $v'=1\rightarrow
v=1$ is isolated by a bandpassfilter, and can be imaged without the usual
background scatter.

\begin{figure}
  \centering
  \includegraphics[width=0.9\textwidth]{figs/energylevels/bgfree.pdf}
  \caption{
  The RROC scheme for \CaF{}. An optical cycle is established, with pumping on
  the $v=0 \rightarrow v'=1$ (gold) and $v=1 \rightarrow v'=0$ (red) transitions.
  Fluorescence from the $v=v'$ transitions (orange) is distinguished from the
  background by a bandpass filter before imaging.
  }
  \label{exper:fig:bgfreelevels}
\end{figure}

\subsection{Results}

% He wrote, cautiously...
