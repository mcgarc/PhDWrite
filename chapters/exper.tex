The chip apparatus was assembled for preliminary tests, but isolated from the 
main \CaF{} experiment. This modest setup was useful for a variety of
measurements, which will be discussed below, including vacuum compatibility,
current testing under vacuum, and looking at the amount of background scatter
from the apparatus during imaging.

\section{Experiment assembly}

The chip flange assembly which was introduced at the end of
chapter~\ref{overview} is shown in \myfigref{exper:fig:flange} (see also
\mysubfigref{overview:fig:chipchamber}{b}). It is designed
to hold the chip in position for loading \CaF{}, whilst also providing current
delivery and heat sinking. Considerations have been made for future experiments
using microwaves, with two high-frequency microwave feedthroughs incorporated
as well. The flange was manufactured by Allectra GmbH, and the heat sink was
machined in the Imperial College workshop.

\begin{figure}
  \centering
  \cm{TODO Exploded view alongside photo of the whole thing assembled (with
  wire etc.)}
  \caption{}
  \label{exper:fig:flange}
\end{figure}

The copper heat sink is mounted to the flange using screws where the thread has
been partially removed. This is done to precent any trapped gas causing virtual
leaks inside the chamber. The heat sink supports the subchip and also
incorporates the large U-trap, which is recessed beneath the chip. It is
electrically isolated from the heat sink by \AlN{} plates. This material is
chosen since it is an electrical insulator, but will still allow the conduction
of heat away from the U-wire whilst being UHV compatible. The subchip is
attached to the heat sink with metal screws. Since the conductor tracks are at
points very close to the mounting holes, we isolate the screws from the surface
using washers made from polyimide (another UHV compatible insulator) which were
manufactured in the College workshop.

The current is delivered to the U-wire through two high-current feedthroughs.
For the chip currents, the 16-pin feedthrough is used, and is connected to the
subchip by kapton-coated wires from LewVac. When connecting the wires to the
subchip we use polyimide bushings (also made in the College workshop) to ensure
that they are electrically isolated from the aluminium core of the PCB.

The flange is mounted in the chip chamber, a DN63 cube chamber from Kurt J.
Lesker Co.\ (KJL) whose configuration is shown in \myfigref{exper:fig:chamber}.
A \cm{turbo pump} is connected to the chamber via a tee, so that there is still
a clear line of sight across the surface of the chip. We install three
AR-coated viewports, also from KJL, as shown in the figure. This provides
optical access across the surface of the chip for illumination, imaging
molecules as they fall and a viewpoint from below looking towards the surface.
We attach a DN63-DN40 adapter opposite the chip flange, this will be where the
chamber is attached to the \CaF{} experiment in the future, but for testing
purposes a \cm{pirani gauge or RGA} is attached instead.

\begin{figure}
  \centering
  \cm{Figure showing the layout of the chamber.}
  \caption{}
  \label{exper:fig:chamber}
\end{figure}


\section{Vacuum testing}

For testing the vacuum compatability of the experiment, the RGA is attached to
the chamber. To reach UHV pressures, it is required to bake the experiment.
Heater tape is applied and the chamber is wrapped in foil, the temperature is
then raised over the course of a \cm{few hours} to \cm{over 100 degrees.}  It
is held at this temperature for \cm{atleast 48 hours}, before being ramped back
down. This removes excess water vapour from inside the chamber, and expells any
water that has been absorbed into the chamber walls.

The chamber was first leak checked~\cite{} and then brought to UHV pressure
with the chip flange assembly replaced with a blank. This provided a baseline
pressure for the chamber. We then swapped in the flange assembly, and repeated
the bakeout process with various iterations of our design. In our design
process we were careful to choose materials that are UHV compatible but it was
important to check that the assembly could reach the pressures required. In
particular we were concerned with the Epoxy Technology glue, since any errors
in the mixing and application procedure could cause outgassing, and the solder,
which was not rated for UHV uses.

To measure the pressure of the chamber an RGA scan is undertaken, with the
resulting partial pressures shown with and without the chip assembly in
\myfigref{exper:fig:rga}. The total pressure (the sum of the partial pressures)
is \SI{5.0E-10}{\milli\bar} and \SI{8.8E-10}{\milli\bar} in each case respectively. The scan with the chip assembly
has a similar shape to the empty chamber scan, but with slightly higher values.
In both, we see the typical peaks for hydrogen (2), water (18) and nitrogen
(28), which dominate the spectrum. This is typical of a scan through a UHV
system~\cite{}, and suggests that the chip assembly does not introduce any
sources of contamination or outgassing into the experiment.

\begin{figure}[h]
  \centering
  \begin{tikzpicture}
    \begin{axis}[
        ymode=log,
        enlargelimits=true,
        xlabel=Mass number,
        ylabel=Partial pressure (\si{\milli\bar}),
        width=0.8\textwidth,
        height = 0.4\textwidth,
        legend pos=north east,
        x tick label style={/pgf/number format/.cd, set thousands separator={}},
        ylabel style={yshift=10pt}
    ]
      \addplot [thick, color=blue] table {figs/exper/rga/emptyData.dat};
      \addlegendentry{Empty chamber};
      \addplot [thick, color=pink] table {figs/exper/rga/chipData.dat};
      \addlegendentry{Chip assembly};
    \end{axis}
  \end{tikzpicture}
  \caption{RGA scan for the chip chamber assembly when empty, and when loaded
  with the chip flange assembly.}
  \label{exper:fig:rga}
\end{figure}

\section{Current testing}
% TODO I think maybe this should go above vacuum tests?

It was also important to test the currents that can be achieved through the
chip trapping wires. These tests were conducted with the chip under vacuum
($P<10^{-6}\si{\milli\bar}$) and using the setup shown in
\myfigref{exper:fig:curtest}. A regulated power supply delivers a current to
the chip, controlled by a FET circuit that is in turn switched by a signal
generator. This gives us control over the pulse length, and allows us to
deliver up to \cm{volts and amps}.

\begin{figure}
  \centering
  \cm{Show diagram of current testing setup}
  \caption{}
  \label{exper:fig:curtest}
\end{figure}

\subsection{Wirebond tests}

We originally tested a number of designs where the chip was connected to the
subchip via wirebonds. In these cases it was found that the wirebonds were
unreliable at such high currents. They were tested using a
\SI{200}{\milli\second} pulse, repeating every \SI{10}{\second}. The results
are shown in \myfigref{}. The wirebonds were universally unable to sustain the
current pusles when they reached \cm{2 amps I think?}. Visual inspection of the
chips showed that the wires were intact, but the wirebond joints had failed.
This was confirmed by checking the electrical continuity across the chip.

The wirebonds failed at a current far below that required to form our traps. We
attempted to improve the maximum current by increasing the number of wirebonds
on each pad, and by improving the quality of the wirebond joints. We found that
the former yielded an improvement to the maximum current capacity, but the
quality of the wirebond joints was difficult to quantify. As a rule of thumb,
the wirebonds were re-done if they could not withstand a light tug from a pair
of tweezers. Ultimately the number of wirebonds that it was possible to produce
was limited by the width of the wirebond pad, and the width of the wirebonder
head. We could achieve \cm{tenish}.

We considered using a ribon wirebond~\cite{}, which promises higher current
throughput by using a ribon-shaped wire rather than a round one. This was not
possible due to the unavailabliity of the the hardware at LCN. Another option
was to use higher-diameter wire, but instead we attempted to directly solder
the chip, as described in the next section.

\subsection{Solder tests}

Since the wirebonds failed far below the required currents for our traps, it
was determined that a



\section{Scatter testing}


\section{Background-free imaging}

\cm{Normally we just image with 606nm light and do light-induced fluorescence
imaging, but this creates a load of scatter which reduces SNR.}

One possible method of reducing background whilst imaging molecules is Raman
Resonance Optical Cycling (RROC), an background-free imaging scheme recently
proposed and demonstrated using \SrF{} in \inlineref{Shaw2021}. In this scheme,
off-diagonal ($v\neq v'$) vibrational transitions are driven to excite the
molecule, which will decay primarily on the diagonal ($v=v'$) transitions. The
off-diagonal transitions are separated from the diagonal ones by
$>\SI{10}{\tera\hertz}$, and so it is possible to use a bandpass filter to
exclude the imaging light from any measurement. For an ideal bandpass filter
this would remove any background from imaging light scattered by the apparatus.

Reference~\cite{Shaw2021} reports a suppersion of scattered light by a factor
of $\sim10^6$, and an average emission of 20 photons per molecule. In this
experiment, a beam of molecules was imaged, and this emission is limited by the
interaction time during travel. For imaging a stationary cloud of molecules we
would expect instead to be limited by the reduced scattering rate in the
off-diagaonal transitions. In this section I will \cm{either just describe the
technique, or do this and show some results (fat chance...)}

\subsection{Implementation in \CaF{}}

% TODO Show name and model of bandpass filter

This scheme for \CaF{} is shown in \myfigref{exper:fig:bgfreelevels}. The
\pewpew{}{01} and \pewpew{}{10} light is used to drive the transitions at
\SI{585}{\nano\meter} and \SI{628}{\nano\meter} respectively (see
\mytableref{overview:table:lasers}). Fluorescenc from the decay on the
\SI{606}{\nano\meter} transitions $v'=0\rightarrow v=0$ and $v'=1\rightarrow
v=1$ is isolated by a bandpassfilter, and can be imaged without the usual
background scatter.

\begin{figure}
  \centering
  \includegraphics[width=0.9\textwidth]{figs/energylevels/bgfree.pdf}
  \caption{
  The RROC scheme for \CaF{}. An optical cycle is established, with pumping on
  the $v=0 \rightarrow v'=1$ (gold) and $v=1 \rightarrow v'=0$ (red) transitions.
  Fluorescence from the $v=v'$ transitions (orange) is distinguished from the
  background by a bandpass filter before imaging.
  }
  \label{exper:fig:bgfreelevels}
\end{figure}

\subsection{Results}

% He wrote, cautiously...
