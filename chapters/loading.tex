We have established that our chip experiment will rely on a loading procedure
from a magnetostatic trap onto the chip. This is not dissimilar to existing
procedures where atoms and molecules can be loaded from beams into magnetic
traps~\cite{}, storage rings~\cite{} or MOTs~\cite{}. However unlike
experiments performed with atoms, where the number of particles is very high
(\cm{$N=??$}~\cite{}) \CaF experiments typically involve only a few thousand
particles \cite{}. It is therefore important that losses during the loading
procedure are minimised.

The chip will incorporate a series of wire traps, which will allow granular
control of the potential through the ramping of currents in each wire. This is
a common technique often employed in atom chips.~\cite{} We also include an
intermediary loading stage between the anti-Helmhotlz coils and the chip: a
large U-wire trap integrated into the \cm{subchip (housing??)}. This will
ensure alignment between the macroscopic and microscopic traps.

Designing the trapping wires on the chip will allow us to ensure a low-loss
% In this cite, is it better to cite his textbook chapter of a paper? Maybe the 
% Lichtenberg book?
loading procedure by phase-matching each stage.~\cite{Crompvoets2005} That is
as we transfer molecules from one trap (be it macroscopic or microscopic) to
the next, that the potentials are suitably overlapped so that the majority of
the cloud remains trapped throughout.

In this chapter, I will further explain the principes of phase matching, and
how the chip has been designed to allow this. I will also present simulations
used to model traps, determining their phase-space acceptance and emmittance.
The final design of the chip trap will be presented.

\section{Theory of phase-matching}

\section{Simulating molecules}

\section{Designing the chip trap and its loading scheme}

\section{Summary}


\cm{The first step towards minimising losses} is to ensure good phase matching
between loading stages.


\cm{To minimise losses} the chip must be designed to ensure that at each stage
of loading good phase matching is achieved. T





Loading and guiding of chip traps is commonly achieved by adiabatically
altering the trapping potential.~\cite{} Chip designs often incorporate
multiple trapping wires

% TODO Do I actually discuss this
As discussed in chapter~\ref{chiptraps}, it is possible to load atoms into
chip traps and then guide them by adiabatically altering the trapping
potential. \cm{Do I need a citation here?} This control is possible by
incorporating multiple trapping wires (or in the case of electrostatic traps,
trapping electrodes) whose currents (voltages) can be \cm{adiabatically}
varied. Bias fields can also be used to control the trapping potential.

A simple example of this is controlling the trapping height above a 2D wire
trap. The trap height is given by \cm{reference equation above?}
%
\begin{equation}
z_0 = \frac{\mu_0I}{2\pi \tilde{B}}.
\end{equation}
%
Increasing the bias field $\tilde{B}$ will therefore decrease the trap height.
To achieve an adiabatic change, the timescale of the change in $\tilde{B}$ must
be grater than the trapping timescale, that is $\tau_\text{ramp} \gg
\tau_\text{trap}$. The trap timescale is related to the trapping frequency
given by \cm{some equation}, $\tau_\text{trap} = 2\pi / \omega_\text{trap}$. So
we have
%
\cm{I think I just want $\omega_\text{trap} = \omega_\text{Lamour} = m_F g
\mu_B B / \hbar$, because any atoms moving on a faster timescale cannot remain
trapped (their state changes faster than the trap??)
}
%
For a typical microfabricated wire trap operating at \SI{1}{\ampere} and trapping
at a height $z_0 = \SI{1}{\milli\meter}$, a ramp to $z_1 =
\SI{10}{\micro\meter}$ will take a time on the order of \cm{TODO}.


