\thesis{
Quantum optical systems are promising candidates for use in a vast range of
technologies, from metrology, to simulation, to probing the fundamentals of
physics. But despite their high controllability and coherence times, they often
lack scalability, requiring ungainly optical systems to function.

Chip traps were introduced as a means of miniaturizing such systems. Magnetic
traps have been formed from fields generated by wires on a substrate and
external bias fields. This forms a robust trapping environment with very high
field gradients. Chips are innately scalable due to their fabrication by
well-understood standard photolithographic methods, and have been routinely
used for experiments with atoms. Atom chips have proven to be particularly
useful in formation of Bose-Einstein condensates. They show promise for
coherent control, and integration with other quantum systems via coupling to
microwave guides also integrated with the chip.

With the development of new technologies for the creation of ultra-cold
molecules, realisation of a molecule chip is now a possibility. Rotational
states of molecules couple strongly to microwave fields, allowing the
exploitation of cavity QED effects for use in a scalable system which can be
integrated with other quantum architectures.
}

In this report, we present ongoing work into development of a molecule chip
operating on a magnetically insensitive transition of \CaF{}. In particular, we
discuss simulations that have allowed us to refine the design of the chip
trap, and the microfabrication techniques used to build a prototype chip.

Simulations of the trajectories of molecules in magnetic traps have been
performed. These are used to analyse the trap design, and to ensure that
loading of molecules is possible through a series of adiabatic transfers
between magnetic potentials. The phase space acceptance of traps is presented.

The microfabrication of an earlier trap design is presented. Through-mask
electroplating has been used to construct a chip with trapping wires with
heights of \SI{3}{\micro\meter}. This is much taller than would be possible
with conventional photolithography methods, and will allow for high currents to
be carried on chip, thus forming deep magnetic traps for our molecules.
