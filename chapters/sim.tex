In this chapter I present simulations of the molecules in the trap. In
particular I show that we have developed a procedure for loading molecules into
the chip trap.

\section{Motion of molecules in a trap}
\label{sim:motion}

We can assume that the motion of the molecules in the trap is classical. They
move in the potential $V(t, \mathbf{q}) = \mu B(t, \mathbf{q})$, where
$\mu\approx\mu_B$ is the magnetic dipole moment of the molecule in the
$\ket{N=0,F=1,m_F=1}$ state.  The motion of any one particle is described by Hamilton's
equations,~\cite{Lichtenberg1969}
%
\begin{align}
  \label{sim:eq:hamilton}
  \dot{\mathbf{q}} =  \frac{\partial H}{\partial \mathbf{p}} &&
  \dot{\mathbf{p}} = -\frac{\partial H}{\partial \mathbf{q}},
\end{align}
%
where $H$ is the classical Hamiltonian of the system
\begin{equation}
  %
  H(t, \mathbf{q}, \mathbf{p}) = \frac{\mathbf{p}^2}{2m} + V(t, \mathbf{q}).
\end{equation}
For now we neglect the time dependence of the potential, so that $V(t,
\mathbf{q}) = V(\mathbf{q})$.

Solving Hamilton's equations tells us the position and momentum of a single
particle. Taken together these two vectors describe a point in a six
dimensional phase-space of position and momentum.
%
For an ensemble of particles we often consider the envelope of the region they
occupy, calling this space the phase-space volume of the ensemble. This is a
powerful tool, since calculating the trajectory of particles on the boundary of
this region is often sufficient to describe the behaviour of the whole
ensemble. This technique is commonly used in particle trapping, and especially
when considering the behaviour of beams of particles, where a four-dimensional
phase-space is used \cite{Hand1998, Lichtenberg1969}.

It is helpful for us to write the phase-space volume $V$ in terms of the
spatial volume occupied by the ensemble ($V_\text{space}$) and its temperature.
This is possible because in thermal equilibrium the momentum distribution is 
%
\begin{equation}
  f(p) = \frac{1}{(2 \pi m k_B T)^\frac{3}{2}}\exp(-\frac{p^2}{2 m k_B T})
\end{equation}
%
whose maximum value
%
\begin{equation}
  \left(\frac{\lambda_\text{dB}(T)}{h}\right)^3 = \frac{1}{(2 \pi m k_B
  T)^\frac{3}{2}}
\end{equation}
%
depends only on the temperature $T$. Here we have introduced the de Broglie
wavelength $\lambda_\text{dB}(T)$, the length-scale for a monatomic gas of
temperature $T$.~\cite{blundell2}

The unitless phase-space volume is
%
\begin{equation}
  V = V_\text{space} \lambda_\text{dB}^{-3}(T).
\end{equation}
%
It is also useful to define a unitless phase-space
density~\cite{PhysRevA.52.1423}
%
\begin{equation}
  \rho = \frac{N}{V} = \frac{N \lambda_\text{dB}(T)^3}{V_\text{space}}.
\end{equation}
%
For a given cloud with uniform phase-space density $\rho$, trap with volume
$V_\text{trap}$ and depth $T_\text{depth}$, the maximum possible number of
particles that can be trapped is $\rho V_\text{trap}/\lambda_\text{dB,
trap}^3$, where $\lambda_\text{dB, trap} = \lambda_\text{dB}(T_\text{depth})$.

A second powerful tool for determining the motion of the particles is
Liouville's theorem~\cite{Landau1982, Hand1998} which states that the phase-space
volume is a conserved quantity, as long as the trapping potential is
conservative. This means that the phase-space density of a trapped molecular
cloud cannot be increased without application of some velocity-dependent
force, such as an optical molasses~\cite{Metcalf1999}. The impact of this for
the chip is that the phase-space density of the cloud at the point of loading
onto the chip determines the maximum number of molecules that can be trapped in
the final trap.

% This calculation in Mathematica notebook 2022-01-20
If we do not introduce any further cooling steps, then the number of molecules
we are able to contain in the final trap will be
%
\begin{equation}
  N_\text{final} = \frac{\rho_\text{\CaF{}}V_\text{trap}}
  {\lambda_\text{dB, trap}^3} = \frac{\rho_\text{\CaF{}} V_\text{trap}(2 \pi m k_B
  T_\text{trap})^\frac{3}{2}}{h^3}.
  \label{sim:eq:psd_N}
\end{equation}
%
Phase space densities of $\rho_\text{\CaF{}} \sim 10^{-8}$ can be achieved by
laser cooling\footnote{Even higher phase-space densities (of the order
  $10^{-4}$) have been achieved using a cross-dipole trap~\cite{Anderegg2019a},
  but for now this is beyond the capabilities of our
experiment}~\cite{PhysRevLett.121.083201}. The smallest trap will have volume
$V_\text{trap}\sim(\SI{10}{\micro\meter})^2\times\SI{1}{\milli\meter}$ and
depth $T_\text{trap}=\SI{4}{\milli\kelvin}$, which can be calculated by
equation~\ref{theory:eqn:depth}.
%
We therefore expect that as many as $10^5$ molecules can be trapped.

In the current experiment we are not able to implement a dipole trap, although
this technique is being developed within our group. Using sub-Doppler cooling
techniques, the maximum phase
space density that we can achieve is much lower, at $\rho_\text{\CaF{}} =
3\times10^{-12}$ meaning the number of molecules trapped will be of order one.
Clearly we require the high phase-space density cloud to efficiently load the
final trap, however it will still be possible to load molecules into the outer
traps with the techniques available to us. By carefully designing a loading
scheme, we hope to be able to maximise the number of molecules that will be
accepted in the final trap.


\subsection{Phase-space acceptance}

For a particle to be trapped it must occupy some spatial region that is within
the trapping potential and it must not have sufficient energy to escape from
said potential. In other words, the total energy of the particle, given by the
classical Hamiltonian $H(\mathbf{q}, \mathbf{p})$, must be less than the trap
depth $k_B T_\text{depth}$.  This relation defines a region of phase-space that
we call the acceptance. Any particles within this region will not have
sufficient energy to escape, and so will remain trapped under classical motion
unless externally influenced (ignoring for example any collisions, decays,
Majorana losses, etc.)~\cite{Lichtenberg1969}.

As an example consider the one-dimensional case of a harmonic trap found in 
\inlineref{Crompvoets2005}
%
\begin{equation}
  V(z) = \begin{cases}
    0 & |z| > l \\
    \frac{1}{2}m\omega^2 z^2 & |z| \leq l.
  \end{cases}
\end{equation}
%
Here, any particle which starts with a position $-l \leq z \leq l$ will be
trapped, so long as its velocity is not sufficiently large that it will escape
into the $|z|> l$ region. To reach $z=l$ from an initial position $z_0$, the
particle must have kinetic energy of at least
%
\begin{equation}
  \frac{1}{2}mv_0^2 = \frac{1}{2}m\omega^2(l^2 - z_0^2),
\end{equation}
%
where $v_0$ is the initial velocity. It is now clear that a particle is trapped
on the condition that
%
\begin{equation}
  (v_0/\omega)^2 + z_0^2 < l^2
\end{equation}
%
which defines an ellipse in phase-space.

% Note that in the simulation I have omega = 1, so taking omega -> 1000 means
% that we end up with the nice circle with v in m/s. The consequence is that
% the simulation ends up lasting 1.2ms rather than 1.2s

This trapping condition can be verified by simulation, as is shown in
\myfigref{sim:fig:psaeg} (a) and (d), where the trap has been simulated for
$m = 1$, $\omega = \SI[parse-numbers=false]{10^3}{\radian\per\second}$ and $l
=\SI{1}{\milli\meter}$. We initialise 2000 particles uniformly distributed with
$|z_0| < \SI{0.2}{\milli\meter}$ and $|v_0|< \SI{1.5}{\meter\per\second}$.
Their trajectories are computed for \SI{1.2}{\milli\second} by the methods
described in the next section. The boundary of the acceptance, called the
separatrix is shown in black. All particles that are initialised inside the
acceptance remain trapped and evolve through time in the usual way for a
harmonic potential.  Particles initialised outside the acceptance are rejected
and lost.

It is also instructive to consider anharmonic potentials, for example
%
\begin{equation}
  V(z) = -V_0\cos(\omega z)
\end{equation}
%
which is also presented in \myfigref{sim:fig:psaeg}, using the same
parameters as for the previous potential. We again see that particles with
sufficiently low energy in the trapping region remain trapped, however in this
case the anharmonicity of the trap results in the particle cloud spiralling
outwards, undergoing an effective increase in the phase-space density. This
means that although the actual phase-space density remains the same, the
envelope of the region that they occupy becomes increasingly convoluted. This
makes it effectively impossible to contain them in any potential with a volume
smaller than that of the trapping potential. This effect is often referred to
as filamentation.

Finally, it is useful to consider the phase-space acceptance of a wire trap.
For simplicity we begin by considering only the acceptance in the $z$ direction
(perpendicular to the chip surface). We calculate the acceptance of a Z-wire
with an axis of \SI{20}{\milli\meter}, current of \SI{40}{\ampere} and a
trapping height of \SI{3}{\milli\meter}. The particles are initialised such
that they have no motion in the $x$ and $y$ directions, and the trajectories
are simulated for \SI{200}{\milli\second}. Figures~\ref{sim:fig:psaeg}~(c)
and (f) show the distribution at the start and end of the simulation.
The accepted molecules (blue) are once again those that start inside the
separatrix. There are also metastable molecules (orange) which remain outside
the acceptance but stay nearby the trap. These are molecules with nearly
sufficient energy to escape the trap, but may take some time to do so.
We call the occupied area of phase-space at the end of the trapping
period the trap's phase-space emittance.

\begin{figure}[htb]
  \centering
  \scalebox{0.8}{\import{figs/simsThesis/}{psa_1D.pgf}}
  \caption[Phase-space acceptance examples]{
    The phase-space acceptance for harmonic (a, d), anharmonic (b, e) and
    Z-wire trap (c, e) are shown. The top row shows the initial positions of
    the particles: those that are accepted by the trap are shaded blue,
    particles that are lost are shaded red, and metastable particles in orange.
    The energy contour of the trap depth is also shown. The end state of the
    simulation is shown in the lower row. Note the filamentation effect in (e)
    and (f), where the effective phase-space density of the particles decreases
    as they explore regions that are energetically accessible. The harmonic and
    anharmonic examples are inspired by the discussion in
    \inlineref{Crompvoets2005}.
  }
  \label{sim:fig:psaeg}
\end{figure}

While this applies to a rapid switching off and on of trapping potentials, we
note that it is possible to adiabatically transform potentials in such a way
that the phase-space density of the ensemble remains unchanged. This can be
achieved because the Liouville theorem also holds for time-dependent
Hamiltonians, as long as the potential remains conservative
throughout~\cite{Hand1998, Lichtenberg1969}. Hence a cloud of trapped particles
can be translated, for example in the transport coils, or towards the surface
of the chip as will be discussed in section~\ref{sim:adiabatic}.

\subsection{Simulating the motion}
\label{sim:motion:simmethods}

The motion of a particle can be simulated by numerically solving
\myeqref{sim:eq:hamilton}, which we do using Python~\cite{python} and the
symplectic Euler method~\cite{Hairer2015, doi:10.1119/1.2034523} provided by
the Desolver package~\cite{desolver}. Unlike other numerical methods,
symplectic integrators guarantee the conservation of energy and momentum.
We are able to simulate any arbitrary potential, such as those already
discussed in the previous section, but it is useful to consider a more
realistic potential given by the sum of the magnetic and gravitational
potentials,
%
\begin{equation}
  V(t, \mathbf{q}) = V_\text{mag} + mg(z_0-z)
\end{equation}
where $g=\SI{9.8}{\meter\per\second\squared}$ is the acceleration due to
gravity, and $z_0$ is an arbitrary point chosen to be the zero of the
gravitational potential. The actual value chosen does not matter as it
constitutes only a linear offset in the entire potential. We choose
$z_0$ to be the minimum of the trapping potential at the start of each
simulation.
%
The Van der Waals attraction between the chip and the molecules is neglected,
since the molecules will (by design) never be close enough to the chip for this
force to be significant.

The magnetic potentials of the wire traps are calculated by considering the
traps to be formed of segments of straight wires, each producing a magnetic
field $\mathbf{B}_\text{seg}^{(i)}$, and the bias field
$\mathbf{B}_\text{bias}(t)$.  The total field is the sum of contributions from
the set of all segments ($S$), and the bias. We also make the approximation
that the field changes slowly compared to the alignment of the spin with the
field\footnote{This means that simulations investigating fast-changing fields
  present an optimistic result. However we will see below that the limiting
  factor for changing the field is the time constant of the classical motion of
  the molecules. Hence this adiabtic approximation is sufficent for our
investigations.}, so that $V=\mathbf{\mu}\cdot\mathbf{B}\approx\mu B$.  The
potential is therefore
%
\begin{equation} V_\text{mag}(t, \mathbf{q}) = \mu B (t, \mathbf{q}) = \mu
\left| \sum_{i\in S} \mathbf{B}_\text{seg}^{(i)}(t, \mathbf{q}) +
\mathbf{B}_\text{bias}(t)\right|,  \end{equation}
%
where the field of each wire segment is~\cite{Griffiths2017}
%
\begin{equation}
    \mathbf{B}_\text{seg}(t, \mathbf{q}) = \frac{\mu_0 I(t)}{4\pi
    s_\text{seg}(\mathbf{q})} (\sin(\theta_2)  -
    \sin(\theta_1))\hat{\mathbf{\phi}}.
    \label{sim:eq:segmentfield}
\end{equation}
%
Here $s_\text{seg}(\mathbf{q})$ represents the shortest distance between the
point $\mathbf{q}$ and the line that runs through the segment off to infinity
in each direction as defined in \myfigref{sim:fig:wiresegment}. We also have
the unit vector $\hat{\mathbf{\phi}} = \mathbf{I}\times\mathbf{q}/(qI)$.

\begin{figure}[h]
\centering
  \begin{tikzpicture}
    % Def coords
    \coordinate (O) at (0, 0);
    \coordinate (L) at (-3, 0);
    \coordinate (R) at (3, 0);
    \coordinate (Q) at (-4, 4);
    \coordinate (P) at (-4, 0);
    % Draw lines
    \draw[line width=0.75mm, ->] (L) -- (R);
    \draw (L) -- (Q);
    \draw (R) -- (Q);
    \draw[<->, densely dotted,shorten >=.5mm,shorten <=.8mm] (Q) -- (P);
    % Draw line parallel to wire
    \draw[-, dashed] (-5, 0) -- (L);
    \draw[-, dashed] (R) -- (5, 0);
    % Draw dot at Q
    \node at (Q)[circle,fill,inner sep=.5mm]{};
    % Draw angels
    \draw pic[->, draw,angle radius=1cm,"$\theta_1$" shift={(-3mm,-1mm)}] {angle=P--Q--L};
    \draw pic[->, draw,angle radius=1.5cm,"$\theta_2$" shift={(2mm,-2mm)}] {angle=P--Q--R};
    % Label
    \node[shift={(4mm,0)}] at (Q) {$\mathbf{q}$};
    \node[shift={(2mm, 4mm)}] at (R) {$I$};
    \node[fill=white] at (-5.0, 1.8) {$s_\text{seg}(\mathbf{q})$};
  \end{tikzpicture}
  \caption[Geometry of a wire segment]{
    Geometry of a wire segment (bold) carrying current $I$, whose field
  can be calculated using \myeqref{sim:eq:segmentfield}. The dotted line
  shows $s_\text{seg}(\mathbf{q})$, the shortest distance from the point at
  which the field is calculated ($\mathbf{q}$) to the line parallel with the
  wire (dashed line).
  }
  \label{sim:fig:wiresegment}
\end{figure}

%In the case of that we are exchanging between
%two potentials, the segments of both traps are included in the summation, with
%the currents becoming functions of time, and similar for exchange between a
%quadrupole and a wire trap, with the quadrupole gradient being a function of
%time.

\subsection{Simulation initialisation}
\label{sim:sim:init}

For all the simulations described below, we begin with a cloud of molecules in
a magnetic quadrupole trap. We use the following initialisation procedure so
that the distribution of our molecules matches that which we expect in the
experiment. 
%
The simulation begins with a cloud of $N$ molecules whose positions are normally distributed
in all three spatial dimensions with a standard deviation of $\sigma_i$
(typically $\sigma_i = \SI{1}{\milli\meter}$). Similarly the velocity components
are normally distributed with a standard deviation of $\sigma_{v_i}$ (typically
\SI{400}{\milli\meter\per\second}, corresponding to a temperature of
\SI{50}{\micro\kelvin}). 

The cloud is initialised in a quadrupole trap with gradient
\SI{10}{\gauss\per\centi\meter}. They are held for \SI{50}{\milli\second}
before the gradient is linearly ramped over \SI{100}{\milli\second} to
\SI{60}{\gauss\per\centi\meter} (the gradient used in the MTT).  The molecules are
then held for a further \SI{50}{\milli\second}, as a stabilisation time. The
increasing trap gradient compresses the distribution. The resulting
distribution is Gaussian in velocity and position, centred on
$z=\SI{3}{\milli\meter}$, as shown in \myfigref{sim:fig:initsum}.

\begin{figure}[htb]
\centering
  \import{figs/simsThesis/}{init_summary.pgf}
  \caption[Phase-space simulation initialisation]{
    The particle distribution in phase-space after initialisation. A
  scatter plot of the $z\,v_z$-phase-space plane the is shown in (a), with
  histograms of $z$ and $v_z$ distributions in (c) and (b) respectively. The
  position distributions for $x$ and $y$ are shown in (d) in orange and green
  respectively.}
  \label{sim:fig:initsum}
\end{figure}

\section{Adiabatic transfer between traps}
\label{sim:adiabatic}

It should be obvious that if we start with molecules in one potential, then
rapidly turn that one off and turn on another with a different shape, is likely
to cause heating and increase the size of the cloud.  However, if we ramp
between the traps adiabatically then we expect the phase-space density to be
conserved. In order to understand how best to load the chip trap it is
important that we understand the extent of heating in the case of rapid
transfer and the timescale required to be adiabatic.

We investigate this with a simple example: consider two ideal quadrupole traps
with gradients of \SI{60}{\gauss\per\centi\meter} separated by
\SI{3}{\milli\meter} in the $x$ direction. We simulate 2000 \CaF{} molecules
using the above-described initialisation procedure, with initial $\sigma_i =
\SI{2}{\milli\meter}$ in all directions, and initial temperature
$T=\SI{50}{\micro\kelvin}$. We ramp linearly between the potentials in a time
$t_\text{ramp}$, which we vary between \SI{1}{\milli\second} and
\SI{100}{\milli\second}. The molecules are then held for a further
\SI{100}{\milli\second}. The state of the system at the end of the simulation
is shown for $t_\text{ramp}\in \{\SI{1}{\milli\second}, \SI{30}{\milli\second},
\SI{100}{\milli\second}\}$ in \myfigref{sim:fig:adia3sim}, along with the
behaviour of the cloud in the \SI{100}{\milli\second} hold time. Note that as
the ramp time increases, the oscillation in cloud size, temperate and position
decreases, as we have said we expect.

\begin{figure}[p]
\centering
  \scalebox{0.8}{\import{figs/simsThesis/}{adiabatic_3sim.pgf}}
  \caption[Simulation of exchange between quadrupole traps]{
    Exchange between two quadrupole traps, separated by \SI{3}{\milli\meter},
    for three different ramp times. The top row shows the $x$-projection of the
    phase-space distributions of the particles at the start of the simulation
    (grey) and after various ramp times (see legend) followed by a
    \SI{100}{\milli\second} hold in the displaced trap. The mean position, size
    and temperature of the cloud are shown throughout the hold time for each
    simulation. Note that for the longer ramps we observe less severe heating,
    oscillation and widening of the cloud.
  }
  \label{sim:fig:adia3sim}
\end{figure}


The question then is at what point do we enter the adiabatic regime?
\myfigref{sim:fig:adiavary} shows the cloud width and temperature at the end
of the simulation for varying $t_\text{ramp}$. After
$t_\text{ramp}\sim\SI{50}{\milli\second}$ the reduction is minimal, and we are
strongly in the adiabatic regime. This is approximately the timescale we
would expect. Although the trap is not harmonic, we can still give the
characteristic period for a particle that has a characteristic amplitude of
motion. Assuming that the particle is under constant acceleration, the
timescale for a particle to traverse the trap is
%
\begin{equation}
  \tau = \sqrt{\frac{m \sigma}{\mu_B B'}},
\end{equation}
%
where $\sigma$ here means the width of the trapped cloud. To be in the
adiabatic regime, we expect to have any particle traverse the trap multiple
times over the duration of a ramp Our typical \CaF{}
clouds
have width $\sigma = \SI{5}{\milli\meter}$, so the timescale will be $\tau
\approx \SI{10}{\milli\second}$. We expect the adiabatic regime to
be reached when a process takes place over a duration that is several multiples
of this timescale, which is in good agreement with the simulations.

\begin{figure}[t!]
\centering
  \scalebox{0.8}{\import{figs/simsThesis/}{adiabatic_varyramps.pgf}}
  \caption[Effect of ramp duration on phase-space distribution]{
    The final temperature and size of particle clouds transferred between
    offset quadrupole traps at various ramp times. Note that the large
    oscillations in width that occur for short ramp times (see
    \myfigref{sim:fig:adia3sim}) can result in the instantaneous width of
    the cloud being small, such as in the shorter ramp times here.
  }
  \label{sim:fig:adiavary}
\end{figure}

\section{Trajectory simulation of the initial loading stages}

In this section we will present simulations of loading molecules from the MTT
into the on-chip Z-wire traps. 
This demonstrates
the two main types of transitions in the loading process: handovers between
traps, and compression within a single trap. The timing of the simulation do
not exactly reflect those that will be used in practice and long holding periods
have been inserted between ramps to better illustrate the cause of particle
loss. It is useful to perform direct trajectory simulations as opposed to
simple analysis of the phase-space emittance and acceptance, since (as we will
see later) brief lowering of trap depth during trap handovers can induce
additional loss of particles.

We will consider the handover in four stages: the particle initialisation, which
was already discussed in section~\ref{sim:sim:init}; followed by the
handover from the MTT to the U-wire; then the U-wire to $\mathrm{Z0_i}$.
Finally we will simulate the compression in the Z-wire trap from
$\mathrm{Z0_i}$ to $\mathrm{Z0_f}$. A full summary of the changing parameters
throughout the simulations is given at the end of the section, see especially
\myfigref{sim:fig:simsum}.

\subsection{MTT - U transfer}
\label{sim:sim:trans_U}

% This is PSD 3.8E-10, which is a lot lower than can reasonably be achieved.
% Might be asked to comment on this
The cloud is initialised by the usual procedure, with $\sigma_i =
\SI{1}{\milli\meter}$ and $T=\SI{50}{\micro\kelvin}$, corresponding to the
typical cloud sizes and temperatures we measure in the tweezer chamber after
transport. Over \SI{100}{\milli\second} the coil current is linearly
reduced to zero and the U-wire current is ramped up to its maximum value of
\SI{100}{\ampere}. The bias fields in the $y$ and $z$ directions (no $x$ bias
is applied) are also ramped linearly, to maintain the trap centre at a height
of \SI{3}{\milli\meter} below the chip. The state of the system before and
after the ramp is shown in \myfigref{sim:fig:mttusim}.

\begin{figure}[p]
\centering
  \scalebox{0.8}{\import{figs/simsThesis/}{mtt_u_summary.pgf}}
  \caption[Simulation of transfer from MTT to U-trap]{
    The simulation of the transfer from the MTT to the U-wire. The state of the
    system is presented at the end of the initialisation procedure (blue) when
    the trapping potential is that of the MTT, and after ramping to the U-trap
    (red). The phase-space distributions are shown in the $z$-$v_z$ plane in
    (a) and (b), with histograms of the $z$ and $v_z$ projections in (c) and
    (e) respectively. The potentials at the start and end of the transfer are
    shown in (d). Subfigure (f) shows the temperature and the particle number
    varying throughout. Transfer occurs during the times that are shaded grey
    (c.f.~\myfigref{sim:fig:simsum}).
  }
  \label{sim:fig:mttusim}
\end{figure}

The phase-space distribution in the $z$-$v_z$ projection is shown for the start
(\mysubfigref{sim:fig:mttusim}{a}) and end
(\mysubfigref{sim:fig:mttusim}{b}) of the simulation. Note that there is
little change to the velocity distribution, but in position the particles shift
to occupy the positive $z$ direction, where the potential is less steep.

Looking at the change in the particle number throughout the ramp (grey region
of \mysubfigref{sim:fig:mttusim}{f}) it is clear that there is some particle
loss. We would naively expect there to be no loss during this transfer, since
the final trap depth is larger than the temperature of the cloud, and there is
good mode-matching of the two potentials. To see why this loss occurs, we  look
at the evolution in phase-space for each projection throughout the ramp, as
shown in \myfigref{sim:fig:mttudetail}.

\begin{figure}[htp]
\centering
  \begin{tabular}{c}
    \scalebox{0.8}{\import{figs/simsThesis/}{mtt_u_detail_x.pgf}} \\
    \scalebox{0.8}{\import{figs/simsThesis/}{mtt_u_detail_y.pgf}} \\
    \scalebox{0.8}{\import{figs/simsThesis/}{mtt_u_detail_z.pgf}}
  \end{tabular}
  \caption[Phase-space distribution throughout MTT-U handover]{
    Each pair of rows shows the phase-space plots for the MTT to U-trap
    handover simulation in the $q_i$-$v_i$ plane, along with the cut-through of
    the potential through the trap centre along $q_i$ at various times. This
    supplements the simulation summary shown in \myfigref{sim:fig:mttusim}.
    The superposition of the quadrupole and U-traps briefly causes a lowering
    of the trapping potential, as can be seen at \SI{75}{\milli\second}. These
    reductions can occur when moving between different types of traps, such as
    from the coil-based to wire trap, or from a quadrupole to Ioffe-Pritchard
    trap. Care must be taken in such cases to ensure the potential is
    sufficiently deep throughout the handover.
}
  \label{sim:fig:mttudetail}
\end{figure}

At $t=\SI{75}{\milli\second}$ there is a noticeable decrease in the trapping
depth in the $-y$ and $+z$ directions. This is the cause of the particle leak,
and occurs due to the due to the superposition of the two fields causing an
overall decrease in the trap depth. When the trap depth is increased again, the
energy of particles in this region is increased, and they can leave the trap.
This only occurs for a brief period during the ramp (lasting approximately
\SI{40}{\milli\second}).  Hence some of the more energetic particles are lost,
as can be seen in the figure. The key finding is that the molecules are able to
be transferred from the MTT to the U-trap with minimal loss by mode-matching of
the potentials.

\subsection{U - Z transfer}
\label{sim:sim:U_to_Z0i}

The next stage of loading is the U to $\mathrm{Z0_i}$ handover. We simulate
this process as a continuation of the MTT-U transfer.  The Z-wire current is
then linearly ramped on over \SI{100}{\milli\second} to \SI{30}{\ampere}, while
the U-wire current is ramped off. The
bias fields are linearly ramped to maintain the trap height at
\SI{3}{\milli\meter}. The simulation is presented in \myfigref{sim:fig:uzsim},
and the ramps are summarised in \myfigref{sim:fig:simsum}.

In the former figure we have the initial (a, blue) and final (b, red) particle
distributions.  With the $z$ velocities and positions summarised in the
histograms in (c) and (e) respectively. We can see that the mean position and
velocity remains the same, but it is clear from (b) that there are a number of
particles that have picked up velocity in the positive $z$ direction, and can
be seen leaving the trap. This is the notable streak to the right of the
subfigure, and can also be seen in the tail of the histogram in (e). The
particle number decreases by 20\% as seen in (f), although the particles do not
fully leave the trap until the next step (see \mysubfigref{sim:fig:simsum}{d}).
This loss is similar to that of the MTT to U-trap handover, with a decrease in
trap depth that we attribute from changing from a quadrupole to Ioffe-Pritchard
trap. This changeover can be seen in (d) where we note that the final trapping
potential's minimum is non-zero. Despite the particle losses there is minimal
change to the particle distribution

\begin{figure}[p]
\centering
  \scalebox{0.8}{\import{figs/simsThesis/}{u_z_summary.pgf}}
  \caption[Simulation of transfer from U-trap to Z-trap]{
    The simulation of the transfer from the U-wire to the Z-wire. The state of
    the system is presented at the end of the initialisation procedure (blue)
    when the trapping potential is that of the U-trap, and after ramping to the
    Z-trap (red). The phase-space distributions are shown in the $z$-$v_z$
    plane in (a) and (b), with histograms of the $z$ and $v_z$ projections in
    (c) and (e) respectively. The potentials at the start and end of the
    transfer are shown in (d). Subfigure (f) shows the temperature and the
    particle number varying throughout the simulation. Transfer occurs during
    the times that are shaded grey (c.f.~\myfigref{sim:fig:simsum}).

  }
  \label{sim:fig:uzsim}
\end{figure}

\subsection{Z compression}

The final stage of the loading procedure is a linear ramp of the trap height
from \SI{3}{\milli\meter} below the chip surface to \SI{1}{\milli\meter} below.
Here only the $y$ bias field is changed so  that the trap depth
increases and the cloud is compressed, providing good localisation of the
molecules. Ramping the height linearly requires the bias to change as a
function of time so that $\tilde{B}_{y} \propto t^{-1}$.

As can be seen in \myfigref{sim:fig:zsim}, there is particle loss,
although much of this is due to particles that had already left the trapping
region in the previous ramp. As can be seen in subfigure~(f), the particle
number is decreasing before the ramp begins. The other major effect of the ramp
is heating, which occurs due to the spatial compression and conservation of
phase-space density.
%
The transfer of the cloud towards the chip surface is successful, as can be seen
in \mysubfigref{sim:fig:zsim}{e}, the particles are highly localised around the
new trap minimum at $z=\SI{1}{\milli\meter}$.

\begin{figure}[p]
\centering
  \scalebox{0.8}{\import{figs/simsThesis/}{z_summary.pgf}}
  \caption[Simulation of compression in Z-trap]{
    The simulation of the compression in the Z-wire trap. The state of the
    system is presented at the end of the initialisation procedure (blue) when
    the trapping potential minimum is held \SI{3}{\milli\meter} from the
    surface, and after compressing to the height of \SI{1}{\milli\meter} (red).
    The phase-space distributions are shown in the $z$-$v_z$ plane in (a) and
    (b), with histograms of the $z$ and $v_z$ projections in (c) and (e)
    respectively. The potentials at the start and end of the transfer are shown
    in (d). Subfigure (f) shows the temperature and the particle number varying
    throughout the simulation. Transfer occurs during the times that are shaded
    grey (c.f.~\myfigref{sim:fig:simsum}).
  }
  \label{sim:fig:zsim}
\end{figure}

\subsection{Summary of the simulation}

The entire simulation from MTT to $\mathrm{Z0_f}$ is summarised in
\myfigref{sim:fig:simsum}. For each stage: initialisation (a), MTT-U
transfer (b), U-Z transfer (c) and Z compression (d) the particle number is
shown. The total temperature and cloud width are also given for the particles
that remain in the trap at the end of the simulation. The profile of parameters
that are changed for each ramp are also shown. 


\begin{figure}[p]
\centering
  \scalebox{0.8}{\import{figs/simsThesis/}{total_summary.pgf}}
  \caption[Summary of loading simulations]{
    Details for each stage of the simulation in turn. The top two rows show
    properties of the molecule cloud and the second two rows show the trapping
    currents and bias fields throughout. Column (a) shows the initialisation
    procedure. Column (b) shows the ramp from the MTT to U-wire, with the ramp
    duration shaded grey.  In column (c) we have the U-Z handover and the $z$
    bias field is switched off. Finally in column (d) is the compression in the
    Z-trap.
  }
  \label{sim:fig:simsum}
\end{figure}

The simulation shows that the loading of the initial chip trap can bring in
approximately 60\% of the initial cloud. We assumed an initial temperature of
\SI{50}{\micro\kelvin}, however refining the transport process could allow us
to load clouds of lower temperatures, which could mean that a higher loading
efficiency is possible. It may also be possible to implement cooling during the
trap transfer, but this is restricted by the limited optical access in the chip
chamber.
%
There is some expansion of the cloud as molecules are lost, particularly caused
during the U-Z handover stage.
%
Heating in the trap is largely due to the compression in the final stage. This
is inevitable, but does not cause any problems, since the trap depth
increases for traps closer to the surface, and the molecules are
well contained.

The trajectory simulation discussed in this section is instructive, since it
tells us about the handovers between different types of traps. Notice
that the largest source of particle loss is when converting from the quadrupole
U-trap to the Ioffe-Pritchard Z-trap. Here it is not sufficient to look just at
the phase-space emittance and acceptance in the final traps, but we must
investigate the potential across the ramp, since a brief lowering of the
potential can induce enormous loss. Fortunately this is not an issue when
compressing and handing over between Z-traps, and the phase-space analysis is
sufficient for the rest of the procedure.

\section{Phase-space acceptance of remaining Z-traps}
\label{sim:transferbetweenzs}

Once trapped in $\mathrm{Z0_f}$, the remaining loading stages are simply
repeated handovers to the smaller wires, with a compression stage carried out
on each wire to bring the molecules closer to the surface. This is a
well-understood and robust procedure that has been used before to load atom
chips with minimal losses~\cite{Reichel2002}. The ramps used are detailed in
table~\ref{overview:table:wires}.  The timing of the ramps could be shorter for
the Z-traps, since the timescale of motion in these traps is higher. This, as
well as the optimal shape of the ramp can be determined empirically.

That said, since our experiment will operate with a significantly lower phase
space density than previous experiments with atoms, it is important to ensure
that our loading procedure will not suffer from any unnecessary losses. These
compressed traps contain hotter molecules, requiring a smaller time step for
simulation. It is therefore easier to directly examine the acceptance of each
trap. We also note now that the trap gradients are sufficiently large the
effects of gravity can be ignored.

The trap acceptances are shown in \myfigref{sim:fig:phasematchinggrid},
starting with $\mathrm{Z0_f}$ in row (a). The region of the trap that we expect
to be occupied (as determined by the simulation above) is marked with a dashed
line.  This cloud of molecules will be adiabatically transferred into
$\mathrm{Z1_i}$.
%
The cloud of molecules will be transferred to $\mathrm{Z1_i}$ with negligible
heating. However, the total acceptance of the trap is less than the expected
phase-space volume of the cloud. We can estimate the spillover loss t be 73\%
on the phase-space volumes in
\myfigref{sim:fig:phasematchinggrid}. Performing the handover adiabatically
will maximise the number of molecules that can be transferred.

Next, $\mathrm{Z1_i}$ is adiabatically compressed to $\mathrm{Z1_f}$, with the
molecules held \SI{100}{\micro\meter} from the surface, as shown in row (b).
The particles will undergo filamentation, exploring all regions that are
energetically accessible and increasing the effective phase-space volume that
they occupy. This expected occupation region is calculated from the
energy-contour in phase-space that matches the temperature of the cloud, and is
marked by the dotted line. There is now handover into $\mathrm{Z2_i}$, with the
particles inside the phase-space acceptance contained in the new trap. We
expect spillover losses of 84\%. In the final stage the trap is compressed,
into $\mathrm{Z2_f}$, during which we expect no loss. The final occupation is
shown in \mysubfigref{sim:fig:phasematchinggrid}{c}.
%
Particle loss during the procedure can be estimated from the mismatching of
these phase-space emittances and acceptances.

\begin{figure}[tbhp]
\centering
  \scalebox{0.8}{\begin{overpic}[page=1]{figs/simsThesis/phase_matching_grid.pdf}
    \put(0,68){(a)}
    \put(0,44.5){(b)}
    \put(0,21){(c)}
  \end{overpic}}
  \caption[Mode-matching on-chip traps]{
    The acceptance (solid) and expected occupation (dashed) of each Z-wire
    trap. This is calculated by finding the acceptance in the weak ($x$)
    trapping direction. Row (a) shows $\mathrm{Z0_f}$, with occupation
    determined by the above simulation. Molecules will be adiabatically
    transferred to $\mathrm{Z1_i}$, which will be fully occupied. Some
    particles will be lost due to the decrease in acceptance but this is to be
    expected, and phase-space density should not increase. In (b) we have the
    acceptance and occupation of $\mathrm{Z1_f}$ following adiabatic
    compression.  Particles will be adiabatically transferred into
    $\mathrm{Z2_i}$, resulting in similar losses to the previous step. Row (c)
    shows the final acceptance and occupation of $\mathrm{Z2_f}$.
  }
  \label{sim:fig:phasematchinggrid}
\end{figure}


The above-described spillover losses can be combined with the loading
efficiency determined by the trajectory simulations. We expect a total loading
efficiency of 2\%. This would deliver only 50 molecules into the final trap in
what is a highly-idealised loading simulation. Notably the analysis of the
phase-space acceptance does not account for decrease in effective phase-space
density by filamenation, which will further reduce the final occupation. This
confirms that we will need to produce a much higher initial phase-space density
to load a sufficiently large number of molecules into the trap. However it is
clear that this procedure can be applied to much higher density clouds, such as
those which can be produced using optical dipole traps. Ongoing efforts to
produce higher density clouds of \CaF{} could improve this final molecule
number even further.

\section{Summary}

In this chapter I have presented the design of the chip experiment, starting
with the source of molecules up to the stage of confinement in a Z-trap
\SI{10}{\micro\meter} from the chip surface. I have presented
a loading procedure that will allow us to load this final trap through a series
of wire traps of decreasing size. This is justified by simulation and
phase-space analysis of the acceptance of the various trapping stages.
%
The design leaves scope for the addition of microwave guides on a second layer
above the trapping wires, as discussed in section~\ref{theory:chips}.
