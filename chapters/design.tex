This chapter begins with an overview of the chip experiment design, from
molecule source to the final chip trap. I will then give an overview of how we
can calculate and simulate the trajectories of trapped \CaF{} molecules. This
will be used to simulate and analyse the loading scheme for the chip.

\thesis{Need to direct to microwaves ch rather than do this...}
This chapter deals mainly with the trapping aspect of the experiment and not
the microwaves. The implementation of microwave guides on the
chip was discussed in the early stage assessment, and has not significantly
changed since then.

\section{Experiment overview}

\thesis{Will likely need more discussion of transport in the thesis. For one
thing it seems like it does cause a bit of heating for tweezers at the moment.}
%
The design of our experiment is largely motivated by the existing apparatus.
The ultracold molecule source described in \inlineref{Truppe2017} has been
adapted to allow for additional experiments to be carried out either in the
same chamber, or in neighbouring chambers. In the latter case, molecules are
transported between chambers by magnetic transport. This is achieved with coils
forming a quadrupole trap mounted on a
transport stage, similar to those described
in~\cite{Lewandowski2003}. Early experiments have shown that \Rb{} atoms can be
transferred with minimal losses and heating.

\thesis{This para is rubbish, but I cba to fix it rn}
%
Currently, there are two experiment chambers, the MOT chamber, where the
molecule MOT is originally formed, and the blue-detuned molasses are applied.
This chamber is also capable of hosting a \Rb{} MOT, which we will be able to
use to test the chip trap. The second chamber is used for experiments with
optical tweezers and dipole traps. Our molecules will be transported through
this chamber into the chip chamber for our experiment, as pictured in
\myfigref{design:fig:vacuumsystem}.

\thesis{Need to fix this fig to put the chip on the right side of the chamber. But
a top-down view of the whole experiment would probably be a better use of my
time.}

\begin{figure}[htb]
  \centering
  \begin{overpic}[width=0.7\textwidth]{figs/vacuum_setup.pdf}
    \put(1,-5){Chip chamber}
    \put(34,-5){Tweezer chamber}
    \put(73,-5){MOT chamber}
  \end{overpic}
  \vspace{1cm}
  \caption{
    The main vacuum chambers are shown. The experiment begins with a buffer gas
    source of \CaF{} molecules (not pictured). This produces a beam of
    molecules which is laser-slowed for capture in the MOT chamber. Here
    blue-detuned molasses are used to cool the molecules to
    \SI{50}{\milli\kelvin}. We can then use transport coils to transfer the
    molecule cloud to the other chambers. Our experiment is conducted in the
    most downstream chamber. The molecule cloud is brought into place directly
    beneath the chip flange (gold).
  }
  \label{design:fig:vacuumsystem}
\end{figure}

The chip trap will be positioned inside this chamber, held in place by a flange
assembly, with the chip surface \SI{4}{\milli\meter}
%
\thesis{be exact here, think it is 3.8??}
%
away from the transport axis. The flange assembly consists of current
feedthroughs and a PCB to deliver trapping currents to the chip, a large copper
heatsink, and microwave feedthroughs. It will be mounted so that the chip
faces the floor, allowing us to trap and then drop the molecules for absorption
imaging~\cite{Willliams2018}.

Once the molecules are brought into the chamber by the transport coils, they
will be magnetically transferred onto the chip trap. The transfer will take
the molecules through a series of magnetic traps of decreasing size until
they are in the smallest trap. The design of these traps and the loading
procedure are the main subjects of this chapter.

The first stage of the transfer is a large U wire embedded in the flange
assembly, and the rest  incorporated into the design of the chip itself (see
\mysubfigref{design:fig:chipexperiment}{b}).
\cm{Mike doesn't understand how the U wire will help align with the transport
coils. Perhaps this could be explained better than just waving away with a
citation.}
The U wire can carry a larger
current than any wires on the chip, and so is suitable to create a deep trap
which will prove useful for alignment with the transport coils~\cite{Ott2001}.
Its current is limited to \SI{100}{\ampere} by the vacuum feedthroughs. Since
the U is embedded in the flange \mt{To me, the flange means the vacuum flange.
But that's not where the U is.}, it is further away from the molecules cloud
than the wires on the chip. Since trap depth is related to the current (see
equations ~\ref{intro:eq:trapbias} and~\ref{intro:eq:trapdepth}, this is what
limits the depth of the trap.

\begin{figure}[ht]
  \centering
  \parbox{\textwidth}{
    \parbox{.5\textwidth}{%
\centering
      \subcaptionbox{}{
\includegraphics[width=0.4\textwidth]{figs/FlangeAssemblyForPresentationTopRight.png}
}
      \vskip3em
      \subcaptionbox{}{
\centering
\includegraphics[height=0.2\textwidth]{figs/fab/subchip1_cropped3.pdf}
}  
    }
    \hskip1em
    \parbox{.5\textwidth}{%
      \subcaptionbox{}{
  \begin{overpic}[width=\hsize,page=1]{figs/chip_des.pdf}
    \put(26,64){{\tiny\SI{100}{\micro\meter}}}
  \end{overpic}
}
    }
  }
  \caption{
    Details of the experiment aparatus. The chip flange (a) \mt{too small and
    needs labelling} hosts the subchip
    (detailed in b \mt{make larger and label} \cm{this fig also looks weird in
    evince?}) and chip (detailed in c) and provides power to the chip via
    feedthroughs. The subchip is a printed circuit board which will carry power
    to the chip. A hole will be milled so that the chip can be glued in place,
    flush with the surface. Electrical contact will be made by wirebonds. The
    subchip hosts three Z wires detailed in \mytableref{design:table:wires}.
    The inset in (c) shows the relative positioning of the wires.
%
    \mt{Figure could be improved to highlight more of the important details,
    e.g. where is the U, how are the microwaves and dc currents brought in etc.
    One helpful thing would be to make (a) large and add labelling to it so
    that readers can understand what they're looking at. Another helpful thing
    would be to show what's underneath the chip carrier.}
  }
  \label{design:fig:chipexperiment}
\end{figure}

Once loaded into the U, the molecules will be transferred through the Z wire
traps on the chip. Each trap must be designed to have a sufficient cross
section to carry the trapping current. In the case of the first Z wire,
the molecules are still \SI{3}{\milli\meter} away from the chip, a
current of order \SI{10}{\ampere} is required to form a suitable trap
%
\thesis{need to figure out what the actual depth is}.
%
We will discuss in chapter~\ref{fab} that the maximum wire height
achievable is \SI{5}{\micro\meter}, and we expect that the wires will be able
to carry a maximum current density of \SI{6E10}{\ampere\per\meter\squared},
as was found for a similar chip design in \inlineref{Treutlein2008}. The Z wire
will therefore have a width $w=\SI{200}{\micro\meter}$. Other wire parameters
are shown in \mytableref{design:table:wires}.

It is widely assumed in the literature and in this work that the trapping
currents are carried by wires which are infinitesimally small compared to the
length scale of the trap. We therefore suppose that the trapping height (the
distance of the cloud centre from the chip) must be much larger than the wire
width at all times.

The axial length  of the wires also decreases to gradually reduce the size of
the trapped cloud in the $x$ direction. The wire layout is shown in
\mysubfigref{design:fig:chipexperiment}{c} and more details along with the
heights are outlined in \mytableref{design:table:wires}. All wires have
been designed to carry twice the current that is required in the loading
scheme.

\thesis{Add bias fields required to this table (as a ramp, like trap height)
maybe just they y bias?? Also CONFIRM THE AXIS LENGTHS.}
%
\begin{table}
  \centering
\begin{tabular}{lrrrrr}
  Name & Axis length (\si{\milli\meter}) & Width (\si{\micro\meter})& Height
  (\si{\micro\meter})& $I_\text{max}$ & Trap height (\si{\micro\meter}) \\
 \hline
  U & 16 & N/A& N/A& 100 & 3000\\
  $\mathrm{Z0}$ & 12 & 200&  5& 60& $3000\rightarrow1000$ \\
  $\mathrm{Z1}$ &  6 & 20&  5& 6& $1000\rightarrow100$ \\
  $\mathrm{Z2}$ &  3 & 2&  5& 0.6& $100\rightarrow10$ \\
 \hline
\end{tabular}
  \caption{Details on the wire dimensions, maximum current, and desired
  trapping heights. The wire design is shown in
  \mysubfigref{design:fig:chipexperiment}{c}. Note that the U wire current is
  limited by vacuum feedthroughs and not by the wire dimensions or cooling.
  The maximum currents have been designed for use at only 50\% of their
  potential maximum ($I_\text{max}$).
  }
  \label{design:table:wires}
\end{table}

Each Z trap will begin trapping at one height before the bias field is
increased to allow for trapping closer to the surface (as per
\myeqref{intro:eq:trapbias}).  To distinguish between the two trap stages for
each wire, we label them $\mathrm{ZX_i}$ for the initial (higher) trap and
$\mathrm{ZX_f}$ for the final (lower) trap, with $\mathrm{ZX}$ corresponding to
the wire labels in \mytableref{design:table:wires}.

We aim to eventually incorporate microwave guides onto the chip. This can be
accomplished by adding an insulating layer on top of the wires, on to which we
can fabricate coplanar waveguides~\cite{1127105}. The flange and subchip have
been designed to allow delivery of microwaves to the chip. \thesis{Direct to
microwaves on a chip chapter.} This was discussed in detail in the early stage
assessment and will not be repeated here.

In the rest of this chapter, I will demonstrate through simulation and analysis
of the trapping potentials that this design will be capable of loading trapped
\CaF{} molecules.

\section{Motion of molecules in a potential}
\label{design:motion}

It is important to review the physics behind the motion of the molecules in the
potentials, which we shall do in this section. We will also briefly describe
the simulation methods.

\subsection{Motion in the traps}
\label{desgin:motion}

We can assume that the motion of the molecules in the trap is classical. They
move in the potential $V(t, \mathbf{q}) = \mu B(t, \mathbf{q})$, where $\mu$ is
the magnetic dipole moment of the molecule.  The motion of any one particle is
described by Hamilton's equations,~\cite{Lichtenberg1969}
%
\begin{align}
  \label{design:eq:hamilton}
  \dot{\mathbf{q}} =  \frac{\partial H}{\partial \mathbf{p}} &&
  \dot{\mathbf{p}} = -\frac{\partial H}{\partial \mathbf{q}},
\end{align}
%
where $H$ is the classical Hamiltonian of the system
\begin{equation}
  %
  H(t, \mathbf{q}, \mathbf{p}) = \frac{\mathbf{p}^2}{2m} + V(t, \mathbf{q}).
\end{equation}
For now we neglect the time dependence of the potential, so that $V(t,
\mathbf{q}) = V(\mathbf{q})$.

Solving Hamilton's equations tells us the position and momentum of a single
particle. Taken together, these two vectors describe a point in a six
dimensional \emph{phase space} of position and momentum. The ensemble of
trapped molecules, each one starting at a different point in phase space, can
be treated by considering the motion of the molecules at its
boundary, and the \thesis{hyphen?} six-dimensional phase space volume that this
boundary encloses~\cite{Hand1998}.

It is helpful for us to write the phase space volume $V$ in terms of the
spatial volume occupied by the ensemble ($V_\text{space}$) and its temperature.
The length-scale for a monatomic gas of temperature $T$ is the thermal de
Broglie wavelength~\cite{blundell2}
%
\begin{equation}
  \lambda_\text{dB}(T) = \frac{h}{\sqrt{2 \pi m k_B T}}.
\end{equation}
And hence the unitless phase space volume is \cm{Think I need to figure out
what the deal is with the unit. Mike seems to think there should be a $h^3$ in
here, but I think for there to be no units this is better.}
%
\begin{equation}
  V = V_\text{space} / \lambda_\text{dB}^3.
\end{equation}
%
It is also useful to define a unitless phase space
density~\cite{PhysRevA.52.1423}
%
\begin{equation}
  \rho = \frac{N}{V} = \frac{N \lambda_\text{dB}^3}{V_\text{space}}.
\end{equation}
%
For a given cloud with phase space density $\rho$, trap with volume
$V_\text{trap}$ and depth $T_\text{depth}$,
the maximum possible number of particles that can be trapped is $\rho
V_\text{trap}/\lambda_\text{dB}^3(T_\text{depth})$.

Furthermore, Liouville's theorem~\cite{Landau1982, Hand1998} states that the
phase space volume is a conserved quantity, as long as the trapping potential
is consevative. This means that the phase space density of a trapped molecular
cloud cannot be increased without application of some velocity-dependent
damping, such as an optical molasses~\cite{Metcalf1999}. The impact of this for the chip
is that the phase space density of the cloud at the point of loading onto the
chip determines the maximum number of molecules that can be trapped in the
final trap.


If we do not introduce any further cooling steps, then the number of molecules
we are able to trap in the final trap will be
%
\begin{equation}
  N_\text{final} = \frac{\rho_\text{\CaF{}}V_\text{trap}}
  {\lambda_\text{trap}^3} = \frac{\rho_\text{\CaF{}} V_\text{trap}(2 \pi m_\text{CaF} k_B
  T_\text{trap})^\frac{3}{2}}{h^3}
  \label{design:eq:psd_N}
\end{equation}
%
where $\rho_\text{\CaF{}} = 3.4\times10^{-12}$ is the phase space density of the \CaF{} cloud
after cooling in the blue-detuned molasses~\cite{Truppe2017},
$V_\text{trap}\sim(\SI{10}{\micro\meter})^3$ and
$T_\text{trap}=\SI{4}{\kelvin}$. The trap depth used here is for the smallest
trap operating at \SI{3}{\ampere}, and can be calculated using
equation~\ref{intro:eq:trapdepth}. We therfore have
%
\begin{equation}
  N_\text{final} \sim 3000.
\end{equation}
%
This upper bound for the number of trapped molecules is calculated with the
assumption that all the trapping stages are deep enough to contain all the
molecules that are within the trapping region. 
%
\cm{As Mike points out, this is clearly not correct. I have also used the phase
space density for the 50uK experiment, so probably off by a factor of
$10^{3/2}$ (need to check that also...) Alex also pointed out that it is hard
to calculate this because not all the constants are available/ obvious, e.g.
the mass. So just need to make this paragraph a lot more clear and put some
care into it.}

This motivates us to introduce the concept of the acceptance of a trap.
This is the phase space volume within which all particles are trapped.
Particles outside of the acceptance are either positioned outside the trap or
are positioned inside the trap but have enough energy to escape it. The
boundary of the acceptance is called the separatrix. An example is shown
in \myfigref{design:fig:psaeg}~\cite{Lichtenberg1969, Hand1998}.

\thesis{Try to recreate this fig myself with my code!}
%
\begin{figure}
  \centering
  \includegraphics[width=0.6\textwidth, page=1]{figs/psa_eg.png}
  \caption{
    Phase space acceptance of a harmonic (left) and anharmonic (right) trap in
    one dimension. The separatrix is shown in both cases. The initial particle
    positions in phase space are shown in the top row, with the accepted
    particles in black and the unaccepted particles in grey. In the anharmonic
    case, the particles expand to fill the acceptance, increasing the effective
    phase space density (a process called filamentation). This figure is
    reproduced from \inlineref{Crompvoets2005}.
  }
  \label{design:fig:psaeg}
\end{figure}

\thesis{I should discuss filamentation in more detail}

An adiabatic change to the potential will not affect the phase space density of
any particles that remain trapped throughout the change~\cite{Hand1998,
Lichtenberg1969}. Hence a cloud of trapped particles can be translated, for
example in the transport coils, or towards the surface of the chip as will be
discussed in section~\ref{design:sim}.

It should be noted that the Liouville theorem also holds for a time-dependent
Hamiltonian. We will see below that the trapping potential will be changed
adiabatically, and make use of the fact that phase space density is conserved
through this process.

\subsection{Simulating the motion}
\label{design:motion:simmethods}

\thesis{When can we ignore VdW forces?}

The particle motion can be simulated by numerically solving
\myeqref{design:eq:hamilton}. This was done on Imperial College London's high
performance computing services~\cite{ICRCS} using Python~\cite{python} and the
symplectic Euler method provided by the Desolver package~\cite{desolver}.
Symplectic methods are powerful tools for numerically solving Hamiltonian
systems whilst conserving energy and momentum \cite{Hairer2015,
doi:10.1119/1.2034523}. A time step of $\dd t = \SI{1E-4}{\second}$ was
sufficient to achieve this for our potentials.

Magnetic potentials were calculated by considering the traps to be formed of
segments of straight wires, each producing a magnetic field
$\mathbf{B}_\text{seg}^{(i)}$, and the bias field $\mathbf{B}_\text{bias}(t)$. The total
field is the sum of contributions from the set of all segments (S), and the
bias. The potential is therefore
%
\begin{equation} V(t, \mathbf{q}) = \mu B (t, \mathbf{q}) = \mu \left|
  \sum_{i\in S}
  \mathbf{B}_\text{seg}^{(i)}(t, \mathbf{q}) + \mathbf{B}_\text{bias}(t)\right|.
\end{equation}

The field due to a segment of straight wire with length carrying current $I(t)$
can be found analytically by integration of the Bio-Savart Law.
This field can be expressed in terms of the angle between the ends of the wire
at $\mathbf{q}$, as is shown in \myfigref{design:fig:wiresegment}. The field
is given by~\cite{Griffiths2017}
%
\begin{equation}
  \mathbf{B}_\text{seg}(t, \mathbf{q}) = \frac{\mu_0 I(t)}{4\pi
  s_\text{seg}(\mathbf{q})} (\sin(\theta_2)  -
  \sin(\theta_1))\mathbf{\mathbf{\phi}}
\label{design:eq:segmentfield}
\end{equation}
here $s_\text{seg}(\mathbf{q})$ is the shortest distance between the segment
and $\mathbf{q}$, and $\hat{\mathbf{\phi}}$ is the unit vector pointing
perpendicular to $\mathbf{q}$ and the wire segment. \cm{Should take more care
indefining this unit vector! (also formatting mathbf and hat?)}

\begin{figure}[h]
\centering
  \begin{tikzpicture}
    % Def coords
    \coordinate (O) at (0, 0);
    \coordinate (L) at (-3, 0);
    \coordinate (R) at (3, 0);
    \coordinate (Q) at (1, 4);
    % Draw lines
    \draw[line width=0.75mm, ->] (L) -- (R);
    \draw (L) -- (Q);
    \draw (R) -- (Q);
    \draw[<->, densely dotted,shorten >=.5mm,shorten <=.8mm] (Q) -- (1, 0);
    % Draw dot at Q
    \node at (Q)[circle,fill,inner sep=.5mm]{};
    % Draw angels
    \draw pic[draw,angle radius=1cm,"$\theta_1$" shift={(7mm,2mm)}] {angle=R--L--Q};
    \draw pic[draw,angle radius=1cm,"$\theta_2$" shift={(-7mm,2mm)}] {angle=Q--R--L};
    % Label
    \node[shift={(4mm,0)}] at (Q) {$\mathbf{q}$};
    \node[shift={(2mm, 0)}] at (R) {$I$};
    \node[fill=white] at (1., 1.8) {$s_\text{seg}(\mathbf{q})$};
  \end{tikzpicture}
  \caption{Geometry of a wire segment (bold) carrying current $I$, whose field
  can be calculated using \myeqref{design:eq:segmentfield}. The dottend line
  shows $s_\text{seg}(\mathbf{q})$, the shortest distance from the point at
  which the field is calculated ($\mathbf{q}$) to the segment.
  }
  \label{design:fig:wiresegment}
\end{figure}

This is everything that is needed to simulate a wire trap, or the handover
between two traps (by including the wire segments of both traps in the field
calculation). The transport trap is assumed to be a perfect quadrupole.

Molecule trajectories are calculated up to a pre-determined end time, or until
they are determined to be lost. In the following simulations a molecule is lost
when it's displacement from the origin in any cardinal direction is greater
than \SI{10}{\milli\meter}.

\subsection{Example simulation: Z trap acceptance}

To illustrate both the simulation methods and the example of phase space
acceptance given in section~\ref{design:motion:simmethods}, we present an
example simulation where a uniform distribution of molecules is spread over the
acceptance of the $\mathrm{Z0_i}$ trap used in our experiment. We should expect
that the molecules inside the acceptance remain trapped for arbitrarty long
time, and those outside are lost.

This is indeed what we see in \myfigref{design:fig:acceptance}, where the
results of the simulation are shown in a trio of plots, each one showing a
projection of phase space into the $(q_i, v_i)$ plane. Here $i\in\{x, y, z\}$
and we have converted from units of momentum to velocity for convenience. 

\cm{There was some confusion in the LSR because I said $10^6$ here, but I
checked and it was $10^5$.}
We simulated $10^5$ \CaF{} molecules, which are initialised with position and
velocity drawn from a uniform distribution inside the range indicated by the
gold rectangle in \myfigref{design:fig:acceptance}. The trajectories are then
integrated for the first \SI{300}{\milli\second} of the motion. The starting
position of those molecules who remained trapped are then plotted.

The separatrix is also shown. It is found by considering the energy at all
points in the phase space plane. The highest energy that forms a closed contour
is the acceptance of the trap. The energy is different in every plane (the $x$
direction is a weaker trapping direction).

\thesis{This sounds like a pretty weak excuse. I better figure out what is
actually going on here. Suspect I am underestimating the size of the
acceptance.}
There are 443 particles that remain trapped throughout the simulation. \mt{This
result tells you the actual phase space acceptance of the trap - it will be
443/10^{6} * initial phase space volume. It would be interesting to compare
this to the phase space volume you calculate analytically.} Of those
that are trapped, the vast majority fall within the calculated acceptance.
However in some edge cases it appears that the particles are outside. This is
due to the acceptance shown being a contour on a cut-through of the potential.
The true acceptance is a hypersurface that cannot easily be displayed in print.
\cm{I am kind of bullshitting here. Mike says the following, which seems more
reasonable: I can see how this results in particles inside the separatrix being
lost. But I can't see how it results in particles outside the separatrix not
being lost. I think more likely is that particles slightly outside the
separatrix are on metastable trajectories meaning they would be lost if you ran
the simulation longer.}

\thesis{Check that the separatrix is calculated sensibly, is it the same in
every direction?}
%
\begin{figure}
  \centering
  \includegraphics[page=1]{figs/sims/example_acceptance.pdf}
  \caption{Example simulation showing the acceptance of the $\mathrm{Z0_i}$.
  The region marked by the gold rectangles is populated with $10^5$ particles.
  The starting positions of the 443 particles that are trapped are marked with a pink dot.
  The acceptance is approximated by a contour of the trapping potential, shown
  here in black. Trapping is weaker in the $x$ direction than the other
  directions, so this contour has a lower value. The majority of the trapped
  particles start within this approximation of the acceptance.}
  \label{design:fig:acceptance}
\end{figure}

As well as demonstrating that the simulation is able to reproduce the expected
results from a simple phase space acceptance problem, we can use it to check
that constants of the motion are indeed conserved. This is shown to be the case
in \myfigref{design:fig:conservation}.  The fluctuations in the values are a
side effect of the symplectic method, but there is no overall trend for the
constants to increase or decrease \cite{doi:10.1119/1.2034523}. The phase
space density is significantly lower due to the nmber of simulated particles
being smaller than we would expect in reality, and the phase space volume of
the initial cloud being significantly larger.

\begin{figure}
  \centering
  \includegraphics[page=1]{figs/sims/example_conservation.pdf}
  \caption{Conservation of energy and phase space density in our simulation.
  Energy is the total energy of the trapped particles, but is given in units of
  \si{\milli\kelvin} for convenience. \cm{Needs to be more clear that this is
  total energy (all particles). Mike was unclear on this.}
  }
  \label{design:fig:conservation}
\end{figure}

\subsection{Simulation initialisation}

For other simulations it is useful to initialise the simulation to replicate
the conditions in the transport trap just as we would have at the beginning of
the loading procedure.

This can be done by initialising a cloud of $N$ molecules whose positions are
normally distributed in all three spatial dimensions with a standard deviation
of \SI{1}{\milli\meter}. Similarly the velocity components are normally
distributed with a standard deviation of \SI{84}{\milli\meter\per\second}
(corresponding to a temperature of \SI{50}{\micro\kelvin}). This is what we
would expect for the cloud after leaving the molasses.

The cloud is immediately placed in a quadrupole trap with gradient
\SI{10}{\gauss\per\centi\meter}. They are held for \SI{50}{\milli\second}
before the gradient is ramped up over \SI{100}{\milli\second} to the actual
transport gradient of \SI{61}{\gauss\per\centi\meter}. This causes the
molecules to compress as they would when initially loaded into the transport
coils in the MOT chamber.

For the purposes of these simulations, the effects of heating in loading
\mt{Not sure which effects you mean here.} \cm{think this is quite vague. we
must get some heating by transporting them, I think that is what I want to
refer to...} are
ignored. The molecules are therefore held for a further \SI{50}{\milli\second}
until $t=0$. It is possible for a very small number of particles to stray too
far from the centre of the trap during initialisation, at which point they are
considered to be lost. There may therefore be a difference between the number
of particles that are initialised and the number present at $t=0$.

\section{Modelling the loading procedure}
\label{design:sim}

To ensure that the chip design will allow for robust loading, the loading
process has been simulated up to the $\mathrm{Z0_f}$ trap. This demonstrates
the two main types of transitions in the loading process: handovers between
traps, and compression within a single trap. The timing of the simulation do
not exactly reflect those that will be used in practice, long holding periods
have been inserted between ramps to better illustrate the cause of particle
loss.

The results of the simulation and the parameters of the ramps are outlined in
\myfigref{design:fig:simparams}. This can be considered in four parts, first
from $t = 0$ to $t=\SI{100}{\milli\second}$ there is a settling period where
the molecules are held in the transport trap. Then there are three
\SI{100}{\milli\second} ramps, each of which is followed by a
\SI{100}{\milli\second} stabilisation period. These three ramps describe the
handover from the transport trap to the U, the U to $\mathrm{Z0_i}$, and the
compression of $\mathrm{Z0_i}$ to $\mathrm{Z0_F}$. They are described in detail
in the following subsections. This simulation used $N=1000$ particles, with one
particle lost during initialisation.

\begin{figure}[htb]
\centering
  \begin{overpic}[page=1]{figs/sims/mtt_u_z_params.pdf}
    \put(32.5,55){(a)}
    \put(45.5,55){(b)}
    \put(63,55){(c)}
    \put(80.5,55){(d)}
  \end{overpic}
  \caption{
    The simulation parameters are shown for loading from the transport trap
    into trap $\mathrm{Z0_f}$. The simulation begins with a stabilisation
    period (a), in which the molecules are held in the transport trap for \SI{100}{\milli\second}. In period (b) the transport
    trap is ramped off and the U is ramped on. There is some heating of
    molecules as they leave the trap, and a reduction in phase
    space density. The transfer to $\mathrm{Z0_i}$ occurs in period (c), where
    more loss occurs due to the trap depth being reduced in the $x$
    direction. Finally in (d) the trap is compressed to $\mathrm{Z0_f}$, when
    the molecules are brought from \SI{3}{\milli\meter} to \SI{1}{\milli\meter}
    of the chip surface. The solid lines show the simulation results when
    collisions with the chip surface are ignored, and the dashed line shows
    results when collision with the surface results in loss of the molecule.
    %
   \mt{Would be useful to show some quantity that measures the size of the
   cloud, along with all the other parameters, e.g. you could plot $\sigma_x 
   \sigma_y  \sigma_z$ as a measure of the volume of the cloud. Then, I would
   expect to see the volume go down in places where the temperature goes up.}
  }
  \label{design:fig:simparams}
\end{figure}

\subsection{Handover to the U trap}
\label{design:sim:trans_U}


\thesis{Ensure language is consistent when talking about chip, so is the U
above or below?}
%
The first step is to transfer the molecules from the transport coils to the U
wire. This U wire has the largest axial length, and so will be the easiest trap
the transport coils to. The U-wire is also capable of carrying more current
than any of the other wires, which is essential as it is positioned further
from the molecule cloud, beneath the chip.

Between $t=\SI{100}{\milli\second}$ and $t=\SI{200}{\milli\second}$ the current
in the transport coils is adiabtically ramped down while the currents in
the U trap and in the bias coils are ramped up. The molecules are then held
until $t=\SI{300}{\milli\second}$ to ensure that they remain trapped after the
ramp. The simulation parameters are shown in
\mysubfigref{design:fig:simparams}{b}.

The particle loss during the ramp is approximately 10\% of the total. There is
also some heating, which decreases the phase space density of the cloud. The
cause of this can be seen in \myfigref{design:fig:trans_U}, which shows the
phase-space positions of the molecules at various times throughout the ramp,
and the trapping potential at those times.

\begin{figure}[p]
\centering
  %\setlength\extrarowheight{-3pt}
  \begin{tabular}{c}
    \includegraphics[page=1]{figs/sims/mtt_u_x.pdf} \\
    \includegraphics[page=1,trim=0 0 0 15, clip]{figs/sims/mtt_u_y.pdf} \\
    \includegraphics[page=1,trim=0 0 0 15, clip]{figs/sims/mtt_u_z.pdf}
  \end{tabular}
  \caption{
    For each dimension, a cut-through of the potential and a plot of
    phase-space positions of the particles is shown. Each column shows a
    different time during the ramp from the transport trap into the U trap. The
    cut-throughs pass through $\mathbf{t}_0 = (0, 0, \SI{3}{\milli\meter})$. In
    the $x$ direction, it is possible to see decompression and recompression as
    the potential varies from $t=\SI{125}{\milli\second}$ to
    $t=\SI{200}{\milli\second}$. In the $y$ direction, the depth is lowered due
    to the superposition of the trapping fields, leading to a significant loss.
    This is partially visible at $t=\SI{175}{\milli\second}$, where we can see
    the drop in the potential and particles with $y<0$, $v_y<0$ escaping the
    trap. In the $z$ direction, some hotter molecules escape as the U trap
    reaches its final depth at $t=\SI{200}{\milli\second}$. Note that this loss
    has ended after the settling time, which can be seen in
    \myfigref{design:fig:U_Z0i}.
  }
  \label{design:fig:trans_U}
\end{figure}

At $t=\SI{175}{\milli\second}$ there is a noticeable decrease in the trapping
depth in the $y$ direction. This is the cause of the particle leak, and occurs
due to the due to the superposition of the two fields causing an overall
decrease in the trap depth in the $x$ and $y$ directions. This only occurs for
a brief period during the ramp (lasting approximately \SI{40}{\milli\second}). 

In addition to the loss during the ramp, we observe a steady loss between
$t=\SI{200}{\milli\second}$ and $t=\SI{300}{\milli\second}$, but only when the
simulation accounts for collision with the chip surface. This loss due to
collision occurs in the U wire because of its position beneath the chip. This
means that the trap is not sufficiently deep to hold our molecules for an
extended period. However the holding time in this simulation is only for
illustration, and the prolonged trapping in the U trap will not be part of the
final experiment. Hence these losses can be ignored.

\subsection{Handover to $\mathrm{Z0_i}$}
\label{design:sim:U_to_Z0i}

The U to $\mathrm{Z0_i}$ handover occurs in our simulation from
$t=\SI{300}{\milli\second}$ to $t=\SI{400}{\milli\second}$, shown in
\mysubfigref{design:fig:simparams}{c}.  The phase space positions of the
molecules and the potentials are shown in \myfigref{design:fig:U_Z0i}.

\begin{figure}[p]
\centering
  %\setlength\extrarowheight{-3pt}
  \begin{tabular}{c}
    \includegraphics[page=1]{figs/sims/u_Z0i_x.pdf} \\
    \includegraphics[page=1,trim=0 0 0 15, clip]{figs/sims/u_Z0i_y.pdf} \\
    \includegraphics[page=1,trim=0 0 0 15, clip]{figs/sims/u_Z0i_z.pdf}
  \end{tabular}
  \caption{
    The same simulation results are displayed as in
    \myfigref{design:fig:trans_U}, this time for the handover to $\mathrm{Z0_i}$.
    The superposition of the U and Z traps causes a reduction of the trap depth
    in the $x$ direction, as seen at time $t=\SI{375}{\milli\second}$.
    Molecules can be seen coming into contact with the chip surface when held
    in the U trap, but after transfer to the Z trap, which is located on the
    chip surface, the cloud is strongly repelled from the surface.
  }
  \label{design:fig:U_Z0i}
\end{figure}

In the ramp between the transport trap and the U trap, we saw that a hole was
introduced in the trapping potential due to interference between the two
trapping fields. We will see a similar effect when transferring from the U trap
to trap $\mathrm{Z0_i}$.  This time the reduction in depth appears in the $x$
direction, and is due to the current in the U and Z wires opposing each other
on the positive $x$ side. This can be seen in \myfigref{design:fig:U_Z0i} at
$t=\SI{375}{\milli\second}$.

In \myfigref{design:fig:simparams} we see that the loss is again about 10\% of
the particles, however there is no significant change in the phase space
density. The acceptance of trap $\mathrm{Z0_i}$ is smaller than the emittance
of the U, so this is to be expected.

\subsection{Compression to $\mathrm{Z0_f}$}

In the final stage of the simulation we look at a compression stage. The
trapping potential is adiabatically altered to bring the field minimum closer
to the chip surface. This is achieved by increasing the bias field, but since
this also increases the trap depth the cloud is simultaneously compressed as it
is heated. Since the phase space density is conserved, the particles heat up in
the trap. This phenomenon is made visible in
\mysubfigref{design:fig:simparams}{d}.

The trajectories and potentials are shown in \myfigref{design:fig:Z0i_Z0f}.
Here the compression in number density and expansion in velocity space is clear
by the changing cloud shape over time. The cloud also clearly follows the
potential minimum without losses.

\thesis{In these figures state the trapping heights $z_0$}
%
\begin{figure}[p]
\centering
  \begin{tabular}{c}
    \includegraphics[page=1]{figs/sims/Z0i_Z0f_x.pdf} \\
    \includegraphics[page=1,trim=0 0 0 15, clip]{figs/sims/Z0i_Z0f_y.pdf} \\
    \includegraphics[page=1,trim=0 0 0 15, clip]{figs/sims/Z0i_Z0f_z.pdf}
  \end{tabular}
  \caption{Presentation of the loading simulation is continued from
  \myfigref{design:fig:trans_U} and \myfigref{design:fig:U_Z0i}. Here we see
  the compression from $\mathrm{Z0_i}$ to $\mathrm{Z0_f}$. The compression in
  the spatial dimensions causes heating, which can be seen as a spread in the
  velocity directions. This is in line with what is expected from Liouville's
  theorem.
  }
  \label{design:fig:Z0i_Z0f}
\end{figure}

\subsection{Transfer between Z traps}
\label{design:transferbetweenzs}

Once trapped in $\mathrm{Z0_f}$, the remaining loading stages are simply
repeated handovers to the smaller wires, with a compression stage carried out
on each wire to bring the molecules closer to the surface. This is a
well-understood and robust procedure that has been used before to load atom
chips with minimal losses~\cite{Reichel2002}.

That said, since our experiment will operate with a significantly lower phase
space density than previous experiments with atoms, it is important to ensure
that our loading procedure will not suffer from any unnecessary losses. These
compressed traps contain hotter molecules, requiring a smaller time step for
simulation. It is therefore easier to directly examine the acceptance of each
trap.

These are shown in \myfigref{design:fig:phasematchinggrid}, starting with
$\mathrm{Z0_f}$ in row (a). The region of the trap that we expect to be
occupied (as determined by the simulation above) is marked with a dashed line.
This cloud of molecules will be adiabatically transferred into $\mathrm{Z1_i}$.

Since the process is adiabatic, we can anticipate that there will be no
reduction in phase space density. The cloud of molecules will be transferred to
$\mathrm{Z1_i}$ with negligible heating. However, the total acceptance of the
trap is less than the expected phase space volume of the cloud, so there
will be particle loss due to spillover.

This was anticipated in section~\ref{design:motion}: phase space density cannot
increase in a conservative potential, and the acceptance of the smallest trap
will be smaller than the initial cloud in the transport trap. So the final
number of particles will be lower than we started with, as calculated in
\myeqref{design:eq:psd_N}.

Next, $\mathrm{Z1_i}$ is adiabatically compressed to $\mathrm{Z1_f}$, with the
molecules held \SI{100}{\micro\meter} from the surface. Again, this
compression is adiabatic, so we expect no change in the phase space density.
Compression increases the trap acceptance, so there is also no loss of
particles. This is further supported by the simulated compression of
$\mathrm{Z0_i}$ to $\mathrm{Z0_f}$ discussed above. The acceptance of
$\mathrm{Z1_I}$ is shown in \mysubfigref{design:fig:phasematchinggrid}{b}, with
the expected occupation shown by the dashed line.

Finally, $\mathrm{Z2}$ is loaded and compressed in exactly the same way as
$\mathrm{Z1}$. The adiabatic handover from $\mathrm{Z1_f}$ to $\mathrm{Z2_i}$
is the final cause of particle loss (due to spillover). Then the trap is
compressed to hold the particles \SI{10}{\micro\meter} from the surface, as
shown in \mysubfigref{design:fig:phasematchinggrid}{c}.


\begin{figure}[htb]
\centering
  \begin{overpic}[page=1]{figs/sims/phase_matching_grid.pdf}
    \put(0,68){(a)}
    \put(0,44.5){(b)}
    \put(0,21){(c)}
  \end{overpic}
  \caption{
    A low bound of the acceptance (solid) and expected occupation (dashed) of
    each Z wire trap. This is calculated by finding the acceptance in the weak
    ($x$) trapping direction. Row (a) shows $\mathrm{Z0_f}$, with occupation
    determined by the above simulation. Molecules will be adiabatically
    transferred to $\mathrm{Z1_i}$, which will be fully occupied. Some
    particles will be lost due to the decrease in acceptance but this is to be
    expected, and phase space density should not increase. In (b) we have the
    acceptance and occupation of $\mathrm{Z1_f}$ following adiabatic
    compression.  Particles will be adiabatically transferred into
    $\mathrm{Z2_i}$, resulting in similar losses to the previous step.  Row (c)
    shows the final acceptance and occupation of $\mathrm{Z2_f}$.
  }
  \label{design:fig:phasematchinggrid}
\end{figure}


\section{Summary}

In this chapter I have presented the design of the chip experiment, starting
with the source of molecules up to the stage of confinement in a Z trap
\SI{10}{\micro\meter} from the chip surface. I have shown that we have created
a loading procedure that will allow us to load this final trap through a series
of wire traps of decreasing size. This is justified by simulation and visual
analysis of the acceptance of the various trapping stages.

The design leaves scope for the addition of microwave guides on a second layer
above the trapping wires, as discussed in chapter~\ref{intro}.
