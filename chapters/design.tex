This chapter begins with an overview of the chip experiment design from
molecule source to the final chip trap. I will then give an overview of how we
can calculate and simulate the trajectories of trapped \CaF{} molecules. This
will be used to simulate and visually analyse the loading scheme for the chip.

\section{Experiment overview}

The design of our experiment is largely motivated by the existing apparatus.
The ultracold molecule source described in \inlineref{Truppe2017} has been
adapted to allow for additional experiments to be carried out either in the
same chamber, or in neighbouring chambers. In the latter case, molecules are
transported between chambers by magnetic transport, such as is described
in~\cite{}. Early experiments have shown that \Rb{} atoms can be transferred
with minimal losses and heating.

Currently, there are two experiment chambers, the MOT chamber (where the
molasses is applied) and a second chamber for a tweezer experiment. Our chip
chamber will be the third chamber, and will be positioned downstream of the
tweezer chamber as pictured in \myfigref{design:fig:vacuumsystem}.

\begin{figure}[htb]
  \centering
  \begin{overpic}[width=0.7\textwidth]{figs/vacuum_setup.pdf}
    \put(1,-5){Chip chamber}
    \put(34,-5){Tweezer chamber}
    \put(73,-5){MOT chamber}
  \end{overpic}
  \vspace{1cm}
  \caption{
    The main vacuum chambers are shown. The experiment begins with a buffer gas
    source of \CaF{} molecules (not pictured). This produces a beam of
    molecules which is laser-slowed for capture in the MOT chamber. Here
    blue-detuned molasses are used to cool the molecules to
    \SI{50}{\milli\kelvin}. We can then use transport coils to transfer the
    molecule cloud to the other chambers. Our experiment is conducted in the
    most downstream chamber. The molecule cloud is brought into place directly
    beneath the chip flange (gold).
  }
  \label{design:fig:vacuumsystem}
\end{figure}

Since the molecules are to be brought into the chamber by the transport coils,
they will then be magnetically transferred onto the chip trap. This requires us
to design the chip such that loading can occur with the minimisation of loss
and heating as the magnetic potentials are varied. The main tools we use for
this are a large U wire positioned away from the chip surface. This can carry a
larger current than any wires on the chip, and so is suitable to create a deep
trap which will prove useful for alignment of the transport
coils.~\cite{Ott2001}

This U wire is embedded in the chip flange, shown in
\mysubfigref{design:fig:chipexperiment}{a}. It hosts the U wire, and the subchip, a PCB
into which a groove is milled for the chip to sit. This is shown in
\mysubfigref{design:fig:chipexperiment}{b} This will serve as power and
eventually microwave delivery to the chip. The flange also incorporates a large
copper heatsink.

\begin{figure}
  \centering
  \ph{chip flange render, subchip view, chip wire layout cartoon}
  \caption{\ph{TODO}}
  \label{design:fig:chipexperiment}
\end{figure}

Once loaded into the U, the molecules will be transferred between a series of
carefully designed traps microfabricated on the chip. The wire layout is shown
in \mysubfigref{design:fig:chipexperiment}{c} and more details along with the
loading ramps are outlined in table \mytableref{design:table:wires}.

In the rest of this chapter, I will present the analysis that has lead us to
this design for our experiment and show that it will be capable of robust
loading of \CaF{} molecules.

\section{Motion of molecules in a potential}

In section ~\ref{design:sim} we will use simulation and visual analysis of the
trapping potentials to show that our chip design and loading procedure are
suitable for transfering molecules into the chip trap. Before we do this it is
important to review the physics behind the motion of the molecules in the
potentials, which we shall do in this section. We will also briefly describe
the simulation methods.

\subsection{Hamiltonian mechanics}

We can assume that the molecules motion in the trap is classical. They move in
the potential $V(t, \mathbf{q}) = \mu B(t, \mathbf{q})$, where $\mu$ is the
magnetic dipole moment of the molecule.  The motion of any one particle is
described by Hamilton's equations,~\cite{Lichtenberg1969}
%
\begin{align}
  \label{design:eq:hamilton}
  \dot{\mathbf{q}} =  \frac{\partial H}{\partial \mathbf{p}} &&
  \dot{\mathbf{p}} = -\frac{\partial H}{\partial \mathbf{q}},
\end{align}
%
where $H$ is the classical Hamiltonian of the system
\begin{equation}
  %
  H(t, \mathbf{q}, \mathbf{p}) = \frac{\mathbf{p}^2}{2m} + V(t, \mathbf{q}).
\end{equation}
For now we neglect the time dependence of the potential, so that $V(t,
\mathrm{q}) = V(\mathrm{q})$ is conservative, and we assume that all particles
are trapped.

Solving Hamilton's equations tells us the position and momentum of a single
particle. Taken together, these two vectors describe a point in a six
dimensional \emph{phase space} of position and momentum. The ensemble of
trapped molecules, each one starting at a different point in phase space, can
be treated by considering the motion of the molecules at its
boundary.~\cite{Hand1998}

It turns out that the volume occupied by the ensemble is a conserved quantity.
This is Liouville's theorem~\cite{Landau1982}, it can be proved by considering
the change in the volume near some infinitesimal element of the boundary which
we label $\dd \mathrm{S}$. At that point on the surface, the velocity of nearby
particles is taken to be $\mathbf{v}$, so the change in volume at an instant in
time is~\cite{Hand1998}
%
\begin{equation}
  \dd V = \mathbf{v} \cdot \dd \mathbf{S}.
\end{equation}
%
Hence the overall change in velocity is
%
\begin{equation}
  \frac{\dd V}{\dd t} = \int_S \mathbf{v} \cdot \dd \mathbf{S} = \int_V \nabla 
  \cdot \mathbf{v} \dd V
\end{equation}
where the second equality makes use of the divergence theorem.

Now consider the integrand, which we rewrite in terms of $\mathbf{q}$ and
$\mathbf{p}$, so
%
\begin{align}
  \nabla \cdot \mathbf{v} &= \left(\frac{\partial}{\partial \mathbf{q}},
  \frac{\partial}{\partial \mathbf{p}}\right) \cdot \left(\dot{\mathbf{q}},
  \dot{\mathbf{p}}\right) \\
  &= \frac{\partial \dot{\mathbf{q}}}{\partial \mathbf{q}} +
  \frac{\partial \dot{\mathbf{p}}}{\partial \mathbf{p}} \\
  & = \frac{\partial^2 H}{\partial \mathbf{q} \partial \mathbf{p}}
  - \frac{\partial^2 H}{\partial \mathbf{p} \partial \mathbf{q}} \\
  & = 0.
\end{align}
In the the third line, we have subsituted Hamilton's equations
(\myeqref{design:eq:hamilton}).  So it is clear that the volume is constant in
time,
%
\begin{equation}
  \frac{\dd V}{\dd t} = 0.
\end{equation}

Now consider the phase space density of the trap. This is can be defined as the
unitless quantity~\cite{PhysRevA.52.1423}
%
% We can also write $\rho = n \lambda_dB^3$, where n is spartial number
% density, and lambda_dB is the thermal deBroglie wavelength of the cloud
\begin{equation}
  \rho = \left(\frac{h}{\sqrt{\pi}}\right)^3 \frac{N}{V}.
\end{equation}
%
Since $V$ is conserved, so is the phase space density. However, we have made
some assumptions that should be addressed.

First, the motion of the particles is classical, but losses can still occur
through quantum effects such as a spin-flip loss~\cite{PhysRevLett.51.1336}.
Such an event would mean that $N$ would be reduced even though $V$ remained
constant, so the phase space density would decrease. Furthermore, we neglected
any change in the potential.  Changing the potential such that the depth is
reduced can allow fast-moving (hot) particles to escape. This also reduces $N$
and hence $\rho$ (though we do not include escaped particles in the calculation
of $V$ so this is unchanged).

This motivates us to introduce the concept of the \emph{acceptance} of a trap.
This is the phase space volume of the potential within which all particles are
trapped. Particles outside of the acceptance are either positioned outside the
trap or are positioned inside the trap but have enough energy to escape it. The
boundary of the acceptance is rather ominously called the \emph{separatrix}. An
example is shown in \myfigref{design:fig:acceptance}.~\cite{Lichtenberg1969,
HanHand1998}

When particles are loaded into a trap, they must be inside its acceptance to
remain trapped. It is common to consider this in terms of a four-dimensional
(two spatial, two momenta) trap for a beam of particles~\cite{Crompvoets2005}.
In this case the change in the potential is instantaneous, so it is obvious
which particles are trapped and which are not (as in
\cm{\myfigref{design:fig:acceptance}}). In our case the loading is achieved by
a series of adiabatic ramps, so the particles which are outside of the phase
space acceptance of the trap are not instantaneously lost, and can be retained
as long as this does not result in an increase in phase space density.~\cite{}
\cm{Big citation needed on this one!} This will be heavily exploited when
transferring between Z wire traps as in section~\ref{design:transferbetweenzs}.

It should be noted that the Liouville theorem also holds for a time-dependent
Hamiltonian.\footnote{The proof given above does not require $H$ to be time
independent, although for the sake of our discussion it was helpful for us to
assume that it was.} We will see below that the trapping potential will be changed
adiabatically, and make use of the fact that phase space density is conserved
through this process.

\subsection{Simulating the motion}
\label{design:motion:simmethods}

\thesis{When can we ignore VdW forces?}

Since the system is Hamiltonian, the particle motion can be simulated by
numerically solving \myeqref{design:eq:hamilton}. This was done on Imperial
College London's high performance computing services~\cite{ICRCS} using
Python~\cite{python} and the symplectic euler method provided by the Desolver
package~\cite{desolver}. Symplectic methods are powerful ways to numerically
integrate Hamiltonian systems whilst conserving energy and
momentum.~\cite{Hairer2015, doi:10.1119/1.2034523} A timestep of $\dd t =
\SI{1E-4}{\second}$ was sufficient to achieve this for our potentials.

Magnetic potentials were calculated by considering the traps to be formed of
segments of straight wires, each producing a magnetic field
$B_\text{seg}^{(i)}$, and the bias field $\mathbf{B}_\text{bias}(t)$. The total
field is the sum of each of these segments (with segments that are sufficiently
far from the trap centre neglected) and the bias. The potential is therefore
%
\begin{equation} V(t, \mathbf{q}) = \mu B (t, \mathbf{q}) = \mu \left| \sum_i
B_\text{seg}^{(i)}(t, \mathbf{q}) + \mathbf{B}_\text{bias}(t)\right|.
\end{equation}

The field due to a segment of straight wire with length carrying current $I(t)$
can be found analytically by integration of the Bio-Savart Law.
This field can be expressed in terms of the angle between the ends of the wire
at $\mathbf{q}$, as is shown in \myfigref{design:fig:wiresegment}. The field
is given by~\cite{Griffiths2017}
%
\begin{equation}
  B_\text{seg}(t, \mathbf{q}) = \frac{\mu_0 I(t)}{4\pi
  s_\text{seg}(\mathbf{q})} (\sin(\theta_2)  - \sin(\theta_1))
\label{design:eq:segmentfield}
\end{equation}
here $s_\text{seg}(\mathbf{q})$ is the shortest distance between the segment
and $\mathbf{q}$.

\begin{figure}[h]
\centering
  \begin{tikzpicture}
    % Def coords
    \coordinate (O) at (0, 0);
    \coordinate (L) at (-3, 0);
    \coordinate (R) at (3, 0);
    \coordinate (Q) at (1, 4);
    % Draw lines
    \draw[line width=0.75mm, ->] (L) -- (R);
    \draw (L) -- (Q);
    \draw (R) -- (Q);
    \draw[<->, densely dotted,shorten >=.5mm,shorten <=.8mm] (Q) -- (1, 0);
    % Draw dot at Q
    \node at (Q)[circle,fill,inner sep=.5mm]{};
    % Draw angels
    \draw pic[draw,angle radius=1cm,"$\theta_1$" shift={(7mm,2mm)}] {angle=R--L--Q};
    \draw pic[draw,angle radius=1cm,"$\theta_2$" shift={(-7mm,2mm)}] {angle=Q--R--L};
    % Label
    \node[shift={(4mm,0)}] at (Q) {$\mathbf{q}$};
    \node[shift={(2mm, 0)}] at (R) {$I$};
    \node[fill=white] at (1., 1.8) {$s_\text{seg}(\mathbf{q})$};
  \end{tikzpicture}
  \caption{Geometry of a wire segment (bold) carrying current $I$, whose field
  can be calculated using \myeqref{design:eq:segmentfield}. The dottend line
  shows $s_\text{seg}(\mathbf{q})$, the shortest distance from the point at
  which the field is calculated ($\mathbf{q}$) to the segment.
  }
  \label{design:fig:wiresegment}
\end{figure}

This is everything that is needed to simulate a wire trap, or the handover
between two traps (by including the wire segments of both traps in the field
calculation). The transport trap is very deep, so is well approximated by a
perfect quadrupole potential.

Molecule trajectories are calculated up to a pre-determined end time, or until
they are determined to be lost. In the following simulations a molecule is lost
when it's displacement from the origin in any cardinal direction is greater
than \SI{10}{\milli\meter}.

\subsection{Example simulation: Z trap acceptance}

To illustrate both the simulation methods and the example of phase space
acceptance given in section~\ref{design:motion:simmethods}, we present an
example simulation where a uniform distribution of molecules is spread over the
acceptance of the $\mathrm{Z0_i}$ trap used in our experiment. We expect that
the molecules inside the acceptance remain trapped for arbitrarty long time,
and those outside are lost.

This is indeed what we see in \myfigref{design:fig:acceptance}. The results of
the simulation are shown in a trio of plots, each one showing a projection of
phase space into the $(q_i, v_i)$ plane. Here $i\in\{x, y, z\}$ and we have
changed from using $\mathbf{p}$ to $\mathbf{v} = \mathbf{p} / m$, since the
mass is always constant. 

We simulated $10^6$ \CaF{} molecules, which are initialised with position and
velocity distributed uniformly inside the gold rectangle. The trajectories are
then integrated for the first \SI{300}{\milli\second} of the motion. The
starting position of those molecules who remained trapped are then plotted.

The acceptance is also shown. It is found by considering the energy at all
points in the phase space plane. The highest energy that forms a closed contour
is the acceptance of the trap. The energy is different in every plane (the $x$
direction is a weaker trapping direction).

There are 443 particles that remain trapped throughout the simulation. Of those
that are trapped, the vast majority fall within the calculated acceptance.
However in some edge cases it appears that the particles are outside. This is
due to the acceptance shown being a contour on a cut-through of the potential.
The true acceptance is a hypersurface that cannot be displayed here.

\begin{figure}
  \centering
  \includegraphics[page=1]{figs/sims/example_acceptance.pdf}
  \caption{Example simulation showing the acceptance of the $\mathrm{Z0_i}$.
  The region marked by the gold rectangles is populated with $10^5$ particles.
  The starting positions of the 443 particles that are trapped are marked with a pink dot.
  The acceptance is approximated by a contour of the trapping potential, shown
  here in black. Trapping is weaker in the $x$ direction than the other
  directions, so this contour has a lower value. The majority of the trapped
  particles start within this approximation of the acceptance.}
  \label{design:fig:acceptance}
\end{figure}

As well as demonstrating that the simulation is able to reproduce the expected
results from a simple phase space acceptance problem, we can use it to check
that the two most important constants of the motion are indeed conserved,
namely energy and phase space density. We expect that they will be, since the
problem is framed in terms of Hamiltonian dynamics and a symplectic solver has
been chosen.

Since particles which are lost cease to be simulated, we examine only the phase
space density and energy of the succesfully trapped particles. These are shown
in \myfigref{design:fig:conservation}. There are fluctuations in both
values, but this is to be expected.~\cite{doi:10.1119/1.2034523} Since the
fluctuations are small, and there is no overall trend we can assume that the
simulation does indeed conserve these quantities.

\begin{figure}
  \centering
  \includegraphics[page=1]{figs/sims/example_conservation.pdf}
  \caption{Conservation of energy and phase space density in our simulation.
  Energy is the total energy of the trapped particles, but is given in units of
  \si{\milli\kelvin} for convenience.
  }
  \label{design:fig:conservation}
\end{figure}

\subsection{Simulation initialisation}

For other simulations it is useful to initialise the simulation to replicate
the conditions in the transport trap just as we would have at the beginning of
the loading procedure.

This can be done by initialising a cloud of $N$ molecules whose positions are
normally distributed in all three spatial dimensions with a standard deviation
of \SI{1}{\milli\meter}. Similarly the velocity components are normally
distributed with a standard deviation of \SI{84}{\milli\meter\per\second}
(corresponding to a temperature of \SI{50}{\micro\kelvin}). This is what we
would expect for the cloud after leaving the molasses.

The cloud is immediately placed in a quadrupole trap with gradient
\SI{10}{\gauss\per\centi\meter}. They are held for \SI{50}{\milli\second}
before the gradient is ramped up over \SI{100}{\milli\second} to the actual
transport gradient of \SI{61}{\gauss\per\centi\meter}. This causes the
molecules to compress as they would when initially loaded into the transport
coils in the MOT chamber.

For the purposes of these simulations, the effects of heating in loading are
ignored. The molecules are therefore held for a further \SI{50}{\milli\second}
until $t=0$. It is possible for a very small number of particles to stray too
far from the centre of the trap during initialisation, at which point they are
considered to be lost. There may therefore be a difference between the number
of particles that are initialised and the number present at $t=0$.

\section{Modelling the loading procedure}
\label{design:sim}

To ensure that the chip design will allow for robust loading, the loading
process has been simulated up to the $\mathrm{Z0_f}$ trap. This demonstrates
the two main types of transitions in the loading process: handovers between
traps, and compression within a single trap.

The results of the simulation and the parameters of the ramps are outlined in
\myfigref{design:fig:simparams}. This can be considered in four parts, first
from $t = 0$ to $t=\SI{100}{\milli\second}$ there is a settling period where
the molecules are held in the transport trap. Then there are three
\SI{100}{\milli\second} ramps, each of which is followed by a
\SI{100}{\milli\second} stabilisation period. These three ramps describe the
handover from the transport trap to the U, the U to $\mathrm{Z0_i}$, and the
compression of $\mathrm{Z0_i}$ to $\mathrm{Z0_F}$. They are described in detail
in the following subsections.

\begin{figure}[htb]
\centering
  \begin{overpic}[page=1]{figs/sims/mtt_u_z_params.pdf}
    \put(32.5,55){(a)}
    \put(45.5,55){(b)}
    \put(63,55){(c)}
    \put(80.5,55){(d)}
  \end{overpic}
  \caption{
    The simulation parameters are shown for loading from the transport trap
    into trap $\mathrm{Z0_f}$. The simulation begins with a stabilisation
    period (a), in which the molecules are held in the transport trap for \SI{100}{\milli\second}. In period (b) the transport
    trap is ramped off and the U is ramped on. There is some heating of
    molecules as they leave the trap, and a reduction in phase
    space density. The transfer to $\mathrm{Z0_i}$ occurs in period (c), where
    more loss occurs due to the trap depth being reduced in the $x$
    direction. Finally in (d) the trap is compressed to $\mathrm{Z0_f}$, when
    the molecules are brought from \SI{3}{\milli\meter} to \SI{1}{\milli\meter}
    of the chip surface. The solid lines show the simulation results when
    collisions with the chip surface are ignored, and the dashed line shows
    results when collision with the surface results in loss of the molecule.
  }
  \label{design:fig:simparams}
\end{figure}

The simulation methods are described in section~\ref{design:motion:simmethods}. This
includes the initialisation procedure which happens before $t=0$. We simulated
$N=1000$ particles, but it is common for a few to be lost during the
initialisation process, hence the starting number for this simulation is
$N(t=0)=999$.

\subsection{Handover to the U trap}
\label{design:sim:trans_U}


\thesis{Ensure language is consistent when talking about chip, so is the U
above or below?}
%
The first step is to transfer the molecules from the transport coils to the U
wire. This U wire has the largest axial length, and so will be the easiest trap
the transport coils to. The U-wire is also capable of carrying more current
than any of the other wires, which is essential as it is positioned further
from the molecule cloud, beneath the chip.

Between $t=\SI{200}{\milli\second}$ and $t=\SI{300}{\milli\second}$ the current
in the transport coils is adiabtically ramped down while the currents in
the U trap and in the bias coils are ramped up. The molecules are then held
until $t=\SI{400}{\milli\second}$ to ensure that they remain trapped after the
ramp. The simulation parameters are shown in
\mysubfigref{design:fig:simparams}{b}.

The particle loss during the ramp is approximately 10\% of the total. There is
also some heating, which decreases the phase space density of the cloud. The
cause of this can be seen in \myfigref{design:fig:trans_U}, which shows the
phase-space positions of the molecules at various times throughout the ramp,
and the trapping potential at those times.

\begin{figure}[p]
\centering
  %\setlength\extrarowheight{-3pt}
  \begin{tabular}{c}
    \includegraphics[page=1]{figs/sims/mtt_u_x.pdf} \\
    \includegraphics[page=1,trim=0 0 0 15, clip]{figs/sims/mtt_u_y.pdf} \\
    \includegraphics[page=1,trim=0 0 0 15, clip]{figs/sims/mtt_u_z.pdf}
  \end{tabular}
  \caption{
    For each dimension, a cut-through of the potential and a plot of
    phase-space positions of the particles is shown. Each column shows a
    different time during the ramp from the transport trap into the U trap. The
    cut-throughs pass through $\mathbf{t}_0 = (0, 0, \SI{3}{\milli\meter})$. In
    the $x$ direction, it is possible to see decompression and recompression as
    the potential varies from $t=\SI{125}{\milli\second}$ to
    $t=\SI{200}{\milli\second}$. In the $y$ direction, the depth is lowered due
    to the superposition of the trapping fields, leading to a significant loss.
    This is partially visible at $t=\SI{175}{\milli\second}$, where we can see
    the drop in the potential and particles with $y<0$, $v_y<0$ escaping the
    trap. In the $z$ direction, some hotter molecules escape as the U trap
    reaches its final depth at $t=\SI{200}{\milli\second}$. Note that this loss
    has ended after the settling time, which can be seen in
    \myfigref{design:fig:U_Z0i}.
  }
  \label{design:fig:trans_U}
\end{figure}

At $t=\SI{175}{\milli\second}$ there is a noticeable decrease in the trapping
depth in the $y$ direction. This is the cause of the particle leak, and occurs
due to the due to the superposition of the two fields causing an overall
decrease in the trap depth in the $x$ and $y$ directions. This only occurs for
a brief period during the ramp (lasting approximately \SI{40}{\milli\second}). 

In addition to the loss during the ramp, we observe a steady loss between
$t=\SI{200}{\milli\second}$ and $t=\SI{300}{\milli\second}$, but only when the
simulation accounts for collision with the chip surface. This loss due to
collision occurs in the U wire because of its position beneath the chip. This
means that the trap is not sufficiently deep to hold our molecules for an
extended period. However the holding time in this simulation is only for
illustration, and the prolonged trapping in the U trap will not be part of the
final experiment. Hence these losses can be ignored.

\subsection{Handover to $\mathrm{Z0_i}$}
\label{design:sim:U_to_Z0i}

The U to $\mathrm{Z0_i}$ handover occurs in our simulation from
$t=\SI{300}{\milli\second}$ to $t=\SI{400}{\milli\second}$, shown in
\mysubfigref{design:fig:simparams}{c}.  The phase space positions of the
molecules and the potentials are shown in \myfigref{design:fig:U_Z0i}.

\begin{figure}[p]
\centering
  %\setlength\extrarowheight{-3pt}
  \begin{tabular}{c}
    \includegraphics[page=1]{figs/sims/u_Z0i_x.pdf} \\
    \includegraphics[page=1,trim=0 0 0 15, clip]{figs/sims/u_Z0i_y.pdf} \\
    \includegraphics[page=1,trim=0 0 0 15, clip]{figs/sims/u_Z0i_z.pdf}
  \end{tabular}
  \caption{
    The same simulation results are displayed as in
    \myfigref{design:fig:trans_U}, this time for the handover to $\mathrm{Z0_i}$.
    The superposition of the U and Z traps causes a reduction of the trap depth
    in the $x$ direction, as seen at time $t=\SI{375}{\milli\second}$.
    Molecules can be seen coming into contact with the chip surface when held
    in the U trap, but after transfer to the Z trap, which is located on the
    chip surface, the cloud is strongly repelled from the surface.
  }
  \label{design:fig:U_Z0i}
\end{figure}

In the ramp between the transport trap and the U trap, we saw that a hole was
introduced in the trapping potential due to interference between the two
trapping fields. We will see a similar effect when transferring from the U trap
to trap $\mathrm{Z0_i}$.  This time the reduction in depth appears in the $x$
direction, and is due to the current in the U and Z wires opposing each other
on the positive $x$ side. This can be seen in \myfigref{design:fig:U_Z0i} at
$t=\SI{375}{\milli\second}$.

In \myfigref{design:fig:simparams} we see that the loss is again about 10\% of
the particles, however there is no significant change in the phase space
density. The acceptance of trap $\mathrm{Z0_i}$ is smaller than the emittance
of the U, so this is to be expected.

\subsection{Compression to $\mathrm{Z0_f}$}

In the final stage of the simulation we look at a compression stage. The
trapping potential is adiabatically altered to bring the field minimum closer
to the chip surface. This is achieved by increasing the bias field, but since
this also increases the trap depth the cloud is simultaneously compressed as it
is heated. Since the phase space density is conserved, the particles heat up in
the trap. This phenomenom is made clear in \mysubfigref{design:fig:simparams}.

The trajectories and potentials are shown in \myfigref{design:fig:Z0i_Z0f}.
Here the compression in number density and expansion in velocity space is clear
by the changing cloud shape over time. The cloud also clearly follows the
potential minimum without losses.

\begin{figure}[p]
\centering
  \begin{tabular}{c}
    \includegraphics[page=1]{figs/sims/Z0i_Z0f_x.pdf} \\
    \includegraphics[page=1,trim=0 0 0 15, clip]{figs/sims/Z0i_Z0f_y.pdf} \\
    \includegraphics[page=1,trim=0 0 0 15, clip]{figs/sims/Z0i_Z0f_z.pdf}
  \end{tabular}
  \caption{Presentation of the loading simulation is continued from
  \myfigref{design:fig:trans_U} and \myfigref{design:fig:U_Z0i}. Here we see
  the compression from $\mathrm{Z0_i}$ to $\mathrm{Z0_f}$. The compression in
  the spartial dimensions causes heating, which can be seen as a spread in the
  velocity directions. This is in line with what is expected from Liouville's
  theorem.
  }
  \label{design:fig:Z0i_Z0f}
\end{figure}

\subsection{Transfer between Z traps}
\label{design:transferbetweenzs}

Once trapped in $\mathrm{Z0_f}$, the remaining loading stages are simply
repeated handovers to the smaller wires, with a compression stage carried out
on each wire to bring the molecules closer to the surface. This is a
well-understood and robust procedure that has been used before to load atom
chips with minimal losses.~\cite{Reichel2002}

That said, since our experiment will operate with a significantly lower phase
space density than previous experiments with atoms, it is important to ensure
that our loading procedure will not suffer from any unnecessary losses. These
compressed traps contain hotter molecules, requiring a smaller timestep for
simulation. It is therefore easier to directly examine the acceptance of each
trap.

These are shown in \myfigref{design:fig:phasematchinggrid}, starting with
$\mathrm{Z0_f}$ in row (a). The region of the trap that we expect to be
occupied (as determined by the simulation above) is marked with a dashed line.
This cloud of molecules will be adiabatically transferred into $\mathrm{Z1_i}$.

Since the process is adiabatic, we can anticipate that there will be no
reduction in phase space density. The cloud of molecules will be transferred to
$\mathrm{Z1_i}$ with negligible heating. However, the total acceptance of the
trap is greater than the expected phase space volume of the cloud, so there
will be particle loss due to spillover.

This is to be anticipated: phase space density cannot increase, and obviously
the acceptance of the smallest trap will be smaller than the intitial cloud in
the transport trap. So the final number of particles will be lower than we
started with. It doesn't matter at which point the loss occurs, so long as
there is no increase to the phase space density.

Next, $\mathrm{Z1_i}$ is adiabatically compressed to $\mathrm{Z1_f}$, with the
molecules held \SI{100}{\micro\meter} from the surface. Again, this
compression is adiabatic, so we expect no change in the phase space density.
Compression increases the trap acceptance, so there is also no loss of
particles. This is further supported by the simulated compression of
$\mathrm{Z0_i}$ to $\mathrm{Z0_f}$ discussed above. The acceptance of
$\mathrm{Z1_I}$ is shown in \mysubfigref{design:fig:phasematchinggrid}{b}, with
the expected occupation shown by the dashed line.

Finally, $\mathrm{Z2}$ is loaded and compressed in exactly the same way as
$\mathrm{Z1}$. The adiabatic handover from $\mathrm{Z1_f}$ to $\mathrm{Z2_i}$
is the final cause of particle loss (due to spillover). Then the trap is
compressed to hold the particles \SI{10}{\micro\meter} from the surface, as
shown in \mysubfigref{design:fig:phasematchinggrid}{c}.


\begin{figure}[htb]
\centering
  \begin{overpic}[page=1]{figs/sims/phase_matching_grid.pdf}
    \put(0,68){(a)}
    \put(0,44.5){(b)}
    \put(0,21){(c)}
  \end{overpic}
  \caption{
    A low bound of the acceptance (solid) and expected occupation (dashed) of
    each Z wire trap. This is calculated by finding the acceptance in the weak
    ($x$) trapping direction. Row (a) shows $\mathrm{Z0_f}$, with occupation
    determined by the above simulation. Molecules will be adiabatically
    transferred to $\mathrm{Z1_i}$, which will be fully occupied. Some
    particles will be lost due to the decrease in acceptance but this is to be
    expected, and phase space density should not increase. In (b) we have the
    acceptance and occupation of $\mathrm{Z1_f}$ following adiabaitc
    compression.  Particles will be adiabatically transferred into
    $\mathrm{Z2_i}$, resulting in similar losses to the previous step.  Row (c)
    shows the final acceptance and occupation of $\mathrm{Z2_f}$.
  }
  \label{design:fig:phasematchinggrid}
\end{figure}


\subsection{Direct loading into $\mathrm{Z0_i}$}

The U trap stage has been included to allow for easy alignment of the transport
coils and the chip trap. If this was not a requirement then it might appear
desirable to load directly into the Z traps. This would have the advantage of
removing the small collision losses discussed in
section~\ref{design:sim:trans_U}, as well as the losses incurred when handing
over from the U to $\mathrm{Z0_i}$ described in
section~\ref{design:sim:U_to_Z0i}.

Results of a simulation of transfer from the transport coils to $\mathrm{Z0_i}$
are shown in \myfigref{design:fig:directzramp}.  Approximately 40\% of
particles were lost, as opposed to approximately 20\% of particles lost in the
transfer via the U described above.  However, it appears that the phase space
density does not decrease as was seen before. The phase space density
calculated here is mainly for diagnosis (see section~\ref{design:motion:simmethods}),
but this suggests that direct loading into $\mathrm{Z0_i}$ fromt the transport
coils may be worth investigating if we are able to reliably achieve good
alignment of the transport coils. 

\begin{figure}[tbh]
\centering
  \includegraphics[page=1]{figs/sims/mtt_z_params.pdf}
  \caption{
    Simulation parameters for direct loading of trap $\mathrm{Z0_i}$ from the
    transport coils. Currents and fields are linearly ramped to and from the
    same values for these traps as were used in the loading via the U trap
    (described above). The ramp occurs between $t=\SI{100}{\milli\second}$ and
    $t=\SI{200}{\milli\second}$. There is more particle loss, but the phase
    space density does not decrease by the same amount.
  }
  \label{design:fig:directzramp}
\end{figure}



\section{Summary}

\ph{TODO, probably just an overview of the final loading procedure}
