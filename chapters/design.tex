The design of the \CaF{} chip experiment was largely motivated by the need to
integrate with the existing apparatus described in \cite{}.
% TODO or above?
The natural way to do so was to extend the magnetic transport of the cold
molecules to a third chamber, where they can be transferred into the
microfabricated trap.

In this chapter I will present how we have designed the chip trap to allow for
this transport and loading scheme, with the ability to introducuce microwave
guides on a later iteration of the experiment. I will begin with an overview of
the entire experimental apparatus, and then focus on the design of the chip
traps, including present simulations of the loading procedure, which have
helped to inform our design choices.

\section{Experiment overview}

\cm{Check that this does not overlap too much with the intro}

\cm{Will likely need more discussion of transport in the thesis. For one
thing it seems like it does cause a bit of heating for tweezers at the moment.}
%

Our ultracold molecules are created using the methods described above and in
\cm{REFERENCES}. In brief, molecules are created in the buffer gas source,
laser slowed, and initially trapped in a MOT. This MOT is housed inside a
vacuum chamber used for experiments where collisions of \CaF{} with \Rb{} atoms
are observed ~\cite{}.  For our purposes, this collisions chamber is where the
sub-Doppler cooling is carried out, producing a cloud of ultracold molecules.

% TODO May need to say something about tranverse cooling and credit Caleb

An additional chamber is connected to the collisions chamber, where we conduct
experiments involving trapping \CaF{} in optical tweezers. Molecules can be
transported into the tweezer chamber by transferring them into a magnetic
transport trap (MTT), a quadrupole trap with coils mounted on a transport stage
outside the vacuum chamber. This transport of the molecules will be discussed
further below and in \cm{transport chapter}.
%
% TODO Can cite \cite{Lewandowski2003} here or later.

The chip experiment will integrate into this setup as shown in
\myfigref{design:fig:vacuumsystem}. A third chamber will be mounted
further downstream of the transport stage, so that molecules can be brought
straight through the tweezer chamber from the collisions chamber.
The chip trap will be positioned chamber, held in place by a flange
assembly, with the chip surface \SI{4}{\milli\meter}
%
\cm{be exact here, think it is 3.8??}
%
away from the transport axis. The flange assembly consists of current
feedthroughs and a PCB to deliver trapping currents to the chip, a large copper
heatsink, and microwave feedthroughs. It will be mounted so that the chip
faces the floor, allowing us to trap and then drop the molecules for imaging.

\begin{figure}[htb]
  \centering
  \cm{Change to autocad drawing, include inset of flange inside the L}
  \begin{overpic}[width=0.7\textwidth]{figs/vacuum_setup.pdf}
    \put(1,-5){Chip chamber}
    \put(34,-5){Tweezer chamber}
    \put(73,-5){MOT chamber}
  \end{overpic}
  \vspace{1cm}
  \caption{
    A top-down view of the planned \CaF{} experiment is shown. Molecules are
    created in the source chamber in a buffer-gas cell. They are then
    laser-slowed by the methods described in~\cite{} and \cm{transverse
    cooling}, then captured in the collisions chamber. Here we can cool the
    molecules by the methods described in \cm{thoery ch} and perform
    experiments in collisions as detailed in \cm{cite}. A magnetic transport
    trap (MTT) can transfer a cloud of molecules to the neighbouring tweezer
    chamber (for the optical tweezer experiment). The chip chamber will be
    positioned after the tweezer chamber, and will receive molecules from the
    MTT in the same way. Inset, details of the chip assembly. Current is
    delievered to the microfabricated chip via an intermediary printed circuit
    board (the sub-chip).
  }
  \label{design:fig:vacuumsystem}
\end{figure}


Once the molecules are brought into the chamber by the transport coils, they
will be magnetically transferred onto the chip trap. The transfer will take
the molecules through a series of magnetic traps of decreasing size until
they are in the smallest trap. The design of these traps and the loading
procedure are the main subjects of this chapter.

The first stage of the transfer is a large U wire embedded in the heatsink (see
\myfigref{design:fig:chipexperiment}).  The idea is to use this as a deep,
macroscopic trap that is well-aligned with the chip (by virtue of being built
into the assembly) and can be easily aligned with the MTT, similarly to other
experiments such as \inlineref{Ott2001}.  Since it is under the chip, it will
of course be further from the molecule cloud than the other wires, hence more
current will be required to create a deep trap (as per
equations~\ref{intro:eq:trapbias} and~\ref{intro:eq:trapdepth}). The current is
limited to \SI{100}{\ampere} by the vacuum feedthroughs, allowing for a
\cm{trap that is what depth in mK??}

\begin{figure}[ht]
  \centering
  \parbox{\textwidth}{
    \parbox{.5\textwidth}{%
\centering
      \subcaptionbox{}{
\includegraphics[width=0.4\textwidth]{figs/FlangeAssemblyForPresentationTopRight.png}
}
      \vskip3em
      \subcaptionbox{}{
\centering
\includegraphics[height=0.2\textwidth]{figs/fab/subchip1_cropped3.pdf}
}  
    }
    \hskip1em
    \parbox{.5\textwidth}{%
      \subcaptionbox{}{
  \begin{overpic}[width=\hsize,page=1]{figs/chip_des.pdf}
    \put(26,64){{\tiny\SI{100}{\micro\meter}}}
  \end{overpic}
}
    }
  }
  \caption{
    \cm{Change this to a view of the (right) chip and exploded view of the
    flange, so we can see the U wire, PCB (left)}
    In (a) we have an exploded view of the chip flange assembly. Note the
    copper heatsink into which the U-wire is embedded. On top of this we have
    the subchip for power delivery to the chip. In (b) we have the chip design,
    with the inset showing a zoomed-in view of the three central wires.
  }
  \label{design:fig:chipexperiment}
\end{figure}

Once loaded into the U, the molecules will be transferred through the Z wire
traps on the chip. The idea is that each Z-wire should be sufficiently large to
maintain the currents required to form a trap at height $z$ below the trap,
whilst having a width and height  $w, h \ll z$ so that that the current is
highly localised compared to the cloud size.  This follows the widely-made
assupmption in the literature that the trapping currents are carried by wires
which are infinitesimally small compared to the length scale of the trap
~\cite{}.

In the case of the first Z wire, the molecules are still \SI{3}{\milli\meter}
away from the chip, a current of order \SI{10}{\ampere} is required to form a
suitable trap
%
\cm{need to figure out what the actual depth is}.
%
We will discuss in chapter~\ref{fab} that the maximum wire height achievable is
\SI{5}{\micro\meter}, and we expect that the wires will be able to carry a
maximum current density of \SI{6E10}{\ampere\per\meter\squared}, as was found
for a similar chip design in \inlineref{Treutlein2008}. The Z wire will
therefore have a width $w=\SI{200}{\micro\meter}$. Other wire parameters are
shown in \mytableref{design:table:wires}.

The axial length  of the wires also decreases to gradually reduce the size of
the trapped cloud in the $x$ direction. The wire layout is shown in
\myfigref{design:fig:chipexperiment} and more details along with the
heights are outlined in \mytableref{design:table:wires}. All wires have
been designed to carry twice the current that is required in the loading
scheme.

\cm{Add bias fields required to this table (as a ramp, like trap height)
maybe just they y bias?? Also CONFIRM THE AXIS LENGTHS.}
%
\begin{table}
  \centering
\begin{tabular}{lrrrrr}
  Name & Axis length (\si{\milli\meter}) & Width (\si{\micro\meter})& Height
  (\si{\micro\meter})& $I_\text{max}$ & Trap height (\si{\micro\meter}) \\
 \hline
  U & 16 & N/A& N/A& 100 & 3000\\
  $\mathrm{Z0}$ & 12 & 200&  5& 60& $3000\rightarrow1000$ \\
  $\mathrm{Z1}$ &  6 & 20&  5& 6& $1000\rightarrow100$ \\
  $\mathrm{Z2}$ &  3 & 2&  5& 0.6& $100\rightarrow10$ \\
 \hline
\end{tabular}
  \caption{Details on the wire dimensions, maximum current, and desired
  trapping heights. The wire design is shown in
  \mysubfigref{design:fig:chipexperiment}{c}. Note that the U wire current is
  limited by vacuum feedthroughs and not by the wire dimensions or cooling.
  The maximum currents have been designed for use at only 50\% of their
  potential maximum ($I_\text{max}$).
  }
  \label{design:table:wires}
\end{table}

Each Z trap will begin trapping at one height before the bias field is
increased to allow for trapping closer to the surface (as per
\myeqref{intro:eq:trapbias}).  To distinguish between the two trap stages for
each wire, we label them $\mathrm{ZX_i}$ for the initial (higher) trap and
$\mathrm{ZX_f}$ for the final (lower) trap, with $\mathrm{ZX}$ corresponding to
the wire labels in \mytableref{design:table:wires}.

We aim to eventually incorporate microwave guides onto the chip. This can be
accomplished by adding an insulating layer on top of the wires, on to which we
can fabricate coplanar waveguides~\cite{1127105}. The flange and subchip have
been designed to allow delivery of microwaves to the chip. \cm{Direct to
microwaves on a chip chapter.} This was discussed in detail in the early stage
assessment and will not be repeated here.

In the rest of this chapter, I will demonstrate through simulation and analysis
of the trapping potentials that this design will be capable of loading trapped
\CaF{} molecules.

\section{Motion of molecules in a potential}
\label{design:motion}

We can assume that the motion of the molecules in the trap is classical. They
move in the potential $V(t, \mathbf{q}) = \mu B(t, \mathbf{q})$, where $\mu$ is
the magnetic dipole moment of the molecule \cm{in what state?}.  The motion of any one particle is
described by Hamilton's equations,~\cite{Lichtenberg1969}
%
\begin{align}
  \label{design:eq:hamilton}
  \dot{\mathbf{q}} =  \frac{\partial H}{\partial \mathbf{p}} &&
  \dot{\mathbf{p}} = -\frac{\partial H}{\partial \mathbf{q}},
\end{align}
%
where $H$ is the classical Hamiltonian of the system
\begin{equation}
  %
  H(t, \mathbf{q}, \mathbf{p}) = \frac{\mathbf{p}^2}{2m} + V(t, \mathbf{q}).
\end{equation}
For now we neglect the time dependence of the potential, so that $V(t,
\mathbf{q}) = V(\mathbf{q})$.

Solving Hamilton's equations tells us the position and momentum of a single
particle. Taken together, these two vectors describe a point in a six
dimensional \emph{phase space} of position and momentum. The ensemble of
trapped molecules, each one starting at a different point in phase space, can
be treated by considering the motion of the molecules at its
boundary, and the six-dimensional phase space volume that this
boundary encloses~\cite{Hand1998}.

It is helpful for us to write the phase space volume $V$ in terms of the
spatial volume occupied by the ensemble ($V_\text{space}$) and its temperature.
The length-scale for a monatomic gas of temperature $T$ is the thermal de
Broglie wavelength~\cite{blundell2}
%
\begin{equation}
  \lambda_\text{dB}(T) = \frac{h}{\sqrt{2 \pi m k_B T}}.
\end{equation}
%
Hence the unitless phase space volume is
%
\begin{equation}
  V = V_\text{space} \lambda_\text{dB}^{-3}(T).
\end{equation}
%
It is also useful to define a unitless phase space
density~\cite{PhysRevA.52.1423}
%
\begin{equation}
  \rho = \frac{N}{V} = \frac{N \lambda_\text{dB}^3}{V_\text{space}}.
\end{equation}
%
For a given cloud with phase space density $\rho$, trap with volume
$V_\text{trap}$ and depth $T_\text{depth}$,
the maximum possible number of particles that can be trapped is $\rho
V_\text{trap}/\lambda_\text{dB}^3(T_\text{depth})$.

Furthermore, Liouville's theorem~\cite{Landau1982, Hand1998} states that the
phase space volume is a conserved quantity, as long as the trapping potential
is consevative. This means that the phase space density of a trapped molecular
cloud cannot be increased without application of some velocity-dependent
damping, such as an optical molasses~\cite{Metcalf1999}. The impact of this for
the chip is that the phase space density of the cloud at the point of loading
onto the chip determines the maximum number of molecules that can be trapped in
the final trap.

If we do not introduce any further cooling steps, then the number of molecules
we are able to trap in the final trap will be
%
\begin{equation}
  N_\text{final} = \frac{\rho_\text{\CaF{}}V_\text{trap}}
  {\lambda_\text{trap}^3} = \frac{\rho_\text{\CaF{}} V_\text{trap}(2 \pi m_\text{CaF} k_B
  T_\text{trap})^\frac{3}{2}}{h^3}
  \label{design:eq:psd_N}
\end{equation}
%
where $\rho_\text{\CaF{}} = 3.4\times10^{-12}$ is the phase space density of
the \CaF{} cloud after cooling in the blue-detuned molasses~\cite{Truppe2017},
$V_\text{trap}\sim(\SI{10}{\micro\meter})^3$ and
$T_\text{trap}=\SI{4}{\kelvin}$. The \CaF{} mass is
$m_\text{\CaF{}}=\SI{9.79E-26}{\kilogram}$. The trap depth used here is for the
smallest trap operating at \SI{3}{\ampere}, and can be calculated using
equation~\ref{intro:eq:trapdepth}. We therefore have
%
\begin{equation}
  N_\text{final} \lesssim 2000
\end{equation}
%
as an upper bound for the number of molecules trapped in the final trap. To
reiterate, this assumes that the entire trap is full, and no excess molecules
have been lost during loading. We will see below that this will not be the
case. A more realistic number of molecules to be trapped can be found by using
two key tools: analysis of the phase space acceptance of the traps, and direct
trajectory simulation of the cloud.

\subsection{Phase space acceptance}

\cm{Probably need to cite Hand 1998}

For a particle to be trapped, it is obvious that it must exist within the
spatial region of the trap, and it must not have sufficient energy to escape
from the trap's potential. In other words, the Hamiltonian of the particle
$H(\mathbf{q}, \mathbf{p})$ must be less than the trap depth $T_\text{depth}$.
This relation defines a region of phase-space that we call the
\emph{acceptance}. Any particles within this region will not have sufficient
energy to escape, and so will remain trapped under classical motion unless
externally influenced (so ignoring for example any collisions, decays, Majorana
losses, etc.)~\cite{Lichtenberg1969}.

As an example consdier the harmonic trap given by the potential found in
\inlineref{Crompvoets2005}
%
\begin{equation}
  V(x) = \begin{cases}
    0 & |x| > x_0 \\
    -\frac{1}{2}m\omega^2 x^2 & |x| \leq x_0.
  \end{cases}
\end{equation}
%
Here, any particle which starts with a position $-x_0 \leq x \leq x_0$ will be
trapped, so long as its velocity is not sufficiently large that it will escape
into the $|x|> x_0$ region. To reach $x_0$ from an initial position
$\tilde{x}$, the particle must have kinetic energy of at least
%
\begin{equation}
  \frac{1}{2}m\tilde{v}^2 = \frac{1}{2}m\omega^2(x_0^2 - \tilde{x}^2)
\end{equation}
%
where $\tilde{v}$ is the initial velocity. It is now clear that a particle is
trapped on the condition that
%
\begin{equation}
  (\tilde{v}/\omega)^2 + \tilde{x}^2 < x_0^2
\end{equation}
%
which defines an ellipse in phase-space.

This trapping region is can be verified by simulation, as is shown in
\mysubfigref{design:fig:psaeg}{a}, where the trap has been simulated for $m =
\omega = 1$, $x_0 =$ \cm{?}. We initialise 3000 particles uniformly distributed
with $|x| < 0.2$ and $|v_x|<2$ \cm{units!}, their trajectories are computed by
the methods described in the next section. The boundary of the acceptance,
called the \emph{separatrix} is shown in black. All particles that are
intialised inside the acceptance remain trapped and evolve through time in the
usual way for a harmonic potential. Particles initialised outside the
acceptance are rejected and lost.

\inlineref{Crompvoets2005} gives a further instructive example of an anharmonic
one-dimensional potential,
%
\begin{equation}
  V(x) = -V_0\cos(\omega x)
\end{equation}
%
which is also presented in \myfigref{design:fig:psaeg}. As well as the same
behaivour for acceptance as the harmonic potential, we see the
\cm{filamentation... actaully explain this...}

Finally, it is useful to consider the phase-space acceptance of a wire trap.
For simplicity we begin by considering only the acceptance in the $z$ direction
(perpendicular to the chip surface). Acceptance of a Z-wire with \cm{init and
trap parameters}
is shown in \mysubfigref{design:fig:psaeg}{c}. Here we can see the accepted
molecules \cm{colour}, as well as those that are rejected \cm{colour} and
metastable molecules \cm{colour}, which remain outside the acceptance but stay
nearby the trap. \cm{TODO: expand}

\begin{figure}
  \centering
  %\includegraphics[width=0.6\textwidth, page=1]{figs/psa_eg.png}
  \cm{My PSA fig with (a) as harmonic, (b) anharmonic, (c) as Z trap. 3 rows
  2 cols, with left col t 0 and right t end}
  \caption{
    The phase-space acceptance is shown for a harmonic (a), anharmonic (b) and
    Z-wire trap (c). The initialised particles are shown in the left-hand
    panes. Accepted particles are shaded black, and those that are rejected are
    shaded grey. In all cases \cm{sim parameters}. Subfigures (a) and (b)
    are re-creations of the examples given by \inlineref{Crompvoets2005}.
  }
  \label{design:fig:psaeg}
\end{figure}

An adiabatic change to the potential will not affect the phase space density of
any particles that remain trapped throughout the change~\cite{Hand1998,
Lichtenberg1969}. Hence a cloud of trapped particles can be translated, for
example in the transport coils, or towards the surface of the chip as will be
discussed in section~\ref{design:sim}.

It should be noted that the Liouville theorem also holds for a time-dependent
Hamiltonian. We will see below that the trapping potential will be changed
adiabatically, and make use of the fact that phase space density is conserved
through this process.

Further, it is possible for us to design our trapping potentials such that the
acceptance of one trap overlaps with the phase space volume occupied by the
molecules in another. This is similar to
the mode-matching of
potentials described in, for example, \inlineref{2011Ac}. We will see below
that the matching of trap acceptances does indeed allow for reliable loading of
the traps, \cm{even in the non-adiabatic r\'egime}.

\subsection{Simulating the motion}
\label{design:motion:simmethods}

\cm{When can we ignore VdW forces?}

The particle motion can be simulated by numerically solving
\myeqref{design:eq:hamilton}. This was done on Imperial College London's high
performance computing services~\cite{ICRCS} using Python~\cite{python} and the
symplectic Euler method provided by the Desolver package~\cite{desolver}.
Symplectic methods are powerful tools for numerically solving Hamiltonian
systems whilst conserving energy and momentum \cite{Hairer2015,
doi:10.1119/1.2034523}. 

We are able to simulate any arbitarty potential, such as those already
discussed in the previous section, but it is most useful to consider a more
realistic potential given by the sum of the magnetic and gravitational
potentials,
%
\begin{equation}
  V(t, \mathbf{q}) = V_\text{mag} + mg(z_0-z)
\end{equation}
where $g=\SI{9.8}{\meter\per\second\squared}$ is the acceleration due to
gravity, and $z_0$ is an arbitrary point chosen to be the zero of the
gravitational potential. The actual value chosen does not matter as it
constitutes only a linear shift in the entire potential, we normally present
the potentials with $z_0$ chosen such that the trap minimum is at zero.

The magnetic potentials are calculated by considering the traps to be formed of
segments of straight wires, each producing a magnetic field
$\mathbf{B}_\text{seg}^{(i)}$, and the bias field $\mathbf{B}_\text{bias}(t)$. The total
field is the sum of contributions from the set of all segments (S), and the
bias. The potential is therefore
%
\begin{equation} V\text{mag}(t, \mathbf{q}) = \mu B (t, \mathbf{q}) = \mu \left|
  \sum_{i\in S}
  \mathbf{B}_\text{seg}^{(i)}(t, \mathbf{q}) +
  \mathbf{B}_\text{bias}(t)\right|.
\end{equation}
%
Where the field of each wire segment is~\cite{Griffiths2017}
%
\begin{equation}
  \mathbf{B}_\text{seg}(t, \mathbf{q}) = \frac{\mu_0 I(t)}{4\pi
  s_\text{seg}(\mathbf{q})} (\sin(\theta_2)  -
  \sin(\theta_1))\hat{\mathbf{\phi}},
\label{design:eq:segmentfield}
\end{equation}
%
$s_\text{seg}(\mathbf{q})$ being the shortest distance between the segment
and the point $\mathbf{q}$, and $\hat{\mathbf{\phi}} =
\mathbf{I}\times\mathbf{q}/(qI)$.

\begin{figure}[h]
\centering
  \begin{tikzpicture}
    % Def coords
    \coordinate (O) at (0, 0);
    \coordinate (L) at (-3, 0);
    \coordinate (R) at (3, 0);
    \coordinate (Q) at (-4, 4);
    % Draw lines
    \draw[line width=0.75mm, ->] (L) -- (R);
    \draw (L) -- (Q);
    \draw (R) -- (Q);
    \draw[<->, densely dotted,shorten >=.5mm,shorten <=.8mm] (Q) -- (-4, 0);
    % Draw line parallel to wire
    \draw[-, dashed] (-5, 0) -- (L);
    \draw[-, dashed] (R) -- (5, 0);
    % Draw dot at Q
    \node at (Q)[circle,fill,inner sep=.5mm]{};
    % Draw angels
    \draw pic[draw,angle radius=1cm,"$\theta_1$" shift={(7mm,2mm)}] {angle=R--L--Q};
    \draw pic[draw,angle radius=1cm,"$\theta_2$" shift={(-7mm,2mm)}] {angle=Q--R--L};
    % Label
    \node[shift={(4mm,0)}] at (Q) {$\mathbf{q}$};
    \node[shift={(2mm, 4mm)}] at (R) {$I$};
    \node[fill=white] at (-5.0, 1.8) {$s_\text{seg}(\mathbf{q})$};
  \end{tikzpicture}
  \caption{Geometry of a wire segment (bold) carrying current $I$, whose field
  can be calculated using \myeqref{design:eq:segmentfield}. The dotted line
  shows $s_\text{seg}(\mathbf{q})$, the shortest distance from the point at
  which the field is calculated ($\mathbf{q}$) to the line parallel with the
  wire (dashed line).
  }
  \label{design:fig:wiresegment}
\end{figure}

For our simulations it is sufficient to consider a U or Z trap made up of three
wire segments, the central (axial) wire and the two long confining wires, which
we consider to extend to infinity. We will also consider quadrupole traps
generated by macroscopic coils to be ideal, with fields given by~\cite{}
% TODO Probably ref. above
%
\begin{equation}
  \mathbf{B}(t, \mathbf{q}) =...
\end{equation}

\subsection{Simulation initialisation}

For all the simulations described below, we begin with a cloud of molecules in
a magnetic quadrupole trap. We use the following initialisation procedure s
that the distribution of our molecules matches that which we expect in the
experiment. 

We begin with a cloud of $N$ molecules whose positions are normally distributed
in all three spatial dimensions with a standard deviation of $\sigma_i$
(typically $\sigma_i = \SI{1}{\milli\meter}$. Similarly the velocity components
are normally distributed with a standard deviation of $\sigma_{v_0}$ (typically
\SI{84}{\milli\meter\per\second}, corresponding to a temperature of
\SI{50}{\micro\kelvin}). 

The cloud is immediately placed in a quadrupole trap with gradient
\SI{10}{\gauss\per\centi\meter}. They are held for \SI{50}{\milli\second}
before the gradient is ramped up over \SI{100}{\milli\second} to the 
\SI{60}{\gauss\per\meter} (the gradient used in the MTT).
The molecules are then held for a further \SI{50}{\milli\second}, as a
stabilisation time.

The increasing trap gradient causes the
molecules to compress as they would when initially loaded into the transport
coils in the MOT chamber, as can be seen in \myfigref{}.
\cm{Need to show histograms in radial and axial, along with comparison to real
data, which I should collect.}

\section{Adiabatic transfer between traps}

% Simple QP ramp (varying time to find adiabatic regime)

In fact, the simple potential of a quadrupole potential can be used to
illustrate a point about transferring molecules between traps. It should be
obvius that if we start with molecules in one potential, then rapidly turn that
one off and turn on another, this will cause heating and increasing its size.
However, if we ramp between the traps adiabatically, then we expect the effect
to the cloud size to be minimal.

In our example, we will consider two ideal quadrupoles with gradients of
\SI{60}{\gauss\per\centi\meter} separated by \SI{3}{\milli\meter} in the $x$
direction. We simulate 2000 \cm{?} molecules using the above-described
initialisation procedure and \cm{sigma and temp values} and ramp between the
potentials in a time $t_\text{ramp}$, which we vary between
\SI{1}{\milli\second} and \SI{100}{\milli\second}. The molecules are then held
for a further \SI{100}{\milli\second}. The state of the system at the end of
the simulation is shown for $t_\text{ramp}\in \{\SI{1}{\milli\second},
\SI{30}{\milli\second}, \SI{100}{\milli\second}\}$ in \cm{figure}, along with
the behaivour of the cloud in the \SI{100}{\milli\second} hold time. Note that
as the ramp time increases, the oscillation in cloud size, temperate and
position decreases, as we have said we expect.

The question then is at what point do we enter the adiabatic r\'egime? \cm{Some
figure} shows the cloud width and temperature at the end of the simulation for
varying $t_\text{ramp}$. After $t_\text{ramp}\sim\SI{50}{\milli\second}$ \cm{we
are pretty adiabatic}.

\cm{In this figure, note that the rapid oscillations in x width probably
account for the weirdness at the start of the width graph.}

% Z transfer (c.f. rapid)

\section{Trajectory simulation of the initial loading stages}

To ensure that the chip design will allow for robust loading, the loading
process has been simulated up to the $\mathrm{Z0_f}$ trap. This demonstrates
the two main types of transitions in the loading process: handovers between
traps, and compression within a single trap. The timing of the simulation do
not exactly reflect those that will be used in practice, long holding periods
have been inserted between ramps to better illustrate the cause of particle
loss.

The results of the simulation and the parameters of the ramps are outlined in
\myfigref{design:fig:simparams}.
%TODO Redo this old nonsense
\cm{This can be considered in four parts, first
from $t = 0$ to $t=\SI{100}{\milli\second}$ there is a settling period where
the molecules are held in the transport trap. Then there are three
\SI{100}{\milli\second} ramps, each of which is followed by a
\SI{100}{\milli\second} stabilisation period. These three ramps describe the
handover from the transport trap to the U, the U to $\mathrm{Z0_i}$, and the
compression of $\mathrm{Z0_i}$ to $\mathrm{Z0_F}$. They are described in detail
in the following subsections. This simulation used $N=1000$ particles, with one
particle lost during initialisation.}
\cm{For this fig. I think I just want all the parameters, temp, ramps,etc in
one go, similar to before.}



\subsection{MTT - U transfer}
\label{design:sim:trans_U}

\begin{figure}[p]
\centering
  \import{figs/simsThesis/}{mtt_u_summary.pgf}
  \caption{
    The particle distribution in the MTT (blue) and after transfer to the U
    trap (red) following the process described in the main text. The
    phase-space distributions are shown in the $z$-$v_z$ plane in (a) and (b),
    with histograms of the $z$ and $v_z$ projections in (c) and (e)
    respectively. The potentials at the start and end of the transfer are shown
    in (d). Subfigure (f) shows the temperature particle number varying
    throughout the fields are varied in the times that are shaded grey (c.f.
    \myfigref{design:fig:simparams}). \cm{Comment on the results.}
  }
  \label{design:fig:mttusum}
\end{figure}




\cm{Ensure language is consistent when talking about chip, so is the U
above or below?}
%
The first step is to transfer the molecules from the transport coils to the U
wire. This U wire has the largest axial length, and so will be the easiest trap
the transport coils to. The U-wire is also capable of carrying more current
than any of the other wires, which is essential as it is positioned further
from the molecule cloud, beneath the chip.

Between $t=\SI{100}{\milli\second}$ and $t=\SI{200}{\milli\second}$ the current
in the transport coils is adiabtically ramped down while the currents in
the U trap and in the bias coils are ramped up. The molecules are then held
until $t=\SI{300}{\milli\second}$ to ensure that they remain trapped after the
ramp. The simulation parameters are shown in
\mysubfigref{design:fig:simparams}{b}.

The particle loss during the ramp is approximately 10\% of the total. There is
also some heating, which decreases the phase space density of the cloud. The
cause of this can be seen in \myfigref{design:fig:trans_U}, which shows the
phase-space positions of the molecules at various times throughout the ramp,
and the trapping potential at those times.

\begin{figure}[p]
\centering
  \begin{tabular}{c}
  \import{figs/simsThesis/}{mtt_u_detail_x.pgf} \\
  \import{figs/simsThesis/}{mtt_u_detail_y.pgf} \\
  \import{figs/simsThesis/}{mtt_u_detail_z.pgf}
  \end{tabular}
  \caption{
    Each pair of rows shows the phase-space plots for the MTT to U trap
    handover simulation in the $q_i$-$v_i$ plane, along with the cut-through of
    the potential through the trap centre along $q_i$ at various times. This
    supplements the simulation summary shown in \myfigref{design:fig:mttusum}.
    The superposition of the quadrupole and U-traps briefly causes a lowering
    of the trapping potential, as can be seen at \cm{time??} in the $y$
    direction. Such reductions are not accounted for when considering the
    simple phase-space handovers discussed in the main text.
  }
  \label{design:fig:mttudetail}
\end{figure}

At $t=\SI{175}{\milli\second}$ there is a noticeable decrease in the trapping
depth in the $y$ direction. This is the cause of the particle leak, and occurs
due to the due to the superposition of the two fields causing an overall
decrease in the trap depth in the $x$ and $y$ directions. This only occurs for
a brief period during the ramp (lasting approximately \SI{40}{\milli\second}). 

In addition to the loss during the ramp, we observe a steady loss between
$t=\SI{200}{\milli\second}$ and $t=\SI{300}{\milli\second}$, but only when the
simulation accounts for collision with the chip surface. This loss due to
collision occurs in the U wire because of its position beneath the chip. This
means that the trap is not sufficiently deep to hold our molecules for an
extended period. However the holding time in this simulation is only for
illustration, and the prolonged trapping in the U trap will not be part of the
final experiment. Hence these losses can be ignored.

\subsection{U - Z transfer}
\label{design:sim:U_to_Z0i}

\begin{figure}[p]
\centering
  \import{figs/simsThesis/}{u_z_summary.pgf}
  \caption{
    The U trap to Z trap handover is summarised, with subfigures and colours as
    in \mysubfigref{design:fig:mttusum}. Here we again see loss effects similar
    to those seen for the MTT to U trap handover, with a decrease in trap depth
    \cm{that may be} due to changing from a quadrupole trap to a
    Ioffe-Pritchard trap.
  }
  \label{design:fig:uzsum}
\end{figure}

The U to $\mathrm{Z0_i}$ handover occurs in our simulation from
$t=\SI{300}{\milli\second}$ to $t=\SI{400}{\milli\second}$, shown in
\mysubfigref{design:fig:simparams}{c}.  The phase space positions of the
molecules and the potentials are shown in \myfigref{design:fig:U_Z0i}.

In the ramp between the transport trap and the U trap, we saw that a hole was
introduced in the trapping potential due to interference between the two
trapping fields. We will see a similar effect when transferring from the U trap
to trap $\mathrm{Z0_i}$.  This time the reduction in depth appears in the $x$
direction, and is due to the current in the U and Z wires opposing each other
on the positive $x$ side. This can be seen in \myfigref{design:fig:U_Z0i} at
$t=\SI{375}{\milli\second}$.

In \myfigref{design:fig:simparams} we see that the loss is again about 10\% of
the particles, however there is no significant change in the phase space
density. The acceptance of trap $\mathrm{Z0_i}$ is smaller than the emittance
of the U, so this is to be expected.

\subsection{Z compression}
\begin{figure}[p]
\centering
  \import{figs/simsThesis/}{z_summary.pgf}
  \caption{
  }
  \label{design:fig:mttusum}
\end{figure}


In the final stage of the simulation we look at a compression stage. The
trapping potential is adiabatically altered to bring the field minimum closer
to the chip surface. This is achieved by increasing the bias field, but since
this also increases the trap depth the cloud is simultaneously compressed as it
is heated. Since the phase space density is conserved, the particles heat up in
the trap. This phenomenon is made visible in
\mysubfigref{design:fig:simparams}{d}.

The trajectories and potentials are shown in \myfigref{design:fig:Z0i_Z0f}.
Here the compression in number density and expansion in velocity space is clear
by the changing cloud shape over time. The cloud also clearly follows the
potential minimum without losses.

\begin{figure}[p]
\centering
  \import{figs/simsThesis/}{total_summary.pgf}
  \caption{
  }
  \label{design:fig:mttusum}
\end{figure}


\section{Phase-space acceptance of remaining Z traps}
\label{design:transferbetweenzs}

\cm{Revisit this spiel.}

Once trapped in $\mathrm{Z0_f}$, the remaining loading stages are simply
repeated handovers to the smaller wires, with a compression stage carried out
on each wire to bring the molecules closer to the surface. This is a
well-understood and robust procedure that has been used before to load atom
chips with minimal losses~\cite{Reichel2002}.

That said, since our experiment will operate with a significantly lower phase
space density than previous experiments with atoms, it is important to ensure
that our loading procedure will not suffer from any unnecessary losses. These
compressed traps contain hotter molecules, requiring a smaller time step for
simulation. It is therefore easier to directly examine the acceptance of each
trap.

These are shown in \myfigref{design:fig:phasematchinggrid}, starting with
$\mathrm{Z0_f}$ in row (a). The region of the trap that we expect to be
occupied (as determined by the simulation above) is marked with a dashed line.
This cloud of molecules will be adiabatically transferred into $\mathrm{Z1_i}$.

Since the process is adiabatic, we can anticipate that there will be no
reduction in phase space density. The cloud of molecules will be transferred to
$\mathrm{Z1_i}$ with negligible heating. However, the total acceptance of the
trap is less than the expected phase space volume of the cloud, so there
will be particle loss due to spillover.

This was anticipated in section~\ref{design:motion}: phase space density cannot
increase in a conservative potential, and the acceptance of the smallest trap
will be smaller than the initial cloud in the transport trap. So the final
number of particles will be lower than we started with, as calculated in
\myeqref{design:eq:psd_N}.

Next, $\mathrm{Z1_i}$ is adiabatically compressed to $\mathrm{Z1_f}$, with the
molecules held \SI{100}{\micro\meter} from the surface. Again, this
compression is adiabatic, so we expect no change in the phase space density.
Compression increases the trap acceptance, so there is also no loss of
particles. This is further supported by the simulated compression of
$\mathrm{Z0_i}$ to $\mathrm{Z0_f}$ discussed above. The acceptance of
$\mathrm{Z1_I}$ is shown in \mysubfigref{design:fig:phasematchinggrid}{b}, with
the expected occupation shown by the dashed line.

Finally, $\mathrm{Z2}$ is loaded and compressed in exactly the same way as
$\mathrm{Z1}$. The adiabatic handover from $\mathrm{Z1_f}$ to $\mathrm{Z2_i}$
is the final cause of particle loss (due to spillover). Then the trap is
compressed to hold the particles \SI{10}{\micro\meter} from the surface, as
shown in \mysubfigref{design:fig:phasematchinggrid}{c}.


\begin{figure}[htb]
\centering
  \begin{overpic}[page=1]{figs/sims/phase_matching_grid.pdf}
    \put(0,68){(a)}
    \put(0,44.5){(b)}
    \put(0,21){(c)}
  \end{overpic}
  \caption{
    A low bound of the acceptance (solid) and expected occupation (dashed) of
    each Z wire trap. This is calculated by finding the acceptance in the weak
    ($x$) trapping direction. Row (a) shows $\mathrm{Z0_f}$, with occupation
    determined by the above simulation. Molecules will be adiabatically
    transferred to $\mathrm{Z1_i}$, which will be fully occupied. Some
    particles will be lost due to the decrease in acceptance but this is to be
    expected, and phase space density should not increase. In (b) we have the
    acceptance and occupation of $\mathrm{Z1_f}$ following adiabatic
    compression.  Particles will be adiabatically transferred into
    $\mathrm{Z2_i}$, resulting in similar losses to the previous step.  Row (c)
    shows the final acceptance and occupation of $\mathrm{Z2_f}$.
  }
  \label{design:fig:phasematchinggrid}
\end{figure}


\section{Summary}

In this chapter I have presented the design of the chip experiment, starting
with the source of molecules up to the stage of confinement in a Z trap
\SI{10}{\micro\meter} from the chip surface. I have shown that we have created
a loading procedure that will allow us to load this final trap through a series
of wire traps of decreasing size. This is justified by simulation and visual
analysis of the acceptance of the various trapping stages.

The design leaves scope for the addition of microwave guides on a second layer
above the trapping wires, as discussed in chapter~\ref{intro}.
