\cm{Need to ensure I cover this in the LSR intro}
\thesis{These first two paragraphs would probably be better off in the intro,
with an explanation of the transport procedure in the `outline' section.}
As discussed above \thesis{TODO} the cold \CaF{} molecules for our experiment
are produced by the MOT and blue-detuned molasses described in
\inlineref{Truppe2017}. The molecules have a phase space density of
$\rho=\SI{3.4(9)e-12}{}$, with temperature $T=\SI{52(2)}{\micro\kelvin}$ and
direct space density $n=\SI{1.1(3)E5}{\per\centi\meter\cubed}$. These molecules
will then be transported by a pair of anti-Helmhotlz coils mounted on a
\ph{track} over a distance of approximately \SI{1}{\meter} to the chip chamber.
The chip is to be mounted such that its centre is \ph{\SI{3}{\milli\meter}}
above the transport axis, as shown in \ph{TODO: fig!}.

Early experiments conducted by other members of the Centre for Cold Matter have
shown that rubidium atoms can be transported in the same manner without
experiencing any significant heating. This is in agreement with similar
experiments in the literature.~\cite{} In addition we may be able to re-apply
blue-detuned molasses to the molecules after transport to re-cool the molecules
to \SI{50}{\micro\kelvin}. We have assumed a worst-case scenario of operating
with a cloud that has heated during transport to \SI{100}{\micro\kelvin}.

The challenge for us is to load as much of this molecular cloud as possible
into the \cm{microscopic trapping stages above the chip}. We will achieve this
first by loading the cloud into a macroscopic wire trap that is aligned with
the chip. We will then transfer the molecules onto the chip, where a series of
wire traps will be used to manipulate the cloud into the desired position
while minimising losses.

This minimisation of losses is achieved by phase-matching (also referred to as
mode-matching) the traps as they hand over to each other. The principles of
phase-matching are described in detail below, but the basic premise is that the
shapes of the potentials should overlap as much as possible between each trap.
The shape of the potentials is largely determined by the shapes of the wires,
and so the design of the chip is motivated by the loading procedure.

In this chapter I will first explain the theory of phase-matching. I will then
present the design for our chip experiment and show how the phase-matching at
each stage will enable efficient loading of our final trap.

\section{Theory of phase-matching}

\cm{TODO: edit para}
We have established that our chip experiment will rely on a loading procedure
from a magnetostatic trap onto the chip. This is not dissimilar to existing
procedures where atoms and molecules can be loaded from beams into magnetic
traps~\cite{}, storage rings~\cite{} or MOTs~\cite{}. However unlike
experiments performed with atoms, where the number of particles is very high
(\cm{$N=??$}~\cite{}) \CaF experiments typically involve only a few thousand
particles \cite{}. It is therefore important that losses during the loading
procedure are minimised.

\cm{
\begin{itemize}
    \item Phase space as a concept
    \item Liouville theorem (metion conservative potential assumption)
    \item Phase-matching
    \item Adiabatic (?) manipulation of potentials to achieve phase-matching
    \item Limitations: need for cooling/ damping to increase PSD
\end{itemize}
}

\section{Chip design and loading scheme}

The arrangement of the trapping wires on our chip is shown in
\myfigref{design:fig:chip}. It has \ph{no. of wires} wires, each with a height
of \SI{5}{\micro\meter}. The widths of the wires are shown in table
\mytableref{design:table:wires}, along with the maximum achievable current. The
currents are calculated assuming a current density of
$j=\SI{6e10}{\ampere\per\meter\squared}$, which is what has been measured for a
similar multi-layer chip described in \inlineref{Treutlein2008}.

\begin{figure}[h]
\vspace{0.8cm}
\centering
  \ph{pseudo-cartoon of chip design}
  \caption{\ph{todo}}
  \label{design:fig:chip}
\end{figure}

The flange-assembley \ph{see some fig} has also been designed to include a
large U-wire positioned below the chip. This U-wire is capable of maintaining
much larger currents than the chip wires. It was orginally intended for
aligning the cloud with the chip, but it has been decided that this step is no
longer necessary. The U may be useful for later experiments if we wish to trap
atoms far from the surface or form a quadrupole trap.

\section{Vacuum system}

\ph{Need to introduce the MOT chamber and Chip chamber (remember the tweezer
cahmber in the middle), the flange setup and the translation stage.}

\section{Science chip design}

\section{Loading the chip}

\subsection{\CaF{} source}

A \CaF{} MOT will produced in the MOT chamber and cooled with the blue-molasses
described in \inlineref{Truppe2017}. The transport coils will then be turned
on, trapping the cloud of ultra-cold molecules. The translation stage will be
activated and the cloud dragged into the chip chamber.  As per the above, we
anticipate very little heating to occur during this process, but maximisation
of the phase-space density is crucial to loading the chip trap due to the small
trapping region.

\cm{Need to significantly expand on this. Might get away with a footnote saying
we don't know how to do this for the LSR...}
For this reason the translation coils will be switched off and the
blue-molasses reapplied in the chip chamber. Due to the restriction of
optical-access the molasses can only be applied in two-dimensions. The molasses
is applied for \SI{5}{\milli\second}, so the molecules will fall by
approximately \SI{100}{\micro\meter} under gravity.

The large z-wire ($\mathrm{Z0_i}$) will then be switched on, along with the
bias fields. This will trap the molecule cloud approximately
\SI{3}{\milli\meter} above the chip surface. The trapping potential is shown
along with the expected particle distribution in \myfigref{design:fig:bigz}.

\begin{figure}[h]
\centering
  \ph{This figure is 6 subplots for x, y, z potentnial cut-throughs and phase
  space plots of the molecules in the big z after loading.}
  \caption{\ph{todo}}
  \label{design:fig:bigz}
\end{figure}

Once trapped on the chip we will adiabatically alter the potentials to bring
the cloud closer to the chip and to reduce the trap length in the $x$ direction
\cm{hyphen here?}. To do this we can first increase the bias field (thereby
reducing the trapping distance from the chip and compressing the trap) and then
handover to a new, smaller trap. These two processes will be repeatedly applied
until the molecules are trapped in the smallest wire trap \cm{should have a
name/label}.

\subsection{Trap compression}

Increasing the bias field will decrease the distance between the centre of the
trap and the chip surface, as per equation \ph{B = I/r equation ref}. We will
perform this process adiabatically to preserve phase-space density of the
trapped molecules. Here we will present an example of compression in the trap
formed by the $\mathrm{Z0}$ wire.


\subsection{Trap handover}

\cm{clumsy wording here}
Following the transition into trap $\mathrm{Z0_f}$ the next step is to handover
to a smaller wire. This serves two purposes: reducing the cloud size in the $x$
direction and moving to a more narrow wire. The latter step is important
because we aim for the trapping current to have a cross-section that is small
compared to the scale of the cloud and the \cm{trapping height}. Wider wires
are required for traps that operate at higher currents (\ph{see section ??}).
% Don't want to mix terms, I am avoiding calling it a trapping height at the
% moment

This subsection details the handover from trap $\mathrm{Z0_f}$ to trap
$\mathrm{Z1_i}$, and serves as an example of how these handovers can be
performed.


\subsection{Summary of the loading procedure}

\section{Simulating molecules}
\label{design:sim}

\cm{This is basically a methods section, not sure where to put it...}

The molecule cloud was simulated by numerically solving Hamilton's equations
for the classical motion in the magnetic potential of the trap. The Hamiltonian
for a single molecule at time $t$ is
%
\begin{equation}
  H(t, \mathbf{r}, \mathbf{p}) = \frac{\mathbf{p}^2}{2m} + V(t, \mathbf{r}).
\end{equation}
Where the particle has magnetic moment $\mu$, position $\mathbf{r}$ and
momentum $\mathbf{p}$.

The potential $V(t, \mathbf{r}) = \mu B(t, \mathbf{r})$ can be calculated by
considering the traps to be formed of segments of straight wires, each
producing a magnetic field $B_\text{seg}^{(i)}$, and the bias field
$\mathbf{B}_\text{bias}(t)$. The total field is the sum of each of these
segments (with segments that are sufficiently far from the trap centre
neglected) and the bias. The potential is therefore
%
\begin{equation}
  V(t, \mathrm{r}) = \mu B (t, \mathrm{r}) = \mu \left| \sum_i
  B_\text{seg}^{(i)}(t, \mathbf{r}) +
  \mathbf{B}_\text{bias}(t)\right|.
\end{equation}

The field due to a segment of straight wire with length carrying current $I(t)$
\cm{can be written in an analytical form} by integration of the Bio-Savart Law.
This field can be expressed in terms of the angle between the ends of the wire
at $\mathbf{r}$, as is shown in \myfigref{TODO}. The field
is given by~\cite{Griffiths2017}
%
\begin{equation}
  B_\text{seg}(t, \mathbf{r}) = \frac{\mu_0 I(t)}{4\pi
  s_\text{seg}(\mathbf{r})} (\sin(\theta_2)  - \sin(\theta_1))
\end{equation}
here $s_\text{seg}(\mathbf{r})$ is the shortest distance between the segment
and $\mathrm{r}$.

By specifying the shape of the segments, the current profiles and the bias
fields, the potential can be calculated at all times. Hamilton's equations, see
equation \cm{above}, can then be solved by \cm{method} to give the path of a
particle. In all \CaF experiments undertaken up to the time of writing, the
phase space density is small enough that collisions can be neglected, and an
ensemble can be simulated by integrating each particle's motion independently.


\section{Summary}

\ph{TODO, probably just an overview of the final loading procedure}
