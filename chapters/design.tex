\cm{Need to ensure I cover this in the LSR intro}
\thesis{These first two paragraphs would probably be better off in the intro,
with an explanation of the transport procedure in the `outline' section.}

Investigation of novel phenomena, such as the strong-coupling of microwaves to
molecules, will be undertaken in a \ph{dimple trap} as desribed above. However
for this to be done we must be sure that we are able to populate this trap with
cold molecules. This must be done whilst minimising losses and heating of the
cloud.

As such, the science chip is not as simple as a single trap on a substrate, but
consists of a series of wire traps, whose currents and bias fields can be
adiabatiacally altered to efficiently load the chip. The design of the science
chip and this loading procedure therefore go hand in hand.

In this chapter I will present the science chip design and the loading scheme.
I will justify our design choices by presenting simulations and analysis of the
potentials. 

\thesis{
In this chapter I will first explain the theory of phase-matching. I will then
present the design for our chip experiment and show how the phase-matching at
each stage will enable efficient loading of our final trap.
}

\section{Theory of phase-matching}

\cm{TODO: edit para}
We have established that our chip experiment will rely on a loading procedure
from a magnetostatic trap onto the chip. This is not dissimilar to existing
procedures where atoms and molecules can be loaded from beams into magnetic
traps~\cite{}, storage rings~\cite{} or MOTs~\cite{}. However unlike
experiments performed with atoms, where the number of particles is very high
(\cm{$N=??$}~\cite{}) \CaF experiments typically involve only a few thousand
particles \cite{}. It is therefore important that losses during the loading
procedure are minimised.

\cm{
\begin{itemize}
    \item Phase space as a concept
    \item Liouville theorem (metion conservative potential assumption)
    \item Phase-matching
    \item Adiabatic (?) manipulation of potentials to achieve phase-matching
    \item Limitations: need for cooling/ damping to increase PSD
\end{itemize}
}

\section{Vacuum system}

\ph{Need to introduce the MOT chamber and Chip chamber (remember the tweezer
cahmber in the middle), the flange setup and the translation stage.}

As discussed above \thesis{TODO} the cold \CaF{} molecules for our experiment
are produced by the MOT and blue-detuned molasses described in
\inlineref{Truppe2017}. The molecules have a phase space density of
$\rho=\SI{3.4(9)e-12}{}$, with temperature $T=\SI{52(2)}{\micro\kelvin}$ and
direct space density $n=\SI{1.1(3)E5}{\per\centi\meter\cubed}$. These molecules
will then be transported by a pair of anti-Helmhotlz coils mounted on a
\ph{track} over a distance of approximately \SI{1}{\meter} to the chip chamber.
The chip is to be mounted such that its centre is \ph{\SI{3}{\milli\meter}}
above the transport axis, as shown in \ph{TODO: fig!}.

Early experiments conducted by other members of the Centre for Cold Matter have
shown that rubidium atoms can be transported in the same manner without
experiencing any significant heating. This is in agreement with similar
experiments in the literature.~\cite{} In addition we may be able to re-apply
blue-detuned molasses to the molecules after transport to re-cool the molecules
to \SI{50}{\micro\kelvin}. We have assumed a worst-case scenario of operating
with a cloud that has heated during transport to \SI{100}{\micro\kelvin}.

\ph{The chip is to be mounted on a flange-assembly...}

The transport trap will align the molecule cloud with the chip trap, with the
cloud centre approximately \SI{3}{\milli\meter} above the surface. From here
the molecules can be loaded onto the chip trap by adiabatically altering the
trapping potential. This process is outline in \ph{table}. 

\section{Science chip design}

The arrangement of the trapping wires on our chip is shown in
\myfigref{design:fig:chip}. It has \ph{no. of wires} wires, each with a height
of \SI{5}{\micro\meter}. The widths of the wires are shown in table
\mytableref{design:table:wires}, along with the maximum achievable current. The
currents are calculated assuming a current density of
$j=\SI{6e10}{\ampere\per\meter\squared}$, which is what has been measured for a
similar multi-layer chip described in \inlineref{Treutlein2008}.

\begin{figure}[h]
\vspace{0.8cm}
\centering
  \ph{pseudo-cartoon of chip design}
  \caption{\ph{todo}}
  \label{design:fig:chip}
\end{figure}

The flange-assembley \ph{see some fig} has also been designed to include a
large U-wire positioned below the chip. This U-wire is capable of maintaining
much larger currents than the chip wires. It was orginally intended for
aligning the cloud with the chip, but it has been decided that this step is no
longer necessary. The U may be useful for later experiments if we wish to trap
atoms far from the surface or form a quadrupole trap.


\section{Simulating the loading ramps}
\label{design:sim}

To ensure that the chip design will allow for robust loading, the loading
process has been simulated up to the $\mathrm{Z0_f}$ trap. This demonstrates
the two main types of transitions in the loading process: handovers between
traps, and compression within a single trap.

The results of the simulation and the parameters of the ramps are outlined in
\myfigref{design:fig:simparams}. This can be considered in four parts, first
from $t = 0$ to $t=\SI{100}{\milli\second}$ there is a settling period where
the molecules are held in the transport trap. Then there are three
\SI{100}{\milli\second} ramps, each of which is followed by a
\SI{100}{\milli\second} stabilisation period. These three ramps describe the
handover from the transport trap to the U, the U to $\mathrm{Z0_i}$, and the
compression of $\mathrm{Z0_i}$ to $\mathrm{Z0_F}$. They are described in detail
in the following subsections.

\begin{figure}[tbh]
\centering
  \begin{overpic}[page=1]{figs/sims/mtt_u_z_params.pdf}
    \put(32.5,55){(a)}
    \put(45.5,55){(b)}
    \put(63,55){(c)}
    \put(80.5,55){(d)}
  \end{overpic}
  \caption{
    The simulation parameters are shown for loading from the transport trap
    into trap $\mathrm{Z0_f}$. The simulation begins with a stabilisation
    period (a), in which the molecules are held in the transport trap for \SI{100}{\milli\second}. In period (b) the transport
    trap is ramped off and the U is ramped on. There is some heating of
    molecules as they leave the trap, and a \cm{ammount} reduction in phase
    space density. The transfer to $\mathrm{Z0_i}$ occurs in period (c), where
    more loss occurs due to the trap depth being reduced in the $x$
    direction. Finally in (d) the trap is compressed to $\mathrm{Z0_f}$, when
    the molecules are brought from \SI{3}{\milli\meter} to \SI{1}{\milli\meter}
    of the chip surface. The solid lines show the simulation results when
    collisions with the chip surface are ignored, and the dashed line shows
    results when collision with the surface results in loss of the molecule.
  }
  \label{design:fig:simparams}
\end{figure}

\cm{Combine math and sim methods into one section abvoe this one?}
The simulation methods are described in section~\ref{design:simmethods}. This
includes the initialisation procedure which happens before $t=0$. We simulated
$N=1000$ particles, but it is common for a few to be lost during the
initialisation process, hence the starting number for this simulation is
$N(t=0)=999$.

\subsection{Handover to the U trap}

\cm{Follman paper says that a big U is helpful for alignment with coils!}

The first step is to transfer the molecules from the transport coils to the U
wire. This U wire has the largest axial length, and so will be the easiest trap
the transport coils to. The U-wire is also capable of carrying more current
than any of the other wires, which is essential as it is positioned further
from the molecule cloud beneath \cm{above?} the chip.

Between $t=\SI{200}{\milli\second}$ and $t=\SI{300}{\milli\second}$ the current
in the transport coils is adiabtically ramped down while the currents in
the U trap and in the bias coils are ramped up. The molecules are then held
until $t=\SI{400}{\milli\second}$ to ensure that they remain trapped after the
ramp. The simulation parameters are shown in
\mysubfigref{design:fig:simparams}{b}.

\cm{Perhaps a detailed plot of the temp and PSD would also be useful?}

The particle loss during the ramp is approximately 10\% of the total. There is
also some heating, which decreases the phase space density of the cloud. The
cause of this can be seen in \myfigref{design:fig:trans_U}, which shows the
phase-space positions of the molecules at various times throughout the ramp,
and the trapping potential at those times.

\begin{figure}[p]
\centering
  %\setlength\extrarowheight{-3pt}
  \begin{tabular}{c}
    \includegraphics[page=1]{figs/sims/mtt_u_x.pdf} \\
    \includegraphics[page=1,trim=0 0 0 15, clip]{figs/sims/mtt_u_y.pdf} \\
    \includegraphics[page=1,trim=0 0 0 15, clip]{figs/sims/mtt_u_z.pdf}
  \end{tabular}
  \caption{
    Lorem Ipsum is simply dummy text of the printing and typesetting industry.
    Lorem Ipsum has been the industry's standard dummy text ever since the
    1500s, when an unknown printer took a galley of type and scrambled it to
    make a type specimen book. It has survived not only five centuries, but
    also the leap into electronic typesetting, remaining essentially unchanged.
    It was popularised in the 1960s with the release of Letraset sheets
    containing Lorem Ipsum passages, and more recently with desktop publishing
    software like Aldus PageMaker including versions of Lorem Ipsum.
  }
  \label{design:fig:trans_U}
\end{figure}

At \cm{$t=??$} there is a noticeable decrease in the trapping depth in the $y$
direction. This is the cause of the particle leak, and occurs due to the 
due to the superposition of the two fields causing an overall decrease in the
trap depth in the $x$ and $y$ directions. This only occurs for a brief period
during the ramp (between \cm{$t=\SI{10000}{\milli\second}$ and
$t=\SI{129412}{\milli\second}$}). 

In addition to the loss during the ramp, we observe a steady loss between
$t=\SI{200}{\milli\second}$ and $t=\SI{300}{\milli\second}$, but only when the
simulation accounts for collision with the chip surface. This loss due to
collision occurs in the U wire because of its position beneath the chip. This
means that the trap is not sufficiently deep to hold our molecules for an
extended period. However the holding time in this simulation is only for
illustration, and the prolonged trapping in the U trap will not be part of the
final experiment. Hence these losses can be ignored.

\subsection{Handover to $\mathrm{Z0_i}$}
\label{design:sim:U_to_Z0i}


The U to $\mathrm{Z0_i}$ handover occurs in our simulation from
$t=\SI{300}{\milli\second}$ to $t=\SI{400}{\milli\second}$, shown in
\mysubfigref{design:fig:simparams}{c}, \cm{with a detailed view in fig?? .}
The phase space positions of the molecules and the potentials are shown in
\myfigref{design:fig:U_Z0i}.

\begin{figure}[p]
\centering
  %\setlength\extrarowheight{-3pt}
  \begin{tabular}{c}
    \includegraphics[page=1]{figs/sims/u_Z0i_x.pdf} \\
    \includegraphics[page=1,trim=0 0 0 15, clip]{figs/sims/u_Z0i_y.pdf} \\
    \includegraphics[page=1,trim=0 0 0 15, clip]{figs/sims/u_Z0i_z.pdf}
  \end{tabular}
  \caption{
    \ph{TODO}
  }
  \label{design:fig:U_Z0i}
\end{figure}

In the ramp between the transport trap and the U trap, we saw that a hole was
introduced in the trapping potential due to interference between the two
trapping fields. We will see a similar effect when transferring from the U trap
to trap $\mathrm{Z0_i}$.  This time the reduction in depth appears in the $x$
direction, and is due to the current in the U and Z wires opposing each other
on the positive $x$ side. This can be seen in \myfigref{design:fig:U_Z0i} at
$t=\SI{375}{\milli\second}$.

In \myfigref{design:fig:simparams} we see that the loss is again about 10\% of
the particles, however there is no significant change in the phase space
density. The acceptance of trap $\mathrm{Z0_i}$ is smaller than the emittance
of the U, so this is to be expected.

\subsection{Compression to $\mathrm{Z0_f}$}

In the final stage of the simulation we look at a compression stage. The
trapping potential is adiabatically altered to bring the field minimum closer
to the chip surface. This is achieved by increasing the bias field, but since
this also increases the trap depth the cloud is simultaneously compressed as it
is heated. Since the phase space density is conserved, the particles heat up in
the trap. This phenomenom is made clear in \mysubfigref{design:fig:simparams}.

The trajectories and potentials are shown in \myfigref{design:fig:Z0i_Z0f}.
Here the compression in number density and expansion in velocity space is clear
by the changing cloud shape over time. The cloud also clearly follows the
potential minimum without losses.

\begin{figure}[p]
\centering
  \begin{tabular}{c}
    \includegraphics[page=1]{figs/sims/Z0i_Z0f_x.pdf} \\
    \includegraphics[page=1,trim=0 0 0 15, clip]{figs/sims/Z0i_Z0f_y.pdf} \\
    \includegraphics[page=1,trim=0 0 0 15, clip]{figs/sims/Z0i_Z0f_z.pdf}
  \end{tabular}
  \caption{\ph{todo}}
  \label{design:fig:Z0i_Z0f}
\end{figure}



\subsection{Further loading into the dimple trap}

Once trapped on the chip we will adiabatically alter the potentials to bring
the cloud closer to the chip and to reduce the trap length in the $x$ direction
\cm{hyphen here?}.  Increasing the bias field will decrease the distance
between the centre of the trap and the chip surface, as per equation \ph{B =
I/r equation ref}. We will perform this process adiabatically to preserve
phase-space density of the trapped molecules. The simulation follows from the
transfer into trap $\mathrm{Z0_i}$ as described in
section~\ref{design:sim:U_to_Z0i}.

\cm{clumsy wording here}
Following the transition into trap $\mathrm{Z0_f}$ the next step is to handover
to a smaller wire. This serves two purposes: reducing the cloud size in the $x$
direction and moving to a more narrow wire. The latter step is important
because we aim for the trapping current to have a cross-section that is small
compared to the scale of the cloud and the \cm{trapping height}. Wider wires
are required for traps that operate at higher currents (\ph{see section ??}).
% Don't want to mix terms, I am avoiding calling it a trapping height at the
% moment

Transfer between Z traps has been shown to be a robust process that can be
underatken with minimal losses.~\cite{} However it is important to ensure that
across each step the phase space emittance of one trap overlaps with the phase
space acceptance of the next. It is sufficient to show that the sepatrix for
the first trap encloses the one it will hand over to.

\subsection{Direct loading into $\mathrm{Z0_i}$}

Let us consider whether we can forgo the U trap entirely and load directly from
the transport trap into $\mathrm{Z0_i}$.

\ph{Do particle loss and PSD plots for direct loading to Z0i and going via the
U WITH the surface losses on for both. Put this here and talk about it.}

Following trapping in $\mathrm{Z0_i}$, the loading procedure would continue
with compression into $\mathrm{Z0_f}$ and onwards as was discussed above.

\section{Simulation methods}
\label{design:simmethods}

\cm{This is basically a methods section, not sure where to put it...}

The molecule cloud was simulated by numerically solving Hamilton's equations
for the classical motion in the magnetic potential of the trap. The Hamiltonian
for a single molecule at time $t$ is
%
\begin{equation}
  H(t, \mathbf{r}, \mathbf{p}) = \frac{\mathbf{p}^2}{2m} + V(t, \mathbf{r}).
\end{equation}
Where the particle has magnetic moment $\mu$, position $\mathbf{r}$ and
momentum $\mathbf{p}$.

The potential $V(t, \mathbf{r}) = \mu B(t, \mathbf{r})$ can be calculated by
considering the traps to be formed of segments of straight wires, each
producing a magnetic field $B_\text{seg}^{(i)}$, and the bias field
$\mathbf{B}_\text{bias}(t)$. The total field is the sum of each of these
segments (with segments that are sufficiently far from the trap centre
neglected) and the bias. The potential is therefore
%
\begin{equation}
  V(t, \mathrm{r}) = \mu B (t, \mathrm{r}) = \mu \left| \sum_i
  B_\text{seg}^{(i)}(t, \mathbf{r}) +
  \mathbf{B}_\text{bias}(t)\right|.
\end{equation}

The field due to a segment of straight wire with length carrying current $I(t)$
\cm{can be written in an analytical form} by integration of the Bio-Savart Law.
This field can be expressed in terms of the angle between the ends of the wire
at $\mathbf{r}$, as is shown in \myfigref{TODO}. The field
is given by~\cite{Griffiths2017}
%
\begin{equation}
  B_\text{seg}(t, \mathbf{r}) = \frac{\mu_0 I(t)}{4\pi
  s_\text{seg}(\mathbf{r})} (\sin(\theta_2)  - \sin(\theta_1))
\end{equation}
here $s_\text{seg}(\mathbf{r})$ is the shortest distance between the segment
and $\mathrm{r}$.

By specifying the shape of the segments, the current profiles and the bias
fields, the potential can be calculated at all times. Hamilton's equations, see
equation \cm{above}, can then be solved by \cm{method} to give the path of a
particle. In all \CaF experiments undertaken up to the time of writing, the
phase space density is small enough that collisions can be neglected, and an
ensemble can be simulated by integrating each particle's motion independently.

\subsection{Cloud initialisation}


\section{Summary}

\ph{TODO, probably just an overview of the final loading procedure}
