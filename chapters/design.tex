\cm{Need to ensure I cover this in the LSR intro}
\thesis{These first two paragraphs would probably be better off in the intro,
with an explanation of the transport procedure in the `outline' section.}
As discussed above \thesis{TODO} the cold \CaF{} molecules for our experiment
are produced by the MOT and blue-detuned molasses described in
\inlineref{Truppe2017}. The molecules have a phase space density of
$\rho=\SI{3.4(9)e-12}{}$, with temperature $T=\SI{52(2)}{\micro\kelvin}$ and
direct space density $n=\SI{1.1(3)E5}{\per\centi\meter\cubed}$. These molecules
will then be transported by a pair of anti-Helmhotlz coils mounted on a
\ph{track} over a distance of approximately \SI{1}{\meter} to the chip chamber.
The chip is to be mounted such that its centre is \ph{\SI{3}{\milli\meter}}
above the transport axis, as shown in \ph{TODO: fig!}.

Early experiments conducted by other members of the Centre for Cold Matter have
shown that rubidium atoms can be transported in the same manner without
experiencing any significant heating. This is in agreement with similar
experiments in the literature.~\cite{} In addition we may be able to re-apply
blue-detuned molasses to the molecules after transport to re-cool the molecules
to \SI{50}{\micro\kelvin}. We have assumed a worst-case scenario of operating
with a cloud that has heated during transport to \SI{100}{\micro\kelvin}.

The challenge for us is to load as much of this molecular cloud as possible
into the \cm{microscopic trapping stages above the chip}. We will achieve this
first by loading the cloud into a macroscopic wire trap that is aligned with
the chip. We will then transfer the molecules onto the chip, where a series of
wire traps will be used to manipulate the cloud into the desired position
while minimising losses.

This minimisation of losses is achieved by phase-matching (also referred to as
mode-matching) the traps as they hand over to each other. The principles of
phase-matching are described in detail below, but the basic premise is that the
shapes of the potentials should overlap as much as possible between each trap.
The shape of the potentials is largely determined by the shapes of the wires,
and so the design of the chip is motivated by the loading procedure.

In this chapter I will first explain the theory of phase-matching. I will then
present the design for our chip experiment and show how the phase-matching at
each stage will enable efficient loading of our final trap.

\section{Theory of phase-matching}

\cm{TODO: edit para}
We have established that our chip experiment will rely on a loading procedure
from a magnetostatic trap onto the chip. This is not dissimilar to existing
procedures where atoms and molecules can be loaded from beams into magnetic
traps~\cite{}, storage rings~\cite{} or MOTs~\cite{}. However unlike
experiments performed with atoms, where the number of particles is very high
(\cm{$N=??$}~\cite{}) \CaF experiments typically involve only a few thousand
particles \cite{}. It is therefore important that losses during the loading
procedure are minimised.

\cm{
\begin{itemize}
    \item Phase space as a concept
    \item Liouville theorem (metion conservative potential assumption)
    \item Phase-matching
    \item Adiabatic (?) manipulation of potentials to achieve phase-matching
    \item Limitations: need for cooling/ damping to increase PSD
\end{itemize}
}

\section{Chip design and loading scheme}

The arrangement of the trapping wires on our chip is shown in
\myfigref{design:fig:chip}. It has \ph{no. of wires} wires, each with a height
of \SI{5}{\micro\meter}. The widths of the wires are shown in table
\mytableref{design:table:wires}, along with the maximum achievable current. The
currents are calculated assuming a current density of
$j=\SI{6e10}{\ampere\per\meter\squared}$, which is what has been measured for a
similar multi-layer chip described in \inlineref{Treutlein2008}.

\begin{figure}[h]
\vspace{0.8cm}
\centering
  \ph{pseudo-cartoon of chip design}
  \caption{\ph{todo}}
  \label{design:fig:chip}
\end{figure}

The flange-assembley \ph{see some fig} has also been designed to include a
large U-wire positioned below the chip. This U-wire is capable of maintaining
much larger currents than the chip wires, and will be used as an alignment
stage between the transport coils and the chip trap.

\subsection{Initial cloud alignment}

The transport coils will bring the cloud of cold molecules from the MOT chamber
into the chip chamber. As per the above, we anticipate very little heating to
occur during this process, but we will assume a factor of two increase to the
temperature in the worst-case.

The first transfer will be from the transport coils to the U trap. The current
in the transport coils will be adiabtically ramped down while the currents in
the U trap and in the bias coils are ramped up. The U-trap operates at
\SI{100}{\ampere} and is \SI{7}{\milli\meter} beneath the cloud centre. The
bias field required to create the trap is $\tilde{B} \approx (0, 22,
14)\,\si{\gauss}$. The current and field ramps are depicted in \ph{some
figure}.

The potential of the transport coils and U-potential are shown in
\myfigref{design:fig:Upot} real space and as a contour-plot in phase space. 
We have assummed that the cloud will be at a temperature of
\SI{100}{\micro\kelvin}, so we can expect the cloud to occupy the entire phase
space volume contained within the \SI{100}{\micro\kelvin} surface of the
potential. In \myfigref{design:fig:Umatch} these surfaces are shown for both
traps by contours in the $(q_i, p_i)$ planes for $i\in\{x, y, z\}$. 

\begin{figure}[h]
\centering
  \ph{This fig needs to show the U and MTT potentials with the acceptance and
  the 100uK surface highlighted.}
  \caption{\ph{todo}}
  \label{design:fig:chip}
\end{figure}

To verify the acceptance of the U trap we have simulated the motion of
particles in the U-potential. The initial positions and velocities of
\ph{$N=?$} particles were uniformly distributed within the region $x \in \{\}$,
\ph{etc.} \ph{$T\in\{??\}$}. This completely covers the supposed acceptance from
the trap potential. The motion of the particles was then simulated as described
in section~\ref{design:sim}. \ph{I hope that... the particles that were trapped
were those whose initial positions were inside of the sepatrix.} The results of
this simulation are shown in \myfigref{design:fig:Usim}.

\begin{figure}[h]
\centering
  \ph{This should show a better version of my initial-final PSA diagrams for
  the U trap as a proof of concept.}
  \caption{\ph{todo}}
  \label{design:fig:Usim}
\end{figure}

\cm{Clumsy wording here}.
It might seem that the phase-matching of these potentials is sufficient for us
to choose a simple ramp of currents and parameters between these two, but we
have to be aware of the path that we take between the two stages. First, the
change should be adiabatic\footnote{An instantaneous change may well
suffice~\cite{} but our coils cannot be reliably ramped on a sufficiently short
timescale.} and second, the molecules must remain trapped around the same point
during the change.

For the transfer from the transport trap to the U-trap a na\"ive choice might
be to perform a linear ramp of currents and fields such as is depicted in
\myfigref{design:fab:badramp} \cm{subfig?}. If we were to do this, then a large
number of particles would be lost due to a second local minimum that
is introduced by the presence of the bias field at the same time as the
anti-Helmholtz coil. Particles trapped in this second minimum can be carried
away from the trap centre and lost. This process is shown in
\myfigref{design:fab:badramp}.

\begin{figure}[h]
\centering
  \ph{Three subfigs: the graph of the ramp params through time, the hole in the
  y potential evolving in time, a shapshot of the particles in (y, ydot) plane
  evolving through time. See notepad.}
  \caption{\ph{todo}}
  \label{design:fig:badramp}
\end{figure}

In this case there are numerous ways to avoid the particle loss. We have
elected to delay the increase in the $z$ bias field. \ph{Expand and point to
fig. which should be similar to the last one...}

\ph{Wrap up about how we have achieved alignment with the chip because we are
now in the U. I might need to make some more points about how this is robust
wrt the position of the MTT.}

% Something like this?
\cm{ I think I need to give better explanation about the parameter ramps, I
chuck this language around without explaining or defining it.}
\ph{
The above discussion serves as an example of how to design a ramp from one trap to
the other. It transpires that the remaining transfers are more simple and a
linear ramp of the parameters is sufficient.}

%TODO
% I Think we can just present the rest of the ramp at this stage and assert
% that everything is OK. Nobody wants to read about 5 different handovers. That
% said I should actually check that all this is OK.

\subsection{Trapping on the science chip}

\ph{TODO, inc. simulating an adiabatic compression}

Show phase matching between Z traps once we are on the thin wires (and dimple
trap). At thsi point, current is the same so lower traps will always be deeper
(unless we choose to lower the current I guess, if so then we always have the
same depth??)



\section{Simulating molecules}
\label{design:sim}

\cm{This is basically a methods section, not sure where to put it...}

The molecule cloud was simulated by numerically solving Hamilton's equations
for the classical motion in the magnetic potential of the trap. The Hamiltonian
for a single molecule at time $t$ is
%
\begin{equation}
  H(t, \mathbf{r}, \mathbf{p}) = \frac{\mathbf{p}^2}{2m} + V(t, \mathbf{r}).
\end{equation}
Where the particle has magnetic moment $\mu$, position $\mathbf{r}$ and
momentum $\mathbf{p}$.

The potential $V(t, \mathbf{r}) = \mu B(t, \mathbf{r})$ can be calculated by
considering the traps to be formed of segments of straight wires, each
producing a magnetic field $B_\text{seg}^{(i)}$, and the bias field
$\mathbf{B}_\text{bias}(t)$. The total field is the sum of each of these
segments (with segments that are sufficiently far from the trap centre
neglected) and the bias. The potential is therefore
%
\begin{equation}
  V(t, \mathrm{r}) = \mu B (t, \mathrm{r}) = \mu \left| \sum_i
  B_\text{seg}^{(i)}(t, \mathbf{r}) +
  \mathbf{B}_\text{bias}(t)\right|.
\end{equation}

The field due to a segment of straight wire with length carrying current $I(t)$
\cm{can be written in an analytical form} by integration of the Bio-Savart Law.
This field can be expressed in terms of the angle between the ends of the wire
at $\mathbf{r}$, as is shown in \myfigref{TODO}. The field
is given by~\cite{Griffiths2017}
%
\begin{equation}
  B_\text{seg}(t, \mathbf{r}) = \frac{\mu_0 I(t)}{4\pi
  s_\text{seg}(\mathbf{r})} (\sin(\theta_2)  - \sin(\theta_1))
\end{equation}
here $s_\text{seg}(\mathbf{r})$ is the shortest distance between the segment
and $\mathrm{r}$.

By specifying the shape of the segments, the current profiles and the bias
fields, the potential can be calculated at all times. Hamilton's equations, see
equation \cm{above}, can then be solved by \cm{method} to give the path of a
particle. In all \CaF experiments undertaken up to the time of writing, the
phase space density is small enough that collisions can be neglected, and an
ensemble can be simulated by integrating each particle's motion independently.


\section{Summary}

\ph{TODO, probably just an overview of the final loading procedure}
