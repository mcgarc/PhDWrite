We have established that our chip experiment will rely on a loading procedure
from a magnetostatic trap onto the chip. This is not dissimilar to existing
procedures where atoms and molecules can be loaded from beams into magnetic
traps~\cite{}, storage rings~\cite{} or MOTs~\cite{}. However unlike
experiments performed with atoms, where the number of particles is very high
(\cm{$N=??$}~\cite{}) \CaF experiments typically involve only a few thousand
particles \cite{}. It is therefore important that losses during the loading
procedure are minimised.

The chip will incorporate a series of wire traps, which will allow granular
control of the potential through the ramping of currents in each wire. This is
a common technique often employed in atom chips.~\cite{} We also include an
intermediary loading stage between the anti-Helmhotlz coils and the chip: a
large U-wire trap integrated into the \cm{subchip (housing??)}. This will
ensure alignment between the macroscopic and microscopic traps.

Designing the trapping wires on the chip will allow us to ensure a low-loss
% In this cite, is it better to cite his textbook chapter of a paper? Maybe the 
% Lichtenberg book?
loading procedure by phase-matching each stage.~\cite{Crompvoets2005} That is
as we transfer molecules from one trap (be it macroscopic or microscopic) to
the next, that the potentials are suitably overlapped so that the majority of
the cloud remains trapped throughout.

In this chapter, I will further explain the principes of phase matching, and
how the chip has been designed to allow this. I will also present simulations
used to model traps, determining their phase-space acceptance and emmittance.
The final design of the chip trap will be presented.

\section{Theory of phase-matching}

\cm{
\begin{itemize}
    \item Phase space as a concept
    \item Liouville theorem (metion conservative potential assumption)
    \item Phase-matching
    \item Adiabatic (?) manipulation of potentials to achieve phase-matching
    \item Limitations: need for cooling/ damping to increase PSD
\end{itemize}
}

\section{Simulating molecules}

The molecule cloud was simulated by numerically solving Hamilton's equations
for the classical motion in the magnetic potential of the trap. The Hamiltonian
for a single molecule at time $t$ is
%
\begin{equation}
  H(t, \mathbf{r}, \mathbf{p}) = \frac{\mathbf{p}^2}{2m} + V(t, \mathbf{r}).
\end{equation}
Where the particle has magnetic moment $\mu$, position $\mathbf{r}$ and
momentum $\mathbf{p}$.

The potential $V(t, \mathbf{r}) = \mu B(t, \mathbf{r})$ can be calculated by
considering the traps to be formed of segments of straight wires, each
producing a magnetic field $B_\text{seg}^{(i)}$, and the bias field
$\mathbf{B}_\text{bias}(t)$. The total field is the sum of each of these
segments (with segments that are sufficiently far from the trap centre
neglected) and the bias. The potential is therefore
%
\begin{equation}
  V(t, \mathrm{r}) = \mu B (t, \mathrm{r}) = \mu \left| \sum_i
  B_\text{seg}^{(i)}(t, \mathbf{r}) +
  \mathbf{B}_\text{bias}(t)\right|.
\end{equation}

The field due to a segment of straight wire with length carrying current $I(t)$
\cm{can be written in an analytical form} by integration of the Bio-Savart Law.
This field can be expressed in terms of the angle between the ends of the wire
at $\mathbf{r}$, as is shown in \myfigref{TODO}. The field
is given by~\cite{Griffiths2017}
%
\begin{equation}
  B_\text{seg}(t, \mathbf{r}) = \frac{\mu_0 I(t)}{4\pi
  s_\text{seg}(\mathbf{r})} (\sin(\theta_2)  - \sin(\theta_1))
\end{equation}
here $s_\text{seg}(\mathbf{r})$ is the shortest distance between the segment
and $\mathrm{r}$.

By specifying the shape of the segments, the current profiles and the bias
fields, the potential can be calculated at all times. Hamilton's equations, see
equation \cm{above}, can then be solved by \cm{method} to give the path of a
particle. In all \CaF experiments undertaken up to the time of writing, the
phase space density is small enough that collisions can be neglected, and an
ensemble can be simulated by integrating each particle's motion independently.

\section{Designing the chip trap}

\cm{
%
Introduce prototype design
%
\begin{itemize}
    \item Overlaid Z traps
    \item Current limited by central section, but hope that the wider sections
      will allow good conduction of heat and higher current density (?) than
      reported by Treutlein
    \item Other benefits, e.g. trap centres stack on top of each other, same
      net current at all time. A very simple design
    \item PSA simulation
    \item Why this design doesn't work: can't get enough current for good
      overlap
    \item Is gradually decreasing the length of the axis even really necessary?
\end{itemize}
%
Introduce version two
%
\begin{itemize}
    \item Wider Z on chip for up to 30A
    \item Why this current? (Original justification was to match the y
      direction bias field but we need to adiabatically change the other bias
      fields, but I think that this is equivalent to having the same depth as
      the lower U)
    \item Show better phase matching U to Z
    \item Show phase matching to the thin Z wires, justify the currents here
      (probably that width of wire must be small compared to height)
\end{itemize}
%
Show phase matching between Z traps once we are on the thin wires (and dimple
trap). At thsi point, current is the same so lower traps will always be deeper
(unless we choose to lower the current I guess, if so then we always have the
same depth??)
%
}

\section{Desiging the loading scheme}

\cm{
%
adiabatic vs. sudden changes (justify with simulation?)
%
Walk through the loading scheme
%
Should I show a simulation of the full loading procedure?
%
}

\section{Summary}
