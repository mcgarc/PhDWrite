\cm{Need to ensure I cover this in the LSR intro}
As discussed above \thesis{TODO} the cold \CaF{} molecules for our experiment
are produced by the MOT and blue-detuned molasses described in
\inlineref{Truppe2017}. The molecules have a phase space density of
$\rho=\SI{3.4(9)e-12}{}$, with temperature $T=\SI{52(2)}{\micro\kelvin}$ and
direct space density $n=\SI{1.1(3)E5}{\per\centi\meter\cubed}$. These molecules
will then be transported by a pair of anti-Helmhotlz coils mounted on a
\ph{track} over a distance of approximately \SI{1}{\meter} to the chip chamber.
The chip is to be mounted such that its centre is \ph{\SI{3}{\milli\meter}}
above the transport axis, as shown in \ph{TODO: fig!}.

\cm{Switching between actigve and passive voice is bad}
This chapter will describe how we have designed the science chip to facilitate
the loading of molecules from the transport coils into a tightly-confined chip
trap. First I will outline the theory of phase-matching, that is optimising the
shape of the potentials to avoid unnecessary loss during loading. Then I will
present the chip design and the loading procedure. \cm{Might want to break this
down a bit...}


\section{Theory of phase-matching}

\cm{TODO: edit para}
We have established that our chip experiment will rely on a loading procedure
from a magnetostatic trap onto the chip. This is not dissimilar to existing
procedures where atoms and molecules can be loaded from beams into magnetic
traps~\cite{}, storage rings~\cite{} or MOTs~\cite{}. However unlike
experiments performed with atoms, where the number of particles is very high
(\cm{$N=??$}~\cite{}) \CaF experiments typically involve only a few thousand
particles \cite{}. It is therefore important that losses during the loading
procedure are minimised.

\cm{
\begin{itemize}
    \item Phase space as a concept
    \item Liouville theorem (metion conservative potential assumption)
    \item Phase-matching
    \item Adiabatic (?) manipulation of potentials to achieve phase-matching
    \item Limitations: need for cooling/ damping to increase PSD
\end{itemize}
}


\section{Trap design}

To achieve good phase-matching, our chip will host a series of wire traps,
which will allow granular control of the potential through the ramping of
currents in each wire and variation of the bias field. This is a common
technique often employed in atom chips.~\cite{} We also include an intermediary
loading stage between the anti-Helmhotlz coils and the chip: a large U-wire
trap integrated into the \cm{subchip (housing??)}. This will ensure alignment
between the macroscopic and microscopic traps.

Designing the trapping wires on the chip will allow us to ensure a low-loss
% In this cite, is it better to cite his textbook chapter of a paper? Maybe the 
% Lichtenberg book? \cm{Probably some completely different cite to be honest.}
loading procedure by phase-matching each stage.~\cite{Crompvoets2005} That is
as we transfer molecules from one trap (be it macroscopic or microscopic) to
the next, that the potentials are suitably overlapped so that the majority of
the cloud remains trapped throughout.

\subsection{From transport coils to a U wire}

\thesis{Update this, probably with my data}
Early experiments conducted by other members of the Centre for Cold Matter have
shown that rubidium atoms can be transported in the same manner without
experiencing any significant heating. This is in agreement with similar
experiments in the literature.~\cite{} In addition we may be able to re-apply
blue-detuned molasses to the molecules after transport to re-cool the molecules
to \SI{50}{\micro\kelvin}. We have assumed a worst-case scenario of operating
with a cloud that has heated during transport to \SI{100}{\micro\kelvin}.


\subsection{From a U wire to the science chip}

\ph{TODO, inc. simulating a phase space acceptance to justify just looking at
the potentials}


\subsection{Compression on the science chip}

\ph{TODO, inc. simulating an adiabatic compression}

Show phase matching between Z traps once we are on the thin wires (and dimple
trap). At thsi point, current is the same so lower traps will always be deeper
(unless we choose to lower the current I guess, if so then we always have the
same depth??)



\section{Simulating molecules}

\cm{This is basically a methods section, not sure where to put it...}

The molecule cloud was simulated by numerically solving Hamilton's equations
for the classical motion in the magnetic potential of the trap. The Hamiltonian
for a single molecule at time $t$ is
%
\begin{equation}
  H(t, \mathbf{r}, \mathbf{p}) = \frac{\mathbf{p}^2}{2m} + V(t, \mathbf{r}).
\end{equation}
Where the particle has magnetic moment $\mu$, position $\mathbf{r}$ and
momentum $\mathbf{p}$.

The potential $V(t, \mathbf{r}) = \mu B(t, \mathbf{r})$ can be calculated by
considering the traps to be formed of segments of straight wires, each
producing a magnetic field $B_\text{seg}^{(i)}$, and the bias field
$\mathbf{B}_\text{bias}(t)$. The total field is the sum of each of these
segments (with segments that are sufficiently far from the trap centre
neglected) and the bias. The potential is therefore
%
\begin{equation}
  V(t, \mathrm{r}) = \mu B (t, \mathrm{r}) = \mu \left| \sum_i
  B_\text{seg}^{(i)}(t, \mathbf{r}) +
  \mathbf{B}_\text{bias}(t)\right|.
\end{equation}

The field due to a segment of straight wire with length carrying current $I(t)$
\cm{can be written in an analytical form} by integration of the Bio-Savart Law.
This field can be expressed in terms of the angle between the ends of the wire
at $\mathbf{r}$, as is shown in \myfigref{TODO}. The field
is given by~\cite{Griffiths2017}
%
\begin{equation}
  B_\text{seg}(t, \mathbf{r}) = \frac{\mu_0 I(t)}{4\pi
  s_\text{seg}(\mathbf{r})} (\sin(\theta_2)  - \sin(\theta_1))
\end{equation}
here $s_\text{seg}(\mathbf{r})$ is the shortest distance between the segment
and $\mathrm{r}$.

By specifying the shape of the segments, the current profiles and the bias
fields, the potential can be calculated at all times. Hamilton's equations, see
equation \cm{above}, can then be solved by \cm{method} to give the path of a
particle. In all \CaF experiments undertaken up to the time of writing, the
phase space density is small enough that collisions can be neglected, and an
ensemble can be simulated by integrating each particle's motion independently.


\section{Summary}

\ph{TODO, probably just an overview of the final loading procedure}
