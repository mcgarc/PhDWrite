\subsection{Design for a molecular chip trap}

We have designed various multiple-layer chips for prototyping. A series of
Z-traps are to be used for loading molecules from an external trap (see
section~\ref{experiment:loading}) and a CPW for control of the molecules' states
is incorporated on an upper layer. The chip with the trapping wires is shown in
\myfigref{experiment:fig:chipdesign} and is referred to as the science chip.

\begin{figure}[tph]
  \includegraphics[width=0.8\textwidth]{./figs/z_trap_fanouts_inset_scale.png}
  \cm{Can't really see this inset, even when zoomed in. Need to redo this fig.
  anyway though.}
  \caption{
    A design of a molecule chip, with six overlaid Z-trapping wires and one
    dimple wire (gold) on the lower layer, and a CPW (insulating section in
    blue) on the upper layer.  The lower layer is to be fabricated by
    photolithography and electroplating, then spin coated with polyimide so that
    the second layer can be fabricated above. The Z-traps have varying lengths
    and thicknesses to allow higher current in larger traps, with compression as
    molecules are transferred into smaller traps. The CPW is tapered down to
    localise the microwave field around the smallest traps. Inset: a zoomed view
    of the trapping region.
  }
  \label{experiment:fig:chipdesign}
\end{figure}

A standard \SI{4}{\inch} wafer has space for twelve science chips (each one being
a square with sides of length \SI{2}{\centi\metre}). In the prototype wafer
design, each one has the same set of trapping wires as depicted in
\myfigref{experiment:fig:chipdesign}, but the CPW designs vary. The parameters
changed are the minimum width of the centre conductor (\SI{10}{\micro\metre},
\SI{20}{\micro\metre}, \SI{50}{\micro\metre}); the taper length
(\SI{1}{\milli\metre}, \SI{3}{\milli\metre}); and the orientation of the CPW
track near the molecules (perpendicular or parallel to the trap axis). The full
wafer is shown in \myfigref{experiment:fig:waferdesign}.

The trapping wire design is a series of seven Z-wire traps. Each one is designed
with large fan-outs to aid in conduction of heat and maximise the achievable
current density. A section of polyimide will be scratched of to reveal these so
that they can be wire-bonded to the holding chip (see below). The outer traps
are to be used for trapping at greater heights, meaning they need not be as
narrow as the inner traps. Increasing the wire width means that a greater
current can be achieved in the outer traps, which is useful for forming deep
traps at heights of several hundred micrometres. The specifications of the traps
are shown in \mytabref{experiment:table:traps}.

\begin{table}
  \centering
  \begin{tabular*}{0.7\textwidth}{| @{\extracolsep{\fill} }c c l|}
   \hline
    Axial length & Width &  Notes \\
    (\si{\milli\metre}) & (\si{\micro\metre}) & \\
    \hline
    30& 60 & Loaded from intermediate U-trap\\
    20& 50 & \\
    10& 40 & \\
    6 & 30 & \\
    3 & 20 & \\
    1 & 10 & \\
    0 & 10 & Dimple trap (\SI{3}{\micro\metre} transverse wire)\\
 \hline
\end{tabular*}
  \caption{The axial lengths and widths of Z-traps on traps on the
  science chip. All wires are fabricated by photolithography followed by
  electroplating, for a height of \SI{1}{\micro\metre}.
  }
  \label{experiment:table:traps}
\end{table}

The goal of this prototype is to characterise the CPWs to check that the
predicted values are met. In particular, we are interested in how the material
under the polyimide (HiRes Si substrate and trapping wires) will affect the
microwave field. We will also be able to test the trapping wires and the
behaviour of the science chip in vacuum. We are optimistic that these prototypes could
ultimately be used for trapping of molecules.

\begin{figure}[tph]
  \includegraphics[width=0.8\textwidth]{./figs/wafer.png}
  \caption{
    The prototype wafer design, combining all variations of the centre conductor
    width, taper lengths and CPW orientations (details in the main text). A
    \SI{4}{\inch} HiRes Si wafer will be used as a substrate. Lower level
    features (yellow) will be fabricated by photolithography followed by
    electroplating. The higher level features (blue) will be fabricated on an
    intermediary layer of polyimide. Note the chip outlines are marked with
    grey, and the alignment features in each corner of the chip to be fabricated
    on the lower layer.
  }
  \label{experiment:fig:waferdesign}
\end{figure}

\thesis{How many molecules can we have in the traps and what temperatures?}

The science chip is to be mounted on a heavily modified vacuum flange, shown in
\myfigref{experiment:fig:flange}. This flange will incorporate feedthroughs for
the trapping currents and microwaves, as well as a large U-wire, which will be
used as an intermediate trap in the loading procedure (detailed below). Currents
will be brought in through a nineteen-pin feedthrough  to power the bias coils
and on-chip trapping wires. These currents will feed onto a holding chip, which
will then connect to the science chip via wirebonds. Separate feedthroughs for
the microwaves and U-wire \thesis{(state supplier)} feed onto CPWs on the
holding chip and the U-wire respectively.

\begin{figure}[ht]
  \includegraphics[width=0.8\textwidth]{./figs/FlangeAssemblyForPresentationTopRight.png}
  \caption{
    Flange assembly for mounting a molecule chip. The science chip will be
    mounted in the centre of the holding chip and wire-bonds will allow current
    to pass between the two. Current will be passed to the holding chip via the
    19-pin feedthrough. There are also high frequency microwave feedthroughs so
    that microwaves can be passed to a CPW on the holding chip, and again onto
    the science chip. Two bias coils will be mounted on the flange, with the
    other required coils attached to the chamber (not pictured). The
    intermediary U-trap is positioned directly beneath the science chip and is
    powered by separate current feedthroughs. The flange will be mounted with
    the chip facing down, so that molecules can be dropped and imaged as they
    fall. This CAD was undertaken by other members of the group.
  }
  \label{experiment:fig:flange}
\end{figure}

\thesis{
\subsection{Fabrication}
Will need full description of fabrication process
\begin{itemize}
  \item Photolith
  \item Electroplating
  \item Multilayers
\end{itemize}
}

\subsection{Loading scheme}
\label{experiment:loading}

\thesis{Will probably want to revisit this whole section, but an important thing
to do will be to ensure any papers published about this in meantime are cited
accordingly.}

\CaF{} molecules will be loaded onto the chip from a MOT similar to
that described in reference~\cite{Truppe2017}. \cm{Is this OK? Maybe make a
macro to deal with it...} In the MOT chamber it is planned
to construct a dipole trap, which will be used to increase phase space density
of the molecule cloud

\thesis{Need to figure out specific numbers for phase space density we can
achieve and those required for loading. Include losses and heating from loading
in this calculation.}

Our source of \CaF{} molecules is a MOT similar to that discussed in
reference~\cite{Truppe2017}. The MOT planned to be used for this experiment is
to be used for separate experiments investigating sympathetic cooling between
\esRb{} atoms and \CaF{} in a dipole trap. This is expected to produce a \CaF{}
cloud suitable for loading onto the chip.

\thesis{This (and probably the rest of this section) will need to be updated and
the tense changed.}
The cloud will then be transferred into a magnetic transport trap (MTT) formed of
a pair of anti-Helmholtz coils external to the chambers. The dipole trap will be
switched off and the current in the coils ramped up to form a quadrupole trap.
The coils are mounted on a transport stage, such that they and hence the centre
of the quadrupole, can be moved across to neighbouring chambers. The trapped
cloud of \CaF{} and \esRb{} will be transported in the magnetic trap.
The MTT has been designed for loading into a neighbouring tweezer experiment,
however this will be extended to allow loading onto the chip. The MOT, MTT and
tweezer experiment are shown in \myfigref{experiment:fig:MTTsetup}. 

\begin{figure}[ht]
  \includegraphics[width=0.8\textwidth]{./figs/existing_chambers.png}
  \caption{
    \cm{Need to do a proper CAD for my chamber.}
    The MOT and tweezer chambers for existing experiments. The magnetic
    transport trap (MTT) is used made up of a pair of transport coils arranged
    in anti-Helmholtz configuration. These generate a quadrupole field, which
    will move with the coils on a translation stage, allowing the transfer of
    molecules trapped in the MOT chamber into the tweezer chamber. The chip
    chamber will be positioned further downstream (left) of the tweezer chamber.
    This CAD was undertaken by Kyle Jarvis.
  }
  \label{experiment:fig:MTTsetup}
\end{figure}

The chip flange will be mounted in a third chamber downstream from the MOT and
tweezer chambers. The flange is to be aligned such that the cloud can pass above
the surface with good clearance, entering on the side of the holding chip opposite
the flange. A macroscopic trap, aligned with the centre of the MTT, can then be
generated by the onboard U-wire and surrounding bias coils. This intermediary
trapping stage will be used to align the cloud with those on the science chip.

The cloud can be brought closer to the surface of the chip by a series of
current ramps in the bias coils and chip wires. By adiabatically changing the
bias fields and currents it is possible to transform the trapping potential
without ever distorting it to the point that it is no longer a trap. An example
of changing trapping height by ramping bias fields is shown in
~\myfigref{experiment:fig:ramptraps}. 

\begin{figure}[ht]
  \centering
  \begin{subfigure}{0.6\textwidth}
    \centering
    \includegraphics[width=\textwidth]{./figs/ramps/x.pdf}
    \caption{}
  \end{subfigure}
  \begin{subfigure}{0.6\textwidth}
    \centering
    \includegraphics[width=\textwidth]{./figs/ramps/y.pdf}
    \caption{}
  \end{subfigure}
  \begin{subfigure}{0.6\textwidth}
    \centering
    \includegraphics[width=\textwidth]{./figs/ramps/z.pdf}
    \caption{}
  \end{subfigure}
  \caption{
    An example of a bias field ramp between the Z1 and Z2 stages of the trapping
    sequence (see \mytabref{experiment:table:loading}). In this stage the
    \SI{3}{\milli\metre} trap is operated at a \SI{3}{\ampere} current. The bias
    field ramp reduces the trapping height from \SI{400}{\micro\metre} (solid)
    to \SI{200}{\micro\metre} (dashed-dots). Two intermediary stages are shown
    at 30\% ramp (dashed) and 60\% ramp (dotted). The potential is transformed
    whilst still forming a trap throughout. Subfigures (a), (b) and (c) show the
    field through the trap centre in the $x$, $y$ and $z$ directions
    respectively.
  }
  \label{experiment:fig:ramptraps}
\end{figure}

A ramp is performed by linearly changing the trapping currents (those in the
Z-wires and bias coils). The time over which a ramp should occur is expected to
be on the order of tens of microseconds, so that the molecules experience an
adiabatic change of the potential.~\cite{Boehi2009} The proposed series of
trapping stages for loading into the dimple is shown in
\mytabref{experiment:table:loading}, along with required trapping currents and
bias fields.

\thesis{Temperature/ compression? Proportion of molecules we expect to be able
to keep? Also just realy need to review this. Often don't need an x bias and
$B_0$ is basically always zero. Also qudrupole trap depth.}

\thesis{This table should be a graph}

\begin{table}
  \centering
  \begin{tabular*}{0.8\textwidth}{| @{\extracolsep{\fill} }l | c c c c c c|}
   \hline
    Trap stage & Axis length  & $h_0$ & $I$  & $\widetilde{B}_x$ &
    $\widetilde{B}_y$ & Depth  \\
    & (\si{\milli\metre}) & (\si{\micro\metre}) & (\si{\ampere}) & (\si{\milli\gauss})
    & (\si{\gauss}) & (\si{\milli\kelvin}) \\
  \hline
    MTT& N/A & $10^4$  & 100 & N/A & N/A & N/A \\
    inter-U1 & 500 & $10^4$ & 100 & 0 & 20 & 1.3 \\
    inter-U2 & 500 & 400 & 5 & -1 & 25 & 1.6 \\
    Z1 & 30 & 400 & 3.3 & 0 & 16.5 & 1.1 \\
    Z2 & 30 & 200 & 3.3 & 0  & 33 & 2.2 \\
    Z3 & 10 & 200 & 2.2 & 0 & 22 & 2.9 \\
    Z4 & 10 & 100 & 2.2 & -2 & 44 & 1.9 \\
    Z5 & 3 & 100 & 1.1 & 0 & 22 &1.5 \\
    Z6 & 3 & 50 & 1.1 & -1 & 44 & 2.9 \\
    Z7 & 1 & 50 & 0.55 & 0 & 22 & 1.5 \\
    Z8 & 1 & 10 & 0.55 & 0 & 22 & 7.3 \\
    dimple & 0 & 10 & 0.5 (0.15) & -20 & 12 & 3.0 \\
 \hline
\end{tabular*}
  \caption{A series of trapping stages in the loading
  procedure. To move between stages, the currents and fields can be
  adiabatically ramped linearly to the next required value. In the case of
  changing between trapping wires, one current is ramped down to zero, while the
  other is ramped up to the new value to form the trap. An example of a ramping
  stage (Z1 $\rightarrow$ Z2) is shown in \myfigref{experiment:fig:ramptraps}.
  The transverse current for the dimple trap is shown in parentheses.}
  \label{experiment:table:loading}
\end{table}
