The design of the \CaF{} chip experiment was motivated by three main factors:
the need to integrate with the existing experiment, the core proposal of
confining molecules close to a microwave guide, and the practicalities of
fabricating the chip. In this chapter I will describe how the first two factors
informed the design choices, with changes due to fabrication discussed in
chapter~\ref{fab}. The design will be further justified by simulation in
chapter~\ref{sim}.
%
We begin with a discussion of the existing experiment, which will be used to
create the ultracold \CaF{} for loading onto the chip. I will then give an
overview and motivation of the design of the chip experiment.


\section{Existing \CaF{} experiment}
\label{overview:existing}

In this section I will present a summary of the process used to produce
ultracold \CaF{} molecules in a magnetic trap, which we intend to load onto the
chip trap. We will consider the various stages of the process, which are
presented in \myfigref{overview:fig:CaFcartoon}. First a beam of \CaF{}
molecules is created using a buffer gas source~\cite{Truppe2018}. A fraction of
the molecules in the beam are slowed to below the capture velocity of the MOT
by radiation pressure from counter-propagating resonant
light~\cite{Truppe2017a}. We can also apply separate light in the transverse direction
to improve collimation of the beam during its flight.
The molecules are captured in a MOT~\cite{Williams2017} and cooled in optical
molasses~\cite{Truppe2017} before being optically pumped into a weak field
seeking state~\cite{WilliamsMagnetic2018}. This allows for magnetic trapping
and transport of the molecules. The experiment is conducted under ultra-high
vacuum (UHV, $P<\SI{1E-9}{\milli\bar}$) conditions to limit background induced
loss. Only the source chamber operates at a higher pressure, but the molecule
beam quickly exits this chamber into the low pressure region. 

\begin{figure}
  \centering
  \includegraphics[width=0.8\textwidth]{figs/overview/cartoon.pdf}
    %\begin{overpic}[width=0.8\textwidth]{figs/overview/cartoon_notext.pdf}
    %  \put(5, -1){Source}
    %  \put(32, -1){Slowing}
    %  \put(70.5, -1){MOT}
    %  \put(78, 55){\color{pink}{To chip}}
    %  \put(36, 52){\color{Lorange}{\pewpew{}{00} and repumps}}
    %  \put(95, 29){\color{Lgreen}{\pewpew{S}{00}}}
    %\end{overpic}
    %\vspace{1cm}
  \caption[\CaF{} source]{
    A schematic of the \CaF{} source, slowing region and MOT chamber.  \CaF{}
    molecules (blue) are produced by the buffer gas cell, and are slowed by
    longitudinal slowing and transverse cooling light (green, \pewpew{S}{00}).
    The molecules are then captured in a MOT (MOT light \pewpew{}{00} and
    repumps combined are shown here in orange) where they can be further cooled
    and transferred to a magnetic trap. The molecules can then be transported
    out of the chamber by the MTT in the direction of the pink arrow. A
    detailed view of the source is shown in \myfigref{overview:fig:source}. This
    figure is based on one in \inlineref{Truppe2017}.}
  \label{overview:fig:CaFcartoon}
\end{figure}

\subsection{\CaF{} energy structure and constants}

The energy structure of \CaF{} is depicted in
\myfigref{overview:fig:CaFenergy}. Here we show only the levels that are
pertinent to this thesis, those being the X, A and B electronic levels up to
vibrational level $v=3$, 2 and 0 respectively. The relevant Hund's cases are
(b), (a) and (b) respectively. For the X state, we will discuss $N=$
\numrange{0}{2} but for clarity $N=2$ is not included in this figure.
%
Figure~\ref{overview:fig:CaFenergy} and \mytableref{overview:table:lasers} also
label the various lasers that are used to addressed the \CaF{} transitions. We
denote these \pewpew{}{vv'} where $v$ ($v'$) is the lower (excited)
vibrational level that the laser addresses. The superscript $S$ is used to
distinguish the $X\rightarrow B$ transition since it is used for the slowing
light.

\begin{figure}
  \centering
  \includegraphics[width=0.98\textwidth]{figs/energylevels/main_sbs_3.pdf}
  \caption[The energy levels of \CaF{}]{
    The energy levels of \CaF{} are shown, along with the lasers that will be
    used to address the various transitions (further information is found in
    \mytableref{overview:table:lasers}). The branching ratios for the allowed
    decays are shown with dashed lines. Note that the $X(N=0)$ level is
    included, but for simplicity the $X(N=2)$ level is omitted. The box shows
    the hyperfine states of $X(N=1, v=0)$, how these arise from the $N$ and
    $J$ angular momenta is described in the main text. This figure is adapted
    from \inlineref{Williams2017}. 
  }
  \label{overview:fig:CaFenergy}
\end{figure}

\begin{table}
  \centering
\begin{tabular}{llll}
  \hline\hline
  Symbol & Ground state & Excited state & Wavelength (\si{\nano\meter}) \\
  \hline
  \pewpew{}{00} & $X(v=0)$ & $A(v=0)$ &  606.3 \\
  \pewpew{S}{00} & $X(v=0)$ & $B(v=0)$ & 531.0 \\
  \pewpew{}{01} & $X(v=0)$ & $A(v=1)$ & 628.6 \\
  \pewpew{}{12} & $X(v=1)$ & $A(v=2)$ & 628.1 \\
  \pewpew{}{23} & $X(v=2)$ & $A(v=3)$ & 628.7 \\
  \pewpew{}{10} & $X(v=1)$ & $A(v=0)$ & 585.4 \\
 \hline
\end{tabular}
\caption[Lasers, transitions and wavelengths]{
  The various lasers used in this thesis.  The laser \pewpew{}{10}
  is included for completion, and will be explained in
  chapter~\ref{exper}.
  }
  \label{overview:table:lasers}
\end{table}

The permitted decay paths are shown by the dashed lines, along with their
branching ratios. One of the reasons \CaF{} is chosen for study over other
molecules is its highly diagonal Franck-Condon factors, so that most decays
occur with $v'=v$. However, as we discussed in section~\ref{theory:coolmols},
for large numbers of scattered photons we must employ the various repump lasers
to avoid pumping into states that are dark to the cooling light. This will be
discussed further when describing slowing of the beam and the MOT.

Hyperfine splitting occurs in \CaF{} due to the spin-half contribution of the
fluorine atom. This is not resolved in the $A$ state, but it accounts for the
hyperfine splitting of the $X$ state as is shown in the box of
\myfigref{overview:fig:CaFenergy}. It is worth explaining why we have the four
$F$ states $F=2,1^+,0,1^-$. These arise due to the combination  of the
hyperfine interaction and a spin rotation interaction of the form
$\mathbf{S}\cdot\mathbf{N}$. Since we are in Hund's (b) case, the $N=1$ state
is split into angular momenta states $\mathbf{J}= \mathbf{N} + \mathbf{S}$.
Here $S=1/2$, so the possible values of $J$ are $N+1/2$ and $N-1/2$ (with the
exception of $N=0$ where we have only $J=1/2$). For $N=1$, we have $J=3/2$ and
$J=1/2$. Each of these is again split by the hyperfine interaction into
$\mathbf{F} = \mathbf{J} + \mathbf{I}$, with $I=1/2$. This gives us the allowed
values $F=2,1^+,0,1^-$.

In the $B$ state, the $F=0$ and $F=1$ states are split by \SI{20}{\mega\hertz},
which can usually be neglected for the purposes of our discussion. A full
description of the \CaF{} energy structure and various constants can be found
in \inlineref{Anderegg2019a}, those that are most useful are presented in
\mytableref{overview:table:constants}.

\begin{table}
  \centering
\begin{tabular}{lll}
  \hline\hline
  Constant & Symbol & Value \\
  \hline
  Mass & m & \SI{59}{\amu}\\
  Electric dipole moment ($X$ state) & $\mu_e$ & $\SI{-3.08}{\debye}$\\
  % Don't confuse EDM with transition moment
  Magnetic dipole moment & $g_F\mu_B m_F$ & \\
  % This is what Ed uses in his notebook, I think it is the same as mu_e...
  % Had to convert form his units, but pretty sure this is right.
  %Chemistry dipole moment & $\mu_c$ & $3\si{\debye}$ \\
 \hline
\end{tabular}
\caption[\CaF{} constants]{
  Various constants for \CaF{} taken from \inlineref{Anderegg2019a}. Note that
  the magnetic dipole moment is given in terms of the g-Factor $g_F$ and the
  quantum number $m_F$, which are state dependent, as well as the Bohr magneton
  $\mu_B = \SI{9.274}{\joule\per\tesla}$.
  \jvg{How does this compare to cavity QED system?}
  \cm{I have a comment here 'Don't confuse EDM with transition moment' I think
    there is some fundamental misunderstanding here and I need to talk to Mike
    about it.}
  }
  \label{overview:table:constants}
\end{table}

\subsection{Buffer gas source}

We begin all our experiments by creating a pulsed beam of \CaF{} molecules
using a buffer gas source. This is pictured in \myfigref{overview:fig:source}
and consists of a copper cell which is cryogenically cooled to \SI{4}{\kelvin}.
%
Helium gas, also cooled to \SI{4}{\kelvin}, and \SFsix{} gas near room
temperature, flow through the cell. A \Ca{} target (shown in blue) is ablated
by a pulsed Nd:YAG laser. The \Ca{} reacts with the \SFsix{} to produce \CaF{}, which is
then flushed out of the cell by the helium flow.

\begin{figure}
  \centering
  \includegraphics[width=0.6\textwidth]{figs/overview/cell.pdf}
  \caption[The buffer gas cell]{
    The buffer gas cell. Helium and \SFsix flow into the cell. \Ca{}
  atoms are ablated from a target (shown in blue) by a Nd:YAG laser. \CaF{} molecules are
formed which thermalise with the \He{} and a molecule beam exits from the aperture.}
  \label{overview:fig:source}
\end{figure}

The cell is designed to optimise the flow of \He{} so as to entrain the 
\CaF{} and guide it to the exit aperture without the creation of vortices,
where the molecules can be trapped~\cite{Truppe2018}. For this reason \He{}
enters from towards the rear of the cell at \SI{4}{\kelvin} and flows towards
the exit aperture.  The \Ca{} target is  mounted on a rotating stage, so that
when one region is depleted another can be targeted.

In order to prevent excess helium entering the slowing and MOT chambers, a
second aperture is positioned between the source and slowing chambers.
Differential pumping ensures that the slowing chamber remains at UHV. A
mechanical shutter at this aperture reduces the time that helium is able to
leave the source chamber to only the time when there is a pulse of \CaF{}. We
also use a copper shield, coated with coconut charcoal and cooled to
\SI{4}{\kelvin} as a helium absorber~\cite{doi:10.1116/1.574141}. This is
thermally cycled overnight when the source is not in use to avoid saturation.
As noted in \inlineref{Jurgilas2021}, this buffer gas source originally
produced up to \SI{5E10}{molecules/steradian} molecules in $X(N=1, v=0)$ per
pulse, with a
mean velocity of \SI{160}{\meter\per\second}~\cite{Truppe2018} but its
performance has degraded since the original report. As a result we now observe
\SIrange{50}{60}{\percent} lower MOT population than at the time
\inlineref{Williams2017} was published.

\subsection{Slowing the beam}

The buffer gas source produces a beam of molecules with mean forward velocity
\SI{160}{\meter\per\second} -- slower than for example, a supersonic
source~\cite{Mathavan2016} but they are still far above the capture velocity of
our MOT, which is approximately \SI{10}{\meter\per\second}. The beam's velocity
can be further reduced by radiation pressure due to a counter-propagating beam
of \pewpew{S}{00} slowing light. This is chirped to account for the change in
Doppler shift of the transition frequency as the molecules slow. The slowing
light is also combined with the \pewpew{}{10} repump light to avoid pumping
into the $X(v=1)$ state. To again account for Doppler shift during slowing, the
repump light is frequency broadened by \SI{300}{\mega\hertz} by a series of
three elecro-optic modulators.

% Time and final velocity according to Hannah's thesis
We apply the light for \SI{6}{\milli\second}, which slows over approximately
\SI{90}{\centi\meter} of travel from the buffer gas cell's exit aperture.
Linear chirps used in previous experiments have been used to slow the molecules
for loading into the MOT, as detailed in \inlineref{Williams2017}. Implementing an
exponential chirp originally proposed in \inlineref{Anderegg2019}  produced a
\SIrange{60}{80}{\percent} improvement in the number of molecules in the
MOT~\cite{Jurgilas2021}.

It is also possible to reduce the transverse velocity of the beam by applying
slowing beams in the perpendicular direction close to the source. This is shown
for one dimension in the `slowing and transverse cooling' section of
\myfigref{overview:fig:CaFcartoon}, however in reality there is a second pair
of beams going into and out of the page. For this cooling step we also use the
$X\rightarrow B$ transition along with the \pewpew{}{10}
repump~\cite{Jurgilas2021}.

\subsection{Capture in a MOT}
\label{overview:MOT}

We aim to capture molecules in a MOT positioned \SI{130}{\centi\meter} from the
cell's exit aperture inside a separate vacuum chamber. The MOT magnetic field
is provided by in-vacuum anti-Helmholtz coils and the MOT light is formed from
a single beam which is reflected along each axis before being retro-reflected
through the entire experiment to provide the restoring beams.
%
We described in section~\ref{theory:coolmols} that laser cooling and trapping of diatomic
molecules requires us to address three key differences to atomic systems. The
first is the repumping of the vibational levels. We apply vibrational repumps (\pewpew{}{10}, \pewpew{}{21}
and \pewpew{}{32}) along with the main cooling light (\pewpew{}{00}).  All of these beams must have the r.f.\ sidebands
to address the hyperfine levels of the $X$ state (recall that the hyperfine
levels of $A$ are unresolved).

The second nuance is the rotational branching, which we avoid by cooling on the
$X(N=1) \rightarrow A(N=0)$ transition. This immediately leads to the third
nuance, which is the remixing of resulting dark states. 
%
The \CaF{} MOT is a type-II MOT~\cite{1367-2630-18-12-123017}, meaning that the
angular momentum of the excited state  is less than that of the ground state .
Unlike a type-I MOT (where $F'>F$), it is possible for a molecule that has been
pumped into the excited state to decay into a dark state, or into a state that
is anti-trapped by the MOT~\cite{Fitch2021}.
%
To resolve this, we employ a dual-frequency MOT.
\myfigureref{overview:fig:dualfreq} illustrates this MOT scheme for the
simplified case of a ground state with $F=1$ and an excited state with $F'=0$.
%
When there is no blue-detuned light, any molecule that decays into the $m_F=-1$
state can be lost from the cycle. When blue-detuned light of opposite
polarisation is present these molecules can be repumped. 

\begin{figure}
  \centering
    \begin{overpic}[abs, width=0.2\textwidth]{figs/overview/typeII.pdf}
      \put(-40, 43){$F=1$}
      \put(-40, 148){$F'=0$}
      \put(90, 43){$m_F=0$}
      \put(90, 81){$m_F=1$}
      \put(90, 5){$m_F=-1$}
      \put(5, 5){\color{blue}{$\sigma^+$}}
      \put(5, 81){\color{pink}{$\sigma^-$}}
    \end{overpic}
  \caption[Dual-frequency cooling scheme]{
    A dual frequency scheme for the type-II case $F'=0$, $F=1$. The molecules
    in each state (circles) can be pumped by their corresponding red- (blue-)
    detuned light. This figure is adapted from one in \inlineref{Williams2018}.
  }
  \label{overview:fig:dualfreq}
\end{figure}

We are typically able to capture on the order of $10^4$ molecules from our
beam.  The MOT population can be estimated by the light-induced fluorescence.
The MOT temperature is reduced to its minimum value by lowering the intensity
of \pewpew{}{00}, since this reduces the effects of Sisyphus
heating~\cite{Truppe2017}. We typically observe MOT temperatures below
\SI{4}{\milli\kelvin}.
%
A complete description of the \CaF{} MOT is far beyond the scope of this
overview, and has been described in detail elsewhere, for example see
\inlineref{Williams2017}. We will not discuss an important alternative scheme
for forming a \CaF{} MOT, the r.f.\ MOT, a description of which can be found in
\inlineref{PhysRevLett.119.103201}.

\subsection{Optical molasses}

Sub-Doppler cooling of \CaF{} is achieved with a blue-detuned molasses. In this
scheme the MOT coils are switched off, and \pewpew{}{00} is now blue-detuned
from the transition frequency. This scheme is somewhat complex, so consider
again the simplified example of a molecules travelling in one dimension and
having an $F=1$ ground state and $F'=0$ excited state, as is shown in
\myfigref{overview:fig:molasses}.  The counter-propagating beams establish
gradients of both polarisation and intensity.  As explained in
\inlineref{Weidem_ller_1994} we will have bright states that have an a.c.\
Stark shift, and dark states which do not.  The a.c.\ Stark shift of the bright
states depends on polarisation, and so as molecules move through the gradient
the bright state energy changes, as is shown in the figure.

\begin{figure}[htb]
  \centering
    \begin{overpic}[width=0.6\textwidth]{figs/overview/molasses.pdf}
      \put(-16, 82){$F'=0$}
      \put(-16, 30){$F=1$}
      \put(105, 19){Dark states}
      \put(105, 5){Polarisation}
      \put(105, 38){Bright states}
    \end{overpic}
    \vspace{1cm}
  \caption[Blue-detuned molasses]{
    Blue-detuned molasses for a type-II system with $F=1$, $F'=0$. Here
    the polarisation gradient across a 1D system (bottom row) causes a change
    in the coupling strength of bright states. A molecule travelling through
    the gradient (blue) will be preferentially excited by \pewpew{}{00} to the
    excited state when the coupling is strong. It will decay into the dark
    states, and adiabatically transfer back to the light states (black arrow)
    when the coupling is weak.  Repeating this process results in a net loss of
  energy for the molecule. This figure is adapted from
\inlineref{1367-2630-18-12-123017}.}
  \label{overview:fig:molasses}
\end{figure}

When a molecule in a bright state moves through the polarisation gradient it
can expend energy climbing the potential. It is preferentially pumped into the
dark state (via $F=0$) at the top of the potential because the probability of
pumping is proportional to the square of the Rabi frequency, which is greatest
at this point. From the dark state, molecules can non-adiabatically transfer
back into the dark state at regions where the energy difference is low. The
molecule can now repeat this cycle, each time losing energy travelling up the
potential. Note that the light must be blue-detuned because in the case of
red-detuning the bright states have lower energy than the dark states, and
there is a heating rather than a cooling effect~\cite{1367-2630-18-12-123017}.

These same principles apply in the three dimensional case, although the
polarisation and intensity structure of the light field is more complicated.
We must also consider the numerous energy levels of \CaF{} and the sideband
structure required to address the hyperfine state. A full treatment requires
solving the optical Bloch equations, which is done in
\inlineref{1367-2630-18-12-123017}. In our experiment we apply this technique
to produce a \CaF{} cloud of temperature
$<\SI{6}{\micro\kelvin}$~\cite{PhysRevLett.123.033202}.

\subsection{Magnetic trapping and transport}

The molecules can now be transferred into a weak-field seeking state by optical
pumping. The weak-field seekers can be confined in a magnetic trap, provided
either by the MOT coils, or by the external transport coils (see
\myfigref{overview:fig:CaFcartoon}. In the case of the
latter, the molecules can then be transferred to the tweezer chamber, or in the
future to the chip for loading.

During the molasses there is no magnetic field to lift the degeneracy between
the Zeeman states. Therefore after the molasses light is turned off the
molecules are distributed between the Zeeman substates of $X(v=0, N=1)$.
Attempting to magnetically trap at this point would cause a significant loss,
so we optically pump into the weak-field seeking state by a procedure originally
proposed for \SrF{} in \inlineref{PhysRevLett.121.013202} % this is Sarunas's ref 118
and described for \CaF{} in \inlineref{Jurgilas2021}.

In this scheme a weak magnetic field (\SI{200}{\milli\gauss}) is applied across
the cloud, and two beams of light are incident on the molecules. Both act on
the $X\rightarrow B$ transition, but their frequency components are tuned to be
close to the $F=2$ and $F=1^-$ hyperfine levels respectively. The $F=2$
($F=1^-$) frequency component propagates parallel (perpendicular) to the
magnetic field and its polarisation is chosen to drive $\sigma^+$ ($\pi$)
transitions. This means that the only state dark to the light is the
magnetically trappable state $\ket{N=1, F=2, m_F=2}$, which is eventually
populated. From here it is possible to transfer to other states by microwave
spectroscopy techniques~\cite{WilliamsMagnetic2018}. Various weak-field seeking
states of \CaF{} are detailed in \myfigref{overview:fig:magtrapstates}.

\begin{figure}
  \centering
  \includegraphics[height=0.45\textwidth]{figs/energylevels/magsplit.pdf}
  \caption[Hyperfine structe in \CaF{}]{
    Hyperfine strucutre of \CaF{} ground states. The stretched states
    are highlighted in blue, and other weak-field seekers are highlighted in
    pink. Adapted from \inlineref{WilliamsMagnetic2018}.}
  \label{overview:fig:magtrapstates}
\end{figure}


At this stage we also introduce three stretched states in \CaF{}. These states
lie in the $X$ electronic level, and are denoted
%
\begin{equation}
  \ket{N}_\text{str} = \ket{N, m_N=N}\ket{S, m_S=S}\ket{I,m_I=I},
  \label{theory:eqn:stretched}
\end{equation}
%
where the degeneracy must be lifted by the application of an external magnetic
field, $m_X$ is then the projection of $X$ onto this field's axis. The
stretched states are of interest because they are not only weak-field seekers,
but the transitions between neighbouring stretched states
($\ket{N}_\text{str}\leftrightarrow\ket{N+1}_\text{str}$) are highly
insensitive to magnetic fields. Of particular note are, the
$\ket{0}_\text{str}\leftrightarrow\ket{1}_\text{str}$ ($\omega_0/(2\pi) =
\SI{20.5}{\giga\hertz}$) and
$\ket{1}_\text{str}\leftrightarrow\ket{2}_\text{str}$ ($\omega_0/(2\pi) =
\SI{41.1}{\giga\hertz}$) transitions observed in
\inlineref{PhysRevLett.124.063001}, which we will discuss further in
chapter~\ref{mws}.

After transferring into the magnetically-trappable states, the molecules can be
contained in a quadrupole magnetic trap generated by anti-Helmholtz coils
inside the vacuum chamber, as also discussed in
\inlineref{WilliamsMagnetic2018}. They can then be used for a variety of
experiments, including the chip experiment, as we will now discuss.

\subsection{Other experiments and transport}

The \CaF{} MOT is the workhorse of our experiment which, after the further
cooling discussed above, can be used to study collisions with \Rb{} atoms (see
\inlinerefs{Jurgilas2021, JurgilasIOP2021, PhysRevLett.126.153401}) or can be
loaded into optical tweezers. Due to the limited optical access available in
each chamber it is convenient to perform some experiments in separate chambers
to that which contains the initial MOT. Hence other members of CCM have
developed a scheme for transporting the molecules from the MOT chamber to a
neighbouring chamber by means of a magnetic transport trap (MTT). This second
chamber is used for experiments with optical tweezers and so is referred to as
the tweezer chamber. The MTT consists of magnetic coils in anti-Helmholtz
configuration, situated outside the vacuum chamber (as opposed to trapping with
internal coils). These coils are mounted on a transport stage, so that they and
hence the field they generate can be translated.

The transportation procedure is depicted in \myfigref{overview:fig:MTT}.
Molecules are initially trapped using internal coils in the MOT chamber. They
are then handed over to the MTT by ramping off the current in the internal
coils, and ramping on the current in the external coils. The coils are then
translated by the stage at speeds sufficiently slow to bring the molecules with
them. The ramping process can then be inverted to hand the molecules back over
to an internal trap, either to coils in the tweezer chamber, or to a wire trap
in a third chamber to be used for the chip (the chip chamber). This transport
of the molecules is similar to that used in
\inlinerefs{Lewandowski2003,PhysRevResearch.1.033035} and elsewhere. Details of
the handover procedure and simulations thereof will be discussed further in
chapter~\ref{sim}. In the next section, we will begin to discuss how the
molecule chip can also be incorporated as an additional experiment.

\begin{figure}
  \centering
  \begin{overpic}[abs, width=0.8\textwidth]{figs/overview/MTTschem.pdf}
      \put(20, -15){MOT chamber}
      \put(150, -15){Tweezer chamber}
      \put(260, -15){Chip chamber}
  \end{overpic}
  \vspace{0.5cm}
  \caption[Magnetic transport scheme]{
  Depiction of the MTT scheme. \CaF{} molecules trapped initially in the MOT
  chamber using internal coils (black, with quadrupole field marked by the
  black cross) are transferred to the external trapping coils (gold, with field
  marked by gold cross). The coils can then be translated to transport the
  molecules to neighbouring chambers. The dashed lines depict transport in
  progress. They can be loaded back into internal coil traps (such as the black
  coils in the tweezer chamber) or onto a wire trap (shown in green in the chip
  chamber), as is shown occurring here.
  }
  \label{overview:fig:MTT}
\end{figure}

\section{Design requirements and overview}
\label{overview:design}

This section will present the design of the molecule chip experiment, and how
it is to be integrated into the existing \CaF{} experiment whilst accounting
for fundamental constraints. I will present computer-aided designs of the chip
chamber, which were created by Kyle Jarvis in CCM, as well as my designs for
the chip itself, and a PCB for power delivery.

As discussed in chapter~\ref{intro} the aim of this project is ultimately to
trap molecules in close proximity to microwave resonators so that we can
perform coherent control of quantum states on the rotational transitions in the
molecules.
%
Following the proposal by \inlineref{Andre2006}, this can be achieved in a chip
architecture, with the molecule trapped as close to the resonator as possible.
%
The ultimate limiting factor will come from the Van der Waals force\footnote{
  One might wonder about the effect of an increase in black body radiation near
  to the chip.  It transpires that the excitation rate for \CaF{} by black body
  radiation remains mostly constant as the molecule approaches a
  surface~\cite{PhysRevA.78.052901}.
  %Further, we consider the Van der Waals force rather than the Casimir-Polder
  %force because the distance to the surface is small compared to the
  %wavelength of the rotational transition.
}, where the fluctuating electric dipole moment of the molecule results in
attraction between the molecule and the chip surface~\cite{2011Ac}. The
associated energy shift is 
%
\begin{equation}
  V_\text{vdW}(z) \approx \frac{\mu_e^2}{4\pi\epsilon_0 z^3}.
\end{equation}
%
We require that the gradient,
%
\begin{equation}
V'_\text{vdW}(z) = \frac{3 \mu_e^2}{4 \pi\epsilon_0 z^4},
\end{equation}
%
to be small compared to the force from a magnetic trap, which can be calculated
for a representative dimple trap using \myeqref{theory:eqn:dimplegrad}. Taking the
operating current to be $I = \SI{1}{\milli\ampere}$. The condition
$V'_\text{vdW}(z) \ll \mu_B B'$ can be written as
%
\begin{equation}
  z \gg \sqrt{\frac{3 \mu_e c^2}{2 \mu_B I}},
\end{equation}
%
limiting the height by $z \gg 10^{-7}\,\si{\meter}$. We therefore take the
minimum possible trap height to be \SI{10}{\micro\meter}. The trapping wires
must be on a smaller scale than this so that the trapping potential is
sufficiently localised. Features of such a size can be easily created by
standard photolithography techniques.

%% I wrote this whole bit, but I think it's actually not necessary, as the
%% 10^-7 limit is sufficient. I only didn't delete it because it's kind of
%% useful but not committed yet..
%% NOTE that this can't just be popped in. It needs proper integrating and
%% cross-referencing across various chapters

%However, this is not the full story, since the Van der Waals shift must also be
%small compared to the vacuum Rabi frequency for the transition driven by a
%on-chip microwave resonator. This quantity can be found as follows: consider a
%microwave resonator with length $\lambda/4$, with $\lambda$ being the
%wavelength of the resonator, and a length scale $L$ in the
%transverse direction. The mode field will occupy a volume $V\approx L^2 \lambda
%/ 4$, and we can write the field amplitude ($E_0$) to its angular frequency ($\omega$) by $\epsilon_0
%E_0^2 V \approx \hbar \omega_0$. We can therefore write the field amplitude as
%%
%\begin{equation}
%  E_0 \approx \frac{2 f}{L}\sqrt{\frac{h}{\epsilon_0 c}}.
%\end{equation}
%%
%The vacuum Rabi frequency ($\Omega$) is related to the matrix element for the
%transition
%\begin{equation}
%  \hbar \Omega = \bra{g}\mathbf{d}\cdot\mathbf{E}\ket{e},
%  \label{overview:eqn:rabi}
%\end{equation}
%%
%where we take the ground and excited states to be the stretched states
%$\ket{0}_\text{str}$ and $\ket{1}_\ket{str}$ respectively (see
%\myeqref{overview:eqn:stretched}). 
%%
%Take the orientation of
%the molecule to be random with respect to the light field, so that the dot
%product averages across the ensemble to give $dE/3$. The matrix element is 
%%
%\begin{equation}
%  \bra{g}d\ket{e} = \frac{\mu_e}{\sqrt{3}},
%\end{equation}
%%
%where the factor of $\sqrt{3}$ arises due to this being a rotational
%transition. This follows from the discussion in section~\ref{theory:transitions}.
%The Rabi frequency is now written as
%%
%\begin{equation}
%  \Omega = /frac{2\omega \mu_e}{L}\sqrt{\frac{1}{h \epsilon_0 c}}.
%\end{equation}

%In chapter~\ref{mws} we will discuss microwave resonators in more detail, but
%for now we will take the length scale $L\sim\SI{10}{\micro\meter}$, which is
%typical of those found in the literature. The requirement $\hbar \Omega \ll
%V_\text{vdW}(z)$ can now be shown to be equivalent to $z \ gg
%\SI{1}{\miro\meter}$. We therefore take the minimum possible trapping height to
%be \SI{10}{\micro\meter}. This introduces the design requirement that trapping
%wires will be on a smaller scale than this so that trapping potential is
%sufficiently localised.  Features of such a size can be easily created by
%standard photolithography techniques.

Additionally, we aim to integrate the chip trap into our existing \CaF{}
experiment. This can be done by the addition of a new chamber to our setup,
shown in \myfigref{overview:fig:vacuumsystem}. The chip chamber will be
positioned along the existing MTT axis, allowing extension of the transport
system and the delivery of molecules to this chamber. \CaF{} molecules can then
be transferred onto the chip trap via the process also outlined in
\myfigref{overview:fig:vacuumsystem}.

\begin{figure}[htb]
\begin{subfigure}{\textwidth}
  \centering
    \begin{tikzpicture}
      \node[anchor=south west,inner sep=0] (image)
{\includegraphics[width=0.8\textwidth]{figs/overview/apparatus_04_crp.png} };
      \begin{scope}[x={(image.south east)},y={(image.north west)}]
        \draw [-stealth] (0.45, 0.35) -- (0.45,0.52);
        \node[] at (0.46,0.3) {\small MOT chamber};
        \draw [-stealth] (0.18,0.23) -- (0.1, 0.3);
        \node[] at (0.21,0.2) {\small Chip chamber};
        \draw [-stealth] (0.15, 0.7) -- (0.2,0.63);
        \node[] at (0.12,0.72) {\small Tweezer chamber};
        \draw [-stealth] (0.7, 0.95) -- (0.52,0.85);
        \node[] at (0.72,0.99) {\small \Rb{} cell};
        \draw [-stealth] (0.95, 0.85) -- (0.8,0.7);
        \node[] at (0.97,0.89) {\small Slowing region};
        \draw [-stealth] (0.6, 0.2) -- (0.7,0.2);
        \node[] at (0.54,0.2) {\small Source};
      \end{scope}
    \end{tikzpicture}
\end{subfigure} \\[0.25cm]

\begin{subfigure}{\textwidth}
  \centering
\begin{tikzpicture}
  \node (buffer) [process] {Buffer gas source};
  \node (slowing) [process, right of=buffer, xshift=3cm] {Laser slowing};
  \draw [farrow] (buffer) -- (slowing);
  \node (MOT) [process, right of=slowing, xshift=3cm] {Capture in MOT};
  \draw [farrow] (slowing) -- (MOT);
  \node (cool) [process, below of=MOT, yshift=-1cm] {Further cooling\\in molasses};
  \draw [farrow] (MOT) -- (cool);
  \node (mag) [process, below of=slowing, yshift=-1cm] {Transfer to\\magnetic trap};
  \draw [farrow] (cool) -- (mag);
  \node (MTT) [process, below of=buffer, yshift=-1cm] {Transfer to\\ external MTT};
  \draw [farrow] (mag) -- (MTT);
  \node (transport) [process, below of=MTT, yshift=-1cm] {Transport to\\ chip chamber};
  \draw [farrow] (MTT) -- (transport);
  \node (loadU) [process, below of=mag, yshift=-1cm] {Load into\\ U-trap};
  \draw [farrow] (transport) -- (loadU);
  \node (loadchip) [process, below of=cool, yshift=-1cm] {Load onto\\ chip};
  \draw [farrow] (loadU) -- (loadchip);

\end{tikzpicture}
\end{subfigure}

  \caption[The \CaF{} and \CaF{} chip experiment]{
    The \CaF{} experiment is shown along with the planned additional chip
    chamber. Not shown: external transport coils and transverse cooling region.
  The flowchart outlines the various stages of the chip experiment.}
  \label{overview:fig:vacuumsystem}
\end{figure}

At this point, we note that the long-lived stretched states discussed in
section~\ref{overview:existing} are promising candidates for the qubit states
in a molecule chip. For this reason it was decided that using magnetic traps,
such as those described in section~\ref{theory:chips} would be preferable to the
electrostatic traps suggested in \inlineref{Andre2006}. We are now faced with
the question of how exactly we can load molecules from the MTT into the
microscopic chip trap.

Fortunately this problem has previously been addressed for atom chips. We
discussed already in chapter~\ref{intro} that atoms can be guided on chips by
changing of trapping currents. We also discussed the transfer of
atoms between on-chip magnetic traps. For the problem of loading from a
macroscopic trap, we can turn to \inlineref{Ott2001}, where transfer from a
transport trap to a microtrap is made easier by the use of an intermediary
macroscopic trap that is well-aligned with the microtrap.
We propose a similar solution: embedding a macroscopic U-wire beneath our chip
trap which is aligned to the chip and makes a large target for loading from the
MTT.

To ensure that molecules are then loaded into the smallest trap efficiently, we
again follow in the footsteps of atom chips, and have designed  a series of
traps of decreasing size~\cite{Reichel1999}. For magnetic traps, the width of
the wires should decrease, so that the molecules remain localised around the
trap centre throughout loading. 
%
Each wire trap will begin trapping at one height before the bias field is
increased to bring the trap centre closer to the surface (as per
\myeqref{theory:eqn:height}). We choose the wires to be Z-traps so as to avoid
any losses by spin-flips, and so we label the stages $\mathrm{ZX_i}$ for
initial (higher) traps and $\mathrm{ZX_f}$ for the final (lower) trap, with
$\mathrm{ZX}$ corresponding to the wire labels in
\mytableref{overview:table:wires}. Bias fields for all traps are to be provided
by external Helmholtz coils.

Each Z-wire should be sufficiently large to maintain the currents required to
form a trap at height $z$ below the trap, whilst having a width and height  $w,
h \ll z$ so that that the current is highly localised compared to the cloud
size.
%
In the case of the first Z-wire, the molecules are still \SI{3}{\milli\meter}
away from the trapping wire. If we demand a trap depth of
$k_B\times\SI{1}{\milli\kelvin}$, then we require a trapping current of
\SI{30}{\ampere} to form a trap of this depth.  We will discuss in
chapter~\ref{fab} that the maximum wire height that can reliably be fabricated
is \SI{5}{\micro\meter}, and we expect that the wires will be able to carry a
maximum current density of \SI{6E10}{\ampere\per\meter\squared}, as was found
for a similar chip design in \inlineref{Treutlein2008}. Consequently, the
required width of the first Z-wire is $w=\SI{200}{\micro\meter}$. The currents
and widths of other wires are calculated similarly.
%
All wires have been designed to
carry twice the current that is required in the loading scheme, so that there
is sufficient headroom for further experiments, and to reduce risk of
accidental damage to the chip during normal operation.
%
The axial length of the wires also decreases to gradually reduce the size of
the trapped cloud in the $x$ direction.  

\begin{table}
  \centering
\begin{tabular}{lrrrrr}
  \hline\hline
  Name & Axis length (\si{\milli\meter}) & Width (\si{\micro\meter})& $I_\text{max}$ & Trap height (\si{\micro\meter}) \\
 \hline
  U & 16 & N/A& 100 & 3000\\
  $\mathrm{Z0}$ & 12 & 200& 60& $3000\rightarrow1000$ \\
  $\mathrm{Z1}$ &  6 & 20& 6& $1000\rightarrow100$ \\
  $\mathrm{Z2}$ &  2 & 9& 2.7& $100\rightarrow10$ \\
 \hline
\end{tabular}
  \caption[Trapping wire properties]{
    Details on the wire dimensions, maximum current, and desired
  trapping heights. The wire design is shown in
  \myfigref{overview:fig:chiplayout}. Note that the U-wire current is
  limited by vacuum feedthroughs and not by the maximum current calculated by
  the wire dimensions.  The maximum currents have been designed for use at only
  50\% of their potential maximum ($I_\text{max}$).
  }
  \label{overview:table:wires}
\end{table}

The final chip design was informed both by the requirements here, the
simulations presented in chapter~\ref{sim} and the restrictions due to the
fabrication process, which will be discussed in chapter~\ref{fab}. We tried
various different designs, but the final one that was chosen is shown in
\myfigref{overview:fig:chiplayout}. It features the wires as stipulated in
\mytableref{overview:table:wires}, fanouts for connection of macroscopic
current delivery wires, and various other features that will also be explained
in chapter~\ref{fab}.

\begin{figure}[ht]
  \centering
    \begin{overpic}[abs, width=0.51\textwidth]{figs/chip_present4.pdf}
      \put(10, 160){\small (i)}
      \put(60, 160){\small(ii)}
      \put(175, 60){\small(iv)}
      \put(110, 137){\small(iii)}
      \put(70, 90){\small \SI{20}{\micro\meter}}
      \put(112, 93){\small\SI{10}{\micro\meter}}
      \put(8, 42){\small $\mathrm{Z_0}$}
      \put(8, 10){\small $\mathrm{Z_1}$}
      \put(8, 200){\small $\mathrm{Z_2}$}
    \end{overpic}
  \caption[Chip schematic]{
    A schematic of the single-layer chip features, with the scaling exaggerated
    for visibility. The three overlapping Z-wires are shown and labeled. The
    gaps between the wires are highlighted.
    %
    Toward the left (i) is the
    electroplating connection pad and various features used for
    characterisation (ii). On Z2 it is possible to see several small pads used
    as anchors, to secure the thin wire to the substrate.  The axis of the
    $\mathrm{Z1}$ wire is labeled for reference (iii) and the other wires are
    similar. All of the above features  will be discussed further in
    chapter~\ref{fab}. The crest of Imperial College London (iv) is also
    included.}
  \label{overview:fig:chiplayout}
\end{figure}

Notice that this design is a single-layer design, with the wires intersecting.
A system has been devised to ensure that the desired currents are achieved in
each wire segment, and is described in chapter~\cm{somehwere}. Furhter, this
inital design does not incorporate microwave guides. These are to be
installed on a second level, separated from the trapping wires by a thin
insulating layer, on which we can fabricate coplanar waveguides~\cite{1127105}.
This stage of the project has not yet been reached, but the planned fabrication
procedure for microwave guides is discussed in section~\ref{fab:planned} and
their operation is discussed in chapters~\ref{mws} and \ref{squeeze}.

To facilitate all of this, the chip is mounted on a flange with supporting
infrastructure, as detailed in \myfigref{overview:fig:chipchamber}. This chip
flange assembly is equipped with a large copper heat sink, a large U-wire to
form the macroscopic alignment trap and a subchip for current and microwave
delivery (detailed in \myfigref{overview:fig:subchip}). It is mounted into a
recess in the subchip so that it is flush with the surface. The chip is mounted
facing downwards so that molecules can be dropped for imaging as they fall. The
flange itself is fitted with two high-current (\SI{100}{\ampere}) feedthroughs,
a 16-pin feedthrough rated for \SI{3}{\ampere} currents, and microwave
feedthroughs. All access to the chip is therefore through this one flange,
allowing for easy access and assembly.  It will be housed in a separate vacuum
chamber (the chip chamber), positioned as shown in
\myfigref{overview:fig:vacuumsystem}. The chip chamber will be arranged so that
molecules can be brought in along the transport axis (shown by the arrow in the
\mysubfigref{overview:fig:chipchamber}{b}) and positioned below the surface of
the chip. This arrangement will be explained in further detail in
chapter~\ref{exper}.

\begin{figure}[htb]
    \centering
    \begin{overpic}[abs,
      width=\textwidth]{figs/exper/labeled_explosion.pdf}
      \put(20, 132){(a)}
      \put(20, -80){(b)} % Yes, I am putting these here. No I am not sorry.
      \put(20, -250){(c)}
    \end{overpic}
    \\
    \begin{overpic}[abs, width=0.5\textwidth]{figs/overview/chamber_xsecarrow.pdf}
      %\put(5, 160){(b)}
    \end{overpic}
  \\
    \begin{overpic}[abs, width=0.5\textwidth]{figs/chip_pic_crop.png}
      %\put(5, 160){(c)}
    \end{overpic}
  \caption[Chip experiment details]{
  The chip experiment is shown in detail. In (a) we show an exploded view of
  the chip flange assembly, with the various components labeled.
  A cross section of the chip chamber is shown in (b). The arrow shows how
  molecules will enter the chamber, brought in by the MTT.
  In (c) we have the chip assembly fully constructed, with a view of the
    aluminium-core PCB (subchip) for current delivery. The
    microwave feedthroughs remain disconnected.
  }
  \label{overview:fig:chipchamber}
\end{figure}

\begin{figure}
  \centering
  \includegraphics[width=0.4\textwidth]{figs/overview/subchip.pdf}
  \caption[Subchip circuit board design]{The subchip layout for power delivery
  to the chip. Traces are shown in pink (the number of traces used depends on
  the number of wires in the chip design and can vary between subchips), and
  mounting holes in blue. The chip is situated in a recessed alcove marked by
  the gold square.}
  \label{overview:fig:subchip}
\end{figure}
