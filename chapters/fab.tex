We have not yet produced a final chip trap, however we have made good progress
in production of a number of prototypes of the trapping wire layer \cm{using
our first design shown in Figure ??}. In this chapter I will outline the
planned fabrication process, reporting on which steps have been undertaken thus
far, and the difficulties which we have yet to overcome.

The microfabrication techniques described below were undertaken at the London
Centre for Nanotechnology (LCN). I would like to thank their technicians and
staff, whose input and assistance has been invaluable to this project.

\section{Overview of the fabrication procedure}

% TODO I grabbed this Madou2002 citation from Treutlein, but I should ensure
% that it is legit
Our trap has been designed so that the size of the wires is small compared to
the size and height of the trapped cloud. As such trapping wires on the chip
are as small as \SI{3}{\micro\meter} in width. Such small features can be
produced using standard photolithography techniques.~\cite{Madou2002} However,
the maximum height of features produced in these procedures is usually of the
order \SI{100}{\nano\meter}.

We must ensure that the wires are capable of carrying the currents that are
outlined in chapter~\ref{design}.  Other experiments with two-layer atom chips
have reported a current density of \cm{??} in the trapping wires. For our wires
this would require a minimum height of \cm{??} to carry our trapping
currents, much higher than can be achieved with photolithography techniques.

Hence, we have begun our protoypting using a combination of microfabrication
techniques and electroplating, the former to produce the footprint of the
trapping wires and the latter to build up to a height suitable for our
purposes. This is a common technique for constructing atom chips and is
described in~\cite{2011Ac, Lev2003}.

The fabrication process starts with a \ph{4 inch silicon wafer of some sort},
on which we can fabricate twelve \ph{2cm by 2cm} chips. This process can be
briefly summarised as follows:
\begin{enumerate}
\item Lay down thin (\SI{60}{\nano\meter}) seed layer of gold for each chip.
\item Spin coat a thick (\SI{6}{\micro\meter}) layer of photoresist.
\item Expose and develop the thick photoresist so as to create a ``mould''
in the desired shape of the wires.
\item Dice the wafer to produce individual chips.
\item Electroplate the chip, such that wires \ph{grow up} in the mould to the
desired height.
\item Chemically etch the chip so as to remove the seed layer and
  \ph{electrically isolate} the wires from each other.
\end{enumerate}

An example of such a chip is shown in \ph{\myfigref{TODO}. (Picture of the chip
with the flying wire.)} This chip has two flaws: one broken wire
\ph{(highlighted in the figure)} and imperfect alignment of the trapping wires
to the chip edge. Both of these are typical problems that can arise, and are
discussed further below.

As dicussed \ph{above (where?)} a future aim of the molecule chip project is to
integrate microwave guides on the chip. These guides must allow good overlap of
the microwave fields and the molecule trapping region. Our design achieves this
by positioning the microwave guides on a second layer, directly above the
trapping wires. \cm{Need to ensure the design is already fully described so
this is coherent.} We have not yet attempted the following stages of
fabrication, but we anticipate that they will be as follows:
\begin{enumerate}[resume]
    \item Spin coat chip with an insulating layer of polyimide.
    \item Perform standard photolithography to lay down microwave guides on the
      chip.
\end{enumerate}
These steps are discussed further in section~\ref{fab:planned}.

In this chapter I will first outline the main fabrication methods that we have
relied on. I will then describe our particular fabrication procedure up to the
point that we have achieved. I will then discuss the planned next steps for
creating a two-layer chip with microwave guides.

\section{Methods}

\cm{Spin coating section? (commented)}

%\subsection{Spin coating}

%Spin coating is a procedure for distributing a uniform layer of liquid such as
%photoresist  across a sample. It is typically followed by baking to
%\ph{solidify} the layer. We use spin coating to apply layers of photoresist to
%our wafers and intend to use it to deposit polyimide

\subsection{Lithography}

Lithography is the process of projecting the image of a design onto the surface
of the surface of the substrate. The usual procedure is to coat the substrate
in a positive (negative) \cm{check sign} photoresist, which can be exposed to
ultraviolet light. Exposed regions of the photoresist will react with the
light. The photoresist can be developed causing the (un)exposed regions to lift
off while the rest of the photoresist remains attached to the wafer.

A substrate can therefore be \cm{patterned} by controlling the region of
exposure. This is commonly done by using a mask to project a desired shadow
during exposure~\cite{}, however we have used a Heidelberg direct
writer~\cite{} to perform photolithography.

The direct writer operates by scanning a \ph{UV laser} over the surface of the
substrate, and printing an uploaded pattern.

\cm{Need things like, power, PR thickness, exposure time, number of writes,
resolution, etc...}

\subsection{Evaporation}

Layers of metal can be applied to the wafer through evaporation. The substrate
is placed into a belljar \cm{can we be specific about type of belljar here?}
with the target side facing an evaporation target, in our case gold or
\cm{chromium} with the latter used as an adhesion layer for the former. The
belljar is pumped down to vacuum of pressure below \ph{??}. A current is then
passed through the target to \cm{induce} evaporation of the metal onto the
substrate. A shutter is used to block deposition onto the substrate until the
desired current has been reached.

It is common to combine this techinque with lithography. A layer of photoresist
between evaporated metals and the substrate surface can be lifted off to reveal
the desired pattern in metal on the substrate. \cm{Expand??} This process is
depicted in \myfigref{fab:fig:photolith}

\begin{figure}
  %\includegraphics{./figs/2019-01-18_stripline_xsection.png}
  \caption{\cm{Show a photolith process as in body text}}
  \label{fab:fig:photolith}
\end{figure}

\cm{Should I have a picture of the belljar?}

The \ph{belljar} incorporates a \ph{measuring device} which reports the rate of
deposition and automatically shuts off deposition once the desired thickness
has been reached by closing the shutter. \cm{What determines/ limits the deposition
rates? What happends when we have too much current?}

The deposition rate of gold is typically \ph{??}.~\cite{} As discussed above 
a thickness of \ph{5um} is required for the chip's trapping wires. Achieving
this with evaporation would take \ph{a long time}. \ph{Other techniques like
sputtering are also no good.}

\subsection{Electroplating}

In electroplating a target is attached to an \cm{anode} and submerged in a
\ph{solution of some sort} along with a cathode. Current is then passed between
the anode and the cathode, causing \cm{ions in the solution} to be deposited on
the sample. \cm{Need a cartoon to show this process and a good citation or
two.}

\cm{Need MUCH MORE technical detail: equations for deposition rate, that sort
of thing.}

In our case we have electroplated gold onto gold targets. The solution used was
\ph{this thing, with this rate and etc.}. The solution which was heated at
\ph{65C} in a water bath. It was found that best electroplating results were
achieved when the solution was agitated by bubbling throughout the process.
\cm{Is this right or was stirring better?}.  The results of our electroplating
process are discussed in section~\ref{fab:procedure}.

This process deposits additional gold on top of gold at rates of \ph{rage of
sensible deposition rates}. Hence it is possible to reach the desired thickness
of our wires in a few minutes.

The downside to electroplating is that gold will be deposited indistinguishably
onto all surfaces which are in good electrical contact with the \cm{anode} and
the solution. The procedure for electroplating a desired pattern is commonly
used for fabrication of atom chips~\cite{}. A thin seed layer of gold is
deposited on the surface and a \cm{`mould'} of photoresist is used to give form
to the wires during electroplating. This is described further in
section~\ref{fab:procedure}.

\subsection{Chemical etching}

\section{Fabrication procedure}
\label{fab:procedure}

\cm{remember to include profilometer stuff}

\subsection{Creating the seed layer}

Fabrication of such a chip begins with a \ph{4 inch silicon wafer}. The wafer
is cleaned first with acetone, then isopropyl alcohol and deionised water. The
wafer is then cleaned with an oxygen plasma and undergoes a dehydration bake.
This ensures that the wafer is clean and \ph{free of any absorbed water} that
might interfere with the later steps.

The first task is to lay down a seed layer of gold, which we can electroplate
the wires onto. This is achieved by evaporation of gold onto the wafer.
However, for a later step in which we will dice the wafer into individual
chips, we require that there are tracks of bare silicon for
dicing.\footnote{The \ph{dicing saw} is capable of cutting through metal, but
there is a high chance that this procedure will fail, so it is more reliable to
leave tracks for dicing.}

\cm{Need more details on what spin coating is}
We therefore begin by spin coating the wafer with a \ph{thin, how thin?} layer
of \ph{SR????} photo-resist. This is a standard procedure devised by LCN staff.
\cm{Better way to say this??} We expose the photoresist using the 

\ph{Spin coat and expose track lines}

\section{Planned fabrication of microwave layer}
\label{fab:planned}

