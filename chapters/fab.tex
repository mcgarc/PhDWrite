We have not yet produced a final chip trap, however we have made good progress
in production of a number of prototypes of the trapping wire layer using a
prototype design shown in \myfigref{fab:fig:design1}. In this chapter I will
outline the planned fabrication process, reporting on which steps have been
undertaken thus far, and the difficulties which we have yet to overcome.

\begin{figure}[h]
    \begin{subfigure}{.5\textwidth}
  \centering
      \includegraphics[width=0.5\textwidth]{figs/wafer.png}
      \caption{}
    \end{subfigure}
    \begin{subfigure}{.5\textwidth}
  \centering
      \includegraphics[width=0.5\textwidth]{figs/z_trap_fanouts_inset_scale.png}
      \caption{}
    \end{subfigure}
  \caption{
    \mt{Make this larger for better use of space}
    Prototype design of a chip that has been used for fabrication. This design
    was originally presented in the ESA, but the wire layout has been refined
    as discussed in chapter~\ref{design}. This design consists of twelve chips
    on a \SI{4}{\inch} wafer, shown in (a). Grey squares represent the seed
    layer of gold for through-mask electroplating. The gold represents the
    trapping wires. The blue represents where microwave guides will be
    positioned in relation to the trapping wires. Detailed view can be seen in
    (b) and the inset.
  }
  \label{fab:fig:design1}
\end{figure}

All the microfabrication techniques described below except for the
electroplating were undertaken at the London Centre for Nanotechnology (LCN). 

\section{Overview of the fabrication procedure}

% TODO I grabbed this Madou2002 citation from Treutlein, but I should ensure
% that it is legit
Our trap has been designed so that the size of the wires is small compared to
the size and height of the trapped cloud. As such trapping wires on the chip
are as small as \SI{3}{\micro\meter} in width. Such small features can be
produced using standard photolithography techniques~\cite{Madou2002}. However,
the maximum height of features produced in these procedures is usually of the
order \SI{100}{\nano\meter}.

We must ensure that the wires are capable of carrying the currents that are
outlined in chapter~\ref{design}.  Other experiments with two-layer atom chips
have reported a current density of $j=\SI{6E10}{\ampere\per\meter\squared}$ in
the trapping wires. For our wires this would require a minimum height of $h
\sim I w/j = \SI{1}{\ampere} \times
\SI{10}{\micro\meter}/\SI{6E10}{\ampere\per\meter\squared} \sim
\SI{1}{\micro\meter}$  to carry our trapping currents.This is much higher than
can be achieved with photolithography techniques.

Hence, we have begun our prototyping using a combination of microfabrication
techniques and electroplating known as through-mask
electroplating~\cite{Ruythooren_2000}.. This is where photolithography is used
to produce a mould on top of a thin seed layer of metal. The metal forms the
anode for electroplating, and the photoresist layer prevents deposition onto
areas that it covers. This allows thick wires to be deposited into the mould.
After electroplating the seed layer can be etched away.  This is a common
technique for constructing atom chips and is described in \inlinerefs{2011Ac,
Lev2003, KOUKHARENKO2004600}. \mt{Mike highlighted this, I don't think the
inlinerefs command is sensible.}

The fabrication process starts with a four inch silicon wafer on which we can
fabricate twelve \SI{2}{\centi\meter} by \SI{2}{\centi\meter}  chips. This
process can be briefly summarised:
\begin{enumerate}
\item Lay down thin (\SI{60}{\nano\meter}) seed layer of gold for each chip.
\item Spin coat a thick (\SI{6}{\micro\meter}) layer of photoresist.
\item Expose and develop the thick photoresist so as to create a ``mould''
in the desired shape of the wires.
\item Dice the wafer to produce individual chips.
\item Electroplate the chip, such that wires are formed in the mould to the
desired height.
\item Remove the photoresist mould.
\item Chemically etch the chip so as to remove the seed layer and
  electrically isolate the wires from each other.
\end{enumerate}
%
\cm{I think this can turn into "electrically isolate for the final design}

\cm{
An example of such a chip is shown in \ph{\myfigref{TODO}. (Picture of the chip
with the flying wire.)} This chip has two flaws: one broken wire
\ph{(highlighted in the figure)} and imperfect alignment of the trapping wires
to the chip edge. Both of these are typical problems that can arise, and are
discussed further below.
}

As discussed in chapter~\ref{intro}, a future aim of the molecule chip project is to
integrate microwave guides on the chip. These guides must allow good overlap of
the microwave fields and the molecule trapping region. Our design achieves this
by positioning the microwave guides on a second layer, directly above the
trapping wires. We have not yet attempted the following stages of
fabrication, but we anticipate that they will be:
\begin{enumerate}[resume]
    \item Spin coat chip with an insulating layer of polyimide.
    \item Perform standard photolithography to lay down microwave guides on the
      chip.
\end{enumerate}
These steps are discussed further in section~\ref{fab:planned}.

In the rest of this chapter I will describe the key microfabrication methods
and the three fabrication stages: preparation of chips for electroplating,
electroplating and the planned fabrication of the microwave layer.

\section{Microfabrication methods}

We use several fabrication techniques multiple times throughout the process.
Here I give an overview of some of them.

\subsection{Spin coating}
\label{fab:spin}

Spin coating is a procedure for distributing a uniform layer of liquid such as
photoresist  across a substrate~\cite{Cohen2011}. It is typically followed by
baking to solidify the layer. We use spin coating to apply layers of
photoresist to our wafers and intend to use it to deposit polyimide.

The substrate is mounted on a spin-coater
%
\cm{such as the one shown in \ph{some figure}}
%
and the liquid is applied to the centre of the substrate by a pipette. The
spin-coater then rotates the sample at a low rotational speed to spread the
liquid, it is then ramped to several thousand revolutions per minute.
This uniformly distributes the liquid across the substrate.

The thickness of the coating is dependent on the properties of the liquid
\cm{the viscosity? Do I need to give an equation here so I can talk about
polyimide later?} and the rotational speed of the substrate during coating.
The maximum speed and the profile of the ramp can be used to control the
resulting thickness.

We use spin-coating to apply photoresists. These processes are described in
further detail in section~\ref{fab:prep}. We also intend to use spin-coating
to apply the insulating layer of polyimide to each chip. Applying the
polyimide is a more involved process because it has to cover large
(\SI{5}{\micro\meter} high) features. In this case multiple coatings are
applied to ensure a smooth surface. \cm{This is discussed further in
section~\ref{fab:planned}.}

\subsection{Lithography}

Lithography is the process of projecting the image of a design onto the surface
of the substrate. The usual procedure is to coat the substrate in a
photoresist, which can be exposed to ultraviolet light. Exposed regions of the
photoresist will react with the light. The photoresist can be developed causing
these regions to lift off while the rest of the photoresist remains attached to
the wafer.\footnote{This describes the procedure for a positive photoresist.
For negative photoresists the unexposed regions are removed on development, but
these are not used in this work.} In this section I will present the standard
procedure for patterning metal onto a substrate with
lithography~\cite{Madou2002}.

A substrate can therefore be patterned by controlling the region of exposure.
This is commonly done by using a mask to project a desired shadow during
exposure~\cite{Madou2002}, however we have used a Heidelberg DWL 66, a direct
writer, to perform photolithography.  \cm{Need a heidelberg cite.} An exposure
pattern is uploaded to the direct writer. An ultraviolet laser is then raster
scanned across the surface to be patterned. The laser is switched on in the
regions where exposure is desired. This process is depicted for a positive
photoresist in \myfigref{fab:fig:methods}{a}. The Heidelberg DWL 66 is capable
of producing features down to \SI{300}{\nano\meter} in size.  Patterning of our
wafer typically takes less than one hour.  Different exposure energies are
required depending on the type of photoresist and its thickness. All our
exposures use one scan across the chip, so the total exposure energy is a
function only of the laser intensity.  \cm{Is it definitely only one scan?} Two
positive photoresists are used in this fabrication procedure to produce thin
and thick layers. The details of both of these are discussed further in
section~\ref{fab:prep}. Some photoresists require rehydration after exposure,
which can take several hours.

\begin{figure}[h]
\vspace{0.8cm}
\centering
  \begin{overpic}[width=0.8\textwidth]{figs/fab/cartoon/lith.pdf}
    \put(10,-5){(a)}
    \put(35,-5){(b)}
    \put(61,-5){(c)}
    \put(86,-5){(d)}
    \put(-1,6){$h$}
    \put(23.2,-0.4){$x$}
  \end{overpic}
  \vspace{10mm}
  \caption{Deposition process of gold onto silicon. A top-down view of the
  target is shown in the top row, with a cross section along the dashed line
  shown in the middle row. The bottom row shows a profile of the target (as
  discussed in section~\ref{fab:profile}). Column (a) shows positive
  photoresist (purple) with the exposed region highlighted (light blue). The
  raster scan of the direct writer laser across the substrate is highlighted in
  the top view.  In column (b) the resist is developed and the bare silicon
  exposed. In column (c) gold (yellow) is deposited across the target, e.g. by
  evaporation (as discussed in section~\ref{fab:evap}). In column (d) liftoff
  is performed, removing the remaining photoresist and gold deposited on top of
  it. This leaves only the desired pattern on the silicon.  The relative
  heights of the materials are exaggerated for the purposes of illustration.
  }
  \label{fab:fig:methods}
\end{figure}


Following patterning of the photoresist, the sample is placed in a developing
solution. This removes the exposed photoresist, exposing the silicon for
deposition. This is shown in \mysubfigref{fab:fig:methods}{b}. Development at
room temperature can take several hours, but robust samples such as the ones
described herein can be placed in a heating sonicator and developed in a few
minutes.

In the next section we will describe how metal can be deposited onto the sample
by evaporation (shown in \mysubfigref{fab:fig:methods}{c}). After this the
unwanted metal is removed along with the remaining photoresist in what is known
as the `lift-off' process. Photoresist remover is applied to the solution for
approximately one minute (depending on the exact photoresist and remover used).
This leaves only the intended pattern of metal on the substrate, as in
\mysubfigref{fab:fig:methods}{d}.

Variations on this photolithography method can be used to achieve different
results. A simple example would be deposition of two metals onto the surface.
We will see in section~\ref{fab:prep} that chromium can be used as an
adhesion layer for gold. We will also show how a thick photoresist mould can be
used to create tall features by electroplating.

\subsection{Evaporation}
\label{fab:evap}

This section details the evaporation procedure. In the previous section we used
the example of direct deposition of gold on to silicon (see
\mysubfigref{fab:fig:methods}{c}). In reality a thin adhesion layer of chromium
is required for good results~\cite{Madou2002}.

We performed evaporation using an Edwards A306 bell jar evaporator.  The
substrate is placed inside the belljar with the target side facing a sample of
the metal to be evaporated. The belljar is then pumped down by a scroll and
turbo to pressures below $10^{-6}\si{\milli\bar}$ for evaporation. Multiple
metals can be placed on a carousel to allow a different metal to be used for
subsequent depositions.

\cm{
%
\begin{figure}
  \centering
  \ph{Photo of bell jar with evaporation cartoon(?)}
  \caption{\ph{caption}}
  \label{fab:fig:belljar}
\end{figure}
%
}

A current is then passed through the target to induce evaporation of the metal
onto the substrate. A shutter is used to block deposition onto the substrate
until the desired current has been reached.

The Edwards bell jar evaporator incorporates a FTM7 deposition monitor, which
reports the rate of deposition and automatically shuts off deposition once the
desired thickness has been reached by closing the shutter. \cm{What
determines/ limits the deposition rates? What happens when we have too much
current?}

The deposition rate of gold is typically \SI{0.2}{\nano\meter\per\second}.
%
\cm{Find a way to cite LCN for this and maybe some other factoids.}
%
As discussed above a thickness of \SI{5}{\micro\meter} is desirable for the
chip's trapping wires. Achieving this with evaporation would take over an hour,
and it would not be possible to load the evaporator with enough gold to last
this long. This is why we must use electroplating to achieve the desired
thickness.

\subsection{Profiling}
\label{fab:profile}

The Bruker DektakXT stylus profiler is used to characterise features on the
substrate. A gold stylus is positioned onto the surface and dragged across. As
the stylus comes into contact with features its height will change, allowing a
profile of the surface to be measured. An example of the profile of a substrate
is given for different stages of the lithography process in
\myfigref{fab:fig:methods}. The Bruker Dektak allows profiling over a wide
range of feature sizes, from \SI{1}{\nano\meter} to \SI{1}{\milli\meter}, and
so is well suited to profiling our chips.
%
\cm{Figure out how to cite this}.

\cm{Maybe some raster scans?}

\section{Preparation for electroplating}
\label{fab:prep}

In this section I will describe the first stage of fabrication, which takes us
up to the electroplating stage. These steps (except for the dicing) have been
performed in a cleanroom.

Fabrication begins with a \SI{4}{\inch} silicon wafer. The wafer is cleaned first
with acetone, then isopropyl alcohol and deionised water. The wafer is then
cleaned with an oxygen plasma and undergoes a dehydration bake.  This ensures
that the wafer is clean and free of any absorbed water that might
interfere with the later steps.

The wafer is then spin coated with a thin ($\sim\SI{500}{\nano\meter}$) layer
of Dupont S1805 photoresist. This is followed by baking on a hotplate at
\SI{115}{\celsius} for one minute, and developing in Microposit MF319
developer, with gentle agitation for one minute (or as required).

The wafer is then placed into the Heidelberg direct writer for exposure, to
create the pattern of seed layers shown in \myfigref{fab:fig:design1} (grey
squares). This pattern is a simple grid of \SI{2}{\centi\meter} by
\SI{2}{\centi\meter} squares which will later be diced to become individual
chips.  It is important that the chips are separated at this stage so that
tracks of bare silicon are exposed for dicing. The required exposure energy is
\SI{20}{\milli\joule\per\centi\meter\squared}.

We use evaporation to deposit a \SI{15}{\nano\meter} thick layer of chromium onto the
wafer. This is an adhesion layer for the \SI{50}{\nano\meter} gold layer which is
deposited next. This gold layer will act as the seed layer for electroplating.
The chromium and gold layer deposition steps are shown in
\mysubfigref{fab:fig:prep}{b} and (c) respectively. The photoresist (not
pictured) is then removed using Microposit 1165 remover.

The next stage is to create the photoresist mould for electroplating. A thick
(\SI{6}{\micro\meter}) layer of Dupont SPR220-7 photoresist is spin coated on to the
wafer. The wafer is then baked at \SI{90}{\celsius} for two minutes, then
immediately afterwards at \SI{120}{\celsius} for three minutes. The sample can
then be exposed to produce the wire pattern shown in
\myfigref{fab:fig:design1} (gold wire layout), with an exposure
energy of \SI{140}{\milli\joule\per\centi\meter\squared}.

After exposure, the sample must undergo rehydration for at least 45 minutes,
although overnight is preferable. A post-exposure bake at the same temperatures
and times as the pre-exposure bake is required, before the final step of
developing in Microposit MF319.

At this stage, the wafer holds twelve chips, with electroplating moulds to
produce the trapping wires shown in \myfigref{fab:fig:design1}. The moulds,
formed of thick photoresist are positioned above a seed layer of gold, as
depicted in \mysubfigref{fab:fig:prep}{d}. These twelve chips can be divided
into separate pieces for electroplating and ultimately for use in the
experiment. This is done using a Disco DAD3220 dicing saw.

At this point in the procedure it is useful to ensure that the creation of
the mould has been successful. This can be achieved by characterisation of the
mould with the stylus profilometer.
%
\cm{Profilometer results pls}

The wafer is then diced and the individual chips are ready to be electroplated.

\begin{figure}[h]
\vspace{0.8cm}
\centering
\begin{tabular}{cccc}
  \begin{overpic}[width=0.22\textwidth]{figs/fab/cartoon/a.pdf}
    \put(0,40){(a)}
  \end{overpic} &
  \begin{overpic}[width=0.22\textwidth]{figs/fab/cartoon/b.pdf}
    \put(0,40){(b)}
  \end{overpic} &
  \begin{overpic}[width=0.22\textwidth]{figs/fab/cartoon/c.pdf}
    \put(0,40){(c)}
  \end{overpic} &
  \begin{overpic}[width=0.22\textwidth]{figs/fab/cartoon/d.pdf}
    \put(0,40){(d)}
  \end{overpic}
\end{tabular}
  \caption{Cross section of the preparation of chips for electroplating. The
  bare wafer (black) is shown in (a). The chromium adhesion layer (grey) is
  applied by evaporation (b), followed by the thicker gold seed layer (yellow,
  c). Finally the thick SPR220-7 photoresist mould for the wires (purple) is
  produced by lithography (d). The photoresist mould pictured here is
  not to be confused with the thin S18105 photoresist pattern used to create
  the seed layer patches.}
  \label{fab:fig:prep}
\end{figure}

\section{Electroplating}

In electroplating a conductive target is connected to a circuit as an anode and
placed into an electrolytic solution along with a cathode. Current passed
through the solution causes ions in the solution to be deposited onto the
target. This is illustrated in \mysubfigref{fab:fig:etch}{a}. An overview
of electroplating can be found in \inlineref{Schlesinger2011}.
%
\cm{This ref again, check it is ok!}

Here we use the through-mask electroplating method to deposit thick gold wires
into the regions that are not covered by the photoresist mould, as shown in
\myfigref{fab:fig:eplate}. This method allows us to produce wires up to the
thickness of the photoresist height. Above this the wires will begin to
``mushroom,'' spreading out across the top of the photoresist and losing their
shape. Since we are able to produce a photoresist layer of \SI{6}{\micro\meter}
in height, the desired wire thickness of \SI{5}{\micro\meter} is
possible~\cite{Ruythooren_2000}..

\cm{include photo of apparatus}
%
\begin{figure}
\vspace{0.8cm}
\centering
  \begin{overpic}[width=0.22\textwidth]{figs/fab/cartoon/eplate.pdf}
    \put(-10,100){(a)}
    \put(28.3,92.5){$I$}
  \end{overpic}
  \caption{
    A chip (c.f.\myfigref{fab:fig:prep}) is submerged in a gold electrolyte
    (light blue) along with an electrode (grey mesh). These are connected to a
    current supply to enable current flow and deposition of gold ions (yellow
    circle)  is depicted. The solution is held at \SI{58}{\celsius} and
    agitated by a stirrer and bubbler.
  %Subfigure (b) shows a photograph of our apparatus. A beaker
  %containing the electrolyte is placed in a water bath for heating during the
  %procedure.
  }
  \label{fab:fig:eplate}
\end{figure}

The height $h$ achieved in a deposition of duration $t$ is given by the Faraday
equation~\cite{Ruythooren_2000}
%
\begin{equation}
  h = \left(\frac{\alpha I M}{nFA\rho}\right)t
\end{equation}
%
where $I$ is the current, $F=\SI{96.5}{\kilo\ampere\second\per\mole}$ is the
Faraday constant, and other parameters with values specific to our gold
deposition are: $\alpha\sim0.9$, the current efficiency; $M =
\SI{197}{\gram\per\mole}$ the molar mass;
$\rho=\SI{19.32}{\gram\per\centi\meter\cubed}$, the density of the deposited
metal; $n=1$, the charge on the deposited ions in units of electron charge; and
$A\sim\SI{1}{\centi\meter\squared}$ is the area for plating.

We therefore have a relationship between the current, the target height and the
time,
%
\begin{equation}
  h \approx \left(
  \SI[per-mode=fraction]{1e-10}{\meter\cubed\per\ampere\per\second} \right)
  \times\frac{It}{A}.
\end{equation}
%
For our electrolytic solution, we have used Metakem Goldbath-SF, which
is rated for currents between \SIrange{1}{15}{\milli\ampere}. This suggests
that we will be able to achieve a few microns of thickness within a few minutes
of electrocoating. However we do not know the exact value of the current
efficiency for our experiment, so our coating time and operating current have
been determined experimentally.
%
\cm{Should cite metakem}

Metakem Goldbath-SF is chosen because it produces very pure (99.99\%) deposits,
and will not react with our photoresist. The effectiveness of this product has
been demonstrated for a similar design in \inlineref{Treutlein2008}.

Our apparatus for the electroplating step is shown in
\mysubfigref{fab:fig:eplate}{b}. The electrolytic solution is placed in a
beaker, which itself is placed in a water bath held at \SI{58}{\celsius}. The
bath is heated using a hotplate with magnetic stirrer. Some time is allowed for
thermalisation and then the target chip and the cathode are placed in the
solution. The chip is held in position by a stiff insulated wire, which also
carries current to the seed layer. The insulation on the wire prevents an
increase in the plating area.  The cathode is a grid of platinised titanium
which has been cut to the size of our beaker. This was also purchased from
Metakem.

A bubbler is placed to agitate the solution near to the chip surface. This in
combination with gentle stirring ensures good circulation of the solution and
hence prevents localised depletion of the ions near to the chip
surface~\cite{Schlesinger2011} \cm{also cite conversation with S Etienne?}.

After electroplating the chips are rinsed with deionised water, dried and stored
for transport to the LCN cleanroom for the final fabrication steps.

We determined experimentally that electroplating at $\SI{15}{\milli\ampere}$
for duration $\SI{400}{\second}$ reliably produced wires of height
\SI{5}{\micro\meter} above the seed layer. This can be confirmed by profiling
the surface as described in section~\ref{fab:profile}, and can be undertaken
before and after the removal of the photoresist mould. 
%
\cm{Profile}

\section{Photoresist liftoff and etching}

After electroplating the photoresist mould is removed with Microposit Remover
1165 as described in section~\ref{fab:prep}. The chip is now as pictured in
\myfigref{fab:fig:etch}{b}, with wires formed to the desired heights but all
connected electrically through the seed layer. The next step is to etch the
seed layer and the adhesion layer so that the wires are separated.

\begin{figure}[h]
\vspace{0.8cm}
\centering
\begin{tabular}{cccc}
  \begin{overpic}[width=0.22\textwidth]{figs/fab/cartoon/e.pdf}
    \put(0,40){(a)}
  \end{overpic} &
  \begin{overpic}[width=0.22\textwidth]{figs/fab/cartoon/f.pdf}
    \put(0,40){(b)}
  \end{overpic} &
  \begin{overpic}[width=0.22\textwidth]{figs/fab/cartoon/g.pdf}
    \put(0,40){(c)}
  \end{overpic} &
  \begin{overpic}[width=0.22\textwidth]{figs/fab/cartoon/h.pdf}
    \put(0,40){(d)}
  \end{overpic}
\end{tabular}
  \caption{Cross section of the cleaning and etching of the chips following
  electroplating.  The electroplated chip (a) is cleaned to remove the
  photoresist layer (b). A gold etch (c) is performed, followed by a chromium
  etch (d) to produce electrically isolated wires. Colours are as in
  \myfigref{fab:fig:prep}.}
  \label{fab:fig:etch}
\end{figure}

\cm{Need to figure out the actual name of the etchant}
%
The gold etch is performed by placing the chip into a beaker of Iodine etchant,
which etches gold at a rate of \SI{5}{\nano\meter\per\second}. This means that
an etch of \SI{10}{\second} will remove the seed layer. After this the chip is
immediately transferred to a beaker of deionised water and then rinsed. Since
the seed layer thickness is significantly smaller than the dimensions of the
wire, the cross-sectional area will not be significantly altered in this time.
This can be confirmed with profiling (see below). At this stage the chip is as
picture in \mysubfigref{fab:fig:etch}{c}.

\cm{Need to find out what the Cr etchant is called}
%
To complete the separation of the wires, the chromium layer must also be etched
in the same way. The process is the same as for the gold etch: the chip is
submerged in a beaker of the etchant and after the pre-determined exposure time
it is transferred to deionised water and then rinsed. This stage of the
fabrication is represented in \mysubfigref{fab:fig:etch}{d}. 

To ensure that the etches have been successful, stylus profiling is once again
performed, results are shown in \myfigref{fab:fig:endprofile}. We ensure that
the wires are of a sufficient height and width to achieve the required currents
as detailed in chapter~\ref{design}. Visual inspection under an optical
microscope is also useful to ensure that the silicon has been completely
exposed. A multimeter is used to confirm that there are no breaks in the wires.



\begin{figure}
\centering
  \includegraphics[width=0.8\textwidth]{figs/fab/wire_profile.pdf}
  \caption{A profile of trapping wires after etching, showing wires
  electroplated up to a height of approximately \SI{3}{\micro\meter}.
  \cm{Mike says I need to make a better version of this figure. This is on my
  todos, but I really want to do it for the new chip design.}
  }
  \label{fab:fig:endprofile}
\end{figure}

\section{Planned fabrication of microwave layer}
\label{fab:planned}

In chapter~\ref{intro} 
we described how a molecule chip could allow strong coupling between \CaF{}
molecules and a microwave field, however for this to be possible there must be
good overlap between the microwave field and the trapped
molecules~\cite{Andre2006}.
%
\cm{better cite needed maybe?}

This overlap has previously been achieved for atoms in a magnetic
trap~\cite{Treutlein2008}. An insulating layer is spin-coated onto the trapping
wires so that a CPW \cm{In final version I need to check that first use of
abbreviation is explained.} can be fabricated by
photolithography. The trap centre can be positioned in the region where the
microwave field is strongest. This has potential to allow coherent control of
the molecules and potentially even sideband cooling into the motional ground
state~\cite{Andre2006}.
%
\cm{more discussion elsewhere?}

We will develop our fabrication procedure further so that we can produce such a
chip. The first stage will be to spin coat the insulating layer on top of an
etched chip, as shown in \mysubfigref{fab:fig:cpw}{a}. Again taking our lead
from \inlineref{Treutlein2008}, we will use polyimide. Polyimide is chosen due
to its low dielectric loss tangent ($\tan\delta_e = 0.016$) which will minimise
conductor losses in the waveguide~\cite{Collin2007, Simons2004}.
\footnote{Although a suitable dielectric is
important to minimise conductor losses in a CPW, this chip will operate at room
temperature and so radiation losses will dominate. This was discussed in detail
in the early stage assessment and will not be repeated here.}
%
\cm{This \emph{will} be repeated in the thesis so I can link this better.}

When spin coating the polyimide it is essential that we are able to produce a
flat surface onto which we can fabricate the microwave layer. We can do this by
applying multiple layers of polyimide on the spin coater, so that any bumps are
smoothed out. This is known as planarisation and is discussed further in
\inlineref{Treutlein2008}.

\begin{figure}[h]
\vspace{0.8cm}
\centering
\begin{tabular}{cc}
  \begin{overpic}[width=0.22\textwidth]{figs/fab/cartoon/i.pdf}
    \put(0,40){(a)}
  \end{overpic} &
  \begin{overpic}[width=0.22\textwidth]{figs/fab/cartoon/j.pdf}
    \put(0,40){(b)}
  \end{overpic}
\end{tabular}
  \caption{Cross section of the fabrication of a two-layer chip. A layer of
  polyimide (teal) is applied to the chip. This insulating
  layer is shown in (a). Note that in this simplified illustration
  planarisation effects are ignored, and the surface may not be completely
  even. This is discussed further in the body text.
  The CPW wires can be fabricated on top of the polyimide (b) by evaporation.
  }
  \label{fab:fig:cpw}
\end{figure}

After the application of a planarised polyimide layer it will be possible to
fabricate microwave guides on the surface by lithography. The end result is
shown in \mysubfigref{fab:fig:cpw}{b}. We will undertake
further work to determine the required height of the CPW features and where
they must be positioned relative to the wires to achieve the strongest
coupling.
