We have not yet produced a final chip trap, however we have made good progress
in production of a number of prototypes of the trapping wire layer \cm{using
our first design shown in Figure ??}. In this chapter I will outline the
planned fabrication process, reporting on which steps have been undertaken thus
far, and the difficulties which we have yet to overcome.

The microfabrication techniques described below were undertaken at the London
Centre for Nanotechnology (LCN). I would like to thank their technicians and
staff, whose input and assistance has been invaluable to this project.

\section{Overview of the fabrication procedure}

% TODO I grabbed this Madou2002 citation from Treutlein, but I should ensure
% that it is legit
Our trap has been designed so that the size of the wires is small compared to
the size and height of the trapped cloud. As such trapping wires on the chip
are as small as \SI{3}{\micro\meter} in width. Such small features can be
produced using standard photolithography techniques.~\cite{Madou2002} However,
the maximum height of features produced in these procedures is usually of the
order \SI{100}{\nano\meter}.

We must ensure that the wires are capable of carrying the currents that are
outlined in chapter~\ref{design}.  Other experiments with two-layer atom chips
have reported a current density of \cm{??} in the trapping wires. For our wires
this would require a minimum height of \cm{??} to carry our trapping
currents, much higher than can be achieved with photolithography techniques.

Hence, we have begun our protoypting using a combination of microfabrication
techniques and electroplating, the former to produce the footprint of the
trapping wires and the latter to build up to a height suitable for our
purposes. This is a common technique for constructing atom chips and is
described in~\cite{2011Ac, Lev2003}.

The fabrication process starts with a \ph{4 inch silicon wafer of some sort},
on which we can fabricate twelve \ph{2cm by 2cm} chips. This process can be
briefly summarised:
\begin{enumerate}
\item Lay down thin (\SI{60}{\nano\meter}) seed layer of gold for each chip.
\item Spin coat a thick (\SI{6}{\micro\meter}) layer of photoresist.
\item Expose and develop the thick photoresist so as to create a ``mould''
in the desired shape of the wires.
\item Dice the wafer to produce individual chips.
\item Electroplate the chip, such that wires \ph{grow up} in the mould to the
desired height.
\item Chemically etch the chip so as to remove the seed layer and
  \ph{electrically isolate} the wires from each other.
\end{enumerate}

An example of such a chip is shown in \ph{\myfigref{TODO}. (Picture of the chip
with the flying wire.)} This chip has two flaws: one broken wire
\ph{(highlighted in the figure)} and imperfect alignment of the trapping wires
to the chip edge. Both of these are typical problems that can arise, and are
discussed further below.

As dicussed \ph{above (where?)} a future aim of the molecule chip project is to
integrate microwave guides on the chip. These guides must allow good overlap of
the microwave fields and the molecule trapping region. Our design achieves this
by positioning the microwave guides on a second layer, directly above the
trapping wires. \cm{Need to ensure the design is already fully described so
this is coherent.} We have not yet attempted the following stages of
fabrication, but we anticipate that they will be:
\begin{enumerate}[resume]
    \item Spin coat chip with an insulating layer of polyimide.
    \item Perform standard photolithography to lay down microwave guides on the
      chip.
\end{enumerate}
These steps are discussed further in section~\ref{fab:planned}.

In the rest of this chapter I will describe the three key fabrication stages:
preparation of chips for electroplating, electroplating and the planned
fabrication of the microwave layer.

\section{Preparation for electroplating}
\label{fab:prep}

I will begin this section with an overview of each of the key methods that we
use (subsections \ref{fab:prep:spincoat}--\ref{fab:prep:profiling}) and then
explain the perparation in detail (subsection \ref{fab:prep:procedure}).

\subsection{Spin coating}
\label{fab:prep:spincoat}

Spin coating is a procedure for distributing a uniform layer of liquid such as
photoresist  across a sample. It is typically followed by baking to
\ph{solidify} the layer. We use spin coating to apply layers of photoresist to
our wafers and intend to use it to deposit polyimide

\subsection{Lithography}

Lithography is the process of projecting the image of a design onto the surface
of the substrate. The usual procedure is to coat the substrate in a positive
(negative) \cm{check sign} photoresist, which can be exposed to ultraviolet
light. Exposed regions of the photoresist will react with the light. The
photoresist can be developed causing the (un)exposed regions to lift off while
the rest of the photoresist remains attached to the wafer.

A substrate can therefore be \cm{patterned} by controlling the region of
exposure. This is commonly done by using a mask to project a desired shadow
during exposure~\cite{}, however we have used a Heidelberg direct
writer~\cite{} to perform photolithography.

The direct writer operates by scanning a \ph{UV laser} over the surface of the
substrate, and printing an uploaded pattern.

\cm{Need things like, power, PR thickness, exposure time, number of writes,
resolution, etc...}

\subsection{Evaporation}

Layers of metal can be applied to the wafer through evaporation. The substrate
is placed into a belljar \cm{can we be specific about type of belljar here?}
with the target side facing an evaporation target, in our case gold or
\cm{chromium} with the latter used as an adhesion layer for the former. The
belljar is pumped down to vacuum of pressure below \ph{??}. A current is then
passed through the target to \cm{induce} evaporation of the metal onto the
substrate. A shutter is used to block deposition onto the substrate until the
desired current has been reached.

It is common to combine this techinque with lithography. A layer of photoresist
between evaporated metals and the substrate surface can be lifted off to reveal
the desired pattern in metal on the substrate. \cm{Expand??} This process is
depicted in \myfigref{fab:fig:photolith}

\begin{figure}
  %\includegraphics{./figs/2019-01-18_stripline_xsection.png}
  \caption{\cm{Show a photolith process as in body text}}
  \label{fab:fig:photolith}
\end{figure}

\cm{Should I have a picture of the belljar?}

The \ph{belljar} incorporates a \ph{measuring device} which reports the rate of
deposition and automatically shuts off deposition once the desired thickness
has been reached by closing the shutter. \cm{What determines/ limits the deposition
rates? What happends when we have too much current?}

The deposition rate of gold is typically \ph{??}.~\cite{} As discussed above 
a thickness of \ph{5um} is required for the chip's trapping wires. Achieving
this with evaporation would take \ph{a long time}. \ph{Other techniques like
sputtering are also no good.}


\subsection{Profiling}
\label{fab:prep:profiling}

\ph{Explanation of Dektak stylus profilometer (Bruker)}

\subsection{Preparation procedure}
\label{fab:prep:procedure}

I will now describe how the above methods have been used to fabricate our
prototype trapping chip.

Fabrication of such a chip begins with a \ph{4 inch silicon wafer}. The wafer
is cleaned first with acetone, then isopropyl alcohol and deionised water. The
wafer is then cleaned with an oxygen plasma and undergoes a dehydration bake.
This ensures that the wafer is clean and \ph{free of any absorbed water} that
might interfere with the later steps.

The wafer is then spin coated with a thin (\ph{??nm}) layer of \ph{Dupont S1805
photoresist}. This is placed into the Heidelberg direct writer for exposure, to
create the pattern shown in \ph{fig??}. This pattern is a simple grid of
\ph{2cm by 2cm} squares which will later be diced to become individual chips.
It is important that the chips are separated at this stage so that tracks of
bare silicon are exposed for dicing.

We use evaporation to deposit a \ph{15nm} thick layer of chromium onto the
wafer. This is an adhesion layer for the \ph{50nm} gold layer which is
deposited next. This gold layer will act as the seed layer for electroplating.
\cm{Depicted in some cartoon.}

The next stage is to create the photoresist mould for electroplating. A thick
(\ph{5um??}) layer of \ph{Dupont S???? photoresist} is spin coated on to the
wafer and exposed \ph{to produce the pattern shown in fig???}. Once diced, each
individual chip will be able to be connected to the anode, and good electrical
contact will be \ph{formed} between the exposed gold and the solution. The gold
will be depositied inside the mould to form wires.

At this point in the procedure it is useful to ensure that the creation of
the mould has been succesful. This can be achieved by characterisation of the
mould with the \ph{stylus profilometer}. \cm{Here I can present results of a
typical good mould on the profilometer...}

The wafer is then diced and the individual chips are ready to be electroplated.

\begin{figure}[h]
\vspace{2cm}
\centering
\begin{tabular}{cccc}
  \begin{overpic}[width=0.22\textwidth]{figs/fab/cartoon/a.pdf}
    \put(0,40){(a)}
  \end{overpic} &
  \begin{overpic}[width=0.22\textwidth]{figs/fab/cartoon/b.pdf}
    \put(0,40){(b)}
  \end{overpic} &
  \begin{overpic}[width=0.22\textwidth]{figs/fab/cartoon/c.pdf}
    \put(0,40){(c)}
  \end{overpic} &
  \begin{overpic}[width=0.22\textwidth]{figs/fab/cartoon/d.pdf}
    \put(0,40){(d)}
  \end{overpic}
\end{tabular}
%\setlength{\tabcolsep}{\oldtabcolsep}
  \caption{Prepartion of chips for electroplating. The bare wafer (black) is
  shown in (a). The chromium adhesion layer (grey) is applied by evaporation
  (b), followed by the thicker gold seed layer (yellow, c). Finally the
  photoresist \cm{what kind?} mould for the wires (purple) is produced by
  lithography (d).}
  \label{fab:fig:prep}
\end{figure}

\section{Electroplating}

\begin{figure}
\vspace{.5cm}
\centering
\begin{tabular}{cccc}
  \begin{overpic}[width=0.22\textwidth]{figs/fab/cartoon/eplate.pdf}
    \put(-10,100){(a)}
    \put(28.3,92.5){$I$}
  \end{overpic}&
  \ph{Picture of the aparatus (b).}
\end{tabular}
  \caption{The electroplating procedure. Subfigure (a) shows a chip (c.f.\
  \myfigref{fab:fig:prep}) is submerged in a gold electrolyte (light blue)
  along with an electrode (grey mesh). These are connected to a current supply
  to enable current flow and deposition of gold ions (yellow circle).  is
  depicted. The solution is held at \ph{temp} and aggitated by a stirrer and
  bubbler. Subfigure (b) shows a photograph of our apparatus. A beaker
  containing the electrolyte is placed in a water bath for heating during the
  procedure.}
  \label{fab:fig:eplate}
\end{figure}

In electroplating a target is attached to an anode and submerged in a
\ph{solution of some sort} along with a cathode. Current is then passed between
the anode and the cathode, causing \cm{ions in the solution} to be deposited on
the sample. \cm{Need a cartoon to show this process and a good citation or
two.}

\cm{Need MUCH MORE technical detail: equations for deposition rate, that sort
of thing.}

In our case we have electroplated gold onto gold targets. The solution used was
\ph{this thing, with this rate and etc.}. The solution which was heated at
\ph{65C} in a water bath. It was found that best electroplating results were
achieved when the solution was agitated by bubbling throughout the process.
\cm{Is this right or was stirring better?}.  The results of our electroplating
process are discussed in section~\ref{fab:procedure}.

This process deposits additional gold on top of gold at rates of \ph{rage of
sensible deposition rates}. Hence it is possible to reach the desired thickness
of our wires in a few minutes.

The downside to electroplating is that gold will be deposited indistinguishably
onto all surfaces which are in good electrical contact with the \cm{anode} and
the solution. The procedure for electroplating a desired pattern is commonly
used for fabrication of atom chips~\cite{}. A thin seed layer of gold is
deposited on the surface and a \cm{`mould'} of photoresist is used to give form
to the wires during electroplating. This is described further in
section~\ref{fab:procedure}. \cm{Or maybe it is shown in fig??}

\section{Cleaning and etching}

\begin{figure}[h]
\vspace{2cm}
\centering
\begin{tabular}{cccc}
  \begin{overpic}[width=0.22\textwidth]{figs/fab/cartoon/e.pdf}
    \put(0,40){(a)}
  \end{overpic} &
  \begin{overpic}[width=0.22\textwidth]{figs/fab/cartoon/f.pdf}
    \put(0,40){(b)}
  \end{overpic} &
  \begin{overpic}[width=0.22\textwidth]{figs/fab/cartoon/g.pdf}
    \put(0,40){(c)}
  \end{overpic} &
  \begin{overpic}[width=0.22\textwidth]{figs/fab/cartoon/h.pdf}
    \put(0,40){(d)}
  \end{overpic}
\end{tabular}
  \caption{Cleaning and etching of the chips following electroplating. The
  electroplated chip (a) is cleaned to remove the photoresist layer (b). A gold
  etch (c) is performed, followed by a chromium etch (d) to produce
  electrically isolated wires. Colours are as in \myfigref{fab:fig:prep}.}
  \label{fab:fig:etch}
\end{figure}


\section{Planned fabrication of microwave layer}
\label{fab:planned}

