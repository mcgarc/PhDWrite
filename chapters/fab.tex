Fabrication of a chip has been an iterative process, where we have
simultaneously improved upon both the chip's design and our fabrication
techniques.
In this chapter I will present the progress that we have made in
learning how to build a chip trap, and how we plan to integrate a microwave
guide in future experiments. 

All the microfabrication techniques described below except for the
electroplating were undertaken at the London Centre for Nanotechnology (LCN). 

\section{Overview of the fabrication procedure}

% TODO I grabbed this Madou2002 citation from Treutlein, but I should ensure
% that it is legit
Our trap has been designed so that the size of the wires is small compared to
the size and height of the trapped cloud. As such trapping wires on the chip
are as small as \SI{3}{\micro\meter} in width. Such small features can be
produced using standard photolithography techniques~\cite{Madou2002}. However,
the maximum height of features produced in these procedures is usually of the
order \SI{100}{\nano\meter}.

% TODO Need to cite this $j$ number and probably revise this based on what we
% actually say in design chapter. I don't think I need this caluclation here
We must ensure that the wires are capable of carrying the currents that are
outlined in chapter~\ref{design}.  Other experiments with two-layer atom chips
have reported a current density of $j=\SI{6E10}{\ampere\per\meter\squared}$ in
the trapping wires. For our wires this would require a minimum height of $h
\sim I w/j = \SI{1}{\ampere} \times
\SI{10}{\micro\meter}/\SI{6E10}{\ampere\per\meter\squared} \sim
\SI{1}{\micro\meter}$  to carry our trapping currents.This is much higher than
can be achieved with photolithography techniques.

To achieve the required height, we have used through-mask
electroplating~\cite{Ruythooren_2000}. First a substate is coated with a thin
seed layer of gold, then photolithography is used to produce a thick (several
mircron high) mould. The mould covers regions where no further deposition is
required. The substrate can then be electroplated, with the seed layer acting
as the anode, allowing thick wires to be deposited into the mould.
After electroplating the seed layer can be etched away. This is a common
technique for constructing atom chips and is described in \inlinerefs{2011Ac,
Lev2003, KOUKHARENKO2004600}.

The fabrication process for our final chip begins with a four inch silicon
wafer, which we dice into individual \SI{20}{\milli\meter} by
\SI{20}{\milli\meter} dies. The remaining process can be briefly summarised:
\cm{Is there a better way to link this list to the following sections?}
\begin{enumerate}
\item Lay down thin (\SI{60}{\nano\meter}) seed layer of chromium and gold.
\item Spin coat a thick (\SI{6}{\micro\meter}) layer of photoresist.
\item Expose and develop the thick photoresist so as to create a ``mould''
in the desired shape of the wires.
\item Electroplate the chip, such that wires are formed in the mould to the
desired height.
\item Remove the photoresist mould.
\item Chemically etch the chip so as to remove the seed layer and
  electrically isolate the wires.
\end{enumerate}
This process is illustrated in \myfigref{fab:fig:process}

\begin{figure}[h]
\vspace{0.8cm}
\centering
\begin{tabular}{cccc}
  %\begin{overpic}[width=0.22\textwidth]{figs/fab/cartoon/01_barewafer.pdf}
  %  \put(0,40){(a)}
  %\end{overpic} &
  \begin{overpic}[width=0.22\textwidth]{figs/fab/cartoon/03_wafercrauwarrows.pdf}
    \put(0,40){(1)}
  \end{overpic} &
  \begin{overpic}[width=0.22\textwidth]{figs/fab/cartoon/03.5_waferpr.pdf}
    \put(0,40){(2)}
  \end{overpic} &
  \begin{overpic}[width=0.22\textwidth]{figs/fab/cartoon/04_wafermould.pdf}
    \put(0,40){(3)}
  \end{overpic} \\[2cm]
  \begin{overpic}[width=0.22\textwidth]{figs/fab/cartoon/e.pdf}
    \put(0,40){(4)}
  \end{overpic} &
  \begin{overpic}[width=0.22\textwidth]{figs/fab/cartoon/f.pdf}
    \put(0,40){(5)}
  \end{overpic} &
  %\begin{overpic}[width=0.22\textwidth]{figs/fab/cartoon/g.pdf}
  %  \put(0,40){(g)}
  %\end{overpic} &
  \begin{overpic}[width=0.22\textwidth]{figs/fab/cartoon/h.pdf}
    \put(0,40){(6)}
  \end{overpic}
\end{tabular}
  \caption{
    Illustration of the fabrication process. We begin with in (1) with a
    silicon die (black), on which we deposit chromium adhesion layer (grey) and
    a seed layer of gold (yellow). In (2) the entire die is spin coated in
    photoresist (purple). The mould is formed by photolithography techniques
    (3) and then electroplating is used to form tall wires (4). The photoresist
    is removed (5), and then a gold etch followed by a chrome etch (6) are used
    to electronically isolate the features.
  }
  \label{fab:fig:process}
\end{figure}

As discussed in chapter~\ref{intro}, a future aim of the molecule chip project is to
integrate microwave guides on the chip. These guides must allow good overlap of
the microwave fields and the molecule trapping region. Our design achieves this
by positioning the microwave guides on a second layer, directly above the
trapping wires. We have not yet attempted the following stages of
fabrication, but we anticipate that they will be:
\begin{enumerate}[resume]
    \item Spin coat chip with an insulating layer of polyimide.
    \item Perform standard photolithography to lay down microwave guides on the
      chip.
\end{enumerate}
These steps are discussed further in section~\ref{fab:planned}.

In the rest of this chapter I will describe the above process in detail,
including the various pitfalls that we found as we itterated towards a complete
\cm{trapping chip}.

\section{Metal evaporation of seed layer}

Before any fabrication, the die must be cleaned and dehydrated to ensure that
there will be good adhesion to the substrate. A solvent clean with acetone and
isopropyl alchol will remove any organic compounds \cm{check this}. The die is
then rinsed with deionised water and dehydrated in an oxygen plasma for ten
minutes.
The die is then ready for metalization, which here is done by metal
evaporation. To further improve adhesion between the gold and silicon, a thin
($<10\si{\micro\meter}$) intermediary chrome layer is deposited first, followed
by \SI{50}{\micro\meter} of gold.

We performed evaporation using an Edwards A306 bell jar evaporator. Typically
we \cm{metalize} four dies at a time. They are loaded into the belljar, along
with gold and chrome, using a boat and rod respectively.  we deposit gold onto
four dies at a time. The dies are positioned with the polished side facing down
towards the metal. This arrangement is shown in \myfigref{fab:fig:belljar}.

\begin{figure}
  \centering
  \begin{subfigure}[b]{0.22\textwidth}
    \centering
    \begin{overpic}[width=\textwidth]{figs/fab/cartoon/evap.pdf}
      \put(48,6){$I$}
    \end{overpic}
    \caption{}
  \end{subfigure}
  \hspace{2cm}
  \begin{subfigure}[b]{0.22\textwidth}
    \centering
    \includegraphics[width=\textwidth]{figs/fab/belljar.png}
    \caption{}
  \end{subfigure}
  \caption{
    Subfigure (a) schematically shows evaporation of gold (yellow) onto a
    silicon (black) die with chromium (grey) adhesion layer. The shutter
    (dashed line) can block the evaporating gold from being deposited when
    the target height is reached. The Edwards belljar evaporator is shown in
    (b), with the belljar removed and a wafer mounted for deposition.
  }
  \label{fab:fig:belljar}
\end{figure}

The belljar is pumped down to pressures below $10^{-6}\si{\milli\bar}$ over a
few hours. The metal for deposition can be selected from a carousel, and heated
by electric current inducing evaporation.  A shutter is used to block
deposition onto the substrate until the desired current has been reached. It is
then opened to begin deposition.
The Edwards bell jar evaporator incorporates a FTM7 deposition monitor, which
reports the rate of deposition and automatically shuts off deposition once the
desired thickness has been reached by closing the shutter. \cm{What
determines/ limits the deposition rates? What happens when we have too much
current?}

The deposition rate of gold is typically \SI{0.2}{\nano\meter\per\second}.
%
\cm{Find a way to cite LCN for this and maybe some other factoids.}
%
As discussed above a thickness of \SI{5}{\micro\meter} is desirable for the
chip's trapping wires. Achieving this with evaporation would take over an hour,
and it would not be possible to load the evaporator with enough gold to last
this long. This is why we must use electroplating to achieve the desired
thickness.

\section{Spin coating of photoresist}
\label{fab:spin}

Spin coating is a procedure for distributing a uniform layer of liquid such as
photoresist  across a substrate~\cite{Cohen2011}. It is typically followed by
baking to solidify the layer. We use spin coating to apply SPR220-7 \cite{}
photoresist, which will form the mould for the wires.

The die is mounted in the \cm{spin coater details}, and \cm{how much? I can
check pippete...} of SPR220-7 is applied. The die undergoes a
\cm{\SI{2}{\second} ramp to \SI{500}{\rpm} where it is held before a ?second
ramp to \SI{4000}{\rpm} and held for \SI{30}{\second}.} This results in a
nominal \SI{6}{\micro\meter} high coating of photoresist \cm{according to LCN,
but we haven't actually varified this}.
SPR220-7 requires a post-application bake, first at \SI{90}{\celsius} for two
minutes, then immediately afterwards at \SI{120}{\celsius}.

Spin coating the photoresist results in a bead at the edge of the die. This
thick region of photoresist may not receive sufficient exposure to fully
develop later. This can cause defects in features near the edge. It is
possible to remove the bead by inserting an intial exposure and development
step before the lithography discussed in the next section. However, since the
only features that are covered by the bead are the robust wire bond pads, this
is deemed unnecessary for our purposes. \cm{Defects due to the presence of a
bead can be seen in \myfigref{}.}

\section{Lithography of the wire mould}

The common and, perhaps, traditional way to perform photolithography is to use
a \cm{mercury} lamp and a chrome-on-glass mask to cast light onto the
substrate, with the mask casting a shadow so as to illuminate only the desired
region~\cite{Madou2002}. This was the method that we began using at the start of the
project, however we found that it was easier to achieve reliable results by
using the the Heidlberg DWL 66, a direct writer~\cite{}. 

Instead of using a mask to cast a shadow, the direct writer uses a tightly
focused ultraviolet laser, whose beam is raster scanned across the surface. The
beam is then switched on and off so as to produce the pattern that is required.
This process is depicted for a in \myfigref{fab:fig:methods}. The Heidelberg
DWL 66 is capable of producing features down to \SI{300}{\nano\meter} in size.
We use the photoresist \cm{Dupont} SPR220-7, which is a positive photoresist,
meaning that areas exposed to the light are those which will be removed on
developing. For \cm{Dupont} SPR220-7 an exposure energy of \cm{??} is required,
which is administered over three passes of the laser at \cm{intensity}. The
whole scan for one die takes around twenty minutes. This is illustrated in
\mysubfigref{fab:fig:tmep}{a}.

Since designs can be directly uploaded to the direct writer, there is no need
to wait for a third party to construct a mask.  Hence the direct writer allows
rapid prototyping. It also has the benefit of making alignment easier, since
this can be performed automatically by the computer, and any issues with
mask-die contact are avoided entirely.



This photoresist requires a rehydration step after exposure, so it is left at
ambient temperature overnight before it is developed with  \cm{what? for how
long?}. This produces the wafer with mould as depicted in
\myfigref{fab:fig:tmep}{b}.

\begin{figure}[h]
\vspace{0.8cm}
\centering
  \begin{overpic}[width=0.8\textwidth]{figs/fab/cartoon/throughmask.pdf}
    \put(10,-5){(a)}
    \put(35,-5){(b)}
    \put(61,-5){(c)}
    \put(86,-5){(d)}
    \put(-1,6){$h$}
    \put(23.2,-0.4){$x$}
  \end{overpic}
  \vspace{10mm}
  \caption{An illustration of how lithography is used to produce tall wires in
  through-mask electroplating. The top row shows a top-down view of the die,
  with a cross section along the dashed line shown in the middle row. The
  bottom row shows a profile of the target (as discussed in
  section~\ref{fab:inspmould}). Column (a) shows photoresist (purple) above the
  seed layer (\cm{colours as in \myfigref{}}) with the exposed region
  highlighted (light blue). The raster scan of the direct writer laser across
  the substrate is highlighted in the top view.  In column (b) the resist is
  developed and the seed layer is exposed. In column (c) gold is deposited via
  electroplating (as discussed in section~\ref{fab:eplate}) to just below the
  height of the mould. In column (d) the photoresist is removed, leaving the
  tall wires (highlighted with black outline) and seed layers, the latter of
  which will be removed by etching.
  }
  \label{fab:fig:tmep}
\end{figure}


\section{Inspecting the photoresist mould}
\label{fab:inspmould}

\cm{Profiling, with examples and contact testing of the wire bond pads to
ensure no residue PR}

So far we have discussed the process up to the point that we have a photoresist
mould on top of the gold seed layer. The next step is to electroplate so as to
form the tall wires. However it is useful to first examine the mould so that we
can ensure any electroplating target is likely to produce a suitable chip.

The die can be imaged with a microscope for inspection of features. This is
useful to give an overview of key areas, and can be used to identify any
regions with potential defects. When these are identified it is often useful to
be able to examine the contours of the region.

This can be achieved with a stylus profilometer, such as the Bruker DektakXT
which was used here. A stylus profilometer operates by positioning a gold
stylus onto the surface of the die and dragging it in one direction. As
the stylus comes into contact with features its height will change, allowing a
profile of the surface to be measured. Profiling of a surface is illustrated
in the lowest row of \myfigref{fab:fig:tmep}

% TODO
% A typical example...

% Some common problems...

% Show the one that I developed again to make sure the mould was celar (this
% was a lettered chip)

%An example of the profile of a substrate
%is given for different stages of the lithography process in
%\myfigref{fab:fig:methods}. The Bruker Dektak allows profiling over a wide
%range of feature sizes, from \SI{1}{\nano\meter} to \SI{1}{\milli\meter}, and
%so is well suited to profiling our chips.
%%
%\cm{Figure out how to cite this}.
%
%\cm{Maybe some raster scans?}
%
%\cm{Profilometer results pls}
%
%The wafer is then diced and the individual chips are ready to be electroplated.


\section{Electroplating the tall wires}

Now the chip is returned to the Blacket Laboratory for electroplating.
Electroplating does not take place in cleanroom conditions, but it is important
that the dies are treated with great care during transport, and kept sealed
until electroplating can begin. We found that the dies were surprisingly
robust, and were able to be kept for several weeks between exposure and
electroplating.

In electroplating a conductive target (here, the die) is connected to an
electric circuit as an anode and placed into an electrolytic solution along
with a cathode. Current passed through the solution causes ions in the solution
to be deposited onto the target. This is illustrated in
\mysubfigref{fab:fig:etch}{a}, and the results of the process are shown
schematically in \mysubfigref{fab:fig:tmep}{c}. An overview of electroplating
can be found in \inlineref{Schlesinger2011}.
%
\cm{This ref again, check it is ok!}

Here we use the through-mask electroplating method to deposit thick gold wires
into the regions that are not covered by the photoresist mould, as shown in
\myfigref{fab:fig:eplate}. This method allows us to produce wires up to the
thickness of the photoresist height. Above this the wires will begin to
``mushroom,'' spreading out across the top of the photoresist and losing their
shape. As per the discussion in \cm{ref chapter}, we require a minimum wire
height of \cm{\SI{5}{\micro\meter}}. We will see below that we have been able
to reliably produce wires up to a height of \SI{6}{\micro\meter}.

\cm{include photo of apparatus}
%
\begin{figure}
\vspace{0.8cm}
\centering
  \begin{subfigure}[b]{0.22\textwidth}
    \centering
  \begin{overpic}[width=\textwidth]{figs/fab/cartoon/eplate.pdf}
    \put(28.7,91.2){$I$}
  \end{overpic}
    \caption{}
  \end{subfigure}
  \hspace{2cm}
  \begin{subfigure}[b]{0.22\textwidth}
    \centering
    \includegraphics[width=\textwidth]{figs/fab/eplate.png}
    \caption{}
  \end{subfigure}
  \caption{
    The elecroplating scheme is shown schematically in (a). A die is submerged in a gold electrolyte
    (light blue) along with an electrode (grey mesh). These are connected to a
    current supply to enable current flow and deposition of gold ions (yellow
    circle)  is depicted. The solution is held at \SI{60}{\celsius} and
    agitated by a stirrer and bubbler. A photograph of our apparatus is shown
    in (b). The beaker containing the electrolyte is submerged in a water bath
    and aggitated with a stirrer and bubbler.
  }
  \label{fab:fig:eplate}
\end{figure}

The height $h$ achieved in a deposition of duration $t$ is given by the Faraday
equation~\cite{Ruythooren_2000}
%
\begin{equation}
  h = \left(\frac{\alpha I M}{nFA\rho}\right)t
\end{equation}
%
where $I$ is the current, $F=\SI{96.5}{\kilo\ampere\second\per\mole}$ is the
Faraday constant, and other parameters with values specific to our gold
deposition are: $\alpha\sim0.9$, the current efficiency; $M =
\SI{197}{\gram\per\mole}$ the molar mass;
$\rho=\SI{19.32}{\gram\per\centi\meter\cubed}$, the density of the deposited
metal; $n=1$, the charge on the deposited ions in units of electron charge; and
$A\sim\SI{1}{\centi\meter\squared}$ is the area for plating.

We therefore have a relationship between the current, the target height and the
time,
%
\begin{equation}
  h \approx \left(
  \SI[per-mode=fraction]{1e-10}{\meter\cubed\per\ampere\per\second} \right)
  \times\frac{It}{A}.
\end{equation}
%
\cm{Ensure I am not word stealing here...}
For our electrolytic solution, we have used Metakem Goldbath-SF.
Metakem Goldbath-SF is chosen because it produces very pure (99.99\%) deposits,
and will not react with our photoresist. The effectiveness of this product has
been demonstrated for a similar design in \inlineref{Treutlein2008}.
%
\cm{Should cite metakem}

Goldbath-SF is suitable for use with current densities in the range
\SIrange{1}{15}{\milli\ampere\per\centi\meter\squared}. Final chip design has a
plating surface area of $S\approx85\si{\milli\meter\squared}$. There is also an
additional contribution to surface area from the clip with which we hold the
die in place during plating. We do therefore do not know the plating area
exactly, but we ensure that the entire clip in the solution every time to
ensure the results are reproducable. We can also estimate the minimum plating
time for wire heights of \SI{3}{\micro\meter}, which we expect to be
\cm{TODO}.

% Can we figure out the chip area from the offset of different plating times?

Our apparatus for the electroplating step is shown in
\mysubfigref{fab:fig:eplate}. The electrolytic solution is placed in a
beaker, which itself is placed in a water bath held at \SI{60}{\celsius}. The
bath is heated using a hotplate with magnetic stirrer. Some time is allowed for
thermalisation, during which it is imporant that the goldbath is covered to
prevent loss by evaporation.

When the goldbath has reached \SI{60}{\celsius} the target chip and the cathode
are submerged. The chip is held in position by a stiff insulated wire, which
also carries current to the seed layer. A multimeter can be used to ensure that
there is good electrical contact from the wire to the holder. The wire is
insulated, and we use the smallest clip possible so as to minimise gold plating
to the chip.  The cathode is a grid of platinised titanium which has been cut
to the size of our beaker.  This was also purchased from Metakem.

% Change this to show that electroplating without the aggitation didn't work
% The whole of the next three paragraphs needs to be turned into more of the
% narative of how we learned to electroplate. Including profiling results

A bubbler is placed to agitate the solution near to the chip surface. This in
combination with gentle stirring ensures good circulation of the solution and
hence prevents localised depletion of the ions near to the chip
surface~\cite{Schlesinger2011} \cm{also cite conversation with S Etienne?}.

After electroplating, dies are rinsed with deionised water, and the photoresist
is removed overnight with \cm{Dupont 1165 photoresist remover}, as illustrated
in \mysubfigref{fab:fig:tmep}{d}. Dies are then
dried and stored for transport to the LCN cleanroom for the inspection and
final fabrication steps.

We determined experimentally that electroplating at $\SI{15}{\milli\ampere}$
for duration $\SI{400}{\second}$ reliably produced wires of height
\SI{5}{\micro\meter} above the seed layer. This can be confirmed by profiling
the surface as described in section~\ref{fab:profile}, and can be undertaken
before and after the removal of the photoresist mould. 
%
\cm{Profile}

\subsection{Troubleshooting}

Building and running an electroplating setup was by far the most challenging
stage of the fabrication process. It is therefore worth noting some of the
complications that we experienced and how they were overcome.

% Doesn't work without aggitation

% Poor plating if the chip doesn't face the cathode (is this definitely a
% thing? I think I need a photo to be able to talk about this) plates only at
% edges

% Comparison between wires is hard, hence the new characterisation features

% Using a large crocodile clip seems to cause localised depletion (need to
% confirm this)

% Trying to remove with IPA and acetone instead of PR remover can cause debris

% Shadowing, as in Treutlein, with profiles and pictures

% Inconsistent heights when changing to fresh solution in order to resolve the
% shadowing (maybe a change in the efficency alpha between solutions??) I can
% show some plots for this too

\section{Etching}

The seed layer electrically connects the trapping wires, which is
essential for electroplating, but they must be separated for operation. This is
achieved by etching the metal.

Any thin films or other debris left on the die can become stuck to the chip
during etching. \cite{} Hence it is essential to ensure that the die is very
well cleaned (having been exposed to a non-cleanroom environment for
electroplating). We found that a solvent clean on its own was insufficient,
and would leave some residue on the die after the etch that we could not
remove, shown in \myfigref{fab:fig:etchres}. This was resolved by also cleaning
for ten minutes in an oxygen plasma, suggesting that this debris may have been
caused by residual photoresist.

\begin{figure}
\centering
  %\includegraphics[width=0.8\textwidth]{figs/fab/wire_profile.pdf}
  \caption{\cm{Chips D and E residue pictures}}
  \label{fab:fig:etchres}
\end{figure}

\cm{Need to figure out the actual name of the etchant, and check numbers}
%
After cleaning the gold etch is performed by placing the chip into a beaker of
Iodine etchant, which etches gold at a rate of \SI{5}{\nano\meter\per\second}.
This means that an etch of \SI{10}{\second} will remove the seed layer. After
this the chip is immediately transferred to a beaker of deionised water and
then rinsed. Since the seed layer thickness is significantly smaller than the
dimensions of the wire, the cross-sectional area will not be significantly
altered in this time.

\cm{Need to find out what the Cr etchant is called}
%
To complete the separation of the wires, the chromium layer must also be etched
in the same way. The process is the same as for the gold etch: the chip is
submerged in a beaker of the etchant and after the pre-determined exposure time
it is transferred to deionised water and then rinsed. This stage of the
fabrication is represented in \mysubfigref{fab:fig:etch}{d}. 

\section{Inspecting the finished die}

To ensure that the etches have been successful, stylus profiling is once again
performed, results are shown in \myfigref{fab:fig:endprofile}. We ensure that
the wires are of a sufficient height and width to achieve the required currents
as detailed in chapter~\ref{design}. Visual inspection under an optical
microscope is also useful to ensure that the silicon has been completely
exposed. A multimeter is used to confirm that there are no breaks in the wires.

\begin{figure}
\centering
  \includegraphics[width=0.8\textwidth]{figs/fab/wire_profile.pdf}
  \caption{A profile of trapping wires after etching, showing wires
  electroplated up to a height of approximately \SI{3}{\micro\meter}.
  \cm{Mike says I need to make a better version of this figure. This is on my
  todos, but I really want to do it for the new chip design.}
  }
  \label{fab:fig:endprofile}
\end{figure}

\section{Scaling fabrication}

In future design iterations it may be useful to scale the fabrication process
up by fabricating on the wafer scale rather than the scale of an individual
die. An entire wafer can be metallised, be spin coated and exposed to
produce the photoresist mask. Electroplating on the wafer scale may prove to be
more complicated than our comparatively small targets, but \cm{...}

\section{Planned fabrication of microwave layer}
\label{fab:planned}

In chapter~\ref{intro} 
we described how a molecule chip could allow strong coupling between \CaF{}
molecules and a microwave field, however for this to be possible there must be
good overlap between the microwave field and the trapped
molecules~\cite{Andre2006}.
%
\cm{better cite needed maybe?}

This overlap has previously been achieved for atoms in a magnetic
trap~\cite{Treutlein2008}. An insulating layer is spin-coated onto the trapping
wires so that a CPW \cm{In final version I need to check that first use of
abbreviation is explained.} can be fabricated by
photolithography. The trap centre can be positioned in the region where the
microwave field is strongest. This has potential to allow coherent control of
the molecules and potentially even sideband cooling into the motional ground
state~\cite{Andre2006}.
%
\cm{more discussion elsewhere?}

We will develop our fabrication procedure further so that we can produce such a
chip. The first stage will be to spin coat the insulating layer on top of an
etched chip, as shown in \mysubfigref{fab:fig:cpw}{a}. Again taking our lead
from \inlineref{Treutlein2008}, we will use polyimide. Polyimide is chosen due
to its low dielectric loss tangent ($\tan\delta_e = 0.016$) which will minimise
conductor losses in the waveguide~\cite{Collin2007, Simons2004}.
\footnote{Although a suitable dielectric is
important to minimise conductor losses in a CPW, this chip will operate at room
temperature and so radiation losses will dominate. This was discussed in detail
in the early stage assessment and will not be repeated here.}
%
\cm{This \emph{will} be repeated in the thesis so I can link this better.}

When spin coating the polyimide it is essential that we are able to produce a
flat surface onto which we can fabricate the microwave layer. We can do this by
applying multiple layers of polyimide on the spin coater, so that any bumps are
smoothed out. This is known as planarisation and is discussed further in
\inlineref{Treutlein2008}.

\begin{figure}[h]
\vspace{0.8cm}
\centering
\begin{tabular}{cc}
  \begin{overpic}[width=0.22\textwidth]{figs/fab/cartoon/i.pdf}
    \put(0,40){(a)}
  \end{overpic} &
  \begin{overpic}[width=0.22\textwidth]{figs/fab/cartoon/j.pdf}
    \put(0,40){(b)}
  \end{overpic}
\end{tabular}
  \caption{Cross section of the fabrication of a two-layer chip. A layer of
  polyimide (teal) is applied to the chip. This insulating
  layer is shown in (a). Note that in this simplified illustration
  planarisation effects are ignored, and the surface may not be completely
  even. This is discussed further in the body text.
  The CPW wires can be fabricated on top of the polyimide (b) by evaporation.
  }
  \label{fab:fig:cpw}
\end{figure}

After the application of a planarised polyimide layer it will be possible to
fabricate microwave guides on the surface by lithography. The end result is
shown in \mysubfigref{fab:fig:cpw}{b}. We will undertake
further work to determine the required height of the CPW features and where
they must be positioned relative to the wires to achieve the strongest
coupling.

\section{Connection to subchip}

\subsection{Gluing}

% Epoxy process and testing

\subsection{Wirebonding}

% Wirebonding to Al subchip
