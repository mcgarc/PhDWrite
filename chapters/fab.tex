We have not yet produced a final chip trap, however we have made good progress
in production of a number of prototypes of the trapping wire layer \cm{using
our first design shown in Figure ??}. In this chapter I will outline the
planned fabrication process, reporting on which steps have been undertaken thus
far, and the difficulties which we have yet to overcome.

All the microfabrication techniques described below except for the
electroplating were undertaken at the London Centre for Nanotechnology (LCN). I
would like to thank their technicians and staff, whose input and assistance has
been invaluable to this project. \cm{move thanks to acks I think...} \cm{The
process is largely based on that discussed in \inlineref{Treutlein2008}.}

\section{Overview of the fabrication procedure}

% TODO I grabbed this Madou2002 citation from Treutlein, but I should ensure
% that it is legit
Our trap has been designed so that the size of the wires is small compared to
the size and height of the trapped cloud. As such trapping wires on the chip
are as small as \SI{3}{\micro\meter} in width. Such small features can be
produced using standard photolithography techniques.~\cite{Madou2002} However,
the maximum height of features produced in these procedures is usually of the
order \SI{100}{\nano\meter}.

We must ensure that the wires are capable of carrying the currents that are
outlined in chapter~\ref{design}.  Other experiments with two-layer atom chips
have reported a current density of \cm{??} in the trapping wires. For our wires
this would require a minimum height of \cm{??} to carry our trapping
currents, much higher than can be achieved with photolithography techniques.

Hence, we have begun our protoypting using a combination of microfabrication
techniques and electroplating known as through-mask
electroplating.~\cite{Ruythooren_2000}. Here we use lithography to produce the
footprint of the trapping wires and then electroplate to build up to a height
suitable for our purposes. This is a common technique for constructing atom
chips and is described in~\cite{2011Ac, Lev2003, KOUKHARENKO2004600}.

The fabrication process starts with a \ph{4 inch silicon wafer of some sort},
on which we can fabricate twelve \ph{2cm by 2cm} chips. This process can be
briefly summarised:
\begin{enumerate}
\item Lay down thin (\SI{60}{\nano\meter}) seed layer of gold for each chip.
\item Spin coat a thick (\SI{6}{\micro\meter}) layer of photoresist.
\item Expose and develop the thick photoresist so as to create a ``mould''
in the desired shape of the wires.
\item Dice the wafer to produce individual chips.
\item Electroplate the chip, such that wires \ph{grow up} in the mould to the
desired height.
\item Chemically etch the chip so as to remove the seed layer and
  \ph{electrically isolate} the wires from each other.
\end{enumerate}

An example of such a chip is shown in \ph{\myfigref{TODO}. (Picture of the chip
with the flying wire.)} This chip has two flaws: one broken wire
\ph{(highlighted in the figure)} and imperfect alignment of the trapping wires
to the chip edge. Both of these are typical problems that can arise, and are
discussed further below.

As dicussed \ph{above (where?)} a future aim of the molecule chip project is to
integrate microwave guides on the chip. These guides must allow good overlap of
the microwave fields and the molecule trapping region. Our design achieves this
by positioning the microwave guides on a second layer, directly above the
trapping wires. \cm{Need to ensure the design is already fully described so
this is coherent.} We have not yet attempted the following stages of
fabrication, but we anticipate that they will be:
\begin{enumerate}[resume]
    \item Spin coat chip with an insulating layer of polyimide.
    \item Perform standard photolithography to lay down microwave guides on the
      chip.
\end{enumerate}
These steps are discussed further in section~\ref{fab:planned}.

In the rest of this chapter I will describe the three key fabrication stages:
preparation of chips for electroplating, electroplating and the planned
fabrication of the microwave layer.

\section{Preparation for electroplating}
\label{fab:prep}

I will begin this section with an overview of each of the key methods that we
use (subsections \ref{fab:spin}--\ref{fab:profile}) and then explain the
perparation in detail (subsection \ref{fab:preppro}).

\subsection{Spin coating}
\label{fab:spin}

Spin coating is a procedure for distributing a uniform layer of liquid such as
photoresist  across a sample. It is typically followed by baking to
\ph{solidify} the layer. We use spin coating to apply layers of photoresist to
our wafers and intend to use it to deposit polyimide

\subsection{Lithography}

Lithography is the process of projecting the image of a design onto the surface
of the substrate. The usual procedure is to coat the substrate in a
photoresist, which can be exposed to ultraviolet light. Exposed regions of the
photoresist will react with the light. The photoresist can be developed causing
these regions to lift off while the rest of the photoresist remains attached to
the wafer.\footnote{This describes the procedure for a positive photoresist.
For negative photoresists the unexposed regions are removed on development, but
these are not used in this work.} In this section I will present the standard
procedure for patterning metal onto a substrate with lithography.

A substrate can therefore be \cm{patterned} by controlling the region of
exposure. This is commonly done by using a mask to project a desired shadow
during exposure~\cite{}, however we have used a Heidelberg direct writer
\cm{and model number!}~\cite{} to perform photolithography.

An exposure pattern is uploaded to the direct writer. An ultraviolet laser is
then raster scanned across the surface to be patterned. The laser is switched
on in the regions where exposure is desired. This process is depicted for a
positive photoresist in \myfigref{fab:fig:methods}{a}.  The \cm{refer to
model no./ equipment name} is capable of producing features of up to \cm{size}.
Patterning of our wafer takes approximately one hour.

\begin{figure}[h]
\vspace{0.8cm}
\centering
  \begin{overpic}[width=0.8\textwidth]{figs/fab/cartoon/lith.pdf}
    \put(10,-5){(a)}
    \put(35,-5){(b)}
    \put(61,-5){(c)}
    \put(86,-5){(d)}
    \put(-1,6){$h$}
    \put(23.2,-0.4){$x$}
  \end{overpic}
  \vspace{10mm}
  \caption{Deposition process of gold onto silicon. A top-down view of the
  target is shown in the top row, with a cross section along the dashed line
  shown in the middle row. The bottom row shows a profile of the target (as
  discussed in section~\ref{fab:profile}). Column (a) shows positive
  photoresist (purple) with the exposed region highlighted (light blue). The
  raster scan of the \cm{direct writer} across the substrate is highlighted in
  the top view.  In column (b) the resist is developed and the bare silicon
  exposed. In column (c) gold (yellow) is deposited across the target, e.g. by
  evaporation (as discussed in section~\ref{fab:evap}). In column (d) liftoff
  is performed, removing the remaining photoresist and gold deposited on top of
  it. This leaves only the desired pattern on the silicon.  The relative
  heights of the materials are exaggerated for the purposes of illustration.
  }
  \label{fab:fig:methods}
\end{figure}

The laser intensity, exposure time and number of scans are dependent on the
type of photoresist used and its thickness. Two positive photoresists are used
in this fabrication procedure to produce thin and thick layers. The details of
both of these are discussed further in section~\ref{fab:preppro}.

\cm{Rehydration step?}
Following patterning of the photoresist the sample is placed in a \cm{developing
solution}. This removes the exposed photoresist, exposing the silicon for
deposition. This is shown in \mysubfigref{fab:fig:methods}{b}. Development at
room temperature can take several hours, but robust samples such as the ones
described herein can be placed in a heating sonicator and developed in a few
minutes.

The next stage is usually to deposit metal onto the sample (shown in
\mysubfigref{fab:fig:methods}{c}). We will describe how we do this with
evaporation in the next section. After this unwanted metal can be removed by
lifting off the remaining photoresist. 

In the next section we will describe how metal can be deposited onto the sample
by evaporation (shown in \mysubfigref{fab:fig:methods}{c}). After this the
unwanted metal is removed along with the remaining photoresist in what is known
as the `lift-off' process. Photoresist remover is applied to the solution for
approximately one minute (depending on the exact photoresist and remover used).
This leaves only the intended pattern of metal on the substrate, as in
\mysubfigref{fab:fig:methods}{d}.

Variations on this photolithography method can be used to achieve different
results. A simple example would be deposition of two metals onto the surface.
We will see in section~\ref{fab:preppro} that chromium can be used as an
adhesion layer for gold. We will also show how a thick photoresist mould can be
used to create tall features by electroplating. \ph{Further discussion can be
found in \inlineref{??}.}

\subsection{Evaporation}
\label{fab:evap}

This section details the evaporation procedure. In the previous section we used
the example of direct deposition of gold on to silicon (see
\mysubfigref{fab:fig:methods}{c}). In reality a thin adhesion layer of chromium
is required for good results.~\cite{}

We performed evaporation using a \ph{tech name of A306 belljar},
pictured in \myfigref{fab:fig:belljar}.
The substrate is placed inside the belljar with the target side facing a sample
of the metal to be evaporated. Multiple metals can be placed on a carousell to
allow a different metal to be used for subsequent depositions.

\begin{figure}
  \centering
  \ph{Picutre of belljar}
  \caption{\ph{caption}}
  \label{fab:fig:belljar}
\end{figure}


The belljar is pumped down to vacuum of pressure below \ph{??}. A current is
then passed through the target to \cm{induce} evaporation of the metal onto the
substrate. A shutter is used to block deposition onto the substrate until the
desired current has been reached.

The \ph{belljar} incorporates a \ph{measuring device} which reports the rate of
deposition and automatically shuts off deposition once the desired thickness
has been reached by closing the shutter. \cm{What determines/ limits the deposition
rates? What happends when we have too much current?}

The deposition rate of gold is typically \ph{??}.~\cite{} As discussed above 
a thickness of \ph{5um} is required for the chip's trapping wires. Achieving
this with evaporation would take \ph{a long time}. \ph{Other techniques like
sputtering are also no good.}

\subsection{Profiling}
\label{fab:profile}

\ph{Explanation of Dektak stylus profilometer (Bruker)}

\subsection{Preparation procedure}
\label{fab:preppro}

I will now describe how the above methods have been used to fabricate our
prototype trapping chip.

Fabrication of such a chip begins with a \ph{4 inch silicon wafer}. The wafer
is cleaned first with acetone, then isopropyl alcohol and deionised water. The
wafer is then cleaned with an oxygen plasma and undergoes a dehydration bake.
This ensures that the wafer is clean and \ph{free of any absorbed water} that
might interfere with the later steps.

The wafer is then spin coated with a thin (\ph{??nm}) layer of \ph{Dupont S1805
photoresist}. This is placed into the Heidelberg direct writer for exposure, to
create the pattern shown in \ph{fig??}. This pattern is a simple grid of
\ph{2cm by 2cm} squares which will later be diced to become individual chips.
It is important that the chips are separated at this stage so that tracks of
bare silicon are exposed for dicing.

We use evaporation to deposit a \ph{15nm} thick layer of chromium onto the
wafer. This is an adhesion layer for the \ph{50nm} gold layer which is
deposited next. This gold layer will act as the seed layer for electroplating.
\cm{Depicted in some cartoon.}

The next stage is to create the photoresist mould for electroplating. A thick
(\ph{5um??}) layer of \ph{Dupont S???? photoresist} is spin coated on to the
wafer and exposed \ph{to produce the pattern shown in fig???}. Once diced, each
individual chip will be able to be connected to the anode, and good electrical
contact will be \ph{formed} between the exposed gold and the solution. The gold
will be depositied inside the mould to form wires.

At this point in the procedure it is useful to ensure that the creation of
the mould has been succesful. This can be achieved by characterisation of the
mould with the \ph{stylus profilometer}. \cm{Here I can present results of a
typical good mould on the profilometer...}

The wafer is then diced and the individual chips are ready to be electroplated.

\begin{figure}[h]
\vspace{0.8cm}
\centering
\begin{tabular}{cccc}
  \begin{overpic}[width=0.22\textwidth]{figs/fab/cartoon/a.pdf}
    \put(0,40){(a)}
  \end{overpic} &
  \begin{overpic}[width=0.22\textwidth]{figs/fab/cartoon/b.pdf}
    \put(0,40){(b)}
  \end{overpic} &
  \begin{overpic}[width=0.22\textwidth]{figs/fab/cartoon/c.pdf}
    \put(0,40){(c)}
  \end{overpic} &
  \begin{overpic}[width=0.22\textwidth]{figs/fab/cartoon/d.pdf}
    \put(0,40){(d)}
  \end{overpic}
\end{tabular}
  \caption{Cross section of the preparation of chips for electroplating. The
  bare wafer (black) is shown in (a). The chromium adhesion layer (grey) is
  applied by evaporation (b), followed by the thicker gold seed layer (yellow,
  c).  Finally the photoresist \cm{what kind?} mould for the wires (purple) is
  produced by lithography (d).}
  \label{fab:fig:prep}
\end{figure}

\section{Electroplating}

In electroplating a conductive target is connected to a circuit as an anode and
placed into an electrolytic solution along with a cathode. Current passed
through the solution causes ions in the solution to be deposited onto the
target. This is illustrated in \mysubfigref{fab:fig:eplate}{a}. An overview
of electroplating can be found in \cm{\inlineref{Schlesinger2011}} and a
discussion of its use in microfabrication, including the through-mask method
employed here, can be found in \cm{\inlineref{Ruythooren_2000}}.

\begin{figure}
\vspace{0.8cm}
\centering
\begin{tabular}{cccc}
  \begin{overpic}[width=0.22\textwidth]{figs/fab/cartoon/eplate.pdf}
    \put(-10,100){(a)}
    \put(28.3,92.5){$I$}
  \end{overpic}&
  \ph{Picture of the aparatus (b).}
\end{tabular}
  \caption{The electroplating procedure. Subfigure (a) shows a chip (c.f.\
  \myfigref{fab:fig:prep}) is submerged in a gold electrolyte (light blue)
  along with an electrode (grey mesh). These are connected to a current supply
  to enable current flow and deposition of gold ions (yellow circle).  is
  depicted. The solution is held at \ph{temp} and aggitated by a stirrer and
  bubbler. Subfigure (b) shows a photograph of our apparatus. A beaker
  containing the electrolyte is placed in a water bath for heating during the
  procedure.}
  \label{fab:fig:eplate}
\end{figure}

The height $h$ acheived in a deposition of duration $t$ is given by the Faraday
equation~\cite{Ruythooren_2000}
%
\begin{equation}
  h = \left(\frac{\alpha I M}{nFA\rho}\right)t
\end{equation}
%
where $I$ is the current, $F=\SI{96.5}{\kilo\ampere\second\per\mole}$ is the
Faraday constant, and other parameters with values specific to our gold
deposition are: $\alpha\sim0.9$, the current efficieny; $M =
\SI{197}{\gram\per\mole}$ the molar mass;
$\rho=\SI{19.32}{\gram\per\centi\meter\cubed}$, the density of the deposited
metal; $n=1$, the charge on the deposited ions in units of electron charge; and
$A\sim\SI{1}{\centi\meter\squared}$ is the area for plating.

We therefore have a relationship between the current, the target height and the
time,
%
\begin{equation}
  h \approx \left(
  \SI[per-mode=fraction]{1e-10}{\meter\cubed\per\ampere\per\second} \right)
  \times\frac{It}{A}.
\end{equation}
%
For our electrolytic solution, we have used Metakem Goldbath-SF~\cite{}, which
is rated for currents between \SIrange{1}{15}{\milli\ampere}. This suggests
that we will be able to achieve a few microns of thickenss within a few minutes
of electrocoating. However we do not know the exact value of the current
efficiency for our experiment, so our coating time and operating current have
been determined experimentally.

Metakem Goldbath-SF is chosen because it produces very pure (99.99\%) deposits,
and will not react with our photoresist. The effectiveness of this product has
been demonstrated for a similar design in \inlineref{Treutlein2008}.

Our apparatus for the electroplating step is shown in
\mysubfigref{fab:fig:eplate}{b}. The electrolytic solution is placed in a
beaker, which itself is placed in a water bath held at \ph{58C}. The bath is
heated using a \ph{heating stir plate}. Some time is allowed for themalisation
and then the target chip and the cathode are placed in the solution. The chip
is held in position by a stiff insulated wire, which also carries current to
the plating contact pad (see \cm{some figure of the design}). The insulation on
the wire prevents an increase in the plating area. The cathode is a grid of
platinised titanium which has been cut to the size of our beaker. This was also
purchased from Metakem.

A bubbler is placed to agitate the solution near to the chip surface. This in
combination with gentle stirring ensures good circulation of the solution and
hence prevents localised depletion of the ions near to the chip
surface.~\cite{Schlesinger2011} \cm{also site conversation with S Etienne?}

After electroplating the chips are rinsed with \cm{DI} water, dried and stored
for transport to the LCN cleanroom for the final fabrication steps.

We determined experimentally that electroplating at
\cm{$I=\SI{15}{\milli\ampere}$?} for duration \cm{$t=\SI{400}{\second}$?}
reliably produced wires of height \cm{h=??} above the seed layer. This can be
confirmed by profiling the surface as described in
section~\ref{fab:profile}, and can be undertaken before and after the
removal of the photoresist mould. The profile of a successfully electroplated
chip is shown in \myfigref{fab:fig:eplateprofile}.

\begin{figure}
\centering
  \ph{This should be two 1d profile graphs, one with PR and one without. Also
  might need a diagram showing where the profile is taken on the surface}
  \caption{\ph{the profile of a chip, so many wires are visible, indication of
  where this is on the chip is essential. Assert that this is typical.}}
  \label{fab:fig:eplateprofile}
\end{figure}

\section{Photoresist liftoff and etching}

After electroplating the photoresist mould is removed with Microposit Remover
1165 as described in section~\ref{fab:preppro}. The chip is now as
pictured in \myfigref{fab:fig:etch}{b}, with wires formed to the desired
heights but all connected electrically through the seed layer. The next step is
to etch the seed layer and the adhesion layer so that the wires are separated.

\begin{figure}[h]
\vspace{0.8cm}
\centering
\begin{tabular}{cccc}
  \begin{overpic}[width=0.22\textwidth]{figs/fab/cartoon/e.pdf}
    \put(0,40){(a)}
  \end{overpic} &
  \begin{overpic}[width=0.22\textwidth]{figs/fab/cartoon/f.pdf}
    \put(0,40){(b)}
  \end{overpic} &
  \begin{overpic}[width=0.22\textwidth]{figs/fab/cartoon/g.pdf}
    \put(0,40){(c)}
  \end{overpic} &
  \begin{overpic}[width=0.22\textwidth]{figs/fab/cartoon/h.pdf}
    \put(0,40){(d)}
  \end{overpic}
\end{tabular}
  \caption{Cross section of the cleaning and etching of the chips following
  electroplating.  The electroplated chip (a) is cleaned to remove the
  photoresist layer (b). A gold etch (c) is performed, followed by a chromium
  etch (d) to produce electrically isolated wires. Colours are as in
  \myfigref{fab:fig:prep}.}
  \label{fab:fig:etch}
\end{figure}

The gold etch is performed by placing the chip into a beaker of \ph{Iodine
something chemical}, which has an etching rate of \ph{xx}. This means that an
etch of \ph{\SI{10}{\second}???} will remove the seed layer. After this the
chip is immediately transferred to a beaker of \ph{DI water} and then rinsed. Since
the seed layer thickness is significantly smaller than the dimensions of the
wire, the cross-sectional area will not be significantly altered in this time.
This can be confirmed with profiling (see below). At this stage the chip is as
picture in \myfigref{fab:fig:etch}{c}.

\cm{Paragraphs too similar?}
%
This does not completely isolate the wires however, as they are still connected
through the chromium adhesion layer. Hence a chromium etch is carried out using
\ph{some chemical}. The etching rate is \ph{some value} so \ph{some time} will
remove the adhesion layer. The process is the same as for the gold etch: the
chip is submerged in a beaker of the etchant and after the pre-determined
exposure time it is transferred to \ph{DI water} and then rinsed. The chip is
then as pictured in \myfigref{fab:fig:etch}{d} and is ready to trap
particles.

To ensure that the etch has been successful, stylus profiling is once again
performed. We ensure that the wires are of a sufficient height and width to
achieve the required currents as detailed \ph{in some section}. Visual
inspection under an optical microscope is also useful to ensure that the
silicon has been completely exposed. A multimeter is used to confirm that there
are no breaks in the wires.

An example of a prototype \cm{of design 1} is shown in
\myfigref{fab:fig:prototype}, along with a stylus profile of the surface. Note
that one of the wires in this prototype has come detached from the substrate.
This may be due to mishandling of the chip during the etch.

\begin{figure}
\centering
  \ph{Image of the good chip, including the broken wire (highlight) and a
  profile.}
  \caption{\ph{Chip prototype of design 1.}}
  \label{fab:fig:protoype}
\end{figure}

\section{Planned fabrication of microwave layer}
\label{fab:planned}

In chapter~\ref{intro} \cm{ensure this is done (both LSR and \thesis{thesis})}
we described how a molecule chip could allow strong coupling between \CaF{}
molecules and a microwave field, however for this to be possible there must be
good overlap between the microwave field and the trapped
molecules.~\cite{Andre2006} \cm{better cite needed maybe?}

This overlap has previously been achieved for atoms in a magnetic
trap.~\cite{Treutlein2008} An insulating layer is spin-coated onto the trapping
wires so that a CPW \cm{ensure defined above} can be fabricated by
photolithography. The trap centre can be positioned in the region where the
microwave field is strongest. This has potential to allow coherent control of
the molecules and potentially even sideband cooling into the motional ground
state.~\cite{Andre2006} This is discussed further in section~\ref{outlook}.
\thesis{more discussion elsewhere?}

We will develop our fabrication procedure further so that we can produce such a
chip. The first stage will be to spin coat the insulating layer on top of an
etched chip, as shown in \myfigref{fab:fig:cpw}{a}. Again taking our lead
from \inlineref{Treutlein2008}, we will use polyimide. Polyimide is chosen due
to its low dielectric loss tangent ($\tan\delta_e = 0.016$) which will minimise
conductor losses in the waveguide.  \footnote{Although a suitable dielectric is
important to minimise conductor losses in a CPW, this chip will operate at room
temperature and so radiation losses will dominate. This was discussed in detail
in the early stage assessment and will not be repeated here.}~\cite{Collin2007,
Simons2004}
%
\thesis{This \emph{will} be repeated in the thesis so I can link this better.}

When spin coating the polyimide it is essential that \cm{we get good
planarisation...}

\begin{figure}[h]
\vspace{0.8cm}
\centering
\begin{tabular}{cc}
  \begin{overpic}[width=0.22\textwidth]{figs/fab/cartoon/i.pdf}
    \put(0,40){(a)}
  \end{overpic} &
  \begin{overpic}[width=0.22\textwidth]{figs/fab/cartoon/j.pdf}
    \put(0,40){(b)}
  \end{overpic}
\end{tabular}
  \caption{Cross section of the fabrication of a two-layer chip. A layer of
  polyimide (teal) \cm{capitalise?} is applied to the chip. This insulating
  layer is shown in (a). Note that in this simplified illustration
  planarisation effects are ignored, and the surface may not be completely
  even. This is discussed further in the body text. \cm{And maybe in a figure?}
  The CPW wires can be fabricated on top of the polyimide (b). The fabrication
  is discussed further in the main text. \cm{main or body? both used in this
  caption}.}
  \label{fab:fig:cpw}
\end{figure}

After the application of a planarised polyimide layer it will be possible to
fabricate microwave guides on the surface by lithography. The end result is
shown in \myfigref{fab:fig:cpw}{b}. We will undertake
further work to determine the required height of the CPW features and where
they must be positioned relative to the wires to achieve the strongest
coupling.
%
\cm{Can I really not do better than this?}
