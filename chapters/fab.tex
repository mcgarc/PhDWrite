We have not yet produced a final chip trap, however we have made good progress
in production of a number of prototypes. In this chapter I will outline the
planned fabrication process, reporting on which steps have been undertaken thus
far, and the difficulties which we have yet to overcome.

\section{Overview of the fabrication procedure}

Trapping wires on the chip are as small as \si{3}{\micro\meter} in width. Such
small features can be produced using standard photolithography
techniques.~\cite{} However, the current desired for trapping requires that
these wires have a cross-section on the order of \ph{???} and therefore must
have a height on the order of \si{100}{\micro\meter}. Such heights are not
achieveable with photolithography, so we employ a hybrid approach of using
photolithography and electroplating, which allows us to achieve the required
height.

Our fabrication process starts from a \ph{4 inch silicon wafer of some
sort}, on which we can fabricate twelve \ph{2cm by 2cm} chips. This process
can be briefly summarised as follows:
\begin{enumerate}
\item Lay down thin (\si{60}{\nano\meter}) seed layer of gold for each chip.
\item Spin coat a thick (\si{6}{\micro\meter}) layer of photoresist.
\item Expose and develop the thick photoresist so as to create a ``mould''
in the desired shape of the wires.
\item Dice the wafer to produce individual chips.
\item Electroplate the chip, such that wires \ph{grow up} in the mould to the
desired height.
\item Chemically etch the chip so as to remove the seed layer and
electrically isolate the wires from each other.
\end{enumerate}
