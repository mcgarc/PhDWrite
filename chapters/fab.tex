We have not yet produced a final chip trap, however we have made good progress
in production of a number of prototypes. In this chapter I will outline the
planned fabrication process, reporting on which steps have been undertaken thus
far, and the difficulties which we have yet to overcome.

\section{Overview of the fabrication procedure}

Trapping wires on the chip are as small as \SI{3}{\micro\meter} in width. Such
small features can be produced using standard photolithography
techniques.~\cite{} However, the current desired for trapping requires that
these wires have a cross-section on the order of \ph{???} and therefore must
have a height on the order of \SI{100}{\micro\meter}. Such heights are not
achieveable with photolithography, so we employ a hybrid approach of using
photolithography and electroplating, which allows us to achieve the required
height.

Our fabrication process starts from a \ph{4 inch silicon wafer of some
sort}, on which we can fabricate twelve \ph{2cm by 2cm} chips. We have been
able to produce prototype chips for trapping only (i.e. without microwave
guides). This process can be briefly summarised as follows:
\begin{enumerate}
\item Lay down thin (\SI{60}{\nano\meter}) seed layer of gold for each chip.
\item Spin coat a thick (\SI{6}{\micro\meter}) layer of photoresist.
\item Expose and develop the thick photoresist so as to create a ``mould''
in the desired shape of the wires.
\item Dice the wafer to produce individual chips.
\item Electroplate the chip, such that wires \ph{grow up} in the mould to the
desired height.
\item Chemically etch the chip so as to remove the seed layer and
  \ph{electrically isolate} the wires from each other.
\end{enumerate}

An example of such a chip is shown in \ph{\myfigref{TODO}. (Picture of the chip
with the flying wire.)} This chip has two flaws: one broken wire
\ph{(highlighted in the figure)} and imperfect alignment of the trapping wires
to the chip edge. Both of these are typical problems that can arise, and are
discussed further below.

As dicussed \ph{above (where?)} a future aim of the molecule chip project is to
integrate microwave guides on the chip. These guides must allow good overlap of
the microwave fields and the molecule trapping region. Our design achieves this
by positioning the microwave guides on a second layer, directly above the
trapping wires. \cm{Need to ensure the design is already fully described so
this is coherent.} We have not yet attempted the following stages of
fabrication, but we anticipate that they will be as follows:
\begin{enumerate}[resume]
    \item Spin coat chip with an insulating layer of polyimide.
    \item Perform standard photolithography to lay down microwave guides on the
      chip.
\end{enumerate}
These steps are discussed further in \ph{section ref}.

\section{Fabrication of a trapping chip}

\section{Planned fabrication of microwave layer}

