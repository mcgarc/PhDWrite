We have not yet produced a final chip trap, however we have made good progress
in production of a number of prototypes of the trapping wire layer \cm{using
our first design shown in Figure ??}. In this chapter I will outline the
planned fabrication process, reporting on which steps have been undertaken thus
far, and the difficulties which we have yet to overcome.

\section{Overview of the fabrication procedure}

% TODO I grabbed this Madou2002 citation from Treutlein, but I should ensure
% that it is legit
Our trap has been designed so that the size of the wires is small compared to
the size and height of the trapped cloud. As such trapping wires on the chip
are as small as \SI{3}{\micro\meter} in width. Such small features can be
produced using standard photolithography techniques.~\cite{Madou2002} However,
the maximum height of features produced in these procedures is usually of the
order \SI{100}{\nano\meter}.

We must ensure that the wires are capable of carrying the currents that are
outlined in chapter~\ref{design}.  Other experiments with two-layer atom chips
have reported a current density of \cm{??} in the trapping wires. For our wires
this would require a minimum height of \cm{??} to carry our trapping
currents, much higher than can be achieved with photolithography techniques.

Hence, we have begun our protoypting using a combination of microfabrication
techniques and electroplating, the former to produce the footprint of the
trapping wires and the latter to build up to a height suitable for our
purposes. This is a common technique for constructing atom chips and is
described in~\cite{2011Ac, Lev2003}.

The fabrication process starts with a \ph{4 inch silicon wafer of some sort},
on which we can fabricate twelve \ph{2cm by 2cm} chips. This process can be
briefly summarised as follows:
\begin{enumerate}
\item Lay down thin (\SI{60}{\nano\meter}) seed layer of gold for each chip.
\item Spin coat a thick (\SI{6}{\micro\meter}) layer of photoresist.
\item Expose and develop the thick photoresist so as to create a ``mould''
in the desired shape of the wires.
\item Dice the wafer to produce individual chips.
\item Electroplate the chip, such that wires \ph{grow up} in the mould to the
desired height.
\item Chemically etch the chip so as to remove the seed layer and
  \ph{electrically isolate} the wires from each other.
\end{enumerate}

An example of such a chip is shown in \ph{\myfigref{TODO}. (Picture of the chip
with the flying wire.)} This chip has two flaws: one broken wire
\ph{(highlighted in the figure)} and imperfect alignment of the trapping wires
to the chip edge. Both of these are typical problems that can arise, and are
discussed further below.

As dicussed \ph{above (where?)} a future aim of the molecule chip project is to
integrate microwave guides on the chip. These guides must allow good overlap of
the microwave fields and the molecule trapping region. Our design achieves this
by positioning the microwave guides on a second layer, directly above the
trapping wires. \cm{Need to ensure the design is already fully described so
this is coherent.} We have not yet attempted the following stages of
fabrication, but we anticipate that they will be as follows:
\begin{enumerate}[resume]
    \item Spin coat chip with an insulating layer of polyimide.
    \item Perform standard photolithography to lay down microwave guides on the
      chip.
\end{enumerate}
These steps are discussed further in section~\ref{fab:planned}.

\section{Fabrication of a trapping chip}

\cm{remember to include profilometer stuff}

\section{Planned fabrication of microwave layer}
\label{fab:planned}

