%%%%%%%%%%%%%%%%%%%%%%%%%%%%%%%%%%%%%%%%%%%%
%                 Preamble                 %
%%%%%%%%%%%%%%%%%%%%%%%%%%%%%%%%%%%%%%%%%%%%

% This LaTeX document uses Cameron's LaTeX template (or some form of it) which
% is ever-evolving.

% --- Requirements ---
%
% Packages from Ubuntu [required for]:
%     texlive-publishers [revtex4-1]
%     texlive-science [siunitx]

\documentclass[a4paper]{article}

\usepackage{graphicx}
\usepackage{amsmath}
\usepackage{siunitx}

% --- Macros ---
%\newcommand{\diff}{\mathrm{d}} USE \dd (physics) instead
\newcommand{\cm}[1]{\textcolor{blue}{#1}} % For my comments
\newcommand{\ph}[1]{\textcolor{green}{#1}} % Placeholders
%\renewcommand{\cm}[1]{} % Uncomment to remove my comments
\newcommand{\incfig}[2]{\includegraphics[width=#1\textwidth]{#2}}

\title{Coplanar waveguides for use on a molecule chip}
\author{Cameron McGarry}


\begin{document}

\maketitle


The original proposal for a microfabricated molecule chip trap by Andr\'e et
al.~\cite{Andre2006} suggested that an integrated coplanar waveguide (CPW) could
be used to drive microwave transitions in trapped molecules. In this document,
we will review the feasibility of this proposal for a simple single-layer chip.

\section{Basic CPW behaviour}

The coplaner waveguide was originally proposed by Cheng P Wen\footnote{Wen
should be awarded with much kudos for his spectacular naming convention.} in
1969~\cite{1127105}. CPWs are formed from a conductor layer on some substrate,
with two parralel channels of conuctor carved out in order to create a centre
conductor strip, with the other conucting regions forming a surrouding ground
plane. A cut-out of a CPW is illustrated in Fig.~\ref{fig:CPWxsec}. The CPW's
geometry is defined by the centre conductor width ($S=2a$) and the channel width
($W=b-a$). In all cases that will be discussed herein, we will assume that the
height of the substrate ($h1$) far exceeds the other distances, that is $h_1
\sim \infty$.

\begin{figure}
  \includegraphics{./figs/2019-01-18_stripline_xsection.png}
  \caption{Cross section of a CPW. Taken from Simons~\cite{Simons2004}}
  \label{fig:CPWxsec}
\end{figure}

It is common to define the planar geometry in terms of the ratio
\begin{equation}
  k_0 = \frac{S}{S+2W} = \frac{a}{b}.
  \label{eqn:kratio}
\end{equation}
For this and all other $k_i$ we have the corresponding value
\begin{equation}
  k'_i = \sqrt{1-k^2_i}.
\end{equation}
CPW demonstrates\cite{1127105} that by a conformal mapping of the CPW geometry
we can describe waveguide in terms of elliptic integrals of the first kind
$K(k)$. In our limit we have that
\begin{equation}
  \frac{K(k'_0)}{K(k_0)} = k_0
\end{equation}
which allows us to write
\begin{align}
  C &= 4\epsilon_0\epsilon_\mathrm{eff} k_0^{-1} \\
  %L &= \frac{\mu_0}{4} k_0\\
  Z &= \frac{1}{C v_\mathrm{ph}} = \frac{1}{4}\sqrt{\frac{\mu_0}{\epsilon_0
    \epsilon-\mathrm{eff}}}k_0
    &\approx \frac{30\pi}{\sqrt{\epsilon_\mathrm{eff}}}k_0 \mathrm{[Ohms]}
\end{align}
where  $\epsilon_\mathrm{eff}$ is the
effective permittivity, related to the relative permittivity of the substrate
($\epsilon_\mathrm{r1}$) and the phase speed in the substrate ($v_\mathrm{ph}$)
by
\begin{equation}
  \epsilon_\mathrm{eff} = \left(\frac{v_\mathrm{ph}}{c}\right)^{-2} = 
  \frac{1 + \epsilon_\mathrm{r1}}{2}.
\end{equation}

% Note that Simons goes into great detail about the various different
% conductor-surrounded CPWs here, but we don't need to worry about those so
% much. See page 47 of my notebook (2019-01-17) and mathematica notebook
% 2019-01-18_* for information on how some of these terms can be neglected. We
% can just use OG CPW's results

In order to impedance match our CPW (probably to $Z=50\si{\ohm}$) we can choose
the permittivity of our substrate (to the extent that we have choice of
substrate) and the waveguide geometry via choice of $S$ and $W$.

\section{Attenuation}

\section{Resonators}


\bibliographystyle{unsrt}
\bibliography{bib}
\end{document}
