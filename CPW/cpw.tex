%%%%%%%%%%%%%%%%%%%%%%%%%%%%%%%%%%%%%%%%%%%%
%                 Preamble                 %
%%%%%%%%%%%%%%%%%%%%%%%%%%%%%%%%%%%%%%%%%%%%

% This LaTeX document uses Cameron's LaTeX template (or some form of it) which
% is ever-evolving.

% --- Requirements ---
%
% Packages from Ubuntu [required for]:
%     texlive-publishers [revtex4-1]
%     texlive-science [siunitx]

\documentclass[a4paper]{article}

% --- Macro dependencies ---
\usepackage{xcolor}

% --- Common ---
\usepackage{graphicx}
\usepackage{amsmath}
\usepackage{siunitx}

% --- Macros ---
%\newcommand{\diff}{\mathrm{d}} USE \dd (physics) instead
\newcommand{\cm}[1]{\textcolor{blue}{#1}} % For my comments
\newcommand{\ph}[1]{\textcolor{green}{#1}} % Placeholders
%\renewcommand{\cm}[1]{} % Uncomment to remove my comments
\newcommand{\incfig}[2]{\includegraphics[width=#1\textwidth]{#2}}

\title{Coplanar waveguides for use on a molecule chip}
\author{Cameron McGarry}


\begin{document}

\maketitle


The original proposal for a microfabricated molecule chip trap by Andr\'e et
al.~\cite{Andre2006} suggested that an integrated coplanar waveguide (CPW) could
be used to drive microwave transitions in trapped molecules. In this document,
we will review the feasibility of this proposal for a simple single-layer chip.

\section{Basic CPW behaviour}

The coplaner waveguide was originally proposed by Cheng P Wen\footnote{Wen
should be awarded with much kudos for his spectacular naming convention.} in
1969~\cite{1127105}. CPWs are formed from a conductor layer on some substrate,
with two parralel channels of conuctor carved out in order to create a centre
conductor strip, with the other conucting regions forming a surrouding ground
plane. A cut-out of a CPW is illustrated in Fig.~\ref{fig:CPWxsec}. The CPW's
geometry is defined by the centre conductor width ($S=2a$) and the channel width
($W=b-a$). In all cases that will be discussed herein, we will assume that the
height of the substrate ($h1$) far exceeds the other distances, that is $h_1
\sim \infty$.

\begin{figure}
  \includegraphics{./figs/2019-01-18_stripline_xsection.png}
  \caption{Cross section of a CPW. Taken from Simons~\cite{Simons2004}}
  \label{fig:CPWxsec}
\end{figure}

It is common to define the planar geometry in terms of the ratio
\begin{equation}
  k_0 = \frac{S}{S+2W} = \frac{a}{b}.
  \label{eqn:k0def}
\end{equation}
For this and all other $k_i$ we have the corresponding value
\begin{equation}
  k'_i = \sqrt{1-k^2_i}.
\end{equation}
CPW demonstrates\cite{1127105} that by a conformal mapping of the CPW geometry
we can describe waveguide in terms of elliptic integrals of the first kind
$K(k)$. In our limit we have that
\begin{equation}
  \frac{K(k'_0)}{K(k_0)} = k_0.
  \label{eqn:k0rat}
\end{equation}

The capacitance of a CPW can then be found by considering the contribution due
to the propgation through the air and through the substrate. Wen gives the
total capacitance as
\begin{equation}
  C = 4\epsilon_0\epsilon_\mathrm{eff} k_0^{-1}
\end{equation}
where
\begin{equation}
  \epsilon_\mathrm{eff} = \frac{1 + \epsilon_\mathrm{r1}}{2}.
  \label{eqn:epsilon_eff_approx}
\end{equation}
However, this is an approximation of the more accurate formula
\begin{equation}
  \epsilon_\mathrm{eff} = 1 + q(\epsilon_\mathrm{r1} - 1)
  \label{eqn:epsilon_eff}
\end{equation}
where $q$ is a filling factor, describing the proportion of the wave that
travels inside the dielectric vs. inside the air. Note that we obtain equation
\ref{eqn:epsilon_eff_approx} from \ref{eqn:epsilon_eff} by assuming $q = 1/2$.

The filling factor can be found by the aforementioned conformal mapping
technique to be
\begin{equation}
  q = \frac{1}{2}\frac{K(k_1)}{K(k'_1)}\frac{K(k'0)}{K(k_0)}.
  \label{eqn:fillfact}
\end{equation}
Here we have introduced
\begin{equation*}
  k_1 = \frac{\sinh (\pi S/ 4h_1)}{\sinh [\pi (S+2W)/4h_1]}
\end{equation*}
whose dependence clearly vanishes in the large $h_1$ limit. We can
computationally show that
\begin{equation}
  R = \lim_{h_1 \to \infty} \frac{K(k_1)}{K(k'_1)} \approx 0.563.
  \label{eqn:kratlim}
\end{equation}
Now, using equations \ref{eqn:k0def} and \ref{eqn:kratlim} we can re-write the
filling factor as
\begin{equation}
  q = \frac{R}{2}k_0.
\end{equation}
We must have $k_0 < 1$ which makes the $q=1/2$ assumption pretty poor.

We can now consider the impedance of the CPW
\begin{align}
  %L &= \frac{\mu_0}{4} k_0\\
  Z &= \frac{1}{C v_\mathrm{ph}} \\
    &= \frac{1}{4}\sqrt{\frac{\mu_0}{\epsilon_0 \epsilon_\mathrm{eff}}}k_0 \\
    &\approx \frac{30\pi}{\sqrt{\epsilon_\mathrm{eff}}}k_0 \, \mathrm{[Ohms]}.
\end{align}
In order to impedance match our CPW (probably to $Z=\SI{50}{\ohm}$) we can choose
the permittivity of our substrate (to the extent that we have choice of
substrate) and the waveguide geometry via choice of $S$ and $W$.

% Note that Simons goes into great detail about the various different
% conductor-surrounded CPWs here, but we don't need to worry about those so
% much. See page 47 of my notebook (2019-01-17) and mathematica notebook
% 2019-01-18_* for information on how some of these terms can be neglected. We
% can just use OG CPW's results


\section{Attenuation}

Ultimately we would like to be able to couple a single photon trapped in a CPW
resonator cavity to a single molecule trapped above the chip, however to fully
understand how this might be achieved we first need an understanding of how a
signal will attenuate as it traverses the CPW.

It is worth noting some basic principles of loss in waveguides \cm{TODO: cite
Collins}
The propogation constant of a transmission line is defined to be the complex
value $\gamma$ such that the amplitude of the signal $A(x)$ satisfies
\begin{equation}
  A(x) = A(0)e^{-\gamma x}.
\end{equation}
In order to describe the attenuation, look at the magnitude of this equation,
writing $\gamma = \alpha +i\beta$,
\begin{equation}
  \lvert\frac{A(x)}{A(0)}\rvert = e^{-\alpha x}
\end{equation}
where $\alpha$ is called the attenuation constant. Taking logs it is clear that
\begin{equation}
  \alpha = \frac{1}{x}\log\lvert\frac{A(0)}{A(x)}\rvert\,\si{\neper}
\end{equation}
where we have used `Neper' as the natural unit to desribe loss. We will normally
set $x$ to something sensible such as \SI{1}{\meter} or one wavelength $\lambda$
so that loss is expressed in $\si{\neper}/\mathrm{length}$.

The Neper corresponds to approximately \SI{8.7}{\dB}, as can be seen readily by
comparing the loss formula in dB (which is log-base ten, with a factor of ten
and based on power, not amplitude)
\begin{align*}
  \alpha' &= \frac{1}{x}20\log_{10}\lvert \frac{A(0)}{A(x)} \rvert \si{\dB}\\
          &= \frac{20}{x}\log_{10}(e^{\alpha x}) \si{\dB} \\
          &= 20 \log_{10}(e) \alpha \si{\dB} \\
          &\approx 8.7 \left(\frac{\si{\dB}}{\si{\neper}}\right) \alpha.
\end{align*}

We can anticipate three main sources of loss, those from the dielectric, the
conductor and radiative losses. \cite{Simons2004} The total loss is described
by the summation of each of the contributing terms:
\begin{equation}
  \alpha = \alpha_d + \alpha_c + \alpha_r.
\end{equation}

\subsection{Dielectric losses}

Collins \cm{TODO: CITE} tells us that the dielectric loss is given by
\begin{equation}
  \alpha_d =
  \frac{\pi}{\lambda_0}\frac{\epsilon_\mathrm{r1}}{\sqrt{\epsilon_\mathrm{eff}}}
  q \tan \delta_e
\end{equation}
where $\lambda_0$ is the wavelength in free space, $\tan \delta_e$ is the
dielectric loss tangent of the substrate, and $q$ is the filling factor from
equation \ref{eqn:fillfact}.

As per the above, $k_0$ is effectively fixed by our impedance matching, so the
dielectric loss will be entirely a function of the properties of the material,
going linearly with $\tan\delta_e$. For an order of magnitude comparison, we can
use the usual $q=1/2$ approximation, and consdier our target frequency
$f\approx\SI{40}{\giga\hertz}$ for
\begin{equation}
  \alpha_d \sim 200\sqrt{2\epsilon_\mathrm{r1}}\tan\delta_e
  \,[\si{\neper\per\meter}].
\end{equation}

For a typical substrate we will expect $\tan\delta_e\leq10^{-3}$ and
$\epsilon_\mathrm{r1} \sim 10$ so the limit on the dielectric loss component is
\begin{equation}
  \alpha_d \leq \SI{0.9}{\neper\per\meter}.
\end{equation}

Compare to the results of  Cao et al.~\cm{TODO: CITE and show Fig. 9}, who
investigated loss in CPWs and other wavguides up to the \si{\tera\hertz}
r\'egime. The measurement of dielectic loss for a $Z=\SI{100}{\ohm}$ shows the
expected linear dependence of dielectric loss on frequency, and has value under
our predicted bound at \SI{40}{\giga\hertz}.

It may be possible to achieve lower losses at lower temperature, as the
tangential dielectric loss has been shown to decrease by around an order of
magnitude at around \SI{4}{\kelvin}. \cite{1717770}

\subsection{Conductor losses}

Conductor losses are well-described by the usual 

\subsection{Radiative losses}

\subsection{Comparison of loss mechanisms}

\section{Resonators}


\bibliographystyle{unsrt}
\bibliography{bib}
\end{document}
