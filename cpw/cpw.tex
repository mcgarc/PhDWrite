%%%%%%%%%%%%%%%%%%%%%%%%%%%%%%%%%%%%%%%%%%%%
%                 Preamble                 %
%%%%%%%%%%%%%%%%%%%%%%%%%%%%%%%%%%%%%%%%%%%%

% This LaTeX document uses Cameron's LaTeX template (or some form of it) which
% is ever-evolving.

% --- Requirements ---
%
% Packages from Ubuntu [required for]:
%     texlive-publishers [revtex4-1]
%     texlive-science [siunitx]

\documentclass[a4paper]{article}

% --- Macro dependencies ---
\usepackage{xcolor}

% --- Common ---
\usepackage{graphicx}
\usepackage{amsmath}
\usepackage{siunitx}

% --- Macros ---
%\newcommand{\diff}{\mathrm{d}} USE \dd (physics) instead
\newcommand{\cm}[1]{\textcolor{blue}{#1}} % For my comments
\newcommand{\ph}[1]{\textcolor{green}{#1}} % Placeholders
%\renewcommand{\cm}[1]{} % Uncomment to remove my comments
\newcommand{\incfig}[2]{\includegraphics[width=#1\textwidth]{#2}}

\title{Coplanar waveguides for use on a molecule chip}
\author{Cameron McGarry}


\begin{document}

\maketitle

The original proposal for a microfabricated molecule chip trap by Andr\'e et
al.~\cite{Andre2006} suggested that an integrated coplanar waveguide (CPW) could
be used to drive microwave transitions in trapped molecules. In this document,
we will review the feasibility of this proposal for a simple single-layer chip.

\section{Basic CPW behaviour}

The coplaner waveguide was originally proposed by Cheng P Wen\footnote{Wen
should be awarded with much kudos for his spectacular naming convention.} in
1969~\cite{1127105}. CPWs are formed from a conductor layer on some substrate,
with two parralel channels of conuctor carved out in order to create a centre
conductor strip, with the other conucting regions forming a surrouding ground
plane. A cut-out of a CPW is illustrated in Fig.~\ref{fig:CPWxsec}. The CPW's
geometry is defined by the centre conductor width ($S=2a$) and the channel width
($W=b-a$). In all cases that will be discussed herein, we will assume that the
height of the substrate ($h1$) far exceeds the other distances, that is $h_1
\sim \infty$.

\begin{figure}
  \includegraphics{./figs/2019-01-18_stripline_xsection.png}
  \caption{Cross section of a CPW. Taken from Simons~\cite{Simons2004}}
  \label{fig:CPWxsec}
\end{figure}

It is common to define the planar geometry in terms of the ratio
\begin{equation}
  k_0 = \frac{S}{S+2W} = \frac{a}{b}.
  \label{eqn:k0def}
\end{equation}
For this and all other $k_i$ we have the corresponding value
\begin{equation}
  k'_i = \sqrt{1-k^2_i}.
\end{equation}
CPW demonstrates\cite{1127105} that by a conformal mapping of the CPW geometry
we can describe waveguide in terms of elliptic integrals of the first kind
$K(k)$. In our limit we have that
\begin{equation}
  \frac{K(k'_0)}{K(k_0)} = k_0.
  \label{eqn:k0rat}
\end{equation}

The capacitance of a CPW can then be found by considering the contribution due
to the propgation through the air and through the substrate. Wen gives the
total capacitance as
\begin{equation}
  C = 4\epsilon_0\epsilon_\mathrm{eff} k_0^{-1}
\end{equation}
where
\begin{equation}
  \epsilon_\mathrm{eff} = \frac{1 + \epsilon_\mathrm{r1}}{2}.
  \label{eqn:epsilon_eff_approx}
\end{equation}
However, this is an approximation of the more accurate formula
\begin{equation}
  \epsilon_\mathrm{eff} = 1 + q(\epsilon_\mathrm{r1} - 1)
  \label{eqn:epsilon_eff}
\end{equation}
where $q$ is a filling factor, describing the proportion of the wave that
travels inside the dielectric vs. inside the air. Note that we obtain equation
\ref{eqn:epsilon_eff_approx} from \ref{eqn:epsilon_eff} by assuming $q = 1/2$.

The filling factor can be found by the aforementioned conformal mapping
technique to be
\begin{equation}
  q = \frac{1}{2}\frac{K(k_1)}{K(k'_1)}\frac{K(k'0)}{K(k_0)}.
  \label{eqn:fillfact}
\end{equation}
Here we have introduced
\begin{equation*}
  k_1 = \frac{\sinh (\pi S/ 4h_1)}{\sinh [\pi (S+2W)/4h_1]}
\end{equation*}
whose dependence clearly vanishes in the large $h_1$ limit. We can
computationally show that
\begin{equation}
  R = \lim_{h_1 \to \infty} \frac{K(k_1)}{K(k'_1)} \approx 0.563.
  \label{eqn:kratlim}
\end{equation}
Now, using equations \ref{eqn:k0def} and \ref{eqn:kratlim} we can re-write the
filling factor as
\begin{equation}
  q = \frac{R}{2}k_0.
\end{equation}
We will typically have $S\sim W$, therefore the $q=1/2$ approximation will  not
usually be correct to an order of magnitude. It is better to take $q\sim
\frac{1}{2} \frac{1}{2} \frac{1}{3} = \frac{1}{12}$.
%We must have $k_0 < 1$ which makes the $q=1/2$ assumption pretty poor.

We can now consider the impedance of the CPW
\begin{align}
  %L &= \frac{\mu_0}{4} k_0\\
  Z &= \frac{1}{C v_\mathrm{ph}} \\
    &= \frac{1}{4}\sqrt{\frac{\mu_0}{\epsilon_0 \epsilon_\mathrm{eff}}}k_0 \\
    &\approx \frac{30\pi}{\sqrt{\epsilon_\mathrm{eff}}}k_0 \, \mathrm{[Ohms]}.
\end{align}
In order to impedance match our CPW (probably to $Z=\SI{50}{\ohm}$) we can choose
the permittivity of our substrate (to the extent that we have choice of
substrate) and the waveguide geometry via choice of $S$ and $W$.

% Note that Simons goes into great detail about the various different
% conductor-surrounded CPWs here, but we don't need to worry about those so
% much. See page 47 of my notebook (2019-01-17) and mathematica notebook
% 2019-01-18_* for information on how some of these terms can be neglected. We
% can just use OG CPW's results


\section{Attenuation}

Ultimately we would like to be able to couple a single photon trapped in a CPW
resonator cavity to a single molecule trapped above the chip, however to fully
understand how this might be achieved we first need an understanding of how a
signal will attenuate as it traverses the CPW.

It is worth noting some basic principles of loss in waveguides\cite{Collin2007}
The propogation constant of a transmission line is defined to be the complex
value $\gamma$ such that the amplitude of the signal $A(x)$ satisfies
\begin{equation}
  A(x) = A(0)e^{-\gamma x}.
\end{equation}
In order to describe the attenuation, look at the magnitude of this equation,
writing $\gamma = \alpha +i\beta$,
\begin{equation}
  \lvert\frac{A(x)}{A(0)}\rvert = e^{-\alpha x}
\end{equation}
where $\alpha$ is called the attenuation constant. Taking logs it is clear that
\begin{equation}
  \alpha = \frac{1}{x}\log\lvert\frac{A(0)}{A(x)}\rvert\,\si{\neper}
\end{equation}
where we have used `Neper' as the natural unit to desribe loss. We will normally
set $x$ to something sensible such as \SI{1}{\meter} or one wavelength $\lambda$
so that loss is expressed in $\si{\neper}/\mathrm{length}$.

The phase constant is the imaginary part of the propogation constant:
\begin{equation}
  \beta = \frac{2\pi}{\lambda}.
\end{equation}

The Neper corresponds to approximately \SI{8.7}{\dB}, as can be seen readily by
comparing the loss formula in dB (which is log-base ten, with a factor of ten
and based on power, not amplitude)
\begin{align*}
  \alpha' &= \frac{1}{x}20\log_{10}\lvert \frac{A(0)}{A(x)} \rvert \si{\dB}\\
          &= \frac{20}{x}\log_{10}(e^{\alpha x}) \si{\dB} \\
          &= 20 \log_{10}(e) \alpha \si{\dB} \\
          &\approx 8.7 \left(\frac{\si{\dB}}{\si{\neper}}\right) \alpha.
\end{align*}

We can anticipate three main sources of loss, those from the dielectric, the
conductor and radiative losses. \cite{Simons2004} The total loss is described
by the summation of each of the contributing terms:
\begin{equation}
  \alpha = \alpha_d + \alpha_c + \alpha_r,
\end{equation}
which are for dielectric, conductor and radiation losses respectively.

It is common to describe resonators in terms of a quality
factor~\cite{1125652}
\begin{equation}
  Q = \omega \frac{\text{Energy stored }}{\text{Power loss}}.
\end{equation}
For our purposes we consider the enrgy stored to be the energy that we keep in
the waveguide after traversing the resonatr's length $L$. Taking this to be the
energy contributed by the electric field, this is
\begin{equation}
  \frac{\epsilon_0}{2} V \lvert A(L) \rvert^2
\end{equation}
with $V$ as the field volume.  The power loss is then all the dissipated power,
which comes from considering the energy flow as per a Poynting
vector~\cite{Jackson1975},
\begin{equation}
  \frac{\epsilon_0 c}{2L}V \left[ \lvert A(0) \rvert^2 - \lvert A(L) \rvert^2
  \right] c/L.
\end{equation}
We therefore have quality factor
\begin{equation}
  Q = \frac{\beta L}{e^{2\alpha L} -1}
\end{equation}
% % Is this all nonsense? It must be related but I don't think I can explain it
% well. See pg. 95
%We can take $L$ to be any characteristic length, for example $L=\lambda$, and
%then express $\alpha\to\alpha'$ in units of Neper per characteristic length. For small
%losses ($\alpha' << 1$) we can expand for
%\begin{equation}
%  Q = \frac{\beta'}{2 \alpha'}
%\end{equation}
%which is a commonly-used form of Q.
For a resonator of length $L = \lambda/2$, this yields
\begin{equation}
  Q = \frac{\pi}{e^{\alpha \lambda} - 1}.
\end{equation}

\subsection{Dielectric losses}

Collins~\cite{Collin2007} tells us that the dielectric loss is given by
\begin{equation}
  \alpha_d =
  \frac{\pi}{\lambda_0}\frac{\epsilon_\mathrm{r1}}{\sqrt{\epsilon_\mathrm{eff}}}
  q \tan \delta_e
\end{equation}
where $\lambda_0$ is the wavelength in free space, $\tan \delta_e$ is the
dielectric loss tangent of the substrate, and $q$ is the filling factor from
equation \ref{eqn:fillfact}.

As per the above, $k_0$ is effectively fixed by our impedance matching, so the
dielectric loss will be entirely a function of the properties of the material,
going linearly with $\tan\delta_e$. For an order of magnitude comparison, we can
use the above approxiamtion $q=1/12$, and consdier our target frequency
$f\approx\SI{40}{\giga\hertz}$ for
\begin{equation}
  \alpha_d \sim 30\sqrt{2\epsilon_\mathrm{r1}}\tan\delta_e
  \,[\si{\neper\per\meter}].
\end{equation}

For a typical substrate we will expect $\tan\delta_e\leq10^{-3}$ and
$\epsilon_\mathrm{r1} \sim 10$ so the limit on the dielectric loss component is
\begin{equation}
  \alpha_d \leq \SI{0.1}{\neper\per\meter},
\end{equation}
or as a Q-factor
\begin{equation}
  Q_d \geq 4000.
\end{equation}

Compare to the results of  Cao et al.~\cite{L.Cao2013}, who
investigated loss in CPWs and other wavguides up to the \si{\tera\hertz}
r\'egime. The measurement of dielectic loss for a $Z=\SI{100}{\ohm}$ shows the
expected linear dependence of dielectric loss on frequency, and has value under
our predicted bound at \SI{40}{\giga\hertz}. The results of their relevant
numerical analysis are reproduced in Fig. \ref{fig:CaoFig9}.

% TODO I'm not sure it's actually worth having this fig...
\begin{figure}
  \ph{Cao Fig. 9 to go here}
  \caption{Here we reproduce numerical results from Cao et al.~\cite{L.Cao2013}
  who evaluate dielectric loss of various wvaguides up to the \si{\tera\hertz}
  r\'egime. Their CPW geometry is not in the same high $h$ limit, and produces
  $Z_0 = \SI{50}{\ohm}$, however the losses they achieve are similar to our
  predictions.
  }
  \label{fig:CaoFig9}
\end{figure}

It may be possible to achieve lower losses at lower temperature, as the
tangential dielectric loss has been shown to decrease by around an order of
magnitude at around \SI{4}{\kelvin}. \cite{1717770}

\subsection{Conductor losses}

Conductor losses are given by the following set of equations\cite{Simons2004}
\begin{equation}
  \alpha_c = \frac{R_c +R_g}{2Z_0}
\end{equation}
where $R_c$ describes the series resistance of the centre strip
\begin{equation}
  R_c = \frac{R_s}{4 S(1-k_0^2)K^2(k_0)}\left[ \pi + \log\left(\frac{4\pi
  S}{t}\right) - k_0\log\left(\frac{1+k_0}{1-k_0}\right) \right],
\end{equation}
and $R_g$ is the corresponding series resistance of the ground place
\begin{equation}
  R_s = \frac{k_0 R_s}{4S(1-k_0^2)K^2(k_0)}\left[\pi +
  \log\left(\frac{4\pi(S+2W)}{t}\right) -
  \frac{1}{k_0}\log\left(\frac{1+k_0}{1-k_0}\right)\right]
\end{equation}
(note that we cannot take a $t\to 0$ as this would physically equate to there
being no conductor and hence no circuit). The skin effect surface is
\begin{equation}
  R_s = \frac{1}{\delta\sigma}
\end{equation}
with $\sigma$ being the conductivity of the conductor, and
\begin{equation}
  \delta = \sqrt{\frac{2}{\omega\mu_0\mu_r\sigma}}
\end{equation}
being the skin depth at the target frequncy.

% TODO
% A few more bits here:
%  When is Rs valid? (Relation of t and skin-depth (delta))
%  Actual dependencies worth nothing? a propto sqrt(f) , sim. for sigma?


As above, we will normally fix $k_0$ so as to choose the waveguide's impedance
($Z_0$). Consider an example case of gold conductor on a silicon substrate. As
per CPW\cite{1127105}  we will have $\epsilon_\mathrm{r1} \approx 10$ so require $k_0
\approx 1/3$ to achieve impedance matching at $Z_0 = \SI{50}{\ohm}$. The only
free parameters are now the conductor thickness $t$, and the size of the centre
width conductor $S$. This is illustrated in Fig.~\ref{fig:conductorQ}.

\begin{figure}
  \includegraphics[width=\textwidth]{./figs/conductor_Q.pdf}
  \caption{Quality factor due to conductor losses for a CPW of fixed impedance
  $Z_0=\SI{50}{\ohm}$ for varying centre conductor width ($S$) with conducotr
  thickness as a parameter.}
  \label{fig:conductorQ}
\end{figure}

Typically we will expect a thickness of no more than a few hundred nanometres,
and the centre conductor to have a width of only a few microns. Setting
$t=\SI{100}{\nano\metre}$ and $S=\SI{1}{\micro\metre}$ we will have
\begin{equation}
  \alpha_c \approx \SI{600}{\neper \per \meter}.
\end{equation}
% sigma Au source: https://hypertextbook.com/facts/2004/JennelleBaptiste.shtml
This completely dominates the dielectric loss, and is far larger than we can
stand. At lower temperature (a few Kelvin), the conductivity of gold will
increase by a factor of 100 which would improve the loss to
% TODO Would probably be good to have an actual cite here.
\begin{equation}
  \alpha_c \approx \SI{60}{\neper \per \meter}.
\end{equation}

This corresponds to $Q\sim5$. It appears the conductor losses are going
to be the dominant source of decay in our waveguide. Other successful CPWs and
CPW resonators at such high frequencies have used a significantly larger
conductor areas~\cite{1127105, doi:10.1063/1.3010859} or have had control of
substrate thickness~\cite{L.Cao2013} so that impedance does not only depend on
$k_0$ but also on $h$.

\subsection{Radiative losses}

The losses due to radiation are of similar order to the dielectric losse. Cao et
al.~\cite{L.Cao2013} report a loss of $\alpha_r = \SI{0.43}{\decibel \per \milli
\meter}$ at $f=\SI{1}{\tera\hertz}$. It is known that~\cite{81658} $\alpha
\propto f^3$; which means we can expect $\alpha_r \sim 10^{-2}$ \cm{units?} at
our frequencies.

\subsection{Comparison of loss mechanisms}

Clearly the conductor losses will be the dominant form of signal attenuation. We
have assumed a gold conductor, and even at low temperatures we anticipate loss
from the conductor of order $\alpha_c \sim 100 \alpha_d \sim
\SI{10}{\neper\meter}$. This is fairly significant, and may prevent us from
implementing CPW resonators with $S\sim\si{\micro\meter}$. This will be
investigated in section \ref{sec:resonators}. Note that existing
CPW resonators on gold have had much larger centre conductor widths, and hence
much lower conductor losses~\cite{1127105}. One possible solution may be to use
superconducting materials to build the CPW.

That said, it should be noted that the high loss should not prevent us from
using the CPW to directly drive microwave transitions in trapped molecules, as
we will still be able to pass signal through the waveguide without the need to
construct a waveguide.

% Incorporate both the below as subsections into new section, something like
% outlook? / what we would like to do after this maybe if possible

\section{Superconducting CPWs}

% Turn this into a very brief overview of how the maths changes for the
% conductor losses

\section{Resonators}
\label{sec:resonators}

% Explain why resonators are tricky. This may mean we have to change discussion
% of Q above.


\bibliographystyle{unsrt}
\bibliography{bib}
\end{document}
