Above we have presented a plan for one possible implementation of a molecule
chip. Going forward our intention is to fabricate the science chip using
facilities at the London Centre for Nanotechnology and to assemble the supporting
infrastructure, most notably the augmented flange shown in
\myfigref{experiment:fig:flange}. The chip can then be tested under vacuum
conditions, to assertain if the desired currents are achievable, and whether
the microwave guides operate as expected. \cm{Alex: how? (transmission
tests...)}

It is possible that changes will have to be made to the experiment, in which
case these will be implemented and further chips fabricated. After the design
has been finalised, we will attempt loading from the \CaF{} MOT by the loading
scheme outlined above. We will attempt to ascertain if molecules are trapped by
dropping them off the chip and attempting to directly image them by fluorescence
imaging.

Once molecules have been trapped on the chip, we will attempt to control them
using the field from the CPW, using the same \SI{41}{\giga\hertz} microwave
transition that has already been measured in the macroscopic trap.

If this stage is reached, then there will be an enormous number of ways to
extend the technology. An experiment that could be conceived without requiring
much more equipment would be using a mixture of \esRb{} and \CaF{} in the chip
trap to perform sympathetic evaporative cooling. This could help offset
adiabatic heating that will occur during chip trap compression. Another option
is to design a chip with multiple traps, and see if they can be loaded and then
coupled reliably.

\cm{Alex: High $T_c$ superconductors?}
Beyond this, if the chip can be cooled to a few kelvin, then it will be possible
to implement a superconducting chip, which would be capable of carrying much
higher trapping currents and hosting CPWs of high enough quality factor to
implement a resonator. True cavity-QED experiments could then be performed,
including investigation of hybrid quantum systems.
