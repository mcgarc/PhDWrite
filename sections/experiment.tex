\cm{TODO: Section intro}

\subsection{Microwaves on a chip}

A key aspect of chip design is incorporating a mechanism for coherent quantum
control of the trapped molecules. As discussed in the previous section~\cm{TODO
check I have done this} control of rotational states of the molecule is possible
by application of microwave fields, with strong coupling between the molecules
and fields, as well as long coherence times of the states.\cite{Blackmore_2018}
Furthermore, microwave engineering is already a well-understood field, and as
will be discussed can be readily incorportated into the chip design. \cm{Wording
here is clunky in last sentence}

In this subsection I will introduce the coplanar waveguide (CPW) as the
prefered method of bringing microwaves onto the chip. I will present how they
can be used to create a microwave resonator and their limitations due to
attenuation.

\subsubsection{The coplanar waveguide}

The coplaner waveguide (CPW) was originally proposed by Cheng P.~Wen in
1969~\cite{1127105}. CPWs are formed from a conductor layer on some dielectric
substrate, with two channels of conuctor carved out in order to create a centre
conductor strip, with the other conucting regions forming a surrouding ground
plane.
%
A segment of a CPW is illustrated in Fig.~\ref{experiment:fig:CPWxsec}. The
CPW's geometry is defined by the centre conductor width ($S$) and the channel
width ($W$). 
\begin{figure}
  \includegraphics{./figs/2019-01-18_stripline_xsection.png}
  \caption{
    Cross section of a coplanar waveguide segment, showing the characteristic
    features of a centre conductor of width $S$, which is isolated from the
    ground plane by a distance of width $W$. Note the similarity to a coaxial
    cable, where the centre conductor takes the place of the central pin and the
    ground plane the shielding. This figure is reporoduced from
    Simons~\cite{Simons2004}}
  \label{experiment:fig:CPWxsec}
\end{figure}

This design is such that it can be microfabricated as part of the chip.  This
provides benefits over, for example, propogating the microwaves through free
space, because the chip can be designed to maximise the overlap between the
region of trapping, and the region in which the microwaves are at a high
intensity.~\cm{TODO: cite} The microwave instensity in and arround a typical CPW
is illustrated in \myfigref{experiment:fig:CPWfield}. Furthermore, the
integration of the microwaves onto the chip improves scalability and robustness
of the system.

\begin{figure}
  \cm{TODO: find this fig.}
  \caption{
    The \cm{relative} electric field strength within the vicinity of a CPW.
    \cm{Note that if we can get the molecules close enough this will be great.}
  }
  \label{experiment:fig:CPWfield}
\end{figure}

As with any waveguide, it is important to impedance match a CPW so that as to
minimise reflections at an interface (for example between CPWs on different
substrates, or beween a CPW and a coaxial cable).~\cite{Jackson1975} In the
remainder of this subsubsection I will outline the basic mathematical
description of a CPW. For our purposes it will be sufficient to consider the
case where the height of the substrate ($h1$) far exceeds the other distances,
that is $h_1 \gg S$, whilst $S \sim W$. The conductor is assumed to be perfect,
and the dielectric has relative permittivity $\epsilon_\mathrm{r1}$.

It is common to define the planar geometry in terms of the
ratio~\cite{1127105, Simons2004}
\begin{equation}
  k_0 = \frac{S}{S+2W} = \sqrt{1-{k'_0}^2}
  \label{eqn:k0def}
\end{equation}
where the second equality defines $k'$.
%
Wen demonstrated~\cite{1127105} that a conformal mapping can be used to describe
a CPW as a transmission line~\cite{Jackson1975} in terms of elliptic integrals
of the first kind $K(k)$. The capacitance of a the dielectric region of the CPW
is given by
\begin{equation}
  C_\mathrm{CPW} = 2\epsilon_0(\epsilon_\mathrm{r1}-1)\frac{K(k_0)}{K(k'_0)}
\end{equation}
and the capacitance of the air region is
\begin{equation}
  C_\mathrm{air} = 4\epsilon_0 \frac{K(k_0)}{K(k'_0)}.
\end{equation}
We can use this to find the effective permittivity 
\begin{align}
  \epsilon_\mathrm{eff} &= \frac{C_\mathrm{CPW}}{C_\mathrm{air}} \\
    &= \frac{1+ \epsilon_\mathrm{r1}}{2} \\
\end{align}
the phase velocity (using $c$ as the speed of light)
\begin{align}
  v_\mathrm{ph} &= \frac{c}{\sqrt{\epsilon_\mathrm{eff}}} \\
    &= \frac{c}{\sqrt{(1 + \epsilon_\mathrm{r1})/2}}
\end{align}
and ultimately the impedance of the CPW
\begin{align}
  Z_0 &= \frac{1}{C_\mathrm{CPW} v_\mathrm{ph}} \\
    &= \frac{30}{\sqrt{(\epsilon_\mathrm{r1}+1)/2}} \frac{K(k_0)}{K(k'_0)}.
\end{align}
\cm{Check this eqn. where does the 30 come from?}
Note that the impedance of the waveguide has dependence only on the geometry in
the form of the ratio $k_0$, and the relative permittivity of the
substrate.~\cite{Simons2004} This means that as long as $k_0$ is held constant,
the CPW can be tapered to change the size of the centre conductor, and hence
control the region the field occupies (c.f. \myfigref{experiment:fig:CPWfield}).
A typical CPW taper is illustrated in
\myfigref{experiment:fig:CPWtaper}.~\cm{TODO: find a nice mw/ CPW cite for this}

\begin{figure}
  \cm{TODO: find this fig.}
  \caption{
    \cm{A CPW taper example}
  }
  \label{experiment:fig:CPWtaper}
\end{figure}

\subsubsection{CPW Resonators}

Andre\'e et al.~\cite{Andre2006} proposed a molecule chip where the trap was
embedded in a resonator formed from a CPW. In this subsubsection I will present
the theory of CPW microwave resonators, with a view to determine whether or not
such an architecture is feasible.

It is common to describe resonators in terms of a quality
factor defined as~\cite{Jackson1975, 1125652} \cm{Need better cites here, how
about my old waves textbook?}
\begin{equation}
  Q = 2\pi \frac{\text{Energy stored }}{\text{Energy loss}} =
  2\pi\frac{U_\mathrm{max}}{U_\mathrm{lost}}.
  \label{experiment:mw:eqn:Qdef}
\end{equation}
This can be found by considering the energy stored in the electric component of
the microwave field. The maximum energy for a field of amplitude $E_0$ stored in
the field volume $V$ is
\begin{equation}
  U_\mathrm{max} = \frac{\epsilon_0E_0^2}{2}V.
  \label{experiment:mw:eqn:Umax}
\end{equation}
Determining the power lost in one cycle requires us to introduce the complex
propogation constant $\gamma = \alpha + i\beta$, where $\alpha$ is the
attenuation constant and $\beta = 2\pi / \lambda$ is the wavenumber. Propogation
through a waveguide induces evolution described by~\cm{TODO: good cite for this
+ Simons}
\begin{equation}
  E(z) = E(0)e^{-\gamma z}.
  \label{experiment:mw:eqn:Eloss}.
\end{equation}
Taking the absolute value, it is clear that $\beta$ represents the evolving
phase of the wave and $\alpha$ represents attentuation, the nature of which will
be discussed in the next section.

Dissipated energy over a round trip for cavity of length $L$ is given by
\begin{equation}
  U_\mathrm{lost} = \left(U(z=0) - U(z=2L)\right)
\end{equation}
where once again, the energy is that of the electric field.
\begin{align}
  U_\mathrm{lost} &= \frac{\epsilon_0 V}{2}\left( \left|E(z=0)\right|^2 -
  \left|E(z=2L)\right|^2 \right) \\
    &= \frac{\epsilon_0 E_0^2 V}{2}\left(1 - e^{-4\alpha L}\right)
  \label{experiment:mw:eqn:Ulost}
\end{align}
where in the second equailty we have introduced the loss defined by
\myeqref{experiment:mw:eqn:Eloss} and set $E(z=0)=E_0$.

Substituting
\myeqrefs~\ref{experiment:mw:eqn:Umax}~and~\ref{experiment:mw:eqn:Ulost}~into~\ref{experiment:mw:eqn:Qdef},
and setting the round trip length such that $2L=\lambda$, we have the quality
factor for a typical cavity
\begin{equation}
  Q = \frac{2\pi}{1 - e^{-2 \alpha \lambda}}.
\end{equation}

\subsubsection{Microwave loss modes}

To understand whether we can achieve high enough quality factors to implement a
resonator, we must consider all modes of loss from the system. For the initial
investigation, we will consider a simple and common example system, where a gold
CPW is fabricated on a high resistivity silicon substrate (HiRes Si). \cm{TODO:
cite}

We can anticipate three main sources of loss: those from the dielectric, the
conductor and radiative losses. \cite{Simons2004} The total loss is described
by the summation of each of the contributing terms:
\begin{equation}
  \alpha = \alpha_d + \alpha_c + \alpha_r,
\end{equation}
which are for dielectric, conductor and radiation losses respectively.

\subsubsection*{Dielectric losses}

Collins~\cite{Collin2007} tells us that the dielectric loss is given by
\begin{equation}
  \alpha_d =
  \frac{\pi}{\lambda_0}\frac{\epsilon_\mathrm{r1}}{\sqrt{\epsilon_\mathrm{eff}}}
  q \tan \delta_e
\end{equation}
where $\lambda_0$ is the wavelength in free space, $\tan \delta_e$ is the
dielectric loss tangent of the substrate, and $q$ is the filling factor from
eqn~\ref{eqn:fillfact}.

As per the above, $k_0$ is effectively fixed by our impedance matching, so the
dielectric loss will be entirely a function of the properties of the material,
going linearly with $\tan\delta_e$. For an order of magnitude comparison, we can
use the above approxiamtion $q=1/12$, and consdier our target frequency
$f\approx\SI{40}{\giga\hertz}$ for
\begin{equation}
  \alpha_d \sim 30\sqrt{2\epsilon_\mathrm{r1}}\tan\delta_e
  \,[\si{\neper\per\meter}].
\end{equation}

For a typical substrate we will expect $\tan\delta_e\leq10^{-3}$ and
$\epsilon_\mathrm{r1} \sim 10$ so the limit on the dielectric loss component is
\begin{equation}
  \alpha_d \leq \SI{0.1}{\neper\per\meter},
\end{equation}
or as a Q-factor
\begin{equation}
  Q_d \geq 4000.
\end{equation}

Compare to the results of  Cao et al.~\cite{L.Cao2013}, who
investigated loss in CPWs and other wavguides up to the \si{\tera\hertz}
r\'egime. The measurement of dielectic loss for a $Z=\SI{100}{\ohm}$ shows the
expected linear dependence of dielectric loss on frequency, and has value under
our predicted bound at \SI{40}{\giga\hertz}. The results of their relevant
numerical analysis are reproduced in Fig.~\ref{fig:CaoFig9}.

% TODO I'm not sure it's actually worth having this fig...
\begin{figure}
  \ph{Cao Fig. 9 to go here}
  \caption{Here we reproduce numerical results from Cao et al.~\cite{L.Cao2013}
  who evaluate dielectric loss of various wvaguides up to the \si{\tera\hertz}
  r\'egime. Their CPW geometry is not in the same high $h$ limit, and produces
  $Z_0 = \SI{50}{\ohm}$, however the losses they achieve are similar to our
  predictions.
  }
  \label{fig:CaoFig9}
\end{figure}

It may be possible to achieve lower losses at lower temperature, as the
tangential dielectric loss has been shown to decrease by around an order of
magnitude at around \SI{4}{\kelvin}. \cite{1717770}

\subsubsection*{Conductor losses}

Conductor losses are given by the following set of equations\cite{Simons2004}
\begin{equation}
  \alpha_c = \frac{R_c +R_g}{2Z_0}
\end{equation}
where $R_c$ describes the series resistance of the centre strip
\begin{equation}
  R_c = \frac{R_s}{4 S(1-k_0^2)K^2(k_0)}\left[ \pi + \log\left(\frac{4\pi
  S}{t}\right) - k_0\log\left(\frac{1+k_0}{1-k_0}\right) \right],
\end{equation}
and $R_g$ is the corresponding series resistance of the ground place
\begin{equation}
  R_s = \frac{k_0 R_s}{4S(1-k_0^2)K^2(k_0)}\left[\pi +
  \log\left(\frac{4\pi(S+2W)}{t}\right) -
  \frac{1}{k_0}\log\left(\frac{1+k_0}{1-k_0}\right)\right]
\end{equation}
(note that we cannot take a $t\to 0$ as this would physically equate to there
being no conductor and hence no circuit). The skin effect surface is
\begin{equation}
  R_s = \frac{1}{\delta\sigma}
\end{equation}
with $\sigma$ being the conductivity of the conductor, and
\begin{equation}
  \delta = \sqrt{\frac{2}{\omega\mu_0\mu_r\sigma}}
\end{equation}
being the skin depth at the target frequncy.

% TODO
% A few more bits here:
%  When is Rs valid? (Relation of t and skin-depth (delta))
%  Actual dependencies worth nothing? a propto sqrt(f) , sim. for sigma?


As above, we will normally fix $k_0$ so as to choose the waveguide's impedance
($Z_0$). Consider an example case of gold conductor on a silicon substrate. As
per CPW\cite{1127105}  we will have $\epsilon_\mathrm{r1} \approx 10$ so require $k_0
\approx 1/3$ to achieve impedance matching at $Z_0 = \SI{50}{\ohm}$. The only
free parameters are now the conductor thickness $t$, and the size of the centre
width conductor $S$. This is illustrated in Fig.~\ref{fig:conductorQ}.

\begin{figure}
  \includegraphics[width=\textwidth]{./figs/conductor_Q.pdf}
  \caption{Quality factor due to conductor losses for a CPW of fixed impedance
  $Z_0=\SI{50}{\ohm}$ for varying centre conductor width ($S$) with conducotr
  thickness as a parameter.}
  \label{fig:conductorQ}
\end{figure}

Typically we will expect a thickness of no more than a few hundred nanometres
\cm{more with electroplating! need to check all my numbers},
and the centre conductor to have a width of only a few microns. Setting
$t=\SI{100}{\nano\metre}$ and $S=\SI{1}{\micro\metre}$ we will have
\begin{equation}
  \alpha_c \approx \SI{600}{\neper \per \meter}.
\end{equation}
% sigma Au source: https://hypertextbook.com/facts/2004/JennelleBaptiste.shtml
This completely dominates the dielectric loss, and is far larger than we can
stand. At lower temperature (a few Kelvin), the conductivity of gold will
increase by a factor of 100 which would improve the loss to
% TODO Would probably be good to have an actual cite here.
\begin{equation}
  \alpha_c \approx \SI{60}{\neper \per \meter}.
\end{equation}

%This corresponds to $Q\sim5$. It appears the conductor losses are going
%to be the dominant source of decay in our waveguide. Other successful CPWs and
%CPW resonators at such high frequencies have used a significantly larger
%conductor areas~\cite{1127105, doi:10.1063/1.3010859} or have had control of
%substrate thickness~\cite{L.Cao2013} so that impedance does not only depend on
%$k_0$ but also on $h$.

\subsubsection*{Radiative losses}

The losses due to radiation are of similar order to the dielectric losse. Cao et
al.~\cite{L.Cao2013} report a loss of $\alpha_r = \SI{0.43}{\decibel \per \milli
\meter}$ at $f=\SI{1}{\tera\hertz}$. It is known that~\cite{81658} $\alpha
\propto f^3$; which means we can expect $\alpha_r \sim 10^{-2}$ \cm{units?} at
our frequencies.

\subsubsection*{Comparison of loss mechanisms}

Clearly the conductor losses will be the dominant form of signal attenuation. We
have assumed a gold conductor, and even at low temperatures we anticipate loss
from the conductor of order $\alpha_c \sim 100 \alpha_d \sim
\SI{10}{\neper\meter}$. This is fairly significant, and may prevent us from
implementing CPW resonators with $S\sim\si{\micro\meter}$. This will be
investigated in section~\ref{sec:resonators}. Note that existing
CPW resonators on gold have had much larger centre conductor widths, and hence
much lower conductor losses~\cite{1127105}. One possible solution may be to use
superconducting materials to build the CPW.

That said, it should be noted that the high loss should not prevent us from
using the CPW to directly drive microwave transitions in trapped molecules, as
we will still be able to pass signal through the waveguide without the need to
construct a waveguide.

% Incorporate both the below as subsections into new section, something like
% outlook? / what we would like to do after this maybe if possible

\subsubsection{Losses for various dielectric substrates}

It has been assumed until this point that the substrate we will be using is
silicon, however for the purposes of a molecule chip, that may not be the case.
It may be advantageous to fabricate the CPW on an insulating layer, such that
DC trapping wires can be positioned directly beneath the CPW, such that the trap
centre is in the region where the microwave field is strongest.

There are various possible methods of fabricating such a device, one that was
shown to be succesful by B\"ohi et al.~\cite{Boehi2009} is to spin coat
polyimide onto a silicon chip with trapping wires, before using photolithography
to lay down the CPW. It may also be possible to use other materials as the
insulating layer, such as aluminium nitride, which is well-known to be a good
substrate for microwave components \cm{TODO: cite} or silicon nitride, which is
commonly used as an insulating layer in microfabrication. \cm{TODO: cite
conversation with VJ?}.

A first estimate towards the feasibility of realising a CPW resonator on such a
substrate is to estimate the losses that will be achieved for a single-layer CPW
structure, with $h_1 \gg S$. Applying the formulae for loss as discussed in the
previous section, we can find the various loss modes for the materials of
interest. The results of these equations are presented in
table~\ref{table:losses}.

\begin{table}[h]
  %\begin{tabularx}{\textwidth}{lp{0.25\linewidth}p{0.1\linewidth}p{0.1\linewidth}}
  \begin{tabularx}{\textwidth}{lXXX}
    Substrate &
    Conductor loss -- $S=\SI{10}{\micro\meter}$ (\si{\neper \per \milli \meter})
    & Conductor loss -- $S=\SI{1}{\micro\meter}$ (\si{\neper \per \milli \meter})
    & Dielectric loss ($10^{-4}\si{\neper \per \milli \meter}$) \\
    \hline
    HiRes Si & 0.19 & 1.36 & 6.9 \\
    Polyimide & 0.07 & 0.46 & 46 \\
    Aluminium nitride & 0.05 & 0.41 & 3.8
  \end{tabularx}
    \caption{
      \cm{Is it OK for me to have these cites here (and only here?)} The sources
      of loss modes for HiRes Si~\cite{1717770, 517417},
      polyimide~\cite{5734805}  and aluminium nitride~\cite{1666171} \cm{find
      bettter cite here}, for CPWs which are impedance matched at \SI{50}{\ohm}.
      Note that conductor loss varies across substrates due to the change of
      $k_0$ that is required to impedance match, and that dielectric loss is
      independent of the change in $S$. \cm{What is the wire height?}}
    \label{table:losses}
\end{table}

From these results, we see that in all cases the conductor losses dominate.  It
  is prudent at this point to note that conductor losses can be significantly
  reduced by using superconducting wires to carry currents, as has been achieved
  for atom chips\cite{PhysRevLett.97.200405, Hattermann2017}. This would also
  allow higher current densities and hence deeper traps, however it was decided
  that it would not be feasible to cool the chip to the required temperatures in
  this experiment.

Also of importance from these results is that the losses are very high on the
  length scale of a resonator. \cm{L is about a cm, so we can get values for Q
  factor, which we will find are not critically damped, so we can't have a
  resonator. However, point out that despite this we can still use a taper
  around a thin region, so we can still pump in a lot of power that we know will
  be dissipated and then drive transitions in trapped molecules directly.}

\subsection{Multi-layer chip}

\subsection{Trapping wire design}

\subsection{Design for a molecule chip trap}
