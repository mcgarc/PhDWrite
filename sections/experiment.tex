\cm{TODO: Section intro}

\subsection{Microwaves on a chip}

A key aspect of chip design is incorporating a mechanism for coherent quantum
control of the trapped molecules. As discussed in the previous section~\cm{TODO
check I have done this} control of rotational states of the molecule is possible
by application of microwave fields, with strong coupling between the molecules
and fields, as well as long coherence times of the states.\cite{Blackmore_2018}
Furthermore, microwave engineering is already a well-understood field, and as
will be discussed can be readily incorportated into the chip design. \cm{Wording
here is clunky in last sentence}

In this subsection I will introduce the coplanar waveguide (CPW) as the
prefered method of bringing microwaves onto the chip. I will present how they
can be used to create a microwave resonator and their limitations due to
attenuation.

\subsubsection{The coplanar waveguide}

The coplaner waveguide (CPW) was originally proposed by Cheng P.~Wen in
1969~\cite{1127105}. CPWs are formed from a conductor layer on some dielectric
substrate, with two channels of conuctor carved out in order to create a centre
conductor strip, with the other conucting regions forming a surrouding ground
plane.
%
A segment of a CPW is illustrated in Fig.~\ref{experiment:fig:CPWxsec}. The
CPW's geometry is defined by the centre conductor width ($S$) and the channel
width ($W$). 
\begin{figure}
  \includegraphics{./figs/2019-01-18_stripline_xsection.png}
  \caption{
    Cross section of a coplanar waveguide segment, showing the characteristic
    features of a centre conductor of width $S$, which is isolated from the
    ground plane by a distance of width $W$. Note the similarity to a coaxial
    cable, where the centre conductor takes the place of the central pin and the
    ground plane the shielding. This figure is reporoduced from
    Simons~\cite{Simons2004}}
  \label{experiment:fig:CPWxsec}
\end{figure}

This design is such that it can be microfabricated as part of the chip.  This
provides benefits over, for example, propogating the microwaves through free
space, because the chip can be designed to maximise the overlap between the
region of trapping, and the region in which the microwaves are at a high
intensity.~\cm{TODO: cite} The microwave instensity in and arround a typical CPW
is illustrated in \myfigref{experiment:fig:CPWfield}. Furthermore, the
integration of the microwaves onto the chip improves scalability and robustness
of the system.

\begin{figure}
  \cm{TODO: find this fig.}
  \caption{
    The \cm{relative} electric field strength within the vicinity of a CPW.
    \cm{Note that if we can get the molecules close enough this will be great.}
  }
  \label{experiment:fig:CPWfield}
\end{figure}

As with any waveguide, it is important to impedance match a CPW so that as to
minimise reflections at an interface (for example between CPWs on different
substrates, or beween a CPW and a coaxial cable).~\cite{Jackson1975} In the
remainder of this subsubsection I will outline the basic mathematical
description of a CPW. For our purposes it will be sufficient to consider the
case where the height of the substrate ($h1$) far exceeds the other distances,
that is $h_1 \gg S$, whilst $S \sim W$. The conductor is assumed to be perfect,
and the dielectric has relative permittivity $\epsilon_\mathrm{r1}$.

It is common to define the planar geometry in terms of the
ratio~\cite{1127105, Simons2004}
\begin{equation}
  k_0 = \frac{S}{S+2W} = \sqrt{1-{k'_0}^2}
  \label{eqn:k0def}
\end{equation}
where the second equality defines $k'_0$.
%
Wen demonstrated~\cite{1127105} that a conformal mapping can be used to describe
a CPW as a transmission line~\cite{Jackson1975} in terms of elliptic integrals
of the first kind $K(k)$. The capacitance of a the dielectric region of the CPW
is given by
\begin{equation}
  C_\mathrm{CPW} = 2\epsilon_0(\epsilon_\mathrm{r1}-1)\frac{K(k_0)}{K(k'_0)}
\end{equation}
and the capacitance of the air region is
\begin{equation}
  C_\mathrm{air} = 4\epsilon_0 \frac{K(k_0)}{K(k'_0)}.
\end{equation}
We can use this to find the effective permittivity 
\begin{align}
  \epsilon_\mathrm{eff} &= \frac{C_\mathrm{CPW}}{C_\mathrm{air}} \\
    &= \frac{1+ \epsilon_\mathrm{r1}}{2} \\
\end{align}
the phase velocity (using $c$ as the speed of light)
\begin{align}
  v_\mathrm{ph} &= \frac{c}{\sqrt{\epsilon_\mathrm{eff}}} \\
    &= \frac{c}{\sqrt{(1 + \epsilon_\mathrm{r1})/2}}
\end{align}
and ultimately the impedance of the CPW\footnote{This result makes use of the
approximation $\sqrt{\mu_0/\epsilon_0}\approx120\pi\mathrm{[Ohms]}$, which is
used commonly in microwave engineering~\cm{TODO: Cite}}
\begin{align}
  Z_0 &= \frac{1}{C_\mathrm{CPW} v_\mathrm{ph}} \\
    &= \frac{30 \pi}{\sqrt{(\epsilon_\mathrm{r1}+1)/2}} \frac{K(k_0)}{K(k'_0)}
    \mathrm{[Ohms]}
\end{align}
Note that the impedance of the waveguide has dependence only on the geometry in
the form of the ratio $k_0$, and the relative permittivity of the
substrate.~\cite{Simons2004} This means that for any substrate we choose the
value of $k_0$ can be chosen to fix the impedance at the standard $Z_0 =
\SI{50}{\ohm}$.

This means that as long as $k_0$ is held constant,
the CPW can be tapered to change the size of the centre conductor, and hence
control the region the field occupies (c.f. \myfigref{experiment:fig:CPWfield}).
A typical CPW taper is illustrated in
\myfigref{experiment:fig:CPWtaper}.~\cm{TODO: find a nice mw/ CPW cite for this}

\begin{figure}
  \cm{TODO: find this fig.}
  \caption{
    \cm{A CPW taper example}
  }
  \label{experiment:fig:CPWtaper}
\end{figure}

\subsubsection{CPW Resonators}

Andre\'e et al.~\cite{Andre2006} proposed a molecule chip where the trap was
embedded in a resonator formed from a CPW. In this subsubsection I will present
the implementation and theory of CPW microwave resonators, with a view to
determine whether or not such an architecture is feasible.

A microwave resonator can be formed from a section of CPW that is capacitively
coupled to another driving segment.~\cite{Day2003} The properties of such a
resonstor are deterimined by the geometry, including its length and the nature
of the capacitor structures, which can be made up of gaps, or overlapping
fingers. An example of such a resonator is shown in
\myfigref{experiment:fig:resonator}.~\cite{doi:10.1063/1.3010859} \cm{also cite
some textbook on resonators more generally?}

\begin{figure}
  \cm{TODO: Gopel fig. 2}
  \caption{
    \cm{A CPW resonator, note the presence of tapers as per the prev. fig.}
  }
  \label{experiment:fig:resonator}
\end{figure}

A CPW resonator of length $L$ has fundamental angular frequency
\begin{equation}
  \omega_0 = \frac{\pi v_\mathrm{ph}}{L} = \frac{\pi
  c}{\sqrt{\epsilon_\text{eff}} L}
\end{equation}

To determine the feasibility of implementing such a resonator, the quality
factor should be considered. By definition the quality factor of a perfect
damped resonator is propotional to the ratio of energy stored and energy lost
per cycle, that is~\cm{cite some standard textbook}
\begin{equation}
  Q = 2\pi\frac{U_\mathrm{max}}{U_\mathrm{lost}}.
  \label{experiment:mw:eqn:Qdef}
\end{equation}
It can readily be shown that the quality factor can be expressed in terms of the
attenuation constant $\alpha$ (defined below) and the resonant frequency
$\omega_0$, so that~\cite{Simons2004}
% Simons pg. 409-410
\begin{equation}
  Q = \frac{\omega_0}{2c\alpha}.
  \label{experiment:mw:eqn:Qalpha}
\end{equation}
It is desirable to maximise $Q$ (minimise damping). It has been shown that CPWs
with $Q$ on the order of $1000$ can be fabricated~\cm{Need a few decent cites
here}. Below we will discuss whether this can be achieved for a multi-layer,
non-superconducting chip.

% Talk about "insertion loss" which is in Simon's 12. (pg. 410??)
In reality, the quality factor has some dependence on the capacitive coupling
into the resonator.~\cite{doi:10.1063/1.3010859} Broadly speaking, a higher
coupling capacitance corresponds to a wider resonance peak, however as we shall
show in the next section, this is not the limiting factor of any CPW resonator
that we will consider implementing at this stage.

\subsubsection{Microwave loss modes}

\cm{Need cites in this subsubsection}

The attentuation constant is defined as the real part of the propogation
constant $\gamma = \alpha + i\beta$, where $\beta = 2\pi / \lambda$ is the wave
number.  Propogation through a waveguide induces evolution described by
\begin{equation}
  \widetilde{E}(z) = \widetilde{E}(0)e^{-\gamma z}.
  \label{experiment:mw:eqn:Eloss}.
\end{equation}
Taking the absolute value, the amplitude falls off as
\begin{equation}
  E(z) = E(0)e^{-\alpha z}
\end{equation}
where the attenuation is given by the sum of attenuation from three loss modes:
dielectric loss ($\alpha_d$), conductor loss ($\alpha_c$) and radiation loss
($\alpha_r$). The propagation constant can be written as
\begin{equation}
  \alpha = \alpha_d + \alpha_c + \alpha_r.
\end{equation}

It is instructive to consider the quality factor of a resonator for one loss
mechanism, ignoring the effect of the others. These are given as
\begin{equation}
  Q_i = \frac{\omega_0}{2c\alpha_i}
\end{equation}
with the total quality factor being
\begin{equation}
  Q = \left(\sum Q_i^{-1} \right)^{-1}.
\end{equation}

In the following \cm{subsub?}sections, we will discuss the quality factor from
each loss mode and determine the value of $Q$ that we expect to be able to
achieve. These propagation constants can be determined by use of the conformal
mapping technique as discussed in references~\cite{1127105}
and~\cite{Collin2007}.

\subsubsection*{Dielectric losses}

Dielectric losses can be described in terms of the dielectric loss tangent of
the substrate ($\tan \delta_e$) ~\cite{Collin2007}
\begin{equation}
  \alpha_d =
  \frac{\omega_0}{4c}\frac{\epsilon_\mathrm{r1}}{\sqrt{\epsilon_\mathrm{eff}}}
  \tan \delta_e.
\end{equation}
A typical microwave substrate is chosen to minimise these dielectric losses. As
such we would expect $\tan\delta_e\leq10^{-3}$ and
$\epsilon_\mathrm{r1} \sim 10$ so the limit on the dielectric loss component is
%
\cm{Need to mention $\omega_0/2\pi = \SI{40}{\giga\hertz}$}
%
\begin{equation}
  \alpha_d \leq \SI{0.9}{\neper\per\meter},
\end{equation}
or as a Q-factor
\begin{equation}
  Q_d \geq 460.
\end{equation}

Higher $Q$ is achievable depending on the substrate that is chosen. For example,
aluminium nitride~\cite{mw101} \cm{Probably need to double check this cite
appears sensibly} ($\epsilon_\mathrm{eff}=5$, $\tan\delta_e = 5\times10^{-4}$)
or high-resistivity silicon (HiRes Si)~\cite{1717770}
($\epsilon_\mathrm{eff}=5$, $\tan\delta_e =2\times10^{-4}$) are both good
choices in the high-frequency r\'egime.  Dielectric quality is usually higher
below room temperature, but experimental limitations prevent us from cooling the
chip at this stage.
% https://www.microwaves101.com/encyclopedias/aluminum-nitride
% https://www.microwaves101.com/encyclopedias/hard-substrate-materials

\subsubsection*{Conductor losses}

Conductor losses arise due to dissipation in the centre conductor and ground
plane of the CPW.~\cite{Simons2004} We define the height of these structures to
be $t$, so that the series resistance of the centre conductor is
\begin{equation}
  R_c = \frac{R_s}{4 S(1-k_0^2)K^2(k_0)}\left[ \pi + \log\left(\frac{4\pi
  S}{t}\right) - k_0\log\left(\frac{1+k_0}{1-k_0}\right) \right],
\end{equation}
and the corresponding series resistance of the ground plane
\begin{equation}
  R_s = \frac{k_0 R_s}{4S(1-k_0^2)K^2(k_0)}\left[\pi +
  \log\left(\frac{4\pi(S+2W)}{t}\right) -
  \frac{1}{k_0}\log\left(\frac{1+k_0}{1-k_0}\right)\right].
\end{equation}
The conductor attenuation constant is
\begin{equation}
  \alpha_c = \frac{R_c +R_g}{2Z_0}.
\end{equation}

Consider an example case of gold conductor on HiRes Si substrate. Gold has
resistivity $\rho_\mathrm{Au} = \SI{2.4E-8}{\ohm\metre}$ at room temperature,
and as above, we will have $\epsilon_\mathrm{r1} \approx 10$. This requires $k_0
\approx 1/3$ to achieve impedance matching at $Z_0 = \SI{50}{\ohm}$. The only
free parameters are the conductor thickness $t$, and the size of the centre
width conductor $S$, both of which must be maximised to reduce loss. Typical
values for the smallest CPWs will be $S\sim\SI{1}{\micro\metre}$, where
this heigh can be achieved by electroplating (see section \cm{??}). Typical
conductor losses are therefore expected to be off order
\begin{equation}
  \alpha_c \sim \SI{400}{\neper\per\metre}
\end{equation}
or as a quality factor
\begin{equation}
  Q_c \sim 1.
\end{equation}

Conductor losses for gold CPWs are large, but it is possible to use
superconductors to produce resonators with $Q$ on the order of
$1000$.~\cite{Booth1999, Wallraff2004} However as discussed above, cooling of
the chip is not currently possible due to experimental restrictions, and hence
we are not able to make use of superconductors to reach these high quality
factors.

\subsubsection*{Radiative losses}

\cm{Rewrite this: losses were reported by \cite{81658} and they are
significantly smaller than the conductor losses so don't worry about it.}

The losses due to radiation are of similar order to the dielectric losses. Cao et
al.~\cite{L.Cao2013} report a loss of $\alpha_r = \SI{0.43}{\decibel \per \milli
\meter}$ at $f=\SI{1}{\tera\hertz}$. It is known that~\cite{81658} $\alpha
\propto f^3$; which means we can expect $\alpha_r \sim 10^{-2}$ \cm{units?} at
our frequencies.

\subsubsection*{Comparison of loss mechanisms}

\cm{Rewrite: we're basically completely screwed and can't make a resonator
without superconductors. This also means we can't integrate an electrostatic
trap as Andre planned but we didn't want to do that anyway. We can also still
make a waveguide and put mws down it, so can achieve state control.}

Clearly the conductor losses will be the dominant form of signal attenuation. We
have assumed a gold conductor, and even at low temperatures we anticipate loss
from the conductor of order $\alpha_c \sim 100 \alpha_d \sim
\SI{10}{\neper\meter}$. This is fairly significant, and may prevent us from
implementing CPW resonators with $S\sim\si{\micro\meter}$. This will be
investigated in section~\ref{sec:resonators}. Note that existing
CPW resonators on gold have had much larger centre conductor widths, and hence
much lower conductor losses~\cite{1127105}. One possible solution may be to use
superconducting materials to build the CPW.

That said, it should be noted that the high loss should not prevent us from
using the CPW to directly drive microwave transitions in trapped molecules, as
we will still be able to pass signal through the waveguide without the need to
construct a waveguide.

\subsection{Chips with multiple layers}

% Talk about CPW on polyimide

\cm{Something like: We need to have the CPW and the trapping wires on the same
chip. The only feasible way of doing this is to stack them on top of each other,
as was done by these people.}

\subsection{Design for a molecular chip trap}

\subsection{Experimental design}

\subsection{Loading scheme}
