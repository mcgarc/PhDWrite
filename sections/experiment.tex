In the previous sections I outlined the state of molecular and chip trapping.
What we know currently suggests that the integration of a magnetic molecular
trap with a microwave transition line for control and measurement is
achievable. In particular, the magnetically insensitive transitions in \cm{CaF}
would be good choices for qubit transitions in such a chip due to the long
coherence times that have already been achieved in macroscopic traps.~\cite{}

Now I will present a plan for implementation of such a chip, beginning with a
discussion of propagating microwaves on the surface of a chip and moving on to
look at integration of the trapping wires and microwave guides by use of
multiple layers. The chip design will then be presented followed by a proposed
loading scheme.

\subsection{Microwaves on a chip}

As discussed in the previous section, it is highly beneficial to be able to
incorporate microwave waveguides onto the chip surface. The structure most
commonly used for this application is the coplanar waveguide (CPW)~\cite{},
which was was originally proposed by Cheng P.~Wen in 1969.~\cite{1127105} CPWs
are formed from a conductor layer on a dielectric substrate, with two channels
of conuctor carved out in order to create a centre conductor strip, and the
other conucting regions forming a surrouding ground plane.  A segment of a CPW
is illustrated in Fig.~\ref{experiment:fig:CPWxsec}.

The CPW's geometry is defined by the centre conductor width ($S$) and the
channel width ($W$). The geometry of the CPW determines the region the microwave
field occupies, as illustrated in \myfigref{experiment:fig:CPWfield}. The CPW
can therefore be designed to maximise overlap between the microwave field and
the trapped molecules. As a rough approximation the molecules should be trapped
at a distance $h\sim S$ above the centre of the CPW to achieve a good
overlap.~\cite{Boehi2009} \cm{Combine this and other Munich thesis into one
ref?}

\begin{figure}
  \includegraphics{./figs/2019-01-18_stripline_xsection.png}
  \caption{
    Cross section of a coplanar waveguide segment, showing the characteristic
    features of a centre conductor of width $S$, which is isolated from the
    ground plane by a distance of width $W$. Note the similarity to a coaxial
    cable, where the centre conductor takes the place of the central pin and the
    ground plane the shielding. This figure is reporoduced from
    reference~\cite{Simons2004}}.
  \label{experiment:fig:CPWxsec}
\end{figure}


\begin{figure}
  \cm{TODO: find this fig.}
  \caption{
    The \cm{relative} electric field strength within the vicinity of a CPW.
    \cm{Note that if we can get the molecules close enough this will be great.}
  }
  \label{experiment:fig:CPWfield}
\end{figure}

We must ensure that coupling between CPW microwaves and trapped molecules is
achievable. In this section we will focus on ensuring that the microwaves will
be able to propogate along the waveguide. This will rely on impedance matching
at CPW interfaces and satisfactory levels of attenuation on the
chip.~\cite{Jackson1975, Simons2004}. We will now outline the basic mathematical
description of a CPW.  For our purposes it will be sufficient to consider the
case where the height of the substrate ($h1$) far exceeds the other distances,
that is $h_1 \gg S$, whilst $S \sim W$. The conductor is assumed to be perfect,
and the dielectric has relative permittivity $\epsilon_\mathrm{r1}$.

\subsubsection{CPW Impedance}

It is common to define the planar geometry in terms of the
ratio~\cite{1127105, Simons2004}
\begin{equation}
  k_0 = \frac{S}{S+2W} = \sqrt{1-{k'_0}^2}
  \label{eqn:k0def}
\end{equation}
where the second equality defines $k'_0$.
%
Wen demonstrated~\cite{1127105} that a conformal mapping can be used to describe
a CPW as a transmission line~\cite{Jackson1975} in terms of elliptic integrals
of the first kind $K(k)$. The capacitance of a the dielectric region of the CPW
is given by
\begin{equation}
  C_\mathrm{CPW} = 2\epsilon_0(\epsilon_\mathrm{r1}-1)\frac{K(k_0)}{K(k'_0)}
\end{equation}
and the capacitance of the air region is
\begin{equation}
  C_\mathrm{air} = 4\epsilon_0 \frac{K(k_0)}{K(k'_0)}.
\end{equation}
We can use this to find the effective permittivity 
\begin{align}
  \epsilon_\mathrm{eff} &= \frac{C_\mathrm{CPW}}{C_\mathrm{air}} \\
    &= \frac{1+ \epsilon_\mathrm{r1}}{2} \\
\end{align}
the phase velocity (using $c$ as the speed of light)
\begin{align}
  v_\mathrm{ph} &= \frac{c}{\sqrt{\epsilon_\mathrm{eff}}} \\
    &= \frac{c}{\sqrt{(1 + \epsilon_\mathrm{r1})/2}}
\end{align}
and ultimately the impedance of the CPW\footnote{This result makes use of the
approximation $\sqrt{\mu_0/\epsilon_0}\approx120\pi\mathrm{[Ohms]}$, which is
used commonly in microwave engineering~\cm{TODO: Cite}}
\begin{align}
  Z_0 &= \frac{1}{C_\mathrm{CPW} v_\mathrm{ph}} \\
    &= \frac{30 \pi}{\sqrt{(\epsilon_\mathrm{r1}+1)/2}} \frac{K(k_0)}{K(k'_0)}
    \mathrm{[Ohms]}
\end{align}
Note that the impedance of the waveguide has dependence only on the geometry in
the form of the ratio $k_0$, and the relative permittivity of the
substrate.~\cite{Simons2004} This means that for any substrate we choose the
value of $k_0$ can be chosen to fix the impedance at the standard $Z_0 =
\SI{50}{\ohm}$.

Therefore as long as $k_0$ is held constant the CPW can be tapered to change the
size of the centre conductor, and hence control the region the field occupies
(c.f. \myfigref{experiment:fig:CPWfield}).  A typical CPW taper is illustrated
in \myfigref{experiment:fig:CPWtaper}.~\cm{TODO: find a nice mw/ CPW cite for
this}

\begin{figure}
  \cm{TODO: find this fig.}
  \caption{
    \cm{A CPW taper example}
  }
  \label{experiment:fig:CPWtaper}
\end{figure}

\subsubsection{CPW Resonators}
\label{experiment:mw:resonator}

Andr\'e et al.~\cite{Andre2006} proposed a molecule chip where the trap was
embedded in a resonator formed from a CPW. In this subsubsection I will present
the implementation and theory of CPW microwave resonators, with a view to
determine whether or not such an architecture is feasible.

A microwave resonator can be formed from a section of CPW that is capacitively
coupled to another driving segment.~\cite{Day2003} The properties of such a
resonstor are deterimined by the geometry, including its length and the nature
of the capacitor structures, which can be made up of gaps, or overlapping
fingers. An example of such a resonator is shown in
\myfigref{experiment:fig:resonator}.~\cite{doi:10.1063/1.3010859} \cm{also cite
some textbook on resonators more generally?}

\begin{figure}
  \cm{TODO: Gopel fig. 2}
  \caption{
    \cm{A CPW resonator, note the presence of tapers as per the prev. fig.}
  }
  \label{experiment:fig:resonator}
\end{figure}

A CPW resonator of length $L$ has fundamental angular frequency
\begin{equation}
  \omega_0 = \frac{\pi v_\mathrm{ph}}{L} = \frac{\pi
  c}{\sqrt{\epsilon_\text{eff}} L}
\end{equation}

To determine the feasibility of implementing such a resonator, the quality
factor should be considered. By definition the quality factor of a perfect
damped resonator is propotional to the ratio of energy stored and energy lost
per cycle, that is~\cm{cite some standard textbook}
\begin{equation}
  Q = 2\pi\frac{U_\mathrm{max}}{U_\mathrm{lost}}.
  \label{experiment:mw:eqn:Qdef}
\end{equation}
It can readily be shown that the quality factor can be expressed in terms of the
attenuation constant $\alpha$ (defined below) and the resonant frequency
$\omega_0$, so that~\cite{Simons2004}
% Simons pg. 409-410
\begin{equation}
  Q = \frac{\omega_0}{2c\alpha}.
  \label{experiment:mw:eqn:Qalpha}
\end{equation}
It is desirable to maximise $Q$ (minimise damping). It has been shown that CPWs
with $Q$ on the order of $1000$ can be fabricated.~\cm{Need a few decent cites
here} Below we will discuss whether this can be achieved in our case.

\subsubsection{Microwave loss modes}

\cm{Need cites in this subsubsection}

The attentuation constant is defined as the real part of the propogation
constant $\gamma = \alpha + i\beta$, where $\beta = 2\pi / \lambda$ is the wave
number.  Propogation through a waveguide induces evolution described by
\begin{equation}
  \widetilde{E}(z) = \widetilde{E}(0)e^{-\gamma z}.
  \label{experiment:mw:eqn:Eloss}.
\end{equation}
Taking the absolute value, the amplitude falls off as
\begin{equation}
  E(z) = E(0)e^{-\alpha z}
\end{equation}
where the attenuation is given by the sum of attenuation from three loss modes:
dielectric loss ($\alpha_d$), conductor loss ($\alpha_c$) and radiation loss
($\alpha_r$). The propagation constant can be written as
\begin{equation}
  \alpha = \alpha_d + \alpha_c + \alpha_r.
\end{equation}

It is instructive to consider the quality factor of a resonator for one loss
mechanism, ignoring the effect of the others. These are given as
\begin{equation}
  Q_i = \frac{\omega_0}{2c\alpha_i}
\end{equation}
with the total quality factor being
\begin{equation}
  Q = \left(\sum Q_i^{-1} \right)^{-1}.
\end{equation}

% Talk about "insertion loss" which is in Simon's 12. (pg. 410??)
Note that we will not consider the dependence of the quality factor the
capacitive coupling into the resonator (insertion loss).~\cite{Simons2004,
doi:10.1063/1.3010859} Broadly speaking, a higher coupling capacitance
corresponds to a wider resonance peak, however in our case this effect is
dominated by other forms of loss.

\subsubsection*{Dielectric losses}

Dielectric losses vary linearly with $\tan \delta_e$, the dielectric loss
tangent of the substrate~\cite{Collin2007}
\begin{equation}
  \alpha_d =
  \frac{\omega_0}{4c}\frac{\epsilon_\mathrm{r1}}{\sqrt{\epsilon_\mathrm{eff}}}
  \tan \delta_e.
\end{equation}
A typical microwave substrate is chosen to minimise these dielectric losses. As
such we would expect $\tan\delta_e\leq10^{-3}$ and
$\epsilon_\mathrm{r1} \sim 10$ so the limit on the dielectric loss component is
%
\cm{Need to mention $\omega_0/2\pi = \SI{40}{\giga\hertz}$}
%
\begin{equation}
  \alpha_d \leq \SI{0.9}{\neper\per\meter},
\end{equation}
or as a Q-factor
\begin{equation}
  Q_d \geq 460.
\end{equation}

Higher $Q$ is achievable depending on the substrate that is chosen. For example,
aluminium nitride~\cite{mw101} \cm{Probably need to double check this cite
appears sensibly} ($\epsilon_\mathrm{eff}=5$, $\tan\delta_e = 5\times10^{-4}$)
or high-resistivity silicon (HiRes Si)~\cite{1717770}
($\epsilon_\mathrm{eff}=5$, $\tan\delta_e =2\times10^{-4}$) are both good
choices in the high-frequency r\'egime.  Dielectric quality is usually higher
below room temperature, but experimental limitations prevent us from cooling the
chip at this stage.
% https://www.microwaves101.com/encyclopedias/aluminum-nitride
% https://www.microwaves101.com/encyclopedias/hard-substrate-materials

\subsubsection*{Conductor losses}

Conductor losses arise due to dissipation in the centre conductor and ground
plane of the CPW.~\cite{Simons2004} We define the height of these structures to
be $t$, so that the series resistance of the centre conductor is
\begin{equation}
  R_c = \frac{R_s}{4 S(1-k_0^2)K^2(k_0)}\left[ \pi + \log\left(\frac{4\pi
  S}{t}\right) - k_0\log\left(\frac{1+k_0}{1-k_0}\right) \right],
\end{equation}
and the corresponding series resistance of the ground plane
\begin{equation}
  R_s = \frac{k_0 R_s}{4S(1-k_0^2)K^2(k_0)}\left[\pi +
  \log\left(\frac{4\pi(S+2W)}{t}\right) -
  \frac{1}{k_0}\log\left(\frac{1+k_0}{1-k_0}\right)\right].
\end{equation}
The conductor attenuation constant is
\begin{equation}
  \alpha_c = \frac{R_c +R_g}{2Z_0}.
\end{equation}

Consider an example case of gold conductor on HiRes Si substrate. Gold has
resistivity $\rho_\mathrm{Au} = \SI{2.4E-8}{\ohm\metre}$ at room temperature,
and as above, we will have $\epsilon_\mathrm{r1} \approx 10$. This requires $k_0
\approx 1/3$ to achieve impedance matching at $Z_0 = \SI{50}{\ohm}$. The only
free parameters are the conductor thickness $t$, and the size of the centre
width conductor $S$, both of which must be maximised to reduce loss. Typical
values for the smallest CPWs will be $S\sim\SI{1}{\micro\metre}$, where
this heigh can be achieved by electroplating (see section \cm{??}). Typical
conductor losses are therefore expected to be of order
\begin{equation}
  \alpha_c \sim \SI{400}{\neper\per\metre}
\end{equation}
or as a quality factor
\begin{equation}
  Q_c \sim 1.
\end{equation}

Conductor losses for gold CPWs are large, but it is possible to use
superconductors to produce resonators with $Q$ on the order of
$1000$.~\cite{Booth1999, Wallraff2004} However as discussed above, cooling of
the chip is not currently possible due to experimental restrictions, and hence
we are not able to make use of superconductors to reach these high quality
factors.

\subsubsection*{Radiative losses}

The radiation losses of a CPW have been calculated and measured by Frankel et
al.~\cite{Frankel1991}, who show that
\begin{equation}
  \alpha_r = \frac{\pi^2 \omega_0^3}{2^7}\left(\frac{\left(1 -
  \frac{\epsilon_\mathrm{eff}}{\epsilon_\mathrm{r1}}\right)^2}{\sqrt{\frac{\epsilon_\mathrm{eff}}{\epsilon_\mathrm{r1}}}}\right)
  \frac{(S+2W)^2\epsilon_\mathrm{r1}^{3/2}}{c^3 K(k'_0)K(k_0)}.
\end{equation}.
Inserting typical values, we retrieve
\begin{equation}
  \alpha_r \sim \SI{0.4}{\neper\per\metre}
\end{equation}
corresponding to a minimum quality factor
\begin{equation}
  Q_r \geq 1000
\end{equation}
which agrees with the measurements made by Frankel et al. which were for CPWs of
similar geometry and frequency domain to our intended usage.

\subsubsection*{Comparison of loss mechanisms}

It is clear that without use of superconductors, the conductor loss will far
dominate the other sources, with an overall quality factor $Q\sim1$. Typical
resonators will operate with much higher $Q$~\cite{}, so it will not be possible
to implement a resonator.

That said, it should be noted that the high loss should not prevent us from
using the CPW to directly drive microwave transitions in trapped molecules, as
we will still be able to pass signal through the waveguide without the need to
construct a waveguide. With loss of $\alpha \approx \alpha_r$, and waveguides of
lengths on the scale of a few centimetre, we can expect a total loss of around
$\SI{400}{\neper\per\metre} \times \SI{0.01}{\metre} = \SI{35}{\decibel}$. This
should be sufficiently small that microwaves will be able to reach molecules
trapped on the chip.~\cite{Treutlein2008}
%
% THESIS: Expand on this
One of the later goals of the project will be to characterise microwave losses
on the chip and compare these with those expected from theory.

A consequence of being unable to implement a microwave resonator is that it
becomes impossible to implement the electrostatic trap proposed by Andr\'e et
al.~\cite{Andre2006} as this relied on the ability to introduce a bias voltage
on the centre conductor of the resonator in order to form the trap.  \cm{Need a
fig. maybe reference above if it's included there?} It is possible to bias the
centre conductor (as detailed in reference~\cite{doi:10.1063/1.3573824}) but it
is not possible to do the same for a waveguide.

\subsection{Chips with multiple layers}
\label{experiment:multilayer}

\cm{Something like: We need to have the CPW and the trapping wires on the same
chip. The only feasible way of doing this is to stack them on top of each other,
as was done by these people.}

As the microwave attenuation prevents implementation of a resonator (and hence
implementation of an electrostatic trap integrated with the waveguide) another
solution for trapping and control of the microwaves must be found. Since
magnetic trapping of atoms above a chip has been extensively researched (see
section~\ref{litrev}), we have chosen this mechanism for trapping of molecules
above the chip. \cm{Need to say that macroscopic magnetic trapping of CaF is a
thing.}

Combining the CPW with the trapping wires on a single layer was extensively
explored, however no geometry was found that would allow for satisfactory
overlap between the microwave field and the molecules. It was determined that a
multi-layer chip should be used, similar to those designed
by Treutlein~\cite{Treutlein2008} and B\"ohi~\cite{rohuta} and discussed in
section ~\ref{litrev:control}. \cm{TODO} A layer of insulating resin can be
spin-coated on top of the trapping wire layer, forming a substrate for the CPW.
Trapping magnetic fields will be used to trap directly above the CPW.

The resin used by Treutlein and B\"ohi~\cite{Treutlein2008, rohtua} was
polyimide, which has the benefit of being highly resistant to cleaning
techniques commonly used in photolithography, including pirranah clean. The
polyimide is also effective as a microwave substrate~\cite{Simons2004,
Boehi2009} ($\epsilon_\mathrm{r1} = 3.3$, $\tan\delta_e = 0.016$) with
conductor losses still dominating dielectric.

Using a multi-layer chip introduces thermal constraints on the currents that can
flow through the trapping wires. We expect that a current density of
\SI{2.8E10}{\ampere\per\metre\squared} will be achievale in the upper layer and
\SI{5.5E10}{\ampere\per\metre\squared} in the lower layer.

\subsection{Design for a molecular chip trap}

We have designed various multiple-layer chips for prototyping. A series of
Z-traps are to be used for loading molecules from an external trap (see
section~\ref{experiment:loading}) and a CPW for control of the molecules' states
is incorporated on an upper layer.  The chip design is shown in
\myfigref{experiment:fig:chipdesign}.

\begin{figure}[tph]
  \includegraphics[width=0.8\textwidth]{./figs/z_trap_fanouts_inset_scale.png}
  \caption{
    A design of a molecule chip, with six overlayed Z-trapping wires and one
    dimple wire (gold) on the lower layer, and a CPW (insulating section in
    blue) on the upper layer.  The lower layer is to be fabricated by
    photolithography and electroplating, then spin coated with polyimide so that
    the second layer can ebe fabricated above. The Z-traps have varying lengths
    and thicknesses to allow higher current in larger traps, with compression as
    molecules are transferred into smaller traps. The CPW is tapered down to
    localise the microwave field around the smallest traps. Inset: a zoomed view
    of the trapping region.
  }
  \label{experiment:fig:chipdesign}
\end{figure}

\cm{difference between designs: CPWs}
A standard \cm{four inch} wafer has space for \cm{twelve} chips (each one being
a square with sides of length \SI{2}{\centi\metre}). In the prototype wafer
design, each one has the same set of trapping wires as depicted in
\myfigref{experiment:fig:chipdesign}, but the CPW designs vary. The parameters
changed are the minimum width of the centre conductor (\SI{10}{\micro\metre},
\SI{20}{\micro\metre}, \SI{50}{\micro\metre}); the taper length
(\SI{1}{\milli\metre}, \SI{3}{\milli\metre}); and the orientation of the CPW
track near the molecules (perpendicular or parallel to the trap axis). The full
wafer is shown in \myfigref{experiment:waferdesign}.

The goal of this prototype is to characterise the CPWs to check that the
predicted values are met. In particular, we are interested in how the material
under the polyimide (HiRes Si substrate and trapping wires) will affect the
microwave field. We will also be able to test the trapping wires and the
behaviour of the chip in vacuum. We are optimistic that these prototypes could
ultimately be used for trapping of molecules.

\begin{figure}[tph]
  \includegraphics[width=0.8\textwidth]{./figs/wafer.png}
  \caption{
    The prototype wafer design, combining all variations of the centre conductor
    width, taper lengths and CPW orientations (details in the main text). A
    \cm{four inch} HiRes Si wafer will be used as a substrate. Lower level
    features (yellow) will be fabricated by photolithography followed by
    electroplating. The higher level features (blue) will be fabricated on an
    intermediary layer of polyimide. Note the chip outlines are marked with
    grey, and the alignment features in each corner of the chip to be fabricated
    on the lower layer.
  }
  \label{experiment:fig:chipdesign}
\end{figure}

\cm{Table of max trap depths and paragraph spiel.}

\cm{How many molecules can we have in the traps and what temperatures?}

The chip is to be mounted on a heavily modified vacuum flange, shown in
\myfigref{experiment:fig:flange}. This flange will incorporate feedthroughs for
the trapping currents and microwaves, as well as a large U-wire, which will be
used as an intermediate trap in the loading procedure (detailed below). Currents
will be brought in through a nineteen-pin feedthrough  to power the bias coils
and on-chip trapping wires. These currents will feed onto a holding-chip, which
will then connect to the science-chip via wirebonds. Seperate feedthroughs for
the microwaves and U-wire (\cm{details}) feed onto CPWs on the holding-chip and
the U-wire respectively.

\begin{figure}[ht]
  \includegraphics[width=0.8\textwidth]{./figs/FlangeAssemblyForPresentationTopRight.png}
  \caption{
    Flange assembly for mounting a molecule chip. The \cm{science chip} will be
    mounted in the centre of the holding-chip and wire-bonds will allow current
    to pass between the two. Current will be passed to the holding-chip via the
    19-pin feedthrough. There are also high frequency microwave feedthroughs so
    that microwaves can be passed to a CPW on the holding-chip, and again onto
    the science chip. Two bias coils will be mounted on the flange, with the
    other required coils attached to the chamber (not pictured). The
    intermediary U-trap is positioned directly beneath the science chip and is
    powered by separate current feedthroughs. \cm{Image credit}
  }
  \label{experiment:fig:flange}
\end{figure}

\thesis{
\subsection{Fabrication}
Will need full description of fabrication process
\begin{itemize}
  \item Photolith
  \item Electroplating
  \item Multilayers
\end{itemize}
}

\subsection{Loading scheme}
\label{experiment:loading}

Calcium flouride molecules will be loaded onto the chip from a MOT similar to
that described in reference~\cite{Truppe2017}. In the MOT chamber it is planned
to construct a dipole trap, which will be used to increase phase space density
of the molecule cloud

\cm{Sympathetic cooling from Rb-CaF collisions}
Our source of \cm{CaF} molecules is a MOT similar to that discussed in
reference~\cite{Truppe2017}. The MOT planned to be used for this experiment is
to be used for separate experiments investigating sympathetic cooling between
\cm{Rb} atoms and \cm{CaF} in a dipole trap. This is expected to produce a CaF
cloud with phase space density of order \cm{What?}. This cloud will be suitable
for loading onto the chip. \cm{need to have set this up above. How much do we
expect to lose in loading?} \cm{TODO: cite}

The cloud will then be transfered into a magnetic transport trap (MTT) formed of
a pair of anti-Helmholtz coils external to the chambers. The dipole trap will be
switched off and the current in the coils ramped up to form a quadrupole trap.
The coils are mounted on a transport stage, such that they and hence the centre
of the quadrupole, can be moved across to neighbouring chambers. The trapped
cloud of \cm{CaF} and \cm{Rb} will be transported in the magnetic trap.~\cite{}
The MTT has been designed for loading into a neighbouring tweezer experiment,
however this will be extended to allow loading onto the chip. The MOT, MTT and
tweezer experiment are shown in \myfigref{experiment:fig:MTTsetup}. 

\begin{figure}[ht]
  \includegraphics[width=0.8\textwidth]{./figs/existing_chambers.png}
  \caption{
    The MOT and tweezer chambers for existing experiments. The magnetic
    transport trap (MTT) is used made up of a pair of transport coils arranged
    in anti-Helmholtz configuration. These generate a quadrupole field, which
    will move with the coils on a translation stage, allowing the transfer of
    molecules trapped in the MOT chamber into the tweezer chamber. The chip
    chamber will be positioned further downstream (left) of the tweezer chamber.
    \cm{Image credit}
  }
  \label{experiment:fig:MTTsetup}
\end{figure}

The chip flange will be mounted in a third chamber downstream from the MOT and
tweezer chambers. The flange is to be aligned such that the cloud can pass above
the surface with good clearance, entering on the side of the sub-chip opposite
the flange. A macroscopic trap, aligned with the centre of the MTT, can then be
generated by the onboard U-wire and surrounding bias coils. This intermediary
trapping stage will be used to align the cloud with those on the science-chip.

The cloud can be brought closer to the surface of the chip by a series of
current ramps in the bias coils and chip wires. The proposed procedure is
outlined in \mytabref{experiment:table:loading}.

\cm{TODO: loading table}
