At the time of the original molecule chip proposal~\cite{Andre2006}, to see atom
chips had been loaded using a magneto-optical trap (MOT) as a
source~\cite{Reichel1999, Ott2001}, however a molecule MOT had not yet been
built, and there was no credible source of loading a chip of this
type.\footnote{\cm{Flyover chips had been created, maybe something about PSD
meant a beam wouldn't work? expand in body}}
%
Today we have access to molecules cooled below the Doppler
temperature~\cite{Truppe2017}, with potential for further sympathetic cooling
with rubidium~\cm{how to cite this?}. Further technologies for loading a
molecule chip, such a optical tweezers~\cite{Liueaar7797} have also been
demonstrated.

These developments have made the implementation of a molecule chip feasible. In
this section, we will review the state of the field, including work on atom
chips and these supporting technologies.

\subsection{Atom chips}

Devlopment of the molecule chip has been heavily inspired by previous work into
atom chips, especially that undertaken by the group at the Max Planck Institute
of Quantum Optics~\cite{rohtua, Treutlein2008, Boehi2009}. We have been able to
draw upon this well-developed field in order to inform our design choices

In developing the molecule chip, we have been able to draw upon the
well-developed field of atom chips to inform our design choices. There are many
examples of interest to us~\cm{add in some citations later that we don't use
further down}, however here we will highlight only the most relevant.

One early atom chip of interest is that created by Folman et
al.~\cite{Folman2000}, who used the chip surface as a mirror to create a
mirror-MOT, from which \cm{7\^ Li} atoms could be loaded directly onto a
magnetic trap.

% Take-away points:
%   Magnetic trap
%   How to get CPW on (Bohi)
%   MTT + big U to load is viable
%   Required PSD???

\subsection{Cold molecule sources}

\subsection{Molecule chips}

