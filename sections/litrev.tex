At the time of the original molecule chip proposal~\cite{Andre2006}, to see atom
chips had been loaded using a magneto-optical trap (MOT) as a
source~\cite{Reichel1999, Ott2001}, however a molecule MOT had not yet been
built, and there was no credible source of loading a chip of this
type.\footnote{\cm{Flyover chips had been created, maybe something about PSD
meant a beam wouldn't work? expand in body}}
%
Today we have access to molecules cooled below the Doppler
temperature~\cite{Truppe2017}, with potential for further sympathetic cooling
with rubidium~\cm{how to cite this?}. Further technologies for loading a
molecule chip, such a optical tweezers~\cite{Liueaar7797} have also been
demonstrated.

These developments have made the implementation of a molecule chip feasible. In
this section, we will review the state of the field, including work on atom
chips and these supporting technologies.

\subsection{Atom chips}

% Take-away points:
%   Magnetic trap
%   How to get CPW on (Bohi)
%   MTT + big U to load is viable
%   Required PSD???

In developing the molecule chip, we have been able to draw upon the
well-developed field of atom chips to inform our design choices. Let us consider
three different aspects of chip design: trapping on the chip, loading into this
trap, manipulation of trapped atoms or molecules.

\subsubsection{Chip traps}

The Andr\'e proposal~\cite{Andre2006} suggests the use of an electrostatic trap
to contain the molecules, however this is not the only option available. In
fact, magnetic trapping of atoms is much more widely explored.~\cite{2011Ac}.
\cm{more cites here!} In addition, we will see in section~\cm{reference CPW
discussion}% TODO
that the electrostatic trap is not achievable in our intended design. As such we
will focus in this section on magnetostatic trapping of atoms.

% TODO Continue this

\subsubsection{Loading}

Folman et al.~\cite{Folman2000} were one of the first to demonstrate mirror-MOT
loading, where the surface of the chip is used as the mirror to from a MOT close
to the surface. The mirror-MOT held $10^8$ \cm{7\^Li} atoms, and was formed of
four laser beams, with two reflected off the chip, making the six beams in total
that are required for trapping. The magnetic field is generated by coils in
anti-Helmholtz configuration, which are switched off in favour of a large
U-shaped wire, which provides a quadrupole field whose zero is aligned with the
chip. This is a very common scheme to bring atoms close to the chip
surface.~\cm{CITE}

After establishing a mirror-MOT, Folman et al. changed the bias field in order
to bring the cloud closer to the surface. They then turn off the lasers so that
the trapping is purely magnetic, and transfer to the trapping
wires on the chip. This adiabatic change of currents (through various wires) and
bias fields is another common step in the loading procedure~\cm{CITE}.

% TODO Is there also adiabatic manipulation?

B\"ohi~\cite{rohtua} follows a similar loading procedure to the above for
\cm{87\^ Rb} atoms in order to load into on-chip Z-wire traps, however in order
to overcome adiabatic heating %TODO introduce above
that occurs when magnetic traps are compressed, they use evaporative cooling...

Direct loading of a chip with an optical tweezer is also a possibility for
transfering molecules directly onto a chip~\cite{Liueaar7797}, with Bernon et
al.~\cite{Bernon2013} having demonstrated the technique with atoms. In this
experiment...


\subsubsection{Manipulation in the trap}


\subsection{Cold molecule sources}

\subsection{Molecule chips}

