% Take-away points:
%   Magnetic trap
%   How to get CPW on (Bohi)
%   MTT + big U to load is viable
%   Required PSD???

At the time of the original molecule chip proposal~\cite{Andre2006}, to see atom
chips had been loaded using a magneto-optical trap (MOT) as a
source~\cite{Reichel1999, Ott2001}, however a molecule MOT had not yet been
built, and there was no credible source of loading a chip of this
type.\footnote{\cm{Flyover chips had been created, maybe something about PSD
meant a beam wouldn't work? expand in body}}
%
Today we have access to molecules cooled below the Doppler
temperature~\cite{Truppe2017}, with potential for further sympathetic cooling
with rubidium~\cm{how to cite this?}. Further technologies for loading a
molecule chip, such a optical tweezers~\cite{Liueaar7797} have also been
demonstrated.

These developments have made the implementation of a molecule chip feasible. In
this section, we will review the state of the field, including work on atom
chips and these supporting technologies.

\subsection{Wire traps}
\cm{need to say something somewhere about how magnetic trapping works}

In developing the molecule chip, we have been able to draw upon the
well-developed field of atom chips to inform our design choices. Let us consider
three different aspects of chip design: trapping on the chip, loading into this
trap, manipulation of trapped atoms or molecules.

The Andr\'e proposal~\cite{Andre2006} suggests the use of an electrostatic trap
to contain the molecules, however this is not the only option available. In
fact, magnetic trapping of atoms is much more widely explored.~\cite{2011Ac}.
\cm{more cites here!} In addition, we will see in section~\cm{reference CPW
discussion}% TODO
that the electrostatic trap is not achievable in our intended design. As such we
will focus in this section on magnetostatic trapping of atoms above a chip.

The basic principle of a magnetic wire trap is explained in Reichel et
al.~\cite{Reichel1999}. As illustrated in \myfigref{litrev:fig:reicheltrap} a
bias field that is applied to the electric field of a straight, current-carrying
wire will create a zero. The field magnitude that must be applied to create the
zero at a height $h$ for a wire carrying current $I$ is given by the magnitude
of the field at that point~\cite{Jackson1975}
%
\begin{equation}
  B = \frac{\mu_0 I}{2\pi h}.
  \label{litrev:eqn:bias}
\end{equation}

\begin{figure}
  \includegraphics[width=0.75\textwidth]{./figs/litrev/reicheltrap.pdf}
  \caption{The combination of the magnetic field due to a straight wire and a
  constant bias are shown to produce a zero, forming a two-dimensional trap for
  weak field seekers. This is the basic principle underlying all magnetic wire
  traps. Reproduced from Reichel et al.~\cite{Reichel1999}
  }
  \label{litrev:fig:reicheltrap}
\end{figure}

Such a field will form a two-dimensional trap for weak-field seeking particles.
To achieve a three-dimensional trap, we can overlay an additional wire,
carrying a smaller current than the first such that there is now some component
of magnetic field along the axis of the original trap. Applying a bias along
this axial direction will re-introduce a field minimum, forming a three-dimensional
\emph{dimple trap} shown in \myfigref{litrev:fig:dimpletrap}.~\cite{2011Ac}
Since the trap centre is now in general a field minimum minimum rather than a
zero, it is a Ioffe-Pritchard trap, which is not susceptible to Majorana
losses~\cite{PhysRevLett.51.1336}.

\begin{figure}
  \includegraphics[width=0.75\textwidth]{./figs/litrev/dimpletrap.png}
  \caption{The field of a straight wire trap (current $I_0$ is perturbed by a
  second wire carrying a current $I_1 \ll I_0 $ and accompanying bias
  ($B_{b,x}$). The resulting field forms a three-dimensional dimple trap.
  }
  \label{litrev:fig:dimpletrap}
\end{figure}

Other variations on the wire trap are possible. Running two perturbing wires
across one axis (with currents $I_1$ and $I_2$) will form an H-trap, shown in
\myfigref{litrev:fig:trapvariations}. Here the minimum will exist between the
two perturbing wires. Note that depending on the relative signs of $I_1$ and
$I_2$ it is possible to form either a Ioffe-Pritchard (currents parallel) or a
quadrupole trap (currents anti-parallel). Approximations of these shapes can be
formed by bending a single wire to create the axial confinement, forming either
a Z-trap (Ioffe-Pritchard) or U-trap (quadrupole), both also shown in
\myfigref{litrev:fig:trapvariations}.

\begin{figure}
  \centering
\begin{tikzpicture}
  % H
  \draw[thick, ->] (-6,0) -- (-3,0);
  \node at (-2.75,-0.25) {$I_0$};
  \draw[thick, ->] (-5.5,-3) -- (-5.5,3);
  \node at (-5.15, 3) {$I_1$};
  \draw[thick, ->] (-3.5,-3) -- (-3.5,3);
  \node at (-3.15, 3) {$I_2$};
  \node at (-6.5, -2.5) {(H)};
  % U
  \draw[thick, ->] (-1,3) -- (-1,0) --(1,0) -- (1,3);
  \node at (1.35,3) {$I$};
  \node at (-1.5, -2.5) {(U)};
  % Z
  \draw[thick, ->] (3,-3) -- (3,0) --(5,0) -- (5,3);
  \node at (5.35,3) {$I$};
  \node at (2.5, -2.5) {(Z)};
\end{tikzpicture}
  \caption{
    A top-down view of the wires forming an H, U and Z trap. The H trap has
    three independent currents, $I_0$, $I_1$ and $I_2$; usually the magnitudes
    of $I_1$ and $I_2$ will be the same, however their relative signs can be
    chosen to form either a quardupole ($I_1 = -I_2$) or Ioffe-Pritchard ($I_1 =
    I_2$) trap. Note that these cases correspond to the U and Z wires
    approximating the currents through the H wire respectively. The bias fields
    requisit for trapping are not shown.
  }
  \label{litrev:fig:trapvariations}
\end{figure}

% Maths for the traps, freq. grad. etc.
The trapping field of the dimple trap close to the centre can be described in
terms of the usual Ioffe-Pritchard equation~\cite{Foot2005}
%
\begin{equation}
  \mathbf{B} = B_0 \begin{bmatrix} 1 \\ 0 \\0 \end{bmatrix}
               + B' \begin{bmatrix} 0 \\ -y \\ z \end{bmatrix}
               + \frac{B''}{2} \begin{bmatrix} 
                  x^2 - \frac{1}{2}(y^2 + z^2) \\
                  -xy \\
                  -xz
               \end{bmatrix},
\end{equation}
%
where we have chosen the origin to be the trap centre (not to be confused with
the centre of the dimple wires). In this case, the field gradient and curvature
are
%
\begin{equation}
  B' = \frac{\mu_0 I_0}{2\pi h^2}
\end{equation}
%
and
%
\begin{equation}
  B'' = \frac{\mu_0 I_1}{\pi h^3}
\end{equation}
%
respectively. The field at the trap centre is
%
\begin{equation}
  B_0 = \widetilde{B}_x + \frac{\mu_0 I_1}{2\pi h}
\end{equation}
%
where $\widetilde{B}_i$ represents the bias field applied in the $i$ direction.
The trap frequencies are given by
\begin{equation}
  \omega_x = \sqrt{\frac{\mu_m}{m}B''}
\end{equation}
and
\begin{equation}
  \omega_\perp = \sqrt{\frac{\mu_m {B'}^2}{m B_0}}
\end{equation}
where $\mu_m$ is the magnetic moment of the trapped particle, and we have taken
$x$ to be the axial direction (aligned with $I_0$).~\cite{2011Ac}
Taking the dimple to be a representative trap type, we can estimate the expected
frequencies that we would expect to achieve based on the current and trapping
height.

\cm{ Need to review these two paragraphs. Perhaps I can just pull the numbers
from literature or straight out of the experiment section. }

The current that can be achieved in the chip will be limited by thermal
properties of the final chip design.  We will see in section \cm{TODO: ref} that
we plan to follow a multi-layer design similar to that found in B\"ohi et
al.~\cite{Boehi2009}, who report~\cite{rohtula} a maximum current density in the
trapping wires of \SI{5.5e10}{\ampere\per\metre\squared}. Trapping wire sizes
will have an area of order $\SI{1}{\micro\metre} \times \SI{1}{\micro\metre}$,
resulting in a total current \SI{55}{\milli\ampere}.

In section \ref{litrev:manipulation} we will discuss how the interaction between
the molecules and microwave fields from an integrated CPW \cm{TODO: Is this
introduced yet?} make $h \sim \SI{10}{\micro\metre}$ a desirable trapping
height. Using \myeqref{litrev:eqn:axisfreq} and \myeqref{litrev:eqn:transfreq}
we can therfore anticipate trapping frequencies of
%
\begin{align}
  \omega_x &= \SI{555555}{\radian \per \second} \\
  \omega_\perp &= \SI{555555}{\radian \per \second}.
\end{align}
\cm{TODO: figure these out}
%
Frequencies for the other trap types should be calculated numerically\cm{TODO:
should I reference my MRes?}, but the above figures are presented as a
representative example of what can be expected from a wire trap.

\subsection{Loading wire traps}

Reichel et al.~\cite{Reichel1999} were one of the first to demonstrate
mirror-MOT loading, where the surface of the chip is used as the mirror to from
a MOT close to the surface. The mirror-MOT held $5\times10^6$ \cm{87\^Rb} atoms,
and was formed of four laser beams, with two reflected off the chip, making the
six beams in total that are required for trapping. The magnetic field is
generated by coils in anti-Helmholtz configuration, which are switched off in
favour of a large U-shaped wire, which provides a quadrupole field whose zero is
aligned with the chip. This is a very common scheme to bring atoms close to the
chip surface~\cite{Folman2000, PhysRevLett.97.200405, 2011Ac, Boehi2009},
however it has also been shown~\cite{0256-307X-25-9-034} that it is possible to
forego the coil MOT and load atoms directly from a MOT formed by a U-wire.

After establishing a mirror-MOT, Reichel et al. changed the bias field in order
to bring the cloud closer to the surface. They then turn off the lasers so that
the trapping is purely magnetic, and transfer to the trapping
wires on the chip. An adiabatic change of the bias fields can then be used to
draw the atoms closer to the surface~\cite{Reichel1999, Folman2000}. This
principle can be exemplified by the straight wire trap, with
\myeqref{litlitrev:eqn:bias} showing that trapping closer to the wire requires a
higher bias field (or a lower current).

This principle can be extended to more complex trap types and varying the
currents through the trapping wires trapping.~\cite{Folman2000} It is also
possible to construct a series of wire traps in close proximity, whose currents
can then be independently controlled such that the trap centre is moved between
them. This allows the use of higher currents through wide wires when the cloud
is far from the surface, but a smaller wire must be used when the atoms are
close to it, so that the approximation of an infinitesimally small wire remains
accurate.

This adiabatic change of currents (through various wires) and bias fields is
another common step in the loading procedure~\cite{Folman2000,2011Ac,
RevModPhys.79.235} and can also be used to transport atoms above the
surface~\cite{Reichel1999, Schwindt2005}. Bringing the trap centre close to the
surface also compresses the atom cloud, and since the compressing force is
conservative~\cm{TODO: cite}, this leads to adiabatic heating of the cloud.
Conservation of phase space density suggests that the final temperature will be
given by~\cite{Metcalf1999}
\cm{TODO: check this cite is legit}
%
\begin{equation}
   T_1 = T_0\left(\frac{V_0}{V_1}\right)^\frac{2}{3}.
\end{equation}

After compression, the likelihood of collisions between the atoms is increased,
which results in the loss of large numbers of atoms due to inelastic collisions.
The solution to this, as presented by B\"ohi et al.~\cite{Boehi2009} is to
evaporatively cool atoms in the larger traps, before transferring them into the
smaller traps. The magnetic traps are compressed to increase the collision
rates and then a radio frequency (RF) ramp is applied to remove the hottest
atoms from the trap, thus reducing the total energy and increasing the number of
atoms that will be accepted into the next (smaller) trap.~\cite{Foot2005,
Metcalf1999}

Also of note is the possibility of forming a MOT from a single laser beam using
a patterned wafer~\cite{Nshii2013}. This eliminates the need for multiple beams
to create a mirror-MOT. This has the benefit of offering a stable and highly
reproducible MOT for loading.

Direct loading of a chip with an optical tweezer is also a possibility for
transferring molecules directly onto a chip~\cite{Liueaar7797}, with Bernon et
al.~\cite{Bernon2013} having demonstrated the technique with atoms. In this
experiment, an optical tweezer transport a cloud of $5\times10^6$ \cm{87\^Rb}
atoms from a MOT onto a chip trap \SI{40}{\milli\metre} away. This has the
benefit of insignificant losses and heating. This is a very attractive option
for loading, but molecular optical tweezers are still in their
infancy~\cite{Anderegg2019}, so this is left as a possibility for future
research.

\subsection{Molecule chips}
\label{litrev:molculechips}

\subsection{Coherent quantum control of atoms and molecules}
\label{litrev:control}

% Start with macroscopic, build up to atom chips w/ CPWs

% Make sure we discuss VdW and Casmir-Poldar (sp?) shift, basically just
% reproduce the start of Ed's notebook.


% Photolith
% Electroplating
% Multilayers
