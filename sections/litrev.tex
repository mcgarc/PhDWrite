\cm{
 Take-away points:
   Magnetic trap
   How to get CPW on (Bohi)
   MTT + big U to load is viable
   Required PSD???
 }

At the time of the original molecule chip proposal~\cite{Andre2006}, atom chips
had been loaded using a magneto-optical trap (MOT) as a
source~\cite{Reichel1999, Ott2001}, however a molecule MOT had not yet been
built, and there was no credible source of loading a chip of this
type.\footnote{Stark declerator molecule chips would be developed shortly
after~\cite{Meek1699}, but Andr\'e et al. presented no method of loading their
chip design with the required phase space density.}  

As discussed above, the field of cold molecules has now developed significantly.
These developments have made the implementation of a molecule chip feasible. In
this section, we will review the state of chip trapping, including work on atom
chips and these supporting technologies.

\subsection{Wire traps}
\label{litrev:wiretraps}

In developing the molecule chip, we have been able to draw upon the
well-developed field of atom chips to inform our design choices. Let us consider
three different aspects of chip design: trapping on the chip, loading into this
trap, manipulation of trapped atoms or molecules.

The Andr\'e proposal~\cite{Andre2006} suggests the use of an electrostatic trap
to contain the molecules, however this is not the only option available. In
fact, magnetic trapping of atoms is much more widely explored.~\cite{2011Ac},
and the discovery of magnetically insensitive transitions in \CaF{} makes this a
promising choice for a molecule chip. As such we will focus in this section on
magnetostatic trapping of atoms above a chip.

The basic principle of a magnetic wire trap is explained in Reichel et
al.~\cite{Reichel1999}. As illustrated in \myfigref{litrev:fig:reicheltrap} a
bias field that is applied to the electric field of a straight, current-carrying
wire will create a zero. The field magnitude that must be applied to create the
zero at a height $h$ for a wire carrying current $I$ is given by the magnitude
of the field at that point~\cite{Jackson1975}
%
\begin{equation}
  B = \frac{\mu_0 I}{2\pi h}.
  \label{litrev:eqn:bias}
\end{equation}
%
As with other magnetic traps, atoms (or molecules) can be optically pumped into
weak-field seeking states such that they favourably occupy the region surround
the trap minimum.~\cite{Metcalf1999}

\begin{figure}
  \includegraphics[width=0.75\textwidth]{./figs/litrev/reicheltrap.pdf}
  \caption{The combination of the magnetic field due to a straight wire and a
  constant bias are shown to produce a zero, forming a two-dimensional trap for
  weak field seekers. This is the basic principle underlying all magnetic wire
  traps. Reproduced from Reichel et al.~\cite{Reichel1999}
  }
  \label{litrev:fig:reicheltrap}
\end{figure}

Such a field will form a two-dimensional trap for weak-field seeking particles.
To achieve a three-dimensional trap, we can overlay an additional wire,
carrying a smaller current than the first such that there is now some component
of magnetic field along the axis of the original trap. Applying a bias along
this axial direction will re-introduce a field minimum, forming a three-dimensional
\emph{dimple trap} shown in \myfigref{litrev:fig:dimpletrap}.~\cite{2011Ac}
Since the trap centre is now in general a field minimum minimum rather than a
zero, it is a Ioffe-Pritchard trap, which is not susceptible to Majorana
losses~\cite{PhysRevLett.51.1336}.

\begin{figure}
  \includegraphics[width=0.75\textwidth]{./figs/litrev/dimpletrap.png}
  \caption{The field of a straight wire trap (current $I_0$ is perturbed by a
  second wire carrying a current $I_1 \ll I_0 $ and accompanying bias
  ($B_{b,x}$). The resulting field forms a three-dimensional dimple trap.
  }
  \label{litrev:fig:dimpletrap}
\end{figure}

Other variations on the wire trap are possible. Running two perturbing wires
across one axis (with currents $I_1$ and $I_2$) will form an H-trap, shown in
\myfigref{litrev:fig:trapvariations}. Here the minimum will exist between the
two perturbing wires. Note that depending on the relative signs of $I_1$ and
$I_2$ it is possible to form either a Ioffe-Pritchard (currents parallel) or a
quadrupole trap (currents anti-parallel). Approximations of these shapes can be
formed by bending a single wire to create the axial confinement, forming either
a Z-trap (Ioffe-Pritchard) or U-trap (quadrupole), both also shown in
\myfigref{litrev:fig:trapvariations}.

\begin{figure}
  \centering
\begin{tikzpicture}
  % H
  \draw[thick, ->] (-6,0) -- (-3,0);
  \node at (-2.75,-0.25) {$I_0$};
  \draw[thick, ->] (-5.5,-3) -- (-5.5,3);
  \node at (-5.15, 3) {$I_1$};
  \draw[thick, ->] (-3.5,-3) -- (-3.5,3);
  \node at (-3.15, 3) {$I_2$};
  \node at (-6.5, -2.5) {(H)};
  % U
  \draw[thick, ->] (-1,3) -- (-1,0) --(1,0) -- (1,3);
  \node at (1.35,3) {$I$};
  \node at (-1.5, -2.5) {(U)};
  % Z
  \draw[thick, ->] (3,-3) -- (3,0) --(5,0) -- (5,3);
  \node at (5.35,3) {$I$};
  \node at (2.5, -2.5) {(Z)};
\end{tikzpicture}
  \caption{
    A top-down view of the wires forming an H, U and Z trap. The H trap has
    three independent currents, $I_0$, $I_1$ and $I_2$; usually the magnitudes
    of $I_1$ and $I_2$ will be the same, however their relative signs can be
    chosen to form either a quadrupole ($I_1 = -I_2$) or Ioffe-Pritchard ($I_1 =
    I_2$) trap. Note that these cases correspond to the U and Z wires
    approximating the currents through the H wire respectively. The bias fields
    requisite for trapping are not shown.
  }
  \label{litrev:fig:trapvariations}
\end{figure}

% Maths for the traps, freq. grad. etc.
The trapping field of the dimple trap close to the centre can be described in
terms of the usual Ioffe-Pritchard equation~\cite{Foot2005}
%
\begin{equation}
  \mathbf{B} = B_0 \begin{bmatrix} 1 \\ 0 \\0 \end{bmatrix}
               + B' \begin{bmatrix} 0 \\ -y \\ z \end{bmatrix}
               + \frac{B''}{2} \begin{bmatrix} 
                  x^2 - \frac{1}{2}(y^2 + z^2) \\
                  -xy \\
                  -xz
               \end{bmatrix},
\end{equation}
%
where we have chosen the origin to be the trap centre (not to be confused with
the centre of the dimple wires). In this case, the field gradient and curvature
are
%
\begin{equation}
  B' = \frac{\mu_0 I_0}{2\pi h^2}
\end{equation}
%
and
%
\begin{equation}
  B'' = \frac{\mu_0 I_1}{\pi h^3}
\end{equation}
%
respectively. The field at the trap centre is
%
\begin{equation}
  B_0 = \widetilde{B}_x + \frac{\mu_0 I_1}{2\pi h}
\end{equation}
%
where $\widetilde{B}_i$ represents the bias field applied in the $i$ direction.
The trap frequencies are given by
\begin{equation}
  \omega_x = \sqrt{\frac{\mu_m}{m}B''}
  \label{litrev:eqn:axisfreq}
\end{equation}
and
\begin{equation}
  \omega_\perp = \sqrt{\frac{\mu_m {B'}^2}{m B_0}}
  \label{litrev:eqn:transfreq}
\end{equation}
where $\mu_m$ is the magnetic moment of the trapped particle, and we have taken
$x$ to be the axial direction (aligned with $I_0$).~\cite{2011Ac}
Taking the dimple to be a representative trap type, we can estimate the expected
frequencies that we would expect to achieve based on the current and trapping
height.

\thesis{I probably want to go through this in a lot more detail including the
maths. This is all in the notebook that Ed Hinds gave me to start with.} A
typical final\footnote{Greater intermediary trapping heights are used as part of
the loading procedure.} trapping height will be around \SI{10}{\micro\metre}.
This allows on-chip microwave fields to interact with trapped atoms without
introducing any energy shifts   or significant attractive force acting on the
molecule from the Casimir effect or Van der Waals force.\footnote{Trapping
beneath this height could allow investigation of these effects on a quantum
scale and is in itself a goal of atom chip
trapping.~\cite{PhysRevA.56.R3350}}~\cite{2011Ac}

\thesis{Probably want to expand this similar to for height}
The current that can be achieved in the chip will be limited by thermal
properties of the final chip design as will be discussed in
section~\ref{experiment:multilayer}. Typical achievable currents are estimated
to be around \SI{60}{\milli\ampere}. 
Using \myeqref{litrev:eqn:transfreq} and
\myeqref{litrev:eqn:axisfreq} we can therefore anticipate trapping frequencies
for \CaF{} of
%
\begin{align}
  \omega_x &= \SI{5e4}{\radian \per \second} \\
  \omega_\perp &= \SI{1e5}{\radian \per \second}.
\end{align}
%
Where we have assumed a minimum value of $B_0 = \SI{1}{\gauss}$ has been chosen.
These figures are in line with what we would normally expect from a wire
trap.~\cite{2011Ac}
Frequencies for the other trap types can be calculated numerically, but the
above figures are presented as a representative example of what can be expected
from a wire trap.

The trap depth is well-approximated by the energy associated with the bias
field. For a trap at height \SI{10}{\micro\metre} this gives a typical trap
depth on the order of a few tens of millikelvin.~\cite{2011Ac}

\subsection{Loading wire traps}

Reichel et al.~\cite{Reichel1999} were one of the first to demonstrate
mirror-MOT loading, where the surface of the chip is used as the mirror to from
a MOT close to the surface. The mirror-MOT held $5\times10^6$ \esRb{} atoms,
and was formed of four laser beams, with two reflected off the chip, making the
six beams in total that are required for trapping. The magnetic field is
generated by coils in anti-Helmholtz configuration, which are switched off in
favour of a large U-shaped wire, which provides a quadrupole field whose zero is
aligned with the chip. This is a very common scheme to bring atoms close to the
chip surface~\cite{Folman2000, PhysRevLett.97.200405, 2011Ac, Boehi2009},
however it has also been shown~\cite{0256-307X-25-9-034} that it is possible to
forego the coil MOT and load atoms directly from a MOT formed by a U-wire.

After establishing a mirror-MOT, Reichel et al. changed the bias field in order
to bring the cloud closer to the surface. They then turn off the lasers so that
the trapping is purely magnetic, and transfer to the trapping
wires on the chip. An adiabatic change of the bias fields can then be used to
draw the atoms closer to the surface~\cite{Reichel1999, Folman2000}. This
principle can be exemplified by the straight wire trap, with
\myeqref{litlitrev:eqn:bias} showing that trapping closer to the wire requires a
higher bias field (or a lower current).

This principle can be extended to more complex trap types and varying the
currents through the trapping wires trapping.~\cite{Folman2000} It is also
possible to construct a series of wire traps in close proximity, whose currents
can then be independently controlled such that the trap centre is moved between
them. This allows the use of higher currents through wide wires when the cloud
is far from the surface, but a smaller wire must be used when the atoms are
close to it, so that the approximation of an infinitesimally small wire remains
accurate.

This adiabatic change of currents (through various wires) and bias fields is
another common step in the loading procedure~\cite{Folman2000,2011Ac,
RevModPhys.79.235} and can also be used to transport atoms above the
surface~\cite{Reichel1999, Schwindt2005}. Bringing the trap centre close to the
surface also compresses the atom cloud, and since the compressing force is
conservative~\cite{Metcalf1999}, this leads to adiabatic heating of the cloud.
Conservation of phase space density suggests that the final temperature will be
given by~\cite{Metcalf1999}
\thesis{Check this cite is legit}
%
\begin{equation}
   T_1 = T_0\left(\frac{V_0}{V_1}\right)^\frac{2}{3}.
\end{equation}

After compression, the likelihood of collisions between the atoms is increased,
which results in the loss of large numbers of atoms due to inelastic collisions.
The solution to this, as presented by B\"ohi et al.~\cite{Boehi2009} is to
evaporatively cool atoms in the larger traps, before transferring them into the
smaller traps. The magnetic traps are compressed to increase the collision
rates and then a radio frequency (RF) ramp is applied to remove the hottest
atoms from the trap, thus reducing the total energy and increasing the number of
atoms that will be accepted into the next (smaller) trap.~\cite{Foot2005,
Metcalf1999}

Also of note is the possibility of forming a MOT from a single laser beam using
a patterned wafer~\cite{Nshii2013}. This eliminates the need for multiple beams
to create a mirror-MOT. This has the benefit of offering a stable and highly
reproducible MOT for loading.

Direct loading of a chip with an optical tweezer is also a possibility for
transferring molecules directly onto a chip~\cite{Liueaar7797}, with Bernon et
al.~\cite{Bernon2013} having demonstrated the technique with atoms. In this
experiment, an optical tweezer transport a cloud of $5\times10^6$ \esRb{}
atoms from a MOT onto a chip trap \SI{40}{\milli\metre} away. This has the
benefit of insignificant losses and heating. This is a very attractive option
for loading, but molecular optical tweezers are still in their
infancy~\cite{Anderegg2019}, so this is left as a possibility for future
research.

\thesis{Should expand this paragraph into a whole section.}
%
Fabrication of atom chips makes use of standard photolithographic procedures to
create wires that form traps.~\cite{2011Ac}. However, this is an inefficient way
to create wires of the desired height (usually several micrometres, whereas
photolithography produces structures tens of nanometres tall). This prevents the
formation of traps high above the surface. In order to achieve the required
heights, it is possible to first lay down a seed layer of gold and then create a
mould made of photoresist in the shape of the desired wires. Electroplating can
then be used to build up the wires before cleaning off the mould and etching off
the seed layer.~\cite{4797887}

\subsection{Molecule chips}
\label{litrev:molculechips}


Molecule chips that have been previously implemented rely on quite different
loading schemes to those discussed for atom chips. In 2008 Meek et
al.~\cite{Meek2008} have loaded \CO{} molecules directly from a supersonic
beam, with on-chip slowing performed by a Stark decelerator integrated into the
chip.

Fabrication of a chip-based Stark decelerators using photolithographic
techniques is significantly less involved than construction of a macroscopic
decelerator, providing excellent scalability; Meek et al. used over 1200
electrodes, as opposed to the 63 stages used in an equivalent macroscopic
experiment~\cite{Bethlem1999}. Varying the electric field above the surface of
the chip was employed to slow of molecules from a \SI{312}{\metre\per\second}
beam to a standstill, constrained in the shape of \SI{20}{\micro\metre} diameter
tubes, \SI{25}{\micro\metre} above the chip surface.~\cite{Meek2009}

On-chip control of the molecules was demonstrated by the same group, using
microwaves propagating through free space~\cite{doi:10.1002/cphc.201001007} and
separately with infrared light~\cite{doi:10.1080/00268976.2012.683885}.  \CO{}
molecules in the $J=1$ rotational state were rotationally excited into the $J=2$
state on the chip, then accelerated off chip for state-selective detection.
On-chip imaging has since been achieved, but has not been fully integrated with
the chip, still requiring external optics.~\cite{Marx2013}

Whilst being a clear demonstration of the potential of
molecule chips, the lack of on-chip measurement and integrated control mean that
these chips are inherently not scalable.  Furthermore, the trapping volumes were
still fairly large, around \SI{0.25}{\milli\metre\cubed}. 
%
\thesis{Why do we want a small trap volume?}

\thesis{PSD of beam? Why did Mike say this was no good for us?}
%
The use of on-chip decelerators was motivated by a lack of laser-cooled
molecules, however this has now been achieved, producing sources of high
phase-space densities~\cite{Truppe2017}, which could be used for loading a
different design of chip such as that proposed by Andr\'e et
al.~\cite{Andre2006}

The Andr\'e proposal outlines an electrostatic trap for molecules embedded in a
microwave resonator to form a cavity-QED system. The high coupling between the
molecule and cavity photons could be used for forming a qubit and sideband
cooling into the ground state of the trap. The microwave guide can be used for
quantum control, and on-chip measurement by applying an off-resonant microwave
pulse. This design promises a robust trap with potential for integration with
other chip-based qubits (either other trapped molecules or different quantum
architectures).

\subsection{Quantum control on macroscopic and microscopic scales}
\label{litrev:control}

\thesis{May want to think about moving this subsection to help with flow.}

\thesis{This next paragraph is very qualitative. I could expand it by
introducing Jayne's Cummings and going through the maths, but this is only
really useful if we can talk about it as a reasonable thing to try (i.e. if we
can do a superconducting chip).}
%
Coherent control of atomic and molecular states has been achieved in macroscopic
traps~\cite{Gross995, Blackmore_2018}, and is an essential tool for quantum
technologies. For example, the ability to address a single atom in an optical
lattice can be used to simulate other quantum systems such as a Mott
insulator.~\cite{Weitenberg2011} Coherent control of atoms on a chip has also
been achieved.~\cite{PhysRevLett.92.063601} Of most relevance to this report is
the experiment by B\"ohi et al.~\cite{Boehi2009} who observed Ramsey fringes on
the hyperfine levels of \esRb{} using microwaves carried across the surface of a
chip. Coherent control achieved in this manner is innately robust and scalable;
and therefore highly desirable for applications in quantum technologies.

\thesis{Need more legit version of this paragraph. For one thing I should be
able to calculate the lifetime (and hence Q factor) that I need.}
%
In order to achieve strong coupling between the trapped atoms (molecules) and the
microwaves there must be good overlap between the microwave field and the cloud.
If the trapped particles experience different field intensity or polarization
across the trap then this will affect the fidelity of any
operations.~\cite{Williams2018} If a microwave resonator is used then it must
have a lifetime long enough to sustain the interaction between the field and the
molecules during any pulse. This has been achieved in atoms for quality factor
$Q\sim2200$.~\cite{Hattermann2017}

\thesis{Is ``qubits'' the right thing to refer to here?}
%
For molecules, it is possible to use microwaves to control the rotational
states~\cite{Blackmore_2018}, which do not exist in atoms. This results in a
coupling between the microwaves and the particles, which is orders of magnitude
stronger for molecules. The Andr\'e proposal~\cite{Andre2006} suggests that this
could be exploited to allow long-range coupling to other qubits, or sideband
cooling into the trap ground state.

\thesis{This paragraph will need updating with cites for the 40GHz paper when
released.}
%
Recently, magnetically insensitive transitions have been discovered in
\CaF{}.~\cite{PhysRevLett.120.163201, Blackmore_201, }  One transition between
the $\ket{N=0, F=1, M_F=1}$ and $\ket{1,2,2}$ states has been found at
\SI{20.5}{\giga\hertz} and another between $\ket{1,2,2}$ and $\ket{2,3,3}$ at
\SI{41}{\giga\hertz} . The \SI{20.5}{\giga\hertz} transition has been shown to
be highly insensitive to the magnetic field ($\dd f / \dd B =
-\SI{105(4)}{\hertz\per\gauss}$). This makes it a promising target for a qubit
transition, as long coherence times of up to \SI{0.61(3)}{\milli\second} have
been observed~\cite{Blackmore_2018}. In as yet unpublished work, the
higher-frequnecy transition has been shown to be even more insensitive ($\dd f /
\dd B = -\SI{4.6(1)}{\hertz\per\gauss}$). 

This makes the $\ket{1,2,2} \leftrightarrow \ket{2,3,3}$  transition in \CaF{} a
promising candidate for quantum control in a magnetic chip trap. The need for an
electrostatic trap is removed by the magnetically insensitive transitions and
the existing atom chip technologies such as those created by B\"ohi et
al.~\cite{Boehi2009} provide a blueprint for implementation.
