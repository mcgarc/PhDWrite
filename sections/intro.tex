\cm{Para on cooling of atoms, along with applications, then introduce need for
atom chips: The ability to laser cool molecules and atoms has opened the
door...}

When Weinstein and Libbrecht first proposed a scheme for microscopic confinement
of atoms using microfabricated magnetic traps~\cite{PhysRevA.52.4004}, it was at
a time when production, ultra-cooling and trapping of atoms was already well
understood and commonly practiced.~\cm{TODO: cites for each of these} Only four
years later, Reichel et al.~\cite{Reichel1999} demonstrated the first atomic
``chip trap,'' a microfabricated device using wires on the surface of a planar
substrate. By the nature of the design and fabrication technique, a chip trap is
more robust and scalable than its macroscopic counterparts~\cm{TODO: cite} and
as such has many applications in quantum technology.

\cm{TODO: find examples and continue talking about applications as separate
paragraph note espeically MW stuff. Need to introduce MOT as a concept. OR maybe
this gives way to opening para}

In 2006, Andre et al.~\cite{Andre2006} published a proposal for a molecular chip
trap. The coupling microwaves to the rotational structure of a diatomic molecule
would be several orders of magnitude stronger than the coupling to hyperfine
transitions in an atom. \cm{Why is this good?}

At the time of proposal, the field of ultracold molecules was still relatively
unexplored, with the first molecular MOT was not created until~\cm{cite molecule
MOT}. Indeed production of cold molecules has shown in many ways to be more
challenging than for atoms, in large part due to the significantly more complex
energy sturcutre they exhibit.~\cm{CITE} Today however we are able to cool
molecules below the Doppler limit~\cite{Truppe2017}, coherently control the
states of such molecules~\cite{PhysRevLett.120.163201} and load molecules into a
tweezer array.~\cite{Anderegg2019}

\cm{TODO: redo this}
The groundwork has been laid to produce a molecular chip trap. In the following
subsections I will discuss further the field of ultra-cooling of atoms
molecules, and the need for a molecule chip. The rest of the report will focus
on chip trapping, the design of our molecule chip and the outlook for the
remainder of the project.

\subsection{Laser cooling of atoms and molecules}
\label{intro:lasercool}

A brief discussion of laser cooling is presented here. More detailed discussion
can be found in references~\cite{Metcalf1999,RevModPhys.70.721,McCarron_2018}.

The basic principle of laser cooling of atoms is to deflect atoms from their
trajectory by transfer of momentum from the laser light (radiation
pressure).~\cite{RevModPhys.70.721} Consider a moving atom with a
pre-chosen cooling transition between two internal states.  When such an atom
encounters a counter-propogating resonant photon absorption can take place,
reducing the atom's momentum. The atom can re-emit the photon in any direction,
but on average the resulting atom will be slower than before the encounter.
This process is known as Doppler cooling, as the light must be detuned from
resonance to account for Doppler shift experienced by the moving
molecule.~\cite{Metcalf1999}

Of course, the assumption of a two-level system is a vast simplification.  A key
example is sodium, where the hyperfine transition $3S_{1/2}\, (F=2) \rightarrow
3P_{3/2}\, (F'=3)$ transition was used to demonstrate this technique. However,
since the excited states of sodium have decay channels into $F=1$, it is
possible for atoms to fall into this state and out of resonance with the
cooling transition.  This can be overcome by introducing repumping light to
transfer atoms in dark states back into light states.~\cite{RevModPhys.70.721}

Most atom sources produce a beam, with typical velocities of a few
\SI{100}{\metre\per\second}.~\cite{Metcalf1999,}  As the atoms are slowed, the
Doppler shift they experience will change, and hence they will move out of
resonance with the transition.~\cite{RevModPhys.70.721} This can be overcome by
introducing chriped light~\cite{Prodan1984} or varying the transition frequency
by control of an external splitting field~\cite{PhysRevLett.48.596}.

An optical molasses can be formed from a pair of counter-propogating laser
beams, both red-detuned from the cooling transition. If the atom moves towards
either beam it will move into resonace and experience a pressure back towards
the trap centre. A set of three perpendicular optical molasses can cool an atom
in each direction, reducing the temperature but not providing any trapping
force~\cite{Metcalf1999} . Further cooling can be achieved in an optical
molasses by choosing the polarization of the light such that there is a
polarization gradient. This sub-Doppler cooling allows atoms to be cooled to the
lowest limit, the recoil temperature of the atoms (typically a few
\cm{\si{\micro\kelvin}}).~\cite{Dalibard:89}

Trapping can be achieved (for example) by optically pumping the atoms into a
weak-field seeking state and aplying a magnetic field to form either a
quadrupole~\cite{PhysRevLett.54.2596} or
Ioffe-Pritchard~\cite{PhysRevLett.51.1336} trap. Magnetically trapped atoms can
be further cooled by evaporation, which can ultimately be used to form a
Bose-Einstein condensate.~\cite{Anderson198} A quadrupole trap combined with an
optical molasses forms a magneto-optical trap (MOT) in which atoms are
simultaneously trapped by the magnetic field and cooled by the
lasers.~\cite{PhysRevLett.59.2631}

The MOT has become a fundamental component of many cold atom experiments, a vast
field which we cannot explore in any level of detail here. Of particular note to
us are experiments involving dipole trapping~\cite{PhysRevA.47.R4567}, such as
those in an optical lattice~\cite{Bakr2009} or tweezer~\cite{Ashkin:86}. Control
of the atomic states in a lattice has been used to perform quantum
simulation~\cite{Gross995} and to precisely measure the
second~\cite{Campbell90}.

The rich energy structure of molecules, having rotational and vibrational
degrees of freedom, makes cooling procedures more involved~\cite{Tarbutt2018},
but there are numerous properties of molecules that could be expolited in a
range of applications. The large electric dipole moment of some diatomic
molecules means that they interact more strongly with electric fields than
atoms, making an optical lattice of molecules, or a molecule on a chip, a good
candidate for scalable quantum simulation or computing~\cite{Micheli2006,
Andre2006}. Collisions between cold molecules could allow investigation of the
nature of quantum chemistry~\cite{Krems2008}, and cooled \cm{YbF} is being
developed to precisely measure the electron electric dipole moment as a means of
testing fundamental theories of physics~\cite{Lim2018}.

The tools developed for creation of cold atoms can be
directly applied to producing cold molecules. A MOT of \cm{$Rb85?$} and
\cm{$Cs133?$} can be used to form \cm{$RbCs$} via an interspecies Feshbach
resonance~\cite{PhysRevA.85.032506, PhysRevA.89.033604} and then transfered into
the ro-vibrational ground state by stimulated Raman adiabatic passage
(STIRAP).~\cite{PhysRevLett.113.255301, RevModPhys.70.1003}.

Alternatively, direct cooling of molecules can be performed, with notable
achivements for diatomic molecules.~\cite{Shuman2010} Beams of \cm{SrF} and
\cm{CaF} have been slowed with chirped light~\cite{PhysRevLett.108.103002,
Truppe2017a} and held in a MOT~\cite{Barry2014, Williams2017}. Sub-Doppler
cooling~\cite{Truppe2017} and coherent control of the rotational
state~\cite{PhysRevLett.120.163201,Blackmore_2018} have been demonstrated for
\cm{CaF}. Early experiments with optical tweezers have also been
performed~\cite{Anderegg2019}.

\subsection{Structure of this report}

In this document, I will outline the ongoing progress in the design and
implementation of a molecular chip trap. In section \ref{litrev} I will present
the existing research that has been undertaken in the fields of atomic and
molecular chip traps. Next, section \ref{experiment} will discuss the design
process for our experiment, and the various engineering challenges that we have
had to overcome. Finally I will present the outlook for the remainder of the
project, including some \cm{stretch goals??}
