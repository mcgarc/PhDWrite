Cold atoms and molecules systems are promising for the implementation of new
quantum technologies. They have been used for devices in fields of
measurement~\cite{PhysRevLett.120.103201}, simulation~\cite{Gross995} and testing of
fundamental physics~\cite{DeMille990}. However, one of the main disadvantages
when compared to, for example, superconductor~\cite{Wallraff2004} or pure
optical~\cite{Browne2017} systems is the lack of scalability and
miniaturization.~\cite{nielsenandchuang}

In 1995, Weinstein and Libbrecht proposed an experiment for
studying cold atoms in environments with extremely high magnetic
gradients.~\cite{PhysRevA.52.4004} These gradients can be obtained by using
microfabricated wires in combination with a biasing field to form a trap above
a surface. Not only does this lead to an interesting environment for studying
new physics, but it introduces miniaturization into the cold-atom
toolkit.~\cite{2011Ac} There is also potential for integration of atomic or
molecular systems on a chip with other chip-based quantum
systems.~\cite{2011Ac, Kubo2011}

The first atomic chip trap was demonstrated four years later by Reichel et
al.~\cite{Reichel1999}. Their chip trap was shown to be a robust environment for
atomic experiments. The field of atom chips has since expanded~\cite{2011Ac},
with use as an architecture for creation of Bose-Einstein condensates being of
particular interest to many physicists.~\cite{Ott2001}

Although laser cooling of atoms has been commonplace since the late
20\textsuperscript{th} century, cooling of molecules has taken longer to develop
due to their more complex internal energy structure. However, this same rich
energy structure and other features of the molecules could be exploited to
produce new interesting technologies.~\cite{Tarbutt2018} We are now able to
produce \CaF{} molecules cooled below the Doppler limit~\cite{Truppe2017},
coherently control the states of such molecules~\cite{Williams2018}
and load molecules into a tweezer array.~\cite{Anderegg2019} These developments
open the door to the realization of a molecule chip.

A proposal by Andr\'e et al. in 2006~\cite{Andre2006} suggested that integrating
an electrostatic trap with a microwave field, both integrated on a chip would be
a powerful architecture, owing to the high coupling between rotational states of
the molecule and the microwave field (orders of magnitude greater than coupling
between microwaves and the hyperfine structure of an atom \cm{Mike: why?}).
Photons coupled to trapped molecules could be used as flying qubits to couple
into other trapped molecules, or qubits in other
architectures.~\cite{PhysRevLett.92.063601}

In the following subsection I will further discuss the field of laser cooling of
atoms and molecules. The rest of the report will focus on chip trapping, the
design of our molecule chip for \CaF{} and the outlook for the remainder of the
project.

\subsection{Laser cooling of atoms and molecules}
\label{intro:lasercool}

A brief discussion of laser cooling is presented here. More detailed discussion
can be found in references~\cite{Metcalf1999,RevModPhys.70.721,McCarron_2018}.

The basic principle of laser cooling of atoms is to deflect atoms from their
trajectory by transfer of momentum from the laser light (radiation
pressure).~\cite{RevModPhys.70.721} Consider a moving atom with a
pre-chosen cooling transition between two internal states.  When such an atom
encounters a counter-propagating resonant photon, absorption can take place,
reducing the atom's momentum. The atom can re-emit the photon in any direction,
but on average the resulting atom will be slower than before the encounter.
This process is known as Doppler cooling, as the light must be detuned from
resonance to account for the Doppler shift experienced by the moving
molecule.~\cite{Metcalf1999}

Of course, the assumption of a two-level system is a vast simplification.  A key
example is sodium, where the transition $3S_{1/2}\, (F=2) \rightarrow 3P_{3/2}\,
(F'=3)$ was used to demonstrate this technique. However, since the excited
states of sodium have decay channels into $F=1$, it is possible for atoms to
fall into this state and out of resonance with the cooling
transition.\footnote{To be precise, $F'=3$ has no decay channels to $F=1$, but
off-resonant excitation into $F'=2$ can result in the population of dark states
by the decay $F'=2 \rightarrow F=1$.} This can be overcome by introducing
repumping light to transfer atoms in dark states back into bright
states.~\cite{RevModPhys.70.721}

Most atom sources produce a beam, with typical velocities of a few
\SI{100}{\metre\per\second}.~\cite{Metcalf1999}  As the atoms are slowed, the
Doppler shift they experience will change, and hence they will move out of
resonance with the transition.~\cite{RevModPhys.70.721} This can be overcome by
introducing chirped light~\cite{Prodan1984} or varying the transition frequency
by control of an external magnetic field~\cite{PhysRevLett.48.596}.

\cm{Mike:  \\
%
``Alternative
approaches change the resonant frequency by introducing a varying electric field
(Stark decelerator)~\cite{Bethlem1999} or magnetic field (Zeeman
decelerator)~\cite{PhysRevLett.48.596}.'' \\
%
This is not correct, there is no laser cooling/ slowing in Stark decelerator.
Electric field is used to remove the energy, not keep atoms resonant. Also
Zeeman mentioned in alst sentence. \\
%
Cameron: I should try and introduce the STark decelerator properly (separate
para (here?)), and name the Zeeman slower above.
}

\cm{Confused tense below.}
An optical molasses can be formed from a pair of counter-propagating laser
beams, both red-detuned from the cooling transition. If the atom moves towards
either beam it will move into resonance and its velocity in that direction
is (on average) reduced by photon scattering. \cm{I don't think this is entirely
clear, but better than before when I implied molasses were a trap...}
A set of three perpendicular optical molasses can cool an atom
in each direction, reducing the temperature but not providing any trapping
force~\cite{Metcalf1999} . Further cooling can be achieved in an optical
molasses by choosing the polarization of the light such that there is a
polarization gradient. This sub-Doppler cooling allows atoms to be cooled to the
lowest limit, the recoil temperature of the atoms (typically a few
microkelvin).~\cite{Dalibard:89}

Trapping can be achieved (for example) by optically pumping the atoms into a
weak-field seeking state and applying a magnetic field to form either a
quadrupole~\cite{PhysRevLett.54.2596} or
Ioffe-Pritchard~\cite{PhysRevLett.51.1336} trap.

\cm{Alex: describe these}

Magnetically trapped atoms can
be further cooled by evaporation, which can ultimately be used to form a
Bose-Einstein condensate.~\cite{Anderson198} A quadrupole trap combined with an
optical molasses forms a magneto-optical trap (MOT) in which atoms are
simultaneously trapped by the magnetic field and cooled by the
lasers.~\cite{PhysRevLett.59.2631}

The MOT has become a fundamental component of many cold atom experiments, a vast
field which we cannot explore in any level of detail here. Of particular note to
us are experiments involving dipole trapping~\cite{PhysRevA.47.R4567}, such as
those in an optical lattice~\cite{Bakr2009} or tweezer~\cite{Ashkin:86}. Control
of the atomic states in a lattice has been used to precisely realise the
second~\cite{PhysRevLett.120.103201} and to perform quantum
simulation~\cite{Gross995}.

The rich energy structure of molecules, having rotational and vibrational
degrees of freedom, makes cooling procedures more involved~\cite{Tarbutt2018},
but there are numerous properties of molecules that could be exploited in a
range of applications. The large electric dipole moment of some diatomic
molecules means that they interact more strongly with electric fields than
atoms, making an optical lattice of molecules, or a molecule on a chip, a good
candidate for scalable quantum simulation or computing~\cite{Micheli2006,
Andre2006}.  Collisions between cold molecules could allow investigation of the
nature of quantum chemistry~\cite{Krems2008}, and cooled \YbF{} is being
developed to precisely measure the electron electric dipole moment as a means of
testing fundamental theories of physics~\cite{Lim2018}.

The tools developed for creation of cold atoms can be directly applied to
producing cold molecules. A MOT of \esRb{} and \ottCs{} can be used to form
\RbCs{} via an interspecies Feshbach resonance~\cite{PhysRevA.85.032506,
PhysRevA.89.033604} and then transferred into the ro-vibrational ground state by
stimulated Raman adiabatic passage.~\cite{PhysRevLett.113.255301,
RevModPhys.70.1003}.

Alternatively, direct cooling of molecules can be performed, with notable
achievements for diatomic molecules.~\cite{Shuman2010} Beams of \SrF{} and
\CaF{} have been slowed with chirped light~\cite{PhysRevLett.108.103002,
Truppe2017a} and held in a MOT~\cite{Barry2014, Williams2017}. Sub-Doppler
cooling~\cite{Truppe2017} and coherent control of the rotational
state~\cite{Williams2018, Blackmore_2018} have been demonstrated for
\CaF{}. Early experiments with optical tweezers have also been
performed~\cite{Anderegg2019}.

\thesis{Will need to re-visit these last two paras as these experiments will
have progressed. Also can probably sort cites}

Ongoing experiments with \CaF{} aim to achieve an increased phase space density by
sympathetic cooling with \esRb{} in a dipole trap.~ Recent
work has also discovered two vibrational transitions which are highly
insensitive to a magnetic field, making them ideal candidates for a qubit
transition in a molecule chip. \thesis{Need to bring in the 40GHz paper
citation}

\subsection{Structure of this report}

In this document, I will outline the ongoing progress in the design and
implementation of a molecular chip trap. In section~\ref{chiptraps} I will present
the existing research that has been undertaken in the fields of atomic and
molecular chip traps. Next, section~\ref{experiment} will discuss the design
process for our experiment, and the various engineering challenges that we have
had to overcome. Finally I will present the outlook for the remainder of the
project, and potential applications of molecule chip traps.
