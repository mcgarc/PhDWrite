\cm{
  The original proposal for a microfabricated molecule chip trap by Andr\'e et
al.~\cite{Andre2006} suggested that an integrated coplanar waveguide (CPW) could
be used to drive microwave transitions in trapped molecules. In this document,
we will review the feasibility of this proposal for a simple single-layer chip.
}

It has been proposed~\cite{Andre2006} that a microfabricated chip trap could be
used to constrain the movement of cold molecules in proximity to a microwave
resonator, allowing direct manipulation of the molecule's rotational energy
structure. Recent work into cooling molecules below the Doppler
limit~\cite{Truppe2017} suggests that such a ``molecule chip'' is now realisable.
Such a chip would be similar to existing atomic
counterparts~\cite{RevModPhys.79.235, 2011Ac}, forming a robust environment for
investigation of quantum effects. Applications are proposed in a variety of
quantum settings, including quantum information~\cite{Folman2000} and
atomic clocks~\cite{RAMIREZMARTINEZ2011247}. Chips are also candidates for
building hybrid systems~\cite{Nirrengarten2006}.

%A chip trap uses currents (voltages) on the surface of an integrated circuit to
%form a magnetic (electric) field above its surface that can be used to constrain
%the movement of particles in its vicinity. Such devices have been used in
%experiments on atoms, showing promise of providing a compact and robust
%environment for investigating quantum effects\cite{RevModPhys.79.235, 2011Ac}.
%Applications have been proposed in a variety of quantum settings , especially with technologies such as
%superconducting qubits\cite{Wallraff2004} whose features of high controllability
%but short decoherence times balance with the atomic system's properties of low
%controllability and long coherence times.
%
%An analogous device using molecules instead of atoms, the molecule chip, was
%originally proposed in 2006 by Andr\'e et al. \cite{Andre2006}. The molecule
%chip would operate using electric dipole transitions between the rotational
%energy levels of the molecule, as opposed to atom chips, which rely on magnetic
%dipole transitions between hyperfine sub-levels\cite{2011Ac}. This offers an
%increase in the coupling strength of several orders of magnitude, which could
%result in a significant increase in system controllability and coherence times.
%
%Cold molecule technologies have traditionally been much more challenging to
%develop than their equivalents in cold atoms due to the more complex energy
%structure typical of molecules\cite{Andre2006}, but recent developments in cold molecule
%sources\cite{Truppe2017} suggest that the creation of a molecule chip is now
%achievable.
%
%In this report, we will present the progress of an ongoing project to design and
%build a molecule chip based on the proposal of Andr\'e et al.\cite{Andre2006}
%and loaded from  the cold molecule source described by Truppe et
%al.\cite{Truppe2017}. We will begin by presenting a review of Andr\'e et al.'s
%proposal, and then move on to describe chip trapping in more detail. Different
%trap types will be discussed, before we cover our proposal and explore how it
%could be used as a molecule trap.

