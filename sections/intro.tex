\cm{Para on cooling of atoms, along with applications, then introduce need for
atom chips: The ability to laser cool molecules and atoms has opened the
door...}

When Weinstein and Libbrecht first proposed a scheme for microscopic confinement
of atoms using microfabricated magnetic traps~\cite{PhysRevA.52.4004}, it was at
a time when production, ultra-cooling and trapping of atoms was already well
understood and commonly practiced.~\cm{TODO: cites for each of these} Only four
years later, Reichel et al.~\cite{Reichel1999} demonstrated the first atomic
``chip trap,'' a microfabricated device using wires on the surface of a planar
substrate. By the nature of the design and fabrication technique, a chip trap is
more robust and scalable than its macroscopic counterparts~\cm{TODO: cite} and
as such has many applications in quantum technology.

\cm{TODO: find examples and continue talking about applications as separate
paragraph note espeically MW stuff. Need to introduce MOT as a concept. OR maybe
this gives way to opening para}

In 2006, Andre et al.~\cite{Andre2006} published a proposal for a molecular chip
trap. The coupling microwaves to the rotational structure of a diatomic molecule
would be several orders of magnitude stronger than the coupling to hyperfine
transitions in an atom. \cm{Why is this good?}

At the time of proposal, the field of ultracold molecules was still relatively
unexplored, with the first molecular MOT was not created until~\cm{cite molecule
MOT}. Indeed production of cold molecules has shown in many ways to be more
challenging than for atoms, in large part due to the significantly more complex
energy sturcutre they exhibit.~\cm{CITE} Today however we are able to cool
molecules below the Doppler limit~\cite{Truppe2017}, coherently control the
states of such molecules~\cite{PhysRevLett.120.163201} and load molecules into a
tweezer array.~\cite{Anderegg2019}

The groundwork has been laid to produce a molecular chip trap. In the following
subsections I will discuss further the field of ultra-cooling of atoms and
molecules, and the need for a molecule chip. The rest of the report will focus
on chip trapping, the design of our molecule chip and the outlook for the
remainder of the project.

%%\cm{
%%  The original proposal for a microfabricated molecule chip trap by Andr\'e et
%%al.~\cite{Andre2006} suggested that an integrated coplanar waveguide (CPW) could
%%be used to drive microwave transitions in trapped molecules. In this document,
%%we will review the feasibility of this proposal for a simple single-layer chip.
%%}
%%
%%It has been proposed~\cite{Andre2006} that a microfabricated chip trap could be
%%used to constrain the movement of cold molecules in proximity to a microwave
%%resonator, allowing direct manipulation of the molecule's rotational energy
%%structure. Recent work into cooling molecules below the Doppler
%%limit~\cite{Truppe2017} suggests that such a ``molecule chip'' is now realisable.
%%Such a chip would be similar to existing atomic
%%counterparts~\cite{RevModPhys.79.235, 2011Ac}, forming a robust environment for
%%investigation of quantum effects. Applications are proposed in a variety of
%%quantum settings, including quantum information~\cite{Folman2000} and
%%atomic clocks~\cite{RAMIREZMARTINEZ2011247}. Chips are also candidates for
%%building hybrid systems~\cite{Nirrengarten2006}.

%A chip trap uses currents (voltages) on the surface of an integrated circuit to
%form a magnetic (electric) field above its surface that can be used to constrain
%the movement of particles in its vicinity. Such devices have been used in
%experiments on atoms, showing promise of providing a compact and robust
%environment for investigating quantum effects\cite{RevModPhys.79.235, 2011Ac}.
%Applications have been proposed in a variety of quantum settings , especially with technologies such as
%superconducting qubits\cite{Wallraff2004} whose features of high controllability
%but short decoherence times balance with the atomic system's properties of low
%controllability and long coherence times.
%
%An analogous device using molecules instead of atoms, the molecule chip, was
%originally proposed in 2006 by Andr\'e et al. \cite{Andre2006}. The molecule
%chip would operate using electric dipole transitions between the rotational
%energy levels of the molecule, as opposed to atom chips, which rely on magnetic
%dipole transitions between hyperfine sub-levels\cite{2011Ac}. This offers an
%increase in the coupling strength of several orders of magnitude, which could
%result in a significant increase in system controllability and coherence times.
%
%Cold molecule technologies have traditionally been much more challenging to
%develop than their equivalents in cold atoms due to the more complex energy
%structure typical of molecules\cite{Andre2006}, but recent developments in cold molecule
%sources\cite{Truppe2017} suggest that the creation of a molecule chip is now
%achievable.
%
%In this report, we will present the progress of an ongoing project to design and
%build a molecule chip based on the proposal of Andr\'e et al.\cite{Andre2006}
%and loaded from  the cold molecule source described by Truppe et
%al.\cite{Truppe2017}. We will begin by presenting a review of Andr\'e et al.'s
%proposal, and then move on to describe chip trapping in more detail. Different
%trap types will be discussed, before we cover our proposal and explore how it
%could be used as a molecule trap.

\subsection{Laser cooling of molecules}
\label{intro:lasercool}

\cm{

Start with atoms, then move on to molecules

Note that every one \cm{check this is the case} of the above loading methods for
chips begins with an atomic MOT. Whilst there do exist atom~\cite{Dekker2000},
and even molecule chips~\cite{Bethlem2000. Meek2008} that don't rely on a MOT,
it seems that having a molecular MOT will vastly increase the loading schemes
available to use. At the time of Andr\'e's proposal~\cite{Andre2006}, this had
not been achieved, but today molecular MOTs, and even sub-Doppler cooling of
molecules has been achieved, as we will discuss below.

}

\subsection{Structure of this report}

In this document, I will outline the ongoing progress in the design and
implementation of a molecular chip trap. In section \ref{litrev} I will present
the existing research that has been undertaken in the fields of atomic and
molecular chip traps. Next, section \ref{experiment} will discuss the design
process for our experiment, and the various engineering challenges that we have
had to overcome. Finally I will present the outlook for the remainder of the
project, including some \cm{stretch goals??}
