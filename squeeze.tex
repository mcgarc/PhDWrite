\documentclass{article}

\usepackage{siunitx}
\usepackage{amsmath}
\usepackage{braket}
\usepackage{graphicx}
\usepackage{amsfonts}
\usepackage{physics}
\usepackage{tikz}
\usepackage[T1]{fontenc}

\newcommand*\meas[1]{\overline{#1}}

% --- Bibliography ---
% Uses biblatex with biber. Don't include url, doi, isbn because we will use
% magic to insert the url as a hyperlink in the title ourselves

% Ubuntu: requires texlive-bibtex-extra
\usepackage[backend=biber,
						url=false,
						doi=false,
						isbn=false,
						eprint=false,
						giveninits=true,
            sorting=none
						]{biblatex}

% This is the macro to do this. see SE:
% https://tex.stackexchange.com/quACestions/23832/biblatex-make-title-hyperlink-to-doi-url-if-available
% and 
% https://tex.stackexchange.com/questions/48400/ACbiblatex-make-title-hyperlink-to-dois-url-or-isbn
\newbibmacro{string+doi}[1]{%
  \iffieldundef{doi}{#1}{\href{http://dx.doi.org/\thefield{doi}}{#1}}}
\DeclareFieldFormat{title}{\usebibmacro{string+doi}{\mkbibemph{#1}}}
\DeclareFieldFormat[article]{title}{\usebibmacro{string+doi}{\mkbibquote{#1}}}

% Manually remove notes
\AtEveryBibitem{%
  \clearfield{note}%
	}

% Bib file
\addbibresource{bib.bib}

\usepackage[colorlinks]{hyperref}

\title{Squeezed states on a molecule chip}

\author{Cameron McGarry}

\begin{document}

\maketitle

We aim to create squeezed spin states on a molecule chip. I think this is
possible by employing a quantum non-demolition (QND) measurement on a two-level
system in the magnetic trap. In this document I will explain in general how
a QND measurement can be performed by coupling a spin-two system to a cavity,
and how squeezed spin states can be created in such a system. I will then
describe how this could be implemented on a chip trap with calcium monofluoride
(CaF) molecules.

\section{Cavity quantum electrodynamics}

\subsection{Single spin in a cavity}

Consider a spin-half system with ground and excited states $\ket{g}$
and $\ket{e}$ respectively, with a transition frequency $\omega_0$. These
states are eigenstates of the spin operator $s_z = \sigma_z /2$ where
$\sigma_j$ represents the $j^\text{th}$ Pauli matrix.

When such a spin interacts with a mode of an optical cavity of frequency
$\omega_c$, the behaviour is described by the familiar Jaynes-Cummings
Hamiltonian~\cite{}
%
\begin{equation}
  H_\text{JC} = \hbar\omega_c a^\dagger a + \frac{\hbar \omega_0}{2}\sigma_z +
  \frac{\hbar\Omega}{2}(a^\dagger \sigma_- + a\sigma_+)
  \label{JCH}
\end{equation}
%
where $a$ ($a^\dagger$) is the annihilation (creation) operator of the photons,
$\Omega = 2g$ is the Rabi frequency of the interaction, and $\sigma_\pm =
\sigma_x \pm i\sigma_y$ are the raising and lowering operators of spin. The
detuning of the cavity resonance from that of the spin is $\Delta = \omega_0 -
\omega_c$.

Light is coupled out of the cavity with some decay rate $\kappa = \omega/Q$,
with $Q$ being the cavity's quality factor. The spin can decay into other modes
at some decay rate $\gamma$. We assume that we are in the strong coupling
r\'egime, so that such decays can be neglected. In this case $g \gg \kappa,
\gamma$.~\cite{PhysRevA.69.062320}.

We shall also assume that we have some control over $\omega_0$ and hence also
the detuning $\Delta$. Consider the dispersive r\'egime, where  $g\ll|\Delta|$.
Reference~\cite{PhysRevA.69.062320} tells us that we can gain insight into the
dispersive behaviour by applying  a unitary transformation
%
\begin{equation} U = \exp \left[\frac{g}{\Delta}(a\sigma_+ -
a^\dagger\sigma_-)\right] \end{equation}
%
up to second order in $g$.
%\footnote{This is quite an interesting reference as
%they do a lot of what we are interested in but with a Cooper pair box instead
%of CaF molecules (for experiment see reference \cite{Wallraff2004}). Note that
%this transformation to get the Hamiltonian is quite different from the
%adiabatic elimination used to reduce a three level system in
%Reference~\cite{SchleierSmith2011}.}
From this we obtain
%
\begin{equation} H= UH_\text{JC}U^\dagger \approx \hbar \omega_c
  a^\dagger a + \frac{\hbar}{2}\left(\Omega +
  \frac{g^2}{\Delta}\right)\sigma_z + \frac{\hbar
  g^2}{\Delta}\sigma_z a^\dagger a.  \end{equation}
%
These three terms describe the oscillation of light in the cavity, the energy
of the spin (which undergoes the usual AC Stark shift) and the interaction of
the photons with the spin. This last term is what will ultimately allow us to
perform QND measurement, since it will enable exchange of information between
the z-component of the spin with the photons.

\subsection{Spin ensemble Hamiltonian}

We are interested in the case where there are $N$ spins in the
cavity. Assume that the coupling between each spin and the resonator photons is
the same (that is, the coupling is homogeneous), then the ensemble is described
by the Tavis-Cummings Hamiltonian~\cite{Kirton2019}
% CITE https://arxiv.org/pdf/1805.09828.pdf
%
\begin{equation}
  H_\text{TC}=  \hbar \omega_c a^\dagger a + \sum_{i=1}^N\left[
    \frac{\hbar\omega_0}{2}\sigma_z^i +\frac{\hbar\Omega}{2\sqrt{N}}(a^\dagger
    \sigma^i_- + a\sigma^i_+)\right]
\end{equation}
%
% TODO proper equation ref here
in direct analogy to eqn~\ref{JCH}, and using $\sigma_*^i$ to refer to the
$i^\text{th}$ spin's operators.

Now define the spin operator for a single spin
%
\begin{equation}
\mathbf{s}^i = \begin{bmatrix} \sigma^i_x \\ \sigma^i_y \\ \sigma^i_z
\end{bmatrix}
\end{equation}
%
and the collective spin operator
%
\begin{equation}
\mathbf{S} = \sum_{i=1}^N \mathbf{s}^i.
\end{equation}
%
We will usually be interested in the eigenstates of $S_z$ which we denote in
the usual way as $\ket{S, m}$, with $S^2 \ket{S,m} = S(S+1)\ket{S, m}$ and $S_z
\ket{S,m} = m\ket{S, m}$. The raising and lowering operators are $S_\pm = S_x
\pm iS_y = \sum_{i=1}^N\sigma_\pm^i/\sqrt{N}$.
%
Now rewrite the Hamiltonian for the ensemble as
%
\begin{equation}
  H_\text{TC} = \hbar\omega_c a^\dagger a + \hbar\frac{\omega_0}{2}S_z +
  \frac{\hbar\Omega}{2}(a^\dagger S_- + aS_+)
\end{equation}
%
which I assert can now be transformed similarly to the single-spin case by the
operator
%
\begin{equation}
  U_\text{E} = \exp \left[\frac{g}{\Delta}(aS_+ - a^\dagger S_-)\right]\
\end{equation}
%
for new Hamiltonian
%
\begin{equation}
  H_\text{E}= U_\text{E}H_\text{TC}U_\text{E}^\dagger \approx 
    \hbar \omega_c a^\dagger a + 
    \frac{\hbar}{2}\left(\Omega + \frac{g^2}{\Delta}\right)S_z + 
    \frac{\hbar g^2}{\Delta}S_z a^\dagger a.
  \label{eqn:He}
\end{equation}

This assertion is valid since the mathematics of the transformation is exactly
the same as for the single-spin case. The operator $\mathbf{S}$ has the same
commutation relations as the analogous $\mathbf{s}^i$. Once again, the final
term will allow the information transfer for our QND measurements.

\subsection{Spin states of the ensemble}

% TODO Better Bloch sphere cite
Consider the spins represented on the Bloch sphere \cite{PhysRevA.47.3554}.
Here the state of the $i^\text{th}$ spin is written as
%
\begin{equation}
  \ket{\psi}_i = \cos\left(\frac{\theta_i}{2}\right)\ket{e}_i +
  e^{i\phi_i}\sin\left(\frac{\theta_i}{2}\right)\ket{g}_i.
  \label{eqn:blochspin}
\end{equation}
%
The values of $\theta$ and $\phi$ uniquely define the points on a sphere of
radius $1$, with $\ket{e}_i$ and $\ket{g}_i$ at the poles. The spin state of
the ensemble is
%
\begin{equation}
  \ket{\Psi} = \bigotimes_{i=1}^N \ket{\psi}_i.
\end{equation}
Such a state can also be represented on a sphere as the sum of all the
individual spin vectors. In this case the radius of the sphere is variable. For
the purposes of this discussion, we will limit ourselves to considering the
case where the spins are in a coherent spin state (CSS), that is they are all
aligned such that 

We can limit ourselves to considering the case where spins are aligned,
so that for all $i$, $\theta_i = \theta$ and $\phi_i = \phi$, in which case the
sphere has radius $N$. Such a state is known as a coherent spin state
(CSS)~\cite{MA201189, Gazeau2009} and
% TODO proper fig. ref
is illustrated in Fig.~\ref{CSSbloch}. The state can be written similarly to
% TODO proper eqn. ref
Eqn.~\ref{eqn:blochspin} as
%
\begin{equation}
  \ket{\Psi} = \cos\left(\frac{\theta}{2}\right)\ket{e} +
  e^{i\phi}\sin\left(\frac{\theta}{2}\right)\ket{g}
\end{equation}
%
where $\ket{e} = \bigotimes_{i=1}^N\ket{e}_i$ and similarly for $\ket{g}$.
However, it is often more useful to denote the state in terms of the
eigenstates of the angular momentum operator. In particular, the eigenstate of
% TODO Need to introduce the eigenstates above with spin operators?
$S_z$, denoted $\ket{S, m}$ can be used to describe a CSS
%
\begin{equation}
  \ket{\Psi} = \sum_{m=-S}^S a_m \ket{S, m}
\end{equation}
where $a_m$ are the probability amplitudes. Note that $\ket{e} = \ket{S, S}$
and $\ket{g} = \ket{S, -S}$.

\begin{figure}
  \centering
  \includegraphics[width=0.4\textwidth]{figs/squeeze/CSS_Bloch_Cox.png}
  %
  \includegraphics[width=0.5\textwidth]{figs/squeeze/squeezed_Bloch_Cox.png}
  \caption{Left: A coherent spin state as represented on the Bloch sphere.
  Right: Comparison of a CSS and squeezed state as represented on the Bloch
  sphere. Note the reduced uncertainty in the z direction.
  Both reproduced from~\cite{Cox2016}}
  \label{CSSbloch}
\end{figure}

Of particular interest to us will be the CSS aligned along the $x$ axis, so
that $\theta = \pi/2$ and $\phi = 0$. We denote such a state as $\ket{+}$, since
%
\begin{equation}
  \ket{+} = \bigotimes_{i=1}^N \ket{+}_i.
\end{equation}
%
This state can be created by initialising the system with all states in the
ground state $\ket{g}$ and then performing a roataional transormation
% TODO: check pulse!
($\pi/2$-pulse) on each spin.

Direct measurement of any individual spin would yeild $\ket{e}_i$ or
$\ket{g}_i$ with probability $p=1/2$. Hence measuring the eigenvalue of $S_z$
will give a value $m$ with probabilty given by the binomial
theorem~\cite{Gazeau2009}
%
\begin{equation}
  P(m) = \frac{1}{2^N} {N \choose m +N/2}
\end{equation}
%
or for large $N$, the binomial is approximated by the Gaussian
%
\begin{equation}
  P(m) \approx\frac{1}{\sqrt{2\pi \Delta_m^2}} e^{-m^2/(2\Delta_m^2)}.
  \label{eqn:CSSmprob}
\end{equation}
%
where the variance is $\Delta_m^2 = N/4$. We therefore rewrite the probability
amplitudes of $\ket{+}$ so that
%
\begin{equation}
  \ket{+} = \sum_{m=-S}^S \sqrt{P(m)}\ket{S,m}.
\end{equation}

Note that such a state has expectation values $\langle S_z \rangle = 0$,
$\langle S_z^2\rangle = \Delta_m^2$, and the uncertainty in $S_z$ is consistent
with the usual formula $\Delta_m^2 = \langle S_z^2 \rangle - \langle
S_z\rangle^2$. This uncertainty is the projection noise of a CSS, and it
corresponsts to to an uncertainty in $\theta$
%
\footnote{This can be shown geometrically: consider the anlge on the Bloch
sphere between the x-axis and the arrow representing the CSS, call this
$\theta'$ and assume it to be small. Then $\sin
\theta' \approx \tan \theta' = m/S \approx \theta'$. Differentiating both sides
with respect to theta gives $\Delta m / \Delta \theta \approx 1/S$, noting that
$\Delta \theta' = \Delta \theta$.
}
%
\begin{equation}
  \Delta\theta = \frac{1}{\sqrt{N}}
\end{equation}
which is known as the standard quantum limit (SQL). By symmety $\Delta \phi =
\Delta \theta$.

Note that these uncertainties hold only for $\ket{+}$ and other states where
$\theta = \pi$ (states on the equator of the Bloch sphere). For example for
$\ket{e}$ and $\ket{g}$, $\Delta \theta = 0$.~\cite{PhysRevA.47.3554}

%Note that the expected value of $S_z$ is
%%
%\begin{align}
%  \langle S_z \rangle &= \bra{+}S_z\ket{+} \\
%  &= \sum_{m,n} \sqrt{P(m)P(n)}m\bra{n}\ket{m} \\
%  &= \sum_m mP(m) \\
%  &\approx \int_{-\infty}^\infty mP(m)\dd m \\
%  &=0.
%\end{align}
%%
%Here we use the fact that $N$ is large, to approximate the summation as an
%integral, and note the final equality since the integrand is odd. Of course,
%this result is intuitvely obvious.
%
%Similarly we have that
%%
%\begin{align}
%  \langle S_z^2 \rangle &= \bra{+}S_z^2\ket{+} \\
%  &\approx \int_{-\infty}^\infty m^2P(m) \dd m \\
%  &=\frac{N}{4}, 
%\end{align}
%%
%so the uncertainty in $m$ is...

\subsection{Quantum non-demoltion measurement of the spin state}

A useful tool for quantum information processing is the quantum non-demolition
(QND) measurement~\cite{}. Such a measurement allows the insepction of the
state of a system (in this case the spin ensemble) while preserving some
quantum coherence. QND measurements can be deployed not only for state readout,
but for the preparation of non-classical states, as will be explained in the
following sections. The QND method described here has been demonstrated to work
in superconducting qubits~\cite{PhysRevA.69.062320} and expands on the proposal
in \cite{Andre2006}.

Consider the spin ensemble coupled to resonator photons as described above. Take
the resonator photons to be in a canonical coherent state~\cite{Gazeau2009}
%
\begin{equation}
  \ket{\alpha} = e^{-\frac{|\alpha|^2}{2}}\sum_{n=0}^\infty \frac{\alpha^n}{\sqrt{n!}} \ket{n}
\end{equation}
%
with $\alpha\in\mathbb{C}$, and $\ket{n}$ here representing the $n^\text{th}$
Fock state of the light~\cite{agarwal2012}. We say that the pulse of light lasting time
$t=T$ and  write the state of the system at time $t=0$ as
%
\begin{equation}
  \ket{\Psi(0)} = \sum_{m=-S}^S a_m \ket{S, m}\ket{\alpha}.
\end{equation}
%
We take this state to evolve in a rotating reference frame, such that
%
\begin{equation}
  \ket{\Psi(T)} = \exp\left(-iH_\text{int}T/\hbar\right)\ket{\Psi(0)}
\end{equation}
%
where the interaction Hamiltonian is
%
\begin{equation}
  H_\text{int} = \hbar \frac{g^2}{\Delta} S_z a^\dagger a.
\end{equation}
%
After the pulse, we therefore have the state
%
\begin{equation}
  \ket{\Psi(T)} = \sum_{m=-S}^S a_m e^{-i\theta_T S_z
  a^\dagger a} \ket{S, m}\ket{\alpha}
\end{equation}
%
with $\theta_T = \frac{g^2}{\Delta} T$. It is now straightforward to show that
the state of the spin ensemble is entangled with the state of the light field,
by writing $\ket{\alpha}$ in terms of the summation over Fock states
%
\begin{align}
  \ket{\Psi(T)} &= e^{-\frac{|\alpha|^2}{2}}\sum_{m=-S}^S \sum_{n=0}^\infty a_m
   \frac{\alpha^n}{\sqrt{n!}} e^{-i\theta_T S_z a^\dagger a} \ket{S, m} \ket{n}
   \\
  &= e^{-\frac{|\alpha|^2}{2}}\sum_{m=-S}^S \sum_{n=0}^\infty a_m
  \frac{\alpha^n}{\sqrt{n!}} e^{-i\theta_T m n} \ket{S, m} \ket{n} \\
  &= \sum_{m=-S}^S a_m \ket{S, m} \left( e^{-\frac{|\alpha|^2}{2}}
  \sum_{n=0}^\infty \frac{(\alpha e^{-i\theta_T m})^n}{\sqrt{n!}}\ket{n}\right)
  \\
  &= \sum_{m=-S}^S a_m \ket{S, m}\ket{\alpha e^{-i\theta_T m}}.
\end{align}

It is now clear measuring the phase of the light leaving the cavity
(for example with a homodyne detector) will therefore tell us something about
the state of the ensemble. Notably, the information gained tells us only about
the entire state of the ensemble, and not about any individual spins. Therfore
the quantum coherence is conserved throughout this process.
% TODO I'm not convinced by this assertion. CITE? Actually check... 
In the case that the measurement is perfectly accurate, the final state of the
% TODO fix abs
system will be $\ket{S, m}$ with probability $P(m) = |a_m|^2$. This constitutes
a QND measurement of the system. %TODO Check/ cite

\subsection{Homodyne measurement}

As mentioned above, the phase of the light can be determined by a homodyne
measurement ~\cite{agarwal2012}, as illustrated in Fig.
\ref{fig:homodyne}.
% TODO Homodyne figure
The phase-shifted light transmitted through the cavity
($\ket{\Psi_\text{out}}$)
% TODO Need to account for transmission coefficient, mention this above?
is then incident on one port (a) of a
% balanced?
beam splitter. On the other port (b) we have a strong local oscillator (LO) in a
coherent state $\ket{\beta}$. We assume the LO has large amplitude ($|\beta|
\gg |\alpha|$) and has phase $-\varphi$, which we set relative to the phase of
alpha, so that $\arg(\alpha)=0$.

The annihilation operators associated with the
input ports are related to those of the output ports (c and d) by the usual
relation for a balanced beam splitter~\cite{agarwal2012}
%
\begin{equation}
  \begin{pmatrix} c \\ d \end{pmatrix} = \frac{1}{\sqrt{2}}\begin{pmatrix}
    1 & i \\ i & 1 
  \end{pmatrix}  \begin{pmatrix} a \\ b \end{pmatrix}.
\end{equation}
%
The difference in the expected photon numbers arriving at each
detector is
% TODO need to use $a' = \sqrt{\kappa} a$ here. IO formalism for transmitted
% light (I think...)
%
\begin{align}
  \langle c^\dagger c - d^\dagger d\rangle &= i
  \bra{\Psi_\text{out}}\bra{\beta}(a^\dagger b-
  ab^\dagger)\ket{\beta}\ket{\Psi_\text{out}} \\
  & = i|\beta|\kappa_T \bra{\Psi}(a^\dagger e^{i\varphi} - a
  e^{-i\varphi})\ket{\Psi} \\
  &= \sqrt{2}|\beta|\kappa_T\bra{\Psi}(X\sin\varphi +
  Y\cos\varphi)\ket{\Psi}.
\end{align}
%
The last equality introduces the canonical quadratures of the light field,
corresponding to its real and imaginary parts. They are defined by
%
\begin{align}
  X = \frac{a + a^\dagger}{\sqrt{2}} && Y = \frac{a - a^\dagger}{\sqrt{2}}.
\end{align}

Hence measuring the intensity of each output of the beamsplitter can give us
information on the phase of the light. We choose $\varphi = \pi/2$ so that the
% TODO FOrgot how to do i.e. in latex
measurement is of the $Y$ quadrature, i.e. we measure the imaginary part of
$\alpha e^{-i\theta_Tm}$ for a result
%
\begin{equation}
  \langle c^\dagger c - d^\dagger d\rangle = \sqrt{2}\kappa |\alpha||\beta|\left\langle\sin(\frac{g^2 T
  m}{\Delta})\right\rangle.
\end{equation}
%
For a short pulse of light, the interrogation time will be the lifetime of the
photons in the cavity, $T = \kappa^{-1}$. In this r\'egime the small angle
approximation can be made so that
%
\begin{equation}
  x \approx \frac{|\alpha|g^2 T}{\Delta}S_z
  \label{eqn:xapprox}
\end{equation}
%
and therefore
% TODO Can I justify this please? Could just sub in suitable values...
%
\begin{equation}
  \langle c^\dagger c - d^\dagger d\rangle = \sqrt{2}|\alpha||\beta|
  \frac{g^2}{\Delta}\langle m\rangle.
  \label{eqn:homomeas}
\end{equation}

To resolve $m$ perfectly, the uncertainty in the measurement must be
$\Delta_\eta \ll \sqrt{2}|\alpha||\beta|g^2/\Delta$. 
%
% TODO Analyse this result
%
% Make this sentence fit with other things
The effects of having larger uncertainty are discussed in the next section.

% TODO
\emph{
Need to consider dephasing of spins, some spins collapsing on measurement
(shrinking the Bloch sphere) and the uncertainty associated with the
measurement.
}

\section{Entangled states in the ensemble}

% TODO: Spiel

\subsection{Spin squeezing}

% TODO: Better description of squeezing here

For the purposes of quantum meterology, it can be useful to reduce the
uncertainty in one of $\theta$ or $\phi$...
% TODO: cite!

\subsection{QND measurement to squeeze the ensemble}

% TODO: Replace this example with a more abstract discussion of the squeezing
% process, with more details on the measurement of the light and how this can
% could create a max. entangled state with precise enough measurement, less
% precision -> less squeezing

% Consider the above case of QND measurement, with Psi(0) = +, in which case
% P(m) is some Gaussian (for large N)...

% TODO Need to have introudced and explained what I mean by perfect QND
% measurment above.
Consider a perfect QND measurement peformed on the state $\ket{\Psi(0)} =
\ket{+}$. The end result will be a state
%
\begin{equation}
  \ket{\Psi_{\meas{m}}} = \ket{S, \meas{m}}
\end{equation}
%
where $\meas{m}$ is the eigenvalue of $S_z$ that was the outcome of our
measurement, which was obtainted with probability $P(\meas{m})$.
$\ket{\Psi_{\meas{m}}}$ is a Dicke state~\cite{Baertschi2019}, the sum of all
states with $N_e = \meas{m} + N/2$ spins in the excited state, and $N_g = N -
N_e$ spins in the ground state.  We can write this in terms of the set $\Pi$ of
combinations of $N_e$ numbers chosen from $[1,N]\cap\mathbb{Z}$, which we
%
\begin{equation}
  \ket{\Psi_{\meas{m}}} = {N \choose N_e}^{-\frac{1}{2}}\sum_{\pi \in \Pi}
  \bigotimes_{i=1}^{N_e} \sigma_x^{\pi_i} \ket{g}
\end{equation}

To borrow an example from
reference~\cite{Cox2016}, consider the case with $N=4$, $\meas{m}=0$ so we have
$N_e = N_g = 2$. Then the state is
%
\begin{equation}
  \ket{\Psi_0} = \frac{1}{\sqrt{6}}(\ket{eegg} + \ket{egeg} + \ket{egge} +
  \ket{geeg} + \ket{gege} + \ket{ggee})
\end{equation}
%
which is a maximally entangled state, represented on the Bloch sphere as a ring
around the equator.
%
Clearly, such a state would have zero uncertainty in $S_z$, it is therefore
%TODO emph?
\emph{maximally squeezed}
and also has infinite uncertainty in $S_x$.

With perfect measurement, it is therfore possible to generate a maximally
squeezed state, simply by peforming QND measurement on  $\ket{+}$. The state
that is produced will be entangled (unless $\meas{m} = \pm S$). We know the
state we have produced by looking at the value of $\meas{m}$, but have no
control over this outcome, for this reason the squeezing is referred to as
% TODO when do I emph new words?
\emph{conditional}.
%
We expect for large $N$ that $|\meas{m}|/N \ll 1$, so it is likely that the
state will have $m$ close to zero.

In the event that the homodyne measurement is not perfect, what is the state
that we are in? If the uncertainty assoicated with the measurement is smaller
than the standard quantum limit, then we expect that we are somewhere between
the maximally squeezed state and the CSS, so that some squeezing has
occured.

The nature of the resulting state is described by the
projector onto the x states:~\cite{Cox2016, Zhang2019}
%
\begin{equation}
  \Pi_\text{light}(\meas{x}) = \frac{1}{\sqrt{2\pi\Delta_\eta^2}}\exp\left[
    -\frac{(x-\meas{x})^2}{2\Delta_\eta^2}\right].
\end{equation}
%
Due to the entanglement of the light with the spins, this is equivalent to
projecting onto the eigenstate of $S_z$, which will act directly on the spin
state. Maintaing the approximation of
% TODO eqn ref
eqn.~\ref{eqn:xapprox}
we have the projector
%
\begin{equation}
  \Pi(\meas{m}) = \frac{1}{\sqrt{2\pi\Delta_{S_z}^2}}\exp\left[
    -\frac{(S_z-\meas{m})^2}{2\Delta_{S_z}^2}\right].
\end{equation}
%
where
%
\begin{equation}
  \Delta_{S_z}^2 = \frac{\Delta_\eta^2}{(|\alpha|\theta_T)^2}.
\end{equation}

Following the above example for a perfect measurement, we again QND measurement
of the state $\ket{+}$. In this case the probability of measuing $m=\meas{m}$
from $\ket{\Psi(T)}$ is given by~\cite{}
%
\begin{align}
  P(m=\meas{m}) &= \bra{+}\Pi(\meas{m})\ket{+}\\
  & = \sum_{m'=-S}^S \sum_{m=-S}^S \sqrt{P(m')P(m)}\bra{S,
  m'}\Pi(\meas{m})\ket{S, m} \\
  & = \sum_{m'=-S}^S \sum_{m=-S}^S \sqrt{P(m')P(m)}
  \frac{1}{\sqrt{2\pi\Delta_{S_z}^2}}\exp\left[-\frac{(m-\meas{m})^2}{2\Delta_{S_z}^2}\right] 
  \bra{S, m'}\ket{S, m} \\
  &= \sum_{m=-S}^{S} P(m) 
  \frac{1}{\sqrt{2\pi\Delta_{S_z}^2}}\exp\left[-\frac{(m-\meas{m})^2}{2\Delta_{S_z}^2}\right] 
\end{align}
%
taking the large $N$ limit, and approximating the summation as an integral,
this last equality becomes a convolution of two Gaussians, so that
%
\begin{equation}
  P(m=\meas{m}) = \frac{1}{\sqrt{2\pi(\Delta_m^2 + \Delta_{S_z}^2)}}\exp\left[
    -\frac{\meas{m}^2}{2(\Delta_m^2 + \Delta_{S_z}^2)}\right].
\end{equation}
%
This becomes eqn.~\ref{eqn:CSSmprob} in the limit that $\Delta_{S_z} \to 0$.
% TODO Extend to not maximally squeezed states

Finally, we can calulate the state after the measurement, which is given
by~\cite{}
%
\begin{equation}
  \ket{\Psi_\text{final}} =
  \frac{\Pi(\meas{m})\ket{+}}{\bra{+}\Pi(\meas{m})\ket{+}}.
\end{equation}
%
We again take the approximation that we are in the limit of large $N$, and also
that $\Delta_{S_z}^2 \ll \Delta_m^2$,
%
\begin{equation}
  \ket{\Psi_\text{final}} = \sum_{m=-S}^S\frac{1}{\sqrt{2\pi\Delta_{S_z}^2}}
  \exp\left[-\frac{(m - \meas{m})^2}{2\Delta_{S_z}^2}\right] \ket{S,m}
\end{equation}
%
% TODO Better fig. and refernce
which is a squeezed state, analogous to that shown in fig~\ref{CSSbloch}.

\emph{Should probably talk about the expectation values, definitely want to
talk about how many spins we can lose to collapse and what we can expect
squeezing paramter to be.}

\subsection{Cat states}

% TODO

\emph{Measuring the other quadrature creates a cat state because we get
$\cos(x) \approx 1 - x^2$ not $\sin(x)\approx x$, which means the sign of
measured $\theta$ is hidden.}

% TODO: Re-gear the below section to be more about the experiment, motivating
% what we will need to do on the chip
% Roll it in with the next one too...



%\section{Performing QND measurements}
%
%Having established that a QND measurement will allow us to create entangled,
%and in particular, squeezed states we now consider how this can be performed
%for a spin ensemble in a cavity. Consider again the Hamiltonian of our system
%given in eqn~\ref{eqn:He}. We have already pointed out that the
%third term will allow for transfer of information about the $z$ spin into the
%cavity light, but notice also that every term commutes with $S_z$. This will
%ensure that performing the measurement will not change the $S_z$ state, but
%note that $S_x$ and $S_y$ are fair game~\cite{SchleierSmith2011}.
%
%Rewriting
%%
%\begin{equation}
%  H_\text{E}= 
%    \hbar \left(\omega_c + \frac{g^2}{\Delta}S_z\right)a^\dagger a + 
%    \frac{\hbar}{2}\left(\Omega + \frac{g^2}{\Delta}\right)S_z
%\end{equation}
%%
%it is now clear that the transition frequency depends on the state of the
%$S_z$. The shift in frequency can be interpreted in a change of the cavity's
%refractive index. Hence the transmission peak of the cavity will be shifted
%from $\omega_c$ so that
%%
%\begin{equation}
%  \omega_c \rightarrow \omega_c + \frac{\hbar g}{\Delta}(N_e - N_g)
%\end{equation}
%%
%where $N_e$ ($N_g$) is the number of spins in the excited (ground) state. We
%can therefore perform a QND measurement for $\delta N = N_e - N_g$ by probing
%the cavity.
%
%\begin{figure}
%  \centering
%  \includegraphics[width=0.5\textwidth]{figs/squeeze/transmission_Blais.png}
%  \caption{Transmission peak shift for a cavity with resonant frequency
%  $\omega_c$ in excited ($\ket{\uparrow}$) and ground ($\ket{\downarrow}$)
%  states.  Reproduced from~\cite{PhysRevA.69.062320}}
%  \label{transmission}
%\end{figure}
%
%It can be shown that this shift is dependent on the number of photons in the
%cavity, in particular the shift begins to decrease above some critical number
%of photons $n_\text{crit} = \langle a^\dagger a\rangle_\text{crit} = \Delta^2/4g^2$.
%Hence the probe power should not exceed $P_\text{max} = n_\text{crit}\hbar
%\omega_c \kappa$ \cite{PhysRevA.69.062320}.
%
%The linewidth of the transition peak is $\kappa$, so to resolve this shift we
%require $g^2/ \Delta \gg \kappa$ for the shift of the frequency to be larger
%than the transmission linewidth and hence resolvable (see
%Fig.~\ref{transmission}). Alternatively, when the frequency of the probe is
%$\omega_c$, the information on $\delta N$ is encoded in the phase of the light,
%producing a shift $N\phi$ where
%%
%\begin{equation}
%  \phi = \arctan \left(\frac{2g^2}{\kappa \Delta}\right).
%\end{equation}
%%
%This can be used to perform QND measurements in the $g^2/ \Delta \ll
%\kappa$ limit~\cite{PhysRevA.69.062320}.\footnote{I'm not sure how this is
%compatible with the idea that we work in a dispersive limit ($\Delta \gg g$),
%but I think we are more interested in the transmission magnitude case anyway.}
%% But why not just "outsside the g^2 D \gg kappa" limit?
%
%%The power of the probe must be low, such that we remain in the disperisive
%%r\'egime, i.e.\ $\braket{a^\dagger a} < \Delta^2/4g^2 = n_\text{crit}$.
%
%It is helpful to consider the 

\section{Implementation on CaF chip}

For our two-level system in CaF choose the states $\ket{g} = \ket{N=0, F=1,
m_F=1}$ and $\ket{e} = \ket{1, 2, 2}$ for their long lifetime in a magnetic
trap. The transition has frequency $\omega_0/2\pi = \SI{20.5}{\giga\hertz}$
\cite{Williams2018}. The idea is to use a high $Q$ microwave cavity on the
chip, with the molecules magnetically trapped near to an antinode to maximise
the coupling. This is similar to the proposal made by Andr\'e et
al.~\cite{Andre2006}, who also discuss the idea of using a cavity for QND
measurement, although not specifically for the creation of entangled states.

The Rabi frequency is $\Omega = 2g = \mathbf{d}.\mathbf{E}/\hbar$ where $\mathbf{d}$
is the dipole moment of the transition, and $\mathbf{E}$ is the electric field
strength. However, this local electric field is a function of the geometry. We
must calculate it by calculating the field from our particular resonator, and
the molecule's relative position to it. Since this calculation is a little
involved, let's just for now take the conservative coupling estimate from
reference~\cite{Andre2006}, which is for a CaBr positioned
\SI{1}{\micro\meter} from the resonator: $g/2\pi = \SI{40}{\kilo\hertz}$.

A typical superconducting microwave cavity will have a quality factor
$Q\sim10^6$. We can choose the cavity resonance to be similar to the
transition)frequency, so that the cavity loss rate is $\kappa/2\pi =
\omega_0/(2\pi Q) \sim \SI{1}{\kilo\hertz}$. 

The procedure to create entangled states on the chip would be as follows:

\begin{enumerate}
  \item Load CaF molecules in $\ket{g}$ into the magnetic trap close to the
    resonator
  \item Use external microwaves (from e.g. a horn) to transfer the molecules
    into CSS $(\ket{e} + \ket{g})/\sqrt{2}$
  \item Bias the trap so as to achieve large molecule-cavity detuning
  \item Weakly probe the cavity to identify the transmission peak
  \item Determine $N$, which tells us something about the nature of the
    squeezed state that has been produced
  \item Un-bias the qubit for precision quantum measurement or other operation
\end{enumerate}

\section{Conclusion}

This document describes how spin squeezed states could be created in CaF
molecules coupled to a microwave resonator on a chip. This is mainly based on
the idea of how QND measurements can be performed in this architecture, as was
presented by Andr\'e et al.~\cite{Andre2006}, but there remain a number of
unanswered questions. For example, I have not addressed losses such as those
that can occur from the enhanced stimulated emission into the cavity mode.
There is also the question of the number of molecules that we would be required
to trap to reliably produce such states.

It would also be interesting to consider the nature of the spin states that
could be produced. What is the maximum squeezing that could occur, and how
exactly would the squeezed state relate to the measured value $M$?

The most pertinent question in my mind is how can one control the detuning
$\Delta$? In our case we have chosen magnetically insensitive states for our
qubit, which is useful for long lifetimes, but for the purpose of performing
QND measurements we will have to detune, either by magnetically biasing the
trap beyond the linear Zeeman shift, or applying some other perturbation
such as an electric field.

%\section{Further thoughts: collision squeezing}
%
%In addition to squeezing by QND measurement, atomic spin squeeze states have
%been generated due to inter-atomic collisions \cite{}. These collisions
%introduce a non-linear Hamiltonian term $H=\chi S_z^2$, which leads to a
%shearing of the coherent state around the Bloch sphere around the $S_z$ state
%\cite{}.
%
%With our knowledge of collisions between atoms and molecules (or molecules and
%molecules) might it be possible to generate squeezed states in atom-molecule
%clouds, or very dense clouds of molecules? The latter could perhaps be explored
%on the new CaF Bose-Einstein condensate apparatus.

\printbibliography

\end{document}
