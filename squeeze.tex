\documentclass{article}

\usepackage{siunitx}
\usepackage{amsmath}
\usepackage{braket}
\usepackage{graphicx}
\usepackage{amsfonts}

% --- Bibliography ---
% Uses biblatex with biber. Don't include url, doi, isbn because we will use
% magic to insert the url as a hyperlink in the title ourselves

% Ubuntu: requires texlive-bibtex-extra
\usepackage[backend=biber,
						url=false,
						doi=false,
						isbn=false,
						eprint=false,
						giveninits=true,
            sorting=none
						]{biblatex}

% This is the macro to do this. see SE:
% https://tex.stackexchange.com/quACestions/23832/biblatex-make-title-hyperlink-to-doi-url-if-available
% and 
% https://tex.stackexchange.com/questions/48400/ACbiblatex-make-title-hyperlink-to-dois-url-or-isbn
\newbibmacro{string+doi}[1]{%
  \iffieldundef{doi}{#1}{\href{http://dx.doi.org/\thefield{doi}}{#1}}}
\DeclareFieldFormat{title}{\usebibmacro{string+doi}{\mkbibemph{#1}}}
\DeclareFieldFormat[article]{title}{\usebibmacro{string+doi}{\mkbibquote{#1}}}

% Manually remove notes
\AtEveryBibitem{%
  \clearfield{note}%
	}

% Bib file
\addbibresource{bib.bib}

\usepackage[colorlinks]{hyperref}

\title{Squeezed states on a molecule chip}

\author{Cameron McGarry}

\begin{document}

\maketitle

We aim to create squeezed spin states on a molecule chip. I think this is
possible by employing a quantum non-demolition (QND) measurement on a two-level
system in the magnetic trap. In this document I will explain in general how
a QND measurement can be performed by coupling a spin-two system to a cavity,
and how squeezed spin states can be created in such a system. I will then
describe how this could be implemented on a chip trap with calcium monofluoride
(CaF) molecules.

\section{Cavity quantum electrodynamics}

\subsection{Single spin in a cavity}

Consider a spin-half system with ground and excited states $\ket{g}$
and $\ket{e}$ respectively, with a transition frequency $\omega_0$. These
states are eigenstates of the spin operator $s_z = \sigma_z /2$ where
$\sigma_j$ represents the $j^\text{th}$ Pauli matrix.

When such a spin interacts with a mode of an optical cavity of frequency
$\omega_c$, the behaviour is described by the familiar Jaynes-Cummings
Hamiltonian~\cite{}
%
\begin{equation}
  H_\text{JC} = \hbar\omega_c a^\dagger a + \frac{\hbar \omega_0}{2}\sigma_z +
  \frac{\hbar\Omega}{2}(a^\dagger \sigma_- + a\sigma_+)
  \label{JCH}
\end{equation}
%
where $a$ ($a^\dagger$) is the annihilation (creation) operator of the photons,
$\Omega = 2g$ is the Rabi frequency of the interaction, and $\sigma_\pm =
\sigma_x \pm i\sigma_y$ are the raising and lowering operators of spin. The
detuning of the cavity resonance from that of the spin is $\Delta = \omega_0 -
\omega_c$.

Light is coupled out of the cavity with some decay rate $\kappa = \omega/Q$,
with $Q$ being the cavity's quality factor. The spin can decay into other modes
at some decay rate $\gamma$. We assume that we are in the strong coupling
r\'egime, so that such decays can be neglected. In this case $g \gg \kappa,
\gamma$.~\cite{PhysRevA.69.062320}.

We shall also assume that we have some control over $\omega_0$ and hence also
the detuning $\Delta$. Consider the dispersive r\'egime, where  $g\ll|\Delta|$.
Reference~\cite{PhysRevA.69.062320} tells us that we can gain insight into the
dispersive behaviour by applying  a unitary transformation
%
\begin{equation} U = \exp \left[\frac{g}{\Delta}(a\sigma_+ -
a^\dagger\sigma_-)\right] \end{equation}
%
up to second order in $g$.\footnote{This is quite an interesting reference as
they do a lot of what we are interested in but with a Cooper pair box instead
of CaF molecules (for experiment see reference \cite{Wallraff2004}). Note that
this transformation to get the Hamiltonian is quite different from the
adiabatic elimination used to reduce a three level system in
Reference~\cite{SchleierSmith2011}.} From this we obtain
%
\begin{equation} H= UH_\text{JC}U^\dagger \approx \hbar \omega_c
  a^\dagger a + \frac{\hbar}{2}\left(\Omega +
  \frac{g^2}{\Delta}\right)\sigma_z + \frac{\hbar
  g^2}{\Delta}\sigma_z a^\dagger a.  \end{equation}
%
These three terms describe the oscillation of light in the cavity, the energy
of the spin (which undergoes the usual AC Stark shift) and the interaction of
the photons with the spin. This last term is what will ultimately allow us to
perform QND measurement, since it will enable exchange of information between
the z-component of the spin with the photons.

\subsection{Spin ensemble Hamiltonian}

We are interested in the case where there are $N$ spins in the
cavity. Assume that the coupling between each spin and the resonator photons is
the same (that is, the coupling is homogeneous), then the ensemble is described
by the Tavis-Cummings Hamiltonian~\cite{Kirton2019}
% CITE https://arxiv.org/pdf/1805.09828.pdf
%
\begin{equation}
  H_\text{TC}=  \hbar \omega_c a^\dagger a + \sum_{i=1}^N\left[
    \frac{\hbar\omega_0}{2}\sigma_z^i +\frac{\hbar\Omega}{2\sqrt{N}}(a^\dagger
    \sigma^i_- + a\sigma^i_+)\right]
\end{equation}
%
% TODO proper equation ref here
in direct analogy to eqn~\ref{JCH}, and using $\sigma_*^i$ to refer to the
$i^\text{th}$ spin's operators.

Now define the spin operator for a single spin
%
\begin{equation}
\mathbf{s}^i = \begin{bmatrix} \sigma^i_x \\ \sigma^i_y \\ \sigma^i_z
\end{bmatrix}
\end{equation}
%
and the collective spin operator
%
\begin{equation}
\mathbf{S} = \sum_{i=1}^N \mathbf{s}^i
\end{equation}
%
with raising and lowering operators $S_\pm = S_x \pm iS_y =
\sum_{i=1}^N\sigma_\pm^i/\sqrt{N}$.
%
We can now re-write the Hamiltonian for the ensemble as
%
\begin{equation}
  H_\text{TC} = \hbar\omega_c a^\dagger a + \hbar\frac{\omega_0}{2}S_z +
  \frac{\hbar\Omega}{2}(a^\dagger S_- + aS_+)
\end{equation}
%
which I assert can now be transformed similarly to the single-spin case by the
operator
%
\begin{equation}
  U_\text{E} = \exp \left[\frac{g}{\Delta}(aS_+ - a^\dagger S_-)\right]\
\end{equation}
%
for new Hamiltonian
%
\begin{equation}
  H_\text{E}= U_\text{E}H_\text{TC}U_\text{E}^\dagger \approx 
    \hbar \omega_c a^\dagger a + 
    \frac{\hbar}{2}\left(\Omega + \frac{g^2}{\Delta}\right)S_z + 
    \frac{\hbar g^2}{\Delta}S_z a^\dagger a.
  \label{eqn:He}
\end{equation}

This assertion is valid since the mathematics of the transformation is exactly
the same as for the single-spin case. The operator $\mathbf{S}$ has the same
commutation relations as the analogous $\mathbf{s}^i$. Once again, the final
term will allow the information transfer for our QND measurements.

\subsection{Coherent spin states}

Consider the spins represented on the Bloch sphere \cite{Cox2016}. Here the
state of the $i^\text{th}$ spin is written as
%
\begin{equation}
  \ket{\psi}_i = \cos\left(\frac{\theta_i}{2}\right)\ket{e}_i +
  e^{i\phi_i}\sin\left(\frac{\theta_i}{2}\right)\ket{g}_i.
\end{equation}
%
The values of $\theta$ and $\phi$ uniquely define the points on a sphere of
radius $1$, with $\ket{e}_i$ and $\ket{g}_i$ at the poles. The spin state of
the entire system is then
%
\begin{equation}
  \ket{\Psi} = \bigotimes_{i=1}^N \ket{\psi}_i.
\end{equation}
Such a state can also be
represented on a sphere, as the sum of each vector $\ket{\psi_i}$ as shown in 
% TODO proper fig. ref
Fig.~\ref{CSSbloch}. In the case shown, all the spins are aligned ($\theta_i
= \theta_j$ and $\phi_i = \phi_j$) we have a coherent spin state (CSS) and the
sphere has radius $N$.

\begin{figure}
  \centering
  \includegraphics[width=0.4\textwidth]{figs/squeeze/CSS_Bloch_Cox.png}
  %
  \includegraphics[width=0.5\textwidth]{figs/squeeze/squeezed_Bloch_Cox.png}
  \caption{Left: A coherent spin state as represented on the Bloch sphere.
  Right: Comparison of a CSS and squeezed state as represented on the Bloch
  sphere. Note the reduced uncertainty in the z direction.
  Both reproduced from~\cite{Cox2016}}
  \label{CSSbloch}
\end{figure}

% TODO: CSS written in z-basis, and especially the probability for the + state

% TODO: Repalce this para with derivation of SQL for + state (and find out how
% J_z variance corresponds to theta variance

For a CSS, there is some inherent uncertainty that arises in $\theta$ and
$\phi$ from the projection of the measurement of each individual atom. When
$\theta=\pi$ is this uncertainty is the standard quantum limit (SQL) and
~\cite{Cox2016}
% What is this uncertainty in cases off the equator?
\begin{equation}
  \Delta \theta = \Delta \phi = \frac{1}{\sqrt{N}}.
\end{equation}

\subsection{Quantum non-demoltion measurement of the spin state}

% TODO: Entangling maths, right up to Homodyne measurement (and the
% preliminaries of that)

% Consider some CSS, which we entangle with coherent light (definte this)
% writing the state as...

A useful tool for quantum information processing is the quantum non-demolition
(QND) measurement~\cite{}. Such a measurement allows the insepction of the
state of a system (in this case the spin ensemble) while preserving some
quantum coherence. QND measurements can be deployed not only for state readout,
but for the preparation of non-classical states, as will be explained in the
following sections. The QND method described here has been demonstrated to work
in superconducting qubits~\cite{PhysRevA.69.062320} and expands on the proposal
in \cite{Andre2006}.

Consider the spin ensemble coupled to resonator photons as described above. Take
the resonator photons to be in a canonical coherent state
%
\begin{equation}
  \ket{\alpha} = e^{-\frac{|\alpha|^2}{2}}\sum_{n=0}^\infty \frac{\alpha^n}{\sqrt{n!}} \ket{n}
\end{equation}
%
with $\alpha\in\mathbb{C}$, and the pulse of light lasting time $t=T$. We write
the state of the system at time $t=0$ as
%
\begin{equation}
  \ket{\Psi(0)} = \sum_{m=-S}^S a_m \ket{\alpha} \ket{S, m}.
\end{equation}
%
We take this state to evolve in a rotating reference frame, such that
%
\begin{equation}
  \ket{\Psi(T)} = \exp\left(-iH_\text{int}T/\hbar\right)\ket{\Psi(0)}
\end{equation}
%
where the interaction Hamiltonian is
%
\begin{equation}
  H_\text{int} = \frac{g^2}{\Delta} S_z a^\dagger a.
\end{equation}
%
After the pulse, we therefore have the state
%
\begin{equation}
  \ket{\Psi(T)} = \sum_{m=-S}^S a_m e^{-i\theta_T S_z
  a^\dagger a} \ket{\alpha}\ket{S, m}
\end{equation}
%
with $\theta_T = \frac{g^2}{\Delta} T$. It is now straightforward to show that
the state of the spin ensemble is entangled with the state of the light field,
%
\begin{align}
  \ket{\Psi(T)} &= e^{-\frac{|\alpha|^2}{2}}\sum_{m=-S}^S \sum_{n=0}^\infty a_m
  \frac{\alpha^n}{\sqrt{n!}} e^{-i\theta_T S_z a^\dagger a} \ket{n}
  \ket{S, m} \\
  &= e^{-\frac{|\alpha|^2}{2}}\sum_{m=-S}^S \sum_{n=0}^\infty a_m
  \frac{\alpha^n}{\sqrt{n!}} e^{-i\theta_T m n} \ket{n}
  \ket{S, m} \\
  &= \sum_{m=-S}^S a_m \left( e^{-\frac{|\alpha|^2}{2}} \sum_{n=0}^\infty 
  \frac{(\alpha e^{-i\theta_T m})^n}{\sqrt{n!}}\ket{n}\right) 
  \ket{S, m} \\
  &= \sum_{m=-S}^S a_m \ket{\alpha
  e^{-i\theta_T m}}\ket{S, m}.
\end{align}

It is now clear measuring the phase of the light leaving the cavity
(for example with a heterodyne detector) will therefore tell us something about
the state of the ensemble. Notably, the information gained tells us only about
the entire state of the ensemble, and not about any individual spins. Therfore
the quantum coherence is conserved throughout this process.
% TODO I'm not convinced by this assertion. CITE? Actually check... 
In the case that the measurement is perfectly accurate, the final state of the
% TODO fix abs
system will be $\ket{S, m}$ with probability $P(m) = |a_m|^2$. This constitutes
a QND measurement of the system. %TODO Check/ cite

\section{Entangled states in the ensemble}

% TODO: Spiel

\subsection{Spin squeezing}

% TODO: Better description of squeezing here

For the purposes of quantum meterology, it can be useful to reduce the
uncertainty in one of these directions.
% TODO: cite!

\subsection{QND measurement to squeeze the ensemble}

% TODO: Replace this example with a more abstract discussion of the squeezing
% process, with more details on the measurement of the light and how this can
% could create a max. entangled state with precise enough measurement, less
% precision -> less squeezing

% Consider the above case of QND measurement, with Psi(0) = +, in which case
% P(m) is some Gaussian (for large N)...


To borrow an example from
reference~\cite{Cox2016}, consider the case with N=4, and we have prepared the
CSS
%
\begin{equation}
  \ket{\Psi} = \bigotimes_{i=1}^4 \frac{\ket{e}_i + \ket{g}_i}{\sqrt{2}}.
\end{equation}
%
If we were to perform a QND measurement, of the number of spins in each state
(but gaining no knowledge of which spin was in which state) then we can produce
an entangled state. For example if the measurement shows that there are two
spins in each state we have
%
\begin{equation}
\ket{\psi} = \frac{1}{\sqrt{6}}(\ket{eegg} + \ket{egeg} + \ket{egge} +
\ket{geeg} + \ket{gege} + \ket{ggee})
\end{equation}
which is a maximally entangled state. For any result other than all spins in
$\ket{e}$ or $\ket{g}$ the result is entangled. This particular state is a
Dicke state, and would be represented by a ring around the equator of the Bloch
sphere. However, creation of such a state is not normally achievable (as this
would require highly accurate measurement of the occupation of the states) and
instead a squeezed state with reduced uncertainty in one direction is produced,
as shown in Fig.~\ref{CSSbloch}.
%TODO Proper fig. ref


In the case of large $N$ the expected number of spins to measure in each state
is $N/2$, but this is a probabilistic process, and as such the state resulting
from the QND measurement is dependent on the measured value.
% Cox gives some probabilistic idea of what theta normally is


% Cox also gives an equation for the proejction operator of the measurement,
% but I think I should come back to that later.
The state following the measurement are ``a bit difficult to write down''
analytically~\cite{Cox2016}, but I will discuss the reduction in uncertainty
further in the context of a potential experiment in the next section.

\subsection{Cat states}

% TODO: What??

% TODO: Re-gear this section to be more about the experiment, motivating what
% we will need to do on the chip
\section{Performing QND measurements}

Having established that a QND measurement will allow us to create entangled,
and in particular, squeezed states we now consider how this can be performed
for a spin ensemble in a cavity. Consider again the Hamiltonian of our system
given in eqn~\ref{eqn:He}. We have already pointed out that the
third term will allow for transfer of information about the $z$ spin into the
cavity light, but notice also that every term commutes with $S_z$. This will
ensure that performing the measurement will not change the $S_z$ state, but
note that $S_x$ and $S_y$ are fair game~\cite{SchleierSmith2011}.

Rewriting
%
\begin{equation}
  H_\text{E}= 
    \hbar \left(\omega_c + \frac{g^2}{\Delta}S_z\right)a^\dagger a + 
    \frac{\hbar}{2}\left(\Omega + \frac{g^2}{\Delta}\right)S_z
\end{equation}
%
it is now clear that the transition frequency depends on the state of the
$S_z$. The shift in frequency can be interpreted in a change of the cavity's
refractive index. Hence the transmission peak of the cavity will be shifted
from $\omega_c$ so that
%
\begin{equation}
  \omega_c \rightarrow \omega_c + \frac{\hbar g}{\Delta}(N_e - N_g)
\end{equation}
%
where $N_e$ ($N_g$) is the number of spins in the excited (ground) state. We
can therefore perform a QND measurement for $\delta N = N_e - N_g$ by probing
the cavity.

\begin{figure}
  \centering
  \includegraphics[width=0.5\textwidth]{figs/squeeze/transmission_Blais.png}
  \caption{Transmission peak shift for a cavity with resonant frequency
  $\omega_c$ in excited ($\ket{\uparrow}$) and ground ($\ket{\downarrow}$)
  states.  Reproduced from~\cite{PhysRevA.69.062320}}
  \label{transmission}
\end{figure}

It can be shown that this shift is dependent on the number of photons in the
cavity, in particular the shift begins to decrease above some critical number
ofo photons $n_\text{crit} = \braket{a^\dagger a}_\text{crit} = \Delta^2/4g^2$.
Hence the probe power should not exceed $P_\text{max} = n_\text{crit}\hbar
\omega_c \kappa$ \cite{PhysRevA.69.062320}.

The linewidth of the transition peak is $\kappa$, so to resolve this shift we
require $g^2/ \Delta \gg \kappa$ for the shift of the frequency to be larger
than the transmission linewidth and hence resolvable (see
Fig.~\ref{transmission}). Alternatively, when the frequency of the probe is
$\omega_c$, the information on $\delta N$ is encoded in the phase of the light,
producing a shift $N\phi$ where
%
\begin{equation}
  \phi = \arctan \left(\frac{2g^2}{\kappa \Delta}\right).
\end{equation}
%
This can be used to perform QND measurements in the $g^2/ \Delta \ll
\kappa$ limit~\cite{PhysRevA.69.062320}.\footnote{I'm not sure how this is
compatible with the idea that we work in a dispersive limit ($\Delta \gg g$),
but I think we are more interested in the transmission magnitude case anyway.}
% But why not just "outsside the g^2 D \gg kappa" limit?

%The power of the probe must be low, such that we remain in the disperisive
%r\'egime, i.e.\ $\braket{a^\dagger a} < \Delta^2/4g^2 = n_\text{crit}$.

It is helpful to consider the 

\section{Implementation on CaF chip}

For our two-level system in CaF choose the states $\ket{g} = \ket{N=0, F=1,
m_F=1}$ and $\ket{e} = \ket{1, 2, 2}$ for their long lifetime in a magnetic
trap. The transition has frequency $\omega_0/2\pi = \SI{20.5}{\giga\hertz}$
\cite{Williams2018}. The idea is to use a high $Q$ microwave cavity on the
chip, with the molecules magnetically trapped near to an antinode to maximise
the coupling. This is similar to the proposal made by Andr\'e et
al.~\cite{Andre2006}, who also discuss the idea of using a cavity for QND
measurement, although not specifically for the creation of entangled states.

The Rabi frequency is $\Omega = 2g = \mathbf{d}.\mathbf{E}/\hbar$ where $\mathbf{d}$
is the dipole moment of the transition, and $\mathbf{E}$ is the electric field
strength. However, this local electric field is a function of the geometry. We
must calculate it by calculating the field from our particular resonator, and
the molecule's relative position to it. Since this calculation is a little
involved, let's just for now take the conservative coupling estimate from
reference~\cite{Andre2006}, which is for a CaBr positioned
\SI{1}{\micro\meter} from the resonator: $g/2\pi = \SI{40}{\kilo\hertz}$.

A typical superconducting microwave cavity will have a quality factor
$Q\sim10^6$. We can choose the cavity resonance to be similar to the
transition)frequency, so that the cavity loss rate is $\kappa/2\pi =
\omega_0/(2\pi Q) \sim \SI{1}{\kilo\hertz}$. 

The procedure to create entangled states on the chip would be as follows:

\begin{enumerate}
  \item Load CaF molecules in $\ket{g}$ into the magnetic trap close to the
    resonator
  \item Use external microwaves (from e.g. a horn) to transfer the molecules
    into CSS $(\ket{e} + \ket{g})/\sqrt{2}$
  \item Bias the trap so as to achieve large molecule-cavity detuning
  \item Weakly probe the cavity to identify the transmission peak
  \item Determine $N$, which tells us something about the nature of the
    squeezed state that has been produced
  \item Un-bias the qubit for precision quantum measurement or other operation
\end{enumerate}

\section{Conclusion}

This document describes how spin squeezed states could be created in CaF
molecules coupled to a microwave resonator on a chip. This is mainly based on
the idea of how QND measurements can be performed in this architecture, as was
presented by Andr\'e et al.~\cite{Andre2006}, but there remain a number of
unanswered questions. For example, I have not addressed losses such as those
that can occur from the enhanced stimulated emission into the cavity mode.
There is also the question of the number of molecules that we would be required
to trap to reliably produce such states.

It would also be interesting to consider the nature of the spin states that
could be produced. What is the maximum squeezing that could occur, and how
exactly would the squeezed state relate to the measured value $M$?

The most pertinent question in my mind is how can one control the detuning
$\Delta$? In our case we have chosen magnetically insensitive states for our
qubit, which is useful for long lifetimes, but for the purpose of performing
QND measurements we will have to detune, either by magnetically biasing the
trap beyond the linear Zeeman shift, or applying some other perturbation
such as an electric field.

%\section{Further thoughts: collision squeezing}
%
%In addition to squeezing by QND measurement, atomic spin squeeze states have
%been generated due to inter-atomic collisions \cite{}. These collisions
%introduce a non-linear Hamiltonian term $H=\chi S_z^2$, which leads to a
%shearing of the coherent state around the Bloch sphere around the $S_z$ state
%\cite{}.
%
%With our knowledge of collisions between atoms and molecules (or molecules and
%molecules) might it be possible to generate squeezed states in atom-molecule
%clouds, or very dense clouds of molecules? The latter could perhaps be explored
%on the new CaF Bose-Einstein condensate apparatus.

\printbibliography

\end{document}
